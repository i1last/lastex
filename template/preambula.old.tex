\documentclass[14pt]{extarticle}
\usepackage[left=3cm,right=1cm,top=2cm,bottom=2cm]{geometry} % Настройка полей

% --- Основные пакеты ---
\usepackage[russian]{babel} % Поддержка русского языка (переносы, названия)
\usepackage{fontspec}       % Для работы со шрифтами в LuaLaTeX/XeLaTeX. Заменяет inputenc и fontenc.

\usepackage{amsmath}
\usepackage{amsfonts}
\usepackage{amssymb}
\usepackage{graphicx}
\usepackage{float}          % Для управления положением плавающих объектов
\usepackage[toc,page]{appendix}
\usepackage{pdflscape}
\usepackage{multirow}

% --- Настройка шрифтов ---
% \setmainfont{Times New Roman} % Установка основного шрифта документа
\setmainfont{TeX Gyre Termes} % Свободный неотличимый от оригинала аналог Times New Roman
\linespread{1.25}             % Межстрочный интервал

% Подписи к рисункам и таблицам
\usepackage{caption}
\captionsetup[figure]{textfont=it, labelsep=period, justification=centering , singlelinecheck=false, format=plain}
\captionsetup[table] {textfont=it, labelsep=period, justification=raggedleft, singlelinecheck=false, format=plain, skip=0pt}

% Настраиваем заголовки
\usepackage{titlesec}
%titleformat{<команда>}{<стиль>}{<номер>}{<отступ>}{<текст до>}
\titleformat{\section}{\bfseries\normalsize}{\thesection.}{1em}{}
\titleformat{\subsection}{\bfseries\normalsize}{\thesubsection.}{1em}{}
\titleformat{\subsubsection}{\bfseries\normalsize}{\thesubsubsection.}{1em}{}
\newcommand{\centeredsection}[1]{%
    \noindent
    % \vspace{1em}
    \begin{center}
        \textbf{\normalsize #1}
    \end{center}
    \par
    % \vspace{1em}
}

% Работа с таблицами
\usepackage{tabularx, array, makecell, booktabs}
\renewcommand{\tabularxcolumn}[1]{m{#1}}
\newcolumntype{C}{>{\centering\arraybackslash}X}
\newcolumntype{L}{>{\arraybackslash}X}
\newcolumntype{R}{>{\raggedleft\arraybackslash}X}

% Работа со ссылками и гиперссылками
\usepackage{hyperref}
\hypersetup{
    colorlinks=true,
    linkcolor=black,
    urlcolor=blue
}


% Пояснения в колонтитулум
\usepackage{enotez}
\makeatletter % Создание нового стиля для сносок на странице
\def\enotez@endnotes@footer{%
    \begin{center}
        \rule{0.5\linewidth}{0.4pt}
        \enotez@theendnotes
    \end{center}
}


% Красная строка
\usepackage{indentfirst}
\setlength{\parindent}{1cm}


% Содержание
\usepackage{tocloft}
\renewcommand{\cftsecleader}{\hfil} % Убираем точки между заголовком и номером страницы
\renewcommand{\cftsubsecleader}{\hfil}
\renewcommand{\cfttoctitlefont}{\normalfont\bfseries\Large\centering}
\renewcommand{\cfttoctitlefont}{\normalfont\Large\bfseries\centering} % Делаем заголовок оглавления без жирного шрифта
\renewcommand{\contentsname}{{\large\uppercase{СОДЕРЖАНИЕ}}}
\renewcommand{\cftsecfont}{\normalfont} % Убираем жирный шрифт с разделов и номеров страниц
\renewcommand{\cftsecpagefont}{\normalfont}
\renewcommand{\cftsubsecfont}{\normalfont}
\renewcommand{\cftsubsecpagefont}{\normalfont}


\usepackage{etoolbox}
\usepackage{lipsum}
\newcommand{\customfourteenpt}{\fontsize{14pt}{16.8pt}\selectfont}
\AtBeginDocument{\customfourteenpt}

% Разрешаем ставить больее длинные пробелы, если иначе не выходит сделать
% более верное разбиение абзаца на строки
\sloppy


% Полезные ресурсы
% https://open-resource.ru/spisok-literatury/
% https://samolisov.blogspot.com/2008/06/latex_09.html