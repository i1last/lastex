\section*{Цель работы}
Исследование широкополосного импульсного усилителя с корректирующими цепями, позволяющими улучшить его амплитудно-частотную характеристику (АЧХ).

\section*{Теоретические сведения}
При усилении широкополосных сигналов, к которым относятся импульсные сигналы, критически важным является сохранение соотношений между амплитудами гармоник спектра и фазовых соотношений. Искажения формы прямоугольных видеоимпульсов характеризуются длительностью фронта $\tau_\text{ф}$ и спадом вершины $\Delta U$.

Длительность фронта, приобретаемого импульсом при прохождении через усилитель, обратно пропорциональна верхней граничной частоте $f_\text{в.гр.}$ АЧХ усилителя:
\begin{equation}
    \tau_\text{ф} = \frac{0,35}{f_\text{в.гр.}}.
    \label{eq:tau_f}
\end{equation}

Спад вершины импульса $\Delta U$ (в процентах от амплитуды $U_m$) (\cref{fig:graph}) связан с нижней граничной частотой $f_\text{н.гр.}$ и длительностью импульса $\tau_\text{и}$:
\begin{equation}
    \frac{\Delta U}{U_m} (\%) = 628 \cdot \tau_\text{и} \cdot f_\text{н.гр.}.
    \label{eq:delta_u}
\end{equation}

\begin{figure}[hbt]
    \centering
    \includegraphics[width=0.5\textwidth]{figs/graph.png}
    \caption{Спад вершини импульса}
    \label{fig:graph}
\end{figure}

Для расширения полосы пропускания применяются методы коррекции АЧХ.

\paragraph{Низкочастотная коррекция (НЧК).}
Осуществляется разделением коллекторного сопротивления $R_k$ на два ($R_{k1}$ и $R_{k2}$), где средняя точка через емкость $C_\text{ф}$ соединяется с землей.
\begin{itemize}
    \item На низких частотах $C_\text{ф}$ имеет большое сопротивление, коэффициент усиления определяется суммой $R_{k1} + R_{k2}$:
    $$ K_U = S(R_{k1} + R_{k2}) $$
    \item На высоких частотах $C_\text{ф}$ шунтирует $R_{k2}$, усиление снижается:
    $$ K_U = S R_{k1} $$
\end{itemize}
Это позволяет поднять усиление в области НЧ, компенсируя спад.

\paragraph{Высокочастотная коррекция.}
Реализуется двумя основными способами:
\begin{enumerate}
    \item \textit{Индуктивная коррекция (ИВЧК).} Последовательно с $R_k$ включается индуктивность $L$. Сопротивление цепи коллектора возрастает с частотой ($Z = \sqrt{R^2 + (\omega L)^2}$), что компенсирует падение усиления из-за шунтирующего действия паразитных емкостей.
    \item \textit{Эмиттерная коррекция (ЭВЧК).} В цепь эмиттера параллельно резистору обратной связи $R_\text{э}$ включается конденсатор небольшой емкости $C_\text{э}$. На высоких частотах $C_\text{э}$ шунтирует $R_\text{э}$, уменьшая глубину отрицательной обратной связи и повышая коэффициент усиления.
\end{enumerate}

\section*{Описание лабораторной установки}
Схема лабораторного макета приведена на \cref{fig:scheme}. Установка включает усилитель на транзисторе VT1 и эмиттерный повторитель на VT2. Коммутация элементов коррекции осуществляется перемычками на лицевой панели.

\begin{figure}[H]
    \centering
    \includegraphics[width=0.8\linewidth]{figs/scheme.png}
    \caption{Принципиальная схема лабораторного макета}
    \label{fig:scheme}
\end{figure}

\section*{Порядок выполнения работы}

\begin{enumerate}
    \item \textit{Подготовка лабораторной установки к работе.}
    \begin{itemize}
        \item \textit{Что сделать:} Подать питание на макет и подготовить измерительные приборы.
        \item \textit{Как сделать:}
        \begin{itemize}
            \item Подключите выходы источника питания $+15$ В и $-15$ В к соответствующим клеммам питания на лабораторном макете. Общую точку источника подключите к клемме «Земля».
            \item Включите генератор сигналов, осциллограф и вольтметры переменного тока.
        \end{itemize}
        \item \textit{Нюансы:} Перед подключением генератора к макету убедитесь, что амплитуда выходного сигнала генератора установлена на минимум. Это предотвратит пробой входных цепей транзистора.
    \end{itemize}

    \item \textit{Исследование АЧХ усилителя без коррекции.}
    \begin{itemize}
        \item \textit{Что сделать:} Собрать базовую схему усилителя и снять зависимость выходного напряжения от частоты.
        \item \textit{Как сделать:}
        \begin{itemize}
            \item С помощью перемычек на лицевой панели соберите схему усилителя, отключив все корректирующие цепи ($C_3$, $L_1/L_2$, $C_\text{ф}$).
            \item Подключите выход генератора ко входу макета.
            \item Подключите первый вольтметр параллельно входу макета для контроля напряжения $U_\text{вх}$.
            \item Подключите осциллограф (или высокочастотный вольтметр) к выходу макета для измерения напряжения $U_\text{вых}$.
            \item На генераторе выберите форму сигнала «Синусоида». Установите частоту $1$ кГц.
            \item Установите входное напряжение $U_\text{вх}$ (например, $0,1$ В). Зафиксируйте это значение: оно должно оставаться строго неизменным на протяжении всех пунктов измерения АЧХ.
            \item Изменяйте частоту генератора согласно списку в первом столбце \cref{tab:afc_data} (от $20$ Гц до $2$ МГц).
            \item На каждой частоте фиксируйте значение $U_\text{вых}$ и записывайте его в столбец «Без коррекции» \cref{tab:afc_data}.
        \end{itemize}
        \item \textit{Нюансы:} Обычные мультиметры имеют резкий спад АЧХ после $10 \dots 100$ кГц. Для измерения $U_\text{вых}$ на высоких частотах обязательно используйте осциллограф, измеряя размах сигнала от пика до пика ($V_{pp}$) и пересчитывая его в действующее значение, либо оперируйте амплитудными значениями для расчетов.
    \end{itemize}

    \item \textit{Исследование влияния сопротивления $R_3$.}
    \begin{itemize}
        \item \textit{Что сделать:} Снять АЧХ при включенном сопротивлении $R_3$.
        \item \textit{Как сделать:}
        \begin{itemize}
            \item Не изменяя настроек генератора, установите перемычку, вводящую в цепь эмиттера сопротивление $R_3$.
            \item Повторите процесс изменения частоты от $20$ Гц до $2$ МГц.
            \item Занесите показания $U_\text{вых}$ в столбец «С $R_3$» \cref{tab:afc_data}.
        \end{itemize}
        \item \textit{Нюансы:} При введении $R_3$ возникает отрицательная обратная связь. Общий уровень выходного сигнала снизится на всех частотах.
    \end{itemize}

    \item \textit{Исследование эмиттерной высокочастотной коррекции (ЭВЧК).}
    \begin{itemize}
        \item \textit{Что сделать:} Снять АЧХ с цепью ЭВЧК.
        \item \textit{Как сделать:}
        \begin{itemize}
            \item Оставьте сопротивление $R_3$ включенным.
            \item Установите перемычку, подключающую конденсатор $C_3$ параллельно резистору $R_3$.
            \item Проведите измерения $U_\text{вых}$ в диапазоне $20$ Гц \dots $2$ МГц.
            \item Занесите данные в столбец «ЭВЧК» \cref{tab:afc_data}.
        \end{itemize}
        \item \textit{Нюансы:} Особое внимание уделите точкам в высокочастотной области ($100$ кГц и выше). Вы должны зафиксировать подъем коэффициента усиления относительно предыдущего опыта.
    \end{itemize}

    \item \textit{Исследование индуктивной высокочастотной коррекции (ИВЧК).}
    \begin{itemize}
        \item \textit{Что сделать:} Снять АЧХ с цепью ИВЧК.
        \item \textit{Как сделать:}
        \begin{itemize}
            \item Отключите эмиттерную коррекцию (снимите перемычку с $C_3$ и $R_3$).
            \item Установите перемычку, включающую индуктивность $L_1$ в коллекторную цепь транзистора.
            \item Проведите измерения $U_\text{вых}$ в диапазоне частот от $20$ кГц до $2$ МГц.
            \item Занесите данные в столбец «ИВЧК» \cref{tab:afc_data}.
        \end{itemize}
        \item \textit{Нюансы:} Измерения на частотах ниже $20$ кГц не требуются, так как индуктивность на низких частотах эквивалентна короткому замыканию. Внимательно пройдите область $1 \dots 2$ МГц, возможен резонансный пик.
    \end{itemize}

    \item \textit{Исследование низкочастотной коррекции (НЧК).}
    \begin{itemize}
        \item \textit{Что сделать:} Снять АЧХ с цепью НЧК.
        \item \textit{Как сделать:}
        \begin{itemize}
            \item Отключите ИВЧК (снимите перемычку с $L_1$).
            \item Установите перемычки, формирующие цепь НЧК (разделение коллекторного сопротивления и подключение конденсатора $C_\text{ф}$ к земле).
            \item Проведите измерения $U_\text{вых}$ в диапазоне частот от $20$ Гц до $20$ кГц.
            \item Занесите данные в столбец «НЧК» \cref{tab:afc_data}.
        \end{itemize}
        \item \textit{Нюансы:} Измерения на частотах выше $20$ кГц не требуются. Ожидается подъем АЧХ в области десятков герц.
    \end{itemize}

    \item \textit{Исследование формы импульсов в области низких частот (Спад вершины).}
    \begin{itemize}
        \item \textit{Что сделать:} Измерить спад плоской вершины прямоугольного импульса.
        \item \textit{Как сделать:}
        \begin{itemize}
            \item Переключите генератор в режим формирования прямоугольных импульсов (меандр).
            \item Установите период следования импульсов $T$ в диапазоне от $3$ до $5$ мс.
            \item Соберите схему без коррекции (все перемычки сняты).
            \item По осциллограмме измерьте амплитуду импульса $U_m$ и абсолютную величину спада вершины $\Delta U$. Занесите значения в первую строку блока «Низкие частоты» \cref{tab:pulse_data}. Зарисуйте осциллограмму.
            \item Введите в схему низкочастотную коррекцию (НЧК).
            \item Повторите измерения $U_m$ и $\Delta U$. Занесите значения во вторую строку \cref{tab:pulse_data}. Зарисуйте новую осциллограмму.
        \end{itemize}
        \item \textit{Нюансы:} Для точного измерения малого значения $\Delta U$ переведите канал осциллографа в режим закрытого входа (AC), увеличьте чувствительность (В/дел) и сместите плоскую вершину импульса в центр экрана.
    \end{itemize}

    \item \textit{Исследование формы импульсов в области высоких частот (Фронт импульса).}
    \begin{itemize}
        \item \textit{Что сделать:} Измерить длительность фронта прямоугольного импульса при различных видах ВЧ коррекции.
        \item \textit{Как сделать:}
        \begin{itemize}
            \item Установите на генераторе период следования импульсов $T$ в диапазоне от $8$ до $10$ мкс.
            \item Соберите схему без коррекции. Измерьте длительность фронта $\tau_\text{ф}$ (время нарастания сигнала от уровня $0,1 U_m$ до $0,9 U_m$). Занесите значение в первую строку блока «Высокие частоты» \cref{tab:pulse_data}. Зарисуйте осциллограмму.
            \item Введите индуктивную высокочастотную коррекцию (ИВЧК). Измерьте $\tau_\text{ф}$, занесите в \cref{tab:pulse_data}, зарисуйте осциллограмму.
            \item Отключите ИВЧК и введите эмиттерную высокочастотную коррекцию (ЭВЧК). Повторите измерения $\tau_\text{ф}$, занесите в \cref{tab:pulse_data}, зарисуйте осциллограмму.
        \end{itemize}
        \item \textit{Нюансы:} Для измерения длительности фронта необходимо использовать быструю развертку осциллографа (порядка долей микросекунды на деление). Зафиксируйте наличие колебательных выбросов на вершине импульса при использовании ИВЧК, если они присутствуют.
    \end{itemize}
\end{enumerate}

\section*{Обработка результатов}
\begin{enumerate}
    \item Постройте графики АЧХ для всех исследованных случаев в логарифмическом масштабе по оси частот.
    \item Определите полосу пропускания для каждого типа коррекции.
    \item Рассчитайте теоретические значения $\tau_\text{ф}$ и $\Delta U / U_m$ по формулам \cref{eq:tau_f} и \cref{eq:delta_u}, используя найденные граничные частоты. Сравните с измеренными осциллографом значениями.
\end{enumerate}