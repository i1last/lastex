\section*{Цель работы}
Исследование спектральных зависимостей фотопроводимости полупроводников CdS и CdSe, а также зависимостей фотопроводимости от уровня оптического облучения (световых характеристик). Определение ширины запрещенной зоны исследуемых материалов.

\section*{Теоретические сведения}

\subsection*{Механизм фотопроводимости}
В темноте полупроводник обладает некоторым начальным сопротивлением, которое обусловлено равновесной концентрацией носителей заряда. При освещении материала фотонами с энергией $h\nu$, превышающей ширину запрещенной зоны $\Delta\text{Э}$, происходит переброс электронов из валентной зоны в зону проводимости. 

Этот процесс называется \textit{внутренним фотоэффектом}. В результате генерации дополнительных (неравновесных) носителей заряда общая проводимость полупроводника возрастает, а сопротивление падает. Это явление носит название \textit{фоторезистивного эффекта}.

Измеряемая на свету проводимость $\gamma_\text{с}$ складывается из темновой проводимости $\gamma_\text{т}$ и добавочной фотопроводимости $\gamma_\text{ф}$:
\begin{equation}
    \gamma_\text{с} = \gamma_\text{т} + \gamma_\text{ф}
\end{equation}
Отсюда искомая фотопроводимость вычисляется как разность:
\begin{equation}
    \gamma_\text{ф} = \gamma_\text{с} - \gamma_\text{т}    
\end{equation}

\subsection*{Спектральная характеристика и красная граница}
Фотопроводимость зависит от длины волны падающего света $\lambda$. Эта зависимость имеет характерный колоколообразный вид:
\begin{itemize}
    \item \textit{Длинноволновая область (справа от пика):} Энергия фотонов слишком мала для преодоления запрещенной зоны ($h\nu < \Delta\text{Э}$). Фотопроводимость стремится к нулю.
    \item \textit{Коротковолновая область (слева от пика):} Энергия фотонов велика, коэффициент поглощения резко возрастает. Свет поглощается в очень тонком поверхностном слое полупроводника. Из-за большого количества дефектов на поверхности скорость рекомбинации носителей крайне высока, поэтому фотопроводимость спадает.
\end{itemize}

Пороговая длина волны, при которой энергии фотона ровно хватает для ионизации, называется \textit{красной границей фотоэффекта} ($\lambda_\text{пор}$). В идеальном случае график имел бы резкий обрыв, но из-за теплового размытия уровней спад происходит плавно. Поэтому на практике за $\lambda_\text{пор}$ принимают длину волны $\lambda_{1/2}$, при которой фотопроводимость падает до половины от своего максимального значения.

Связь между шириной запрещенной зоны и красной границей выражается формулой:
\begin{equation}
    \Delta\text{Э} = \frac{hc}{\lambda_\text{пор}}
\end{equation}
где $h$ — постоянная Планка, $c$ — скорость света. При подстановке констант в электрон-вольтах и микрометрах формула принимает удобный для расчета вид:
\begin{equation}
    \Delta\text{Э} \approx \frac{1.24}{\lambda_\text{пор}} \text{ эВ}
\end{equation}

\subsection*{Световая характеристика}
Световая характеристика — это зависимость фотопроводимости от интенсивности падающего света (в данной работе интенсивность регулируется шириной щели монохроматора $d$).
При малых интенсивностях концентрация генерируемых носителей прямо пропорциональна световому потоку. При высоких интенсивностях возрастает вероятность встречи электрона и дырки, доминирующим становится процесс рекомбинации, и рост фотопроводимости замедляется.

\section*{Описание экспериментальной установки}
Основой установки является монохроматор (см. \cref{fig:scheme}). Свет от лампы $E$ проходит через входную щель $F_1$, диспергирующее устройство $\Pi$ и выходит через щель $F_2$, попадая на исследуемый образец $R$. Изменение сопротивления образца фиксируется цифровым омметром $PR$.

\begin{figure}[hbt]
    \centering
    \includegraphics[width=0.6\linewidth]{figs/scheme.pdf}
    \caption{Схема установки для исследования фотоэлектрических свойств}
    \label{fig:scheme}
\end{figure}

\section*{Порядок обработки результатов}

\subsection*{Расчет спектральной характеристики}
Для заполнения \cref{tab:spectral_cds} и \cref{tab:spectral_cdse} необходимо выполнить следующие шаги для каждой строки:
\begin{enumerate}
    \item Включить цифровой омметр PR и дать ему прогреться в течение 5 мин. Открыть полностью щель $F_1$, для чего микрометрическим винтом на входе монохроматора установить ширину щели, равную 4 мм. Перед измерениями спектральных характеристик измерить темновое сопротивление обоих образцов. Включить лампу E.
    \item Установить барабан монохроматора на деление (около 600), начиная с которого наблюдается снижение сопротивления исследуемого образца. Изменяя положение диспергирующего устройства П поворотом барабана от 600 до 3500 делений, измерять установившееся значение сопротивления первого полупроводника $CdS$ на свету $R_\text{с}$ через каждые 100 делений барабана. (Значения $\lambda$ и $\text{Э}_\lambda$ определяются по градуировочной таблице монохроматора (выдается преподавателем или содержится в методичке --- табл. 3.2. на стр. 28) на основе показаний барабана.)
    \item Проводимость на свету вычисляется как обратное сопротивление:
    $$\gamma_\text{с} = \frac{1}{R_\text{с}}$$
    \textit{Внимание к размерностям:} если $R_\text{с}$ измерено в МОм ($10^6$ Ом), то $\gamma_\text{с}$ будет получено в мкСм ($10^{-6}$ См).
    \item Темновая проводимость вычисляется аналогично: $\gamma_\text{т} = 1 / R_\text{т}$.
    \item Фотопроводимость: $\gamma_\text{ф} = \gamma_\text{с} - \gamma_\text{т}$.
    \item Приведенная фотопроводимость (учитывает неравномерность энергии излучения лампы по спектру):
    $$\gamma'_\text{ф} = \frac{\gamma_\text{ф}}{\text{Э}_\lambda}$$
    \item Относительная фотопроводимость: $\gamma'_\text{ф} / \gamma'_{\text{ф}\max}$, где $\gamma'_{\text{ф}\max}$ — максимальное значение приведенной фотопроводимости в столбце.
\end{enumerate}

По полученным данным строятся графики зависимости $\gamma'_\text{ф} / \gamma'_{\text{ф}\max} = f(\lambda)$. На графике необходимо найти пик (значение $1.0$ по оси ординат), спуститься по правому (длинноволновому) склону до уровня $0.5$ и определить соответствующую абсциссу. Это значение и есть $\lambda_\text{пор}$. По нему рассчитывается $\Delta\text{Э}$.

\subsection*{Расчет световой характеристики}
Для заполнения \cref{tab:intensity}:
\begin{enumerate}
    \item Установить барабан монохроматора в положение, соответствующее минимальному значению сопротивления полупроводника. Микрометрический винт, регулирующий ширину щели монохроматора, поставить на нуль. Изменяя положение микрометрического винта от нуля до 4 мм, измерять установившиеся значения сопротивлений $R_\text{c}$.
    \item Проводимости $\gamma_\text{с}$ и $\gamma_\text{ф}$ рассчитываются аналогично предыдущему пункту.
    \item Вычисляется отношение $d/d_{\max}$, где $d_{\max} = 4.0$ мм.
\end{enumerate}

По результатам строится график в логарифмических координатах: $\lg \gamma_\text{ф}$ от $\lg(d/d_{\max})$. Логарифмический масштаб позволяет визуально разделить линейный и сублинейный участки рекомбинации.