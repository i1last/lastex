\begin{table}[H]
    \centering
    \caption{Зависимость сопротивления полупроводниковых материалов от температуры}
    \label{tab:semiconductors-src}
    \begin{tabularx}{\linewidth}{ C C C C C }
        $t, ^\circ\text{C}$ & $R_{Si}, \text{Ом}$ & $R_{Ge}, \text{Ом}$ & $R_{SiC}, \text{Ом}$ & $R_{InSb}, \text{Ом}$ \\
        \midrule
        25  & 110,7 & 292   & 4874 & 59,2 \\
        35  & 114,4 & 303   & 4556 & --- \\
        45  & 117,5 & 319   & 4205 & 53,8 \\
        55  & 120,7 & 326   & 3722 & 50,7 \\
        65  & 123,8 & 339   & 3090 & 47,8 \\
        75  & 127,9 & 328   & 2600 & 44,3 \\
        85  & 131,0 & 324   & 2225 & 38,0 \\
        95  & 134,3 & 312   & 1965 & 35,5 \\
        105 & 138,1 & 285   & 1658 & 33,8 \\
        115 & 141,8 & 262,2 & 1462 & 34,9 \\
        127 & 145,0 & 211   & 1142 & 33,8 \\
    \end{tabularx}
\end{table}

\begin{table}[H]
    \centering
    \caption{Геометрические параметры образцов и температурные зависимости подвижности}
    \label{tab:setup_parameters}
    \begin{tabularx}{\linewidth}{ L c c L }
        Материал & $L$, см & $S$, мм$^2$ & Температурная зависимость подвижности $\mu(T)$, м$^2$/(В$\cdot$с) \\
        \midrule
        Si (Кремний) & 3,0 & 0,2 & $\mu_n = 0,15 \cdot (T/300)^{-2,5}$ \\
        \addlinespace
        Ge (Германий) & 3,0 & 0,2 & $\mu_n = 0,39 \cdot (T/300)^{-1,66}$ \\
        \addlinespace
        SiC (Карбид кремния) & 1,0 & 1,2 & $\mu_n = 0,01 \cdot (T/300)^{-1}$ \\
        \addlinespace
        InSb (Антимонид индия) & 2,0 & 0,1 & $\begin{aligned} \mu_n &= 7,8 \cdot (T/300)^{-1,6} \\ \mu_p &= 0,075 \cdot (T/300)^{-2,1} \end{aligned}$ \\
    \end{tabularx}
\end{table}