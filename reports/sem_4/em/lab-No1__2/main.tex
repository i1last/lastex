\section*{Цель}
Сравнение температурных зависимостей сопротивления полупроводников с различной шириной запрещенной зоны; определение ширины запрещенной зоны и энергии ионизации легирующих примесей в материалах.

\section*{Теоретические сведения}

Электрофизические свойства полупроводников определяются концентрацией носителей заряда и их подвижностью. В собственных полупроводниках концентрация электронов $n_i$ и дырок $p_i$ совпадает. Температурная зависимость собственной концентрации описывается выражением:
\begin{equation}
    n_i = p_i \sim \exp\left( - \frac{\Delta \text{Э}}{2kT} \right),
    \label{eq:ni_temp}
\end{equation}
где $\Delta \text{Э}$ — ширина запрещенной зоны, $k$ — постоянная Больцмана, $T$ — абсолютная температура.

В примесных полупроводниках при низких температурах концентрация носителей определяется ионизацией примесей. Для $n$-типа концентрация электронов $n_{\text{пр}}$ зависит от энергии ионизации доноров $\Delta \text{Э}_{\text{пр}}$:
\begin{equation}
    n_{\text{пр}} \sim \exp\left( - \frac{\Delta \text{Э}_{\text{пр}}}{2kT} \right).
    \label{eq:n_impurity}
\end{equation}

Удельная электрическая проводимость $\gamma$ связана с концентрацией $n$ и подвижностью $\mu$ носителей заряда соотношением:
\begin{equation}
    \gamma = q n \mu,
    \label{eq:conductivity}
\end{equation}
где $q$ — элементарный заряд.

Температурная зависимость проводимости $\ln \gamma = f(1/T)$ позволяет выделить области примесной проводимости (низкие температуры), истощения примеси (плато) и собственной проводимости (высокие температуры). Наклон линейных участков графика в координатах $\ln n$ от $1/T$ пропорционален энергиям активации:
$$ |\tg \alpha| \sim \frac{\Delta \text{Э}}{2k}. $$

\section*{Описание экспериментальной установки и образцов}

Исследование проводится на установке, включающей термостат с исследуемыми образцами и измерительный модуль.
\begin{itemize}
    \item Образцы: Имеют форму параллелепипедов длиной $l$ и поперечным сечением $S$.
    \item Контакты: Два омических контакта на торцах для подключения к омметру.
    \item Термометрия: Температура контролируется термопарой, шкала проградуирована в градусах Цельсия.
    \item Нагрев: Регулируется переключателем ступеней нагрева.
\end{itemize}

Для расчетов используются справочные параметры полупроводниковых материалов, приведенные в \cref{tab:semiconductors}.

\begin{table}[hbt]
    \centering
    \caption{Справочные параметры полупроводников (при $T=300$ К)}
    \label{tab:semiconductors}
    \begin{tabularx}{\linewidth}{ L C C C C C }
        \toprule
        Материал & $\Delta \text{Э}$, эВ & $\mu_n$, \newline $\text{м}^2/(\text{В}\cdot\text{с})$ & $\mu_p$, \newline $\text{м}^2/(\text{В}\cdot\text{с})$ & $N_c \cdot 10^{-25}$, \newline $\text{м}^{-3}$ & $N_\nu \cdot 10^{-25}$, \newline $\text{м}^{-3}$ \\
        \midrule
        Si  & 1,12 & 0,13 & 0,05  & 2,74 & 1,05 \\
        Ge  & 0,66 & 0,39 & 0,19  & 1,02 & 0,61 \\
        InSb& 0,18 & 7,8  & 0,075 & $3,7 \cdot 10^{-3}$ & 0,63 \\
        SiC & 2,90 & 0,04 & 0,006 & 1,44 & 1,93 \\
        \bottomrule
    \end{tabularx}
\end{table}

\section*{Порядок выполнения расчетов}

\subsection*{Обработка экспериментальных данных}

1.  Расчет удельного сопротивления. Для каждой температурной точки рассчитать $\rho$ по измеренному сопротивлению $R$:
    \begin{equation}
        \rho = \frac{R \cdot S}{l}.
        \label{eq:rho_calc}
    \end{equation}

2.  Расчет проводимости. Вычислить экспериментальную удельную проводимость:
    $$ \gamma_{\text{эксп}} = 1 / \rho. $$

3.  Построение графика. Построить зависимость $\ln \gamma_{\text{эксп}} = f(10^3/T)$, где $T$ — температура в Кельвинах.

\subsection*{Расчет параметров полупроводников}

1.  Теоретическая оценка при 300 К. Рассчитать собственную концентрацию носителей $n_i$ и собственную проводимость $\gamma_i$ для всех материалов:
    \begin{equation}
        n_i = \sqrt{N_c N_v} \exp\left( - \frac{\Delta \text{Э}}{2kT} \right),
        \label{eq:ni_300}
    \end{equation}
    \begin{equation}
        \gamma_i = q n_i (\mu_n + \mu_p).
        \label{eq:gamma_i}
    \end{equation}
    Сравнить полученное значение $\gamma_i$ с экспериментальным $\gamma_{\text{эксп}}$ при 300 К для определения типа проводимости (собственная или примесная).

2.  Расчет концентрации носителей. Используя экспериментальные значения проводимости $\gamma_{\text{эксп}}$, рассчитать концентрацию носителей $n_{\text{эксп}}$ для всех температурных точек:
    \begin{equation}
        n_{\text{эксп}} = \frac{\gamma_{\text{эксп}}}{q(\mu_n + \mu_p)}.
        \label{eq:n_exp}
    \end{equation}

3.  Определение энергии ионизации примеси ($\Delta \text{Э}_{\text{пр}}$). Если на графике наблюдается низкотемпературный участок роста проводимости (область неполной ионизации), рассчитать $\Delta \text{Э}_{\text{пр}}$ по двум точкам ($T_1, n_1$) и ($T_2, n_2$) на этом участке:
    \begin{equation}
        \Delta \text{Э}_{\text{пр}} = 2k \frac{T_2 T_1}{T_2 - T_1} \ln \frac{n(T_2)}{n(T_1)}.
        \label{eq:delta_E_pr}
    \end{equation}

4.  Определение ширины запрещенной зоны ($\Delta \text{Э}$). Для высокотемпературного участка (область собственной проводимости) рассчитать $\Delta \text{Э}$ с учетом температурной зависимости эффективной плотности состояний ($N_c, N_v \sim T^{3/2}$):
    \begin{equation}
        \Delta \text{Э} = 2k \frac{T_2 T_1}{T_2 - T_1} \left[ \ln \frac{n(T_2)}{n(T_1)} - \frac{3}{2} \ln \frac{T_2}{T_1} \right].
        \label{eq:delta_E_bandgap}
    \end{equation}

\section*{Задание}
Необходимо определить температурные диапазоны реализации различных механизмов проводимости (ионизация примеси, истощение примеси, собственная проводимость) для каждого материала, основываясь на графиках $\ln \gamma_{\text{эксп}}(1/T)$.

\newpage
\centeredsection{ОБРАБОТКА РЕЗУЛЬТАТОВ}
\subsection*{ \boxed{\text{ Задание 1. }} }
\begin{quote}
    Рассчитать удельное сопротивление исследуемых полупроводниковых материалов по экспериментальным данным для каждой температурной точки \cref{tab:semiconductors-src} по формуле $\rho = RS/l$, где $R$ --- сопротивление образца; $S$ --- площадь поперечного сечения образца; $l$ --- длина образца. Вычислить соответствующие удельные проводимости образцов по формуле $\gamma_\text{эксп} = 1/\rho$. Результаты занести в \cref{tab:semiconductors-results-si,tab:semiconductors-results-ge,tab:semiconductors-results-sic,tab:semiconductors-results-insb}.
\end{quote}

Приведем вычисления для кремния при $25^\circ$C:
\begin{gather*}
    \rho = \frac{110.7 \text{ Ом} \cdot 0.2 \text{ мм}^2}{3 \text{ см}} =
    \frac{110.7 \text{ Ом} \cdot 0.2 \cdot 10^{-6} \text{ м}^2}{0.03 \text{ м}} =
    0.000738 \text{ Ом}\cdot\text{м}
    \\
    \gamma_\text{эксп} = \frac{1}{0.000738} \approx 1355 \text{ См}/\text{м}
\end{gather*}

\begin{table}[H]
    \centering
    \caption{Удельные сопротивления и проводимости кремния}
    \label{tab:semiconductors-results-si}
    \begin{tblr}{
    colspec = { l *{6}{X[c]} },
    rows = {m},
    }
        Материал & {$T$,\\ К} & {$T^{-1}\cdot 10^3$,\\ К${}^{-1}$} & {$R$, \\Ом} & {$\rho,\\ \text{Ом}\cdot\text{м}$} & {$\gamma_{\text{эксп}},\\ \text{См}/\text{м}$} & {$\ln\gamma_{\text{эксп}}$} \\
        \midrule
        {Si} 
        & 298 & 3,36 & 110,7 & 0,000738 & 1355,0 & 7,21 \\
        & 308 & 3,25 & 114,4 & 0,000763 & 1311,2 & 7,18 \\
        & 318 & 3,14 & 117,5 & 0,000783 & 1276,6 & 7,15 \\
        & 328 & 3,05 & 120,7 & 0,000805 & 1242,8 & 7,13 \\
        & 338 & 2,96 & 123,8 & 0,000825 & 1211,6 & 7,10 \\
        & 348 & 2,87 & 127,9 & 0,000853 & 1172,8 & 7,07 \\
        & 358 & 2,79 & 131,0 & 0,000873 & 1145,0 & 7,04 \\
        & 368 & 2,72 & 134,3 & 0,000895 & 1116,9 & 7,02 \\
        & 378 & 2,65 & 138,1 & 0,000921 & 1086,2 & 6,99 \\
        & 388 & 2,58 & 141,8 & 0,000945 & 1057,8 & 6,96 \\
        & 400 & 2,50 & 145,0 & 0,000967 & 1034,5 & 6,94 \\
    \end{tblr}
\end{table}

\begin{table}[H]
    \centering
    \caption{Удельные сопротивления и проводимости германия}
    \label{tab:semiconductors-results-ge}
    \begin{tblr}{
    colspec = { l *{6}{X[c]} },
    rows = {m},
    }
        Материал & {$T$,\\ К} & {$T^{-1}\cdot 10^3$,\\ К${}^{-1}$} & {$R$, \\Ом} & {$\rho,\\ \text{Ом}\cdot\text{м}$} & {$\gamma_{\text{эксп}},\\ \text{См}/\text{м}$} & {$\ln\gamma_{\text{эксп}}$} \\
        \midrule
        {Ge} 
        & 298 & 3,36 & 292 & 0,001947 & 513,7 & 6,24 \\
        & 308 & 3,25 & 303 & 0,002020 & 495,0 & 6,20 \\
        & 318 & 3,14 & 319 & 0,002127 & 470,2 & 6,15 \\
        & 328 & 3,05 & 326 & 0,002173 & 460,1 & 6,13 \\
        & 338 & 2,96 & 339 & 0,002260 & 442,5 & 6,09 \\
        & 348 & 2,87 & 328 & 0,002187 & 457,3 & 6,13 \\
        & 358 & 2,79 & 324 & 0,002160 & 463,0 & 6,14 \\
        & 368 & 2,72 & 312 & 0,002080 & 480,8 & 6,18 \\
        & 378 & 2,65 & 285 & 0,001900 & 526,3 & 6,27 \\
        & 388 & 2,58 & 262,2 & 0,001748 & 572,1 & 6,35 \\
        & 400 & 2,50 & 211 & 0,001407 & 710,9 & 6,57 \\
        
    \end{tblr}
\end{table}

\begin{table}[H]
    \centering
    \caption{Удельные сопротивления и проводимости карбида кремния}
    \label{tab:semiconductors-results-sic}
    \begin{tblr}{
    colspec = { l *{6}{X[c]} },
    rows = {m},
    }
        Материал & {$T$,\\ К} & {$T^{-1}\cdot 10^3$,\\ К${}^{-1}$} & {$R$, \\Ом} & {$\rho,\\ \text{Ом}\cdot\text{м}$} & {$\gamma_{\text{эксп}},\\ \text{См}/\text{м}$} & {$\ln\gamma_{\text{эксп}}$} \\
        \midrule
        {SiC} 
        & 298 & 3,36 & 4874 & 0,5849 & 1,71 & 0,54 \\
        & 308 & 3,25 & 4556 & 0,5467 & 1,83 & 0,60 \\
        & 318 & 3,14 & 4205 & 0,5046 & 1,98 & 0,68 \\
        & 328 & 3,05 & 3722 & 0,4466 & 2,24 & 0,81 \\
        & 338 & 2,96 & 3090 & 0,3708 & 2,70 & 0,99 \\
        & 348 & 2,87 & 2600 & 0,3120 & 3,21 & 1,17 \\
        & 358 & 2,79 & 2225 & 0,2670 & 3,75 & 1,32 \\
        & 368 & 2,72 & 1965 & 0,2358 & 4,24 & 1,44 \\
        & 378 & 2,65 & 1658 & 0,1990 & 5,03 & 1,62 \\
        & 388 & 2,58 & 1462 & 0,1754 & 5,70 & 1,74 \\
        & 400 & 2,50 & 1142 & 0,1370 & 7,30 & 1,99 \\
    \end{tblr}
\end{table}

\begin{table}[H]
    \centering
    \caption{Удельные сопротивления и проводимости антимонида индия}
    \label{tab:semiconductors-results-insb}
    \begin{tblr}{
    colspec = { l *{6}{X[c]} },
    rows = {m},
    }
        Материал & {$T$,\\ К} & {$T^{-1}\cdot 10^3$,\\ К${}^{-1}$} & {$R$, \\Ом} & {$\rho,\\ \text{Ом}\cdot\text{м}$} & {$\gamma_{\text{эксп}},\\ \text{См}/\text{м}$} & {$\ln\gamma_{\text{эксп}}$} \\
        \midrule
        {InSb} 
        & 298 & 3,36 & 59,2 & 0,000296 & 3378,4 & 8,13 \\
        & 318 & 3,14 & 53,8 & 0,000269 & 3717,5 & 8,22 \\
        & 328 & 3,05 & 50,7 & 0,000254 & 3944,8 & 8,28 \\
        & 338 & 2,96 & 47,8 & 0,000239 & 4184,1 & 8,34 \\
        & 348 & 2,87 & 44,3 & 0,000222 & 4514,7 & 8,42 \\
        & 358 & 2,79 & 38,0 & 0,000190 & 5263,2 & 8,57 \\
        & 368 & 2,72 & 35,5 & 0,000178 & 5633,8 & 8,64 \\
        & 378 & 2,65 & 33,8 & 0,000169 & 5917,2 & 8,69 \\
        & 388 & 2,58 & 34,9 & 0,000175 & 5730,7 & 8,65 \\
        & 400 & 2,50 & 33,8 & 0,000169 & 5917,2 & 8,69 \\
    \end{tblr}
\end{table}

\subsection*{ \boxed{\text{ Задание 2. }} }
\begin{quote}
    По данным \cref{tab:semiconductors-results-si,tab:semiconductors-results-ge,tab:semiconductors-results-sic,tab:semiconductors-results-insb} построить температурные зависимости удельной проводимости полупроводников, откладывая по оси абсцисс параметр $T^{-1}$, а по оси ординат --- экспериментальные значения $\ln\gamma_{\text{эксп}}$. 

    Зависимости $\ln\gamma_{\text{эксп}} = f(T^{-1})$ для всех исследованных полупроводниковых материалов привести на одном графике. 
\end{quote}

\begin{figure}[H]
    \centering
    %% Creator: Matplotlib, PGF backend
%%
%% To include the figure in your LaTeX document, write
%%   \input{<filename>.pgf}
%%
%% Make sure the required packages are loaded in your preamble
%%   \usepackage{pgf}
%%
%% Also ensure that all the required font packages are loaded; for instance,
%% the lmodern package is sometimes necessary when using math font.
%%   \usepackage{lmodern}
%%
%% Figures using additional raster images can only be included by \input if
%% they are in the same directory as the main LaTeX file. For loading figures
%% from other directories you can use the `import` package
%%   \usepackage{import}
%%
%% and then include the figures with
%%   \import{<path to file>}{<filename>.pgf}
%%
%% Matplotlib used the following preamble
%%   \def\mathdefault#1{#1}
%%   \everymath=\expandafter{\the\everymath\displaystyle}
%%   \IfFileExists{scrextend.sty}{
%%     \usepackage[fontsize=14.000000pt]{scrextend}
%%   }{
%%     \renewcommand{\normalsize}{\fontsize{14.000000}{16.800000}\selectfont}
%%     \normalsize
%%   }
%%   \usepackage{fontspec}\setmainfont{Times New Roman}\usepackage[english,russian]{babel}\usepackage{amsmath}
%%   \ifdefined\pdftexversion\else  % non-pdftex case.
%%     \usepackage{fontspec}
%%     \setmainfont{times.ttf}[Path=\detokenize{/usr/local/share/fonts/truetype/times-new-roman/}]
%%     \setsansfont{DejaVuSans.ttf}[Path=\detokenize{/usr/share/matplotlib/mpl-data/fonts/ttf/}]
%%     \setmonofont{DejaVuSansMono.ttf}[Path=\detokenize{/usr/share/matplotlib/mpl-data/fonts/ttf/}]
%%   \fi
%%   \makeatletter\@ifpackageloaded{underscore}{}{\usepackage[strings]{underscore}}\makeatother
%%
\begingroup%
\makeatletter%
\begin{pgfpicture}%
\pgfpathrectangle{\pgfpointorigin}{\pgfqpoint{5.680000in}{3.680541in}}%
\pgfusepath{use as bounding box, clip}%
\begin{pgfscope}%
\pgfsetbuttcap%
\pgfsetmiterjoin%
\definecolor{currentfill}{rgb}{1.000000,1.000000,1.000000}%
\pgfsetfillcolor{currentfill}%
\pgfsetlinewidth{0.000000pt}%
\definecolor{currentstroke}{rgb}{1.000000,1.000000,1.000000}%
\pgfsetstrokecolor{currentstroke}%
\pgfsetdash{}{0pt}%
\pgfpathmoveto{\pgfqpoint{0.000000in}{0.000000in}}%
\pgfpathlineto{\pgfqpoint{5.680000in}{0.000000in}}%
\pgfpathlineto{\pgfqpoint{5.680000in}{3.680541in}}%
\pgfpathlineto{\pgfqpoint{0.000000in}{3.680541in}}%
\pgfpathlineto{\pgfqpoint{0.000000in}{0.000000in}}%
\pgfpathclose%
\pgfusepath{fill}%
\end{pgfscope}%
\begin{pgfscope}%
\pgfsetbuttcap%
\pgfsetmiterjoin%
\definecolor{currentfill}{rgb}{1.000000,1.000000,1.000000}%
\pgfsetfillcolor{currentfill}%
\pgfsetlinewidth{0.000000pt}%
\definecolor{currentstroke}{rgb}{0.000000,0.000000,0.000000}%
\pgfsetstrokecolor{currentstroke}%
\pgfsetstrokeopacity{0.000000}%
\pgfsetdash{}{0pt}%
\pgfpathmoveto{\pgfqpoint{0.460969in}{0.576229in}}%
\pgfpathlineto{\pgfqpoint{5.630000in}{0.576229in}}%
\pgfpathlineto{\pgfqpoint{5.630000in}{3.604118in}}%
\pgfpathlineto{\pgfqpoint{0.460969in}{3.604118in}}%
\pgfpathlineto{\pgfqpoint{0.460969in}{0.576229in}}%
\pgfpathclose%
\pgfusepath{fill}%
\end{pgfscope}%
\begin{pgfscope}%
\pgfpathrectangle{\pgfqpoint{0.460969in}{0.576229in}}{\pgfqpoint{5.169031in}{3.027889in}}%
\pgfusepath{clip}%
\pgfsetbuttcap%
\pgfsetroundjoin%
\pgfsetlinewidth{0.501875pt}%
\definecolor{currentstroke}{rgb}{0.501961,0.501961,0.501961}%
\pgfsetstrokecolor{currentstroke}%
\pgfsetstrokeopacity{0.500000}%
\pgfsetdash{{1.850000pt}{0.800000pt}}{0.000000pt}%
\pgfpathmoveto{\pgfqpoint{1.245077in}{0.576229in}}%
\pgfpathlineto{\pgfqpoint{1.245077in}{3.604118in}}%
\pgfusepath{stroke}%
\end{pgfscope}%
\begin{pgfscope}%
\pgfsetbuttcap%
\pgfsetroundjoin%
\definecolor{currentfill}{rgb}{0.000000,0.000000,0.000000}%
\pgfsetfillcolor{currentfill}%
\pgfsetlinewidth{0.803000pt}%
\definecolor{currentstroke}{rgb}{0.000000,0.000000,0.000000}%
\pgfsetstrokecolor{currentstroke}%
\pgfsetdash{}{0pt}%
\pgfsys@defobject{currentmarker}{\pgfqpoint{0.000000in}{-0.048611in}}{\pgfqpoint{0.000000in}{0.000000in}}{%
\pgfpathmoveto{\pgfqpoint{0.000000in}{0.000000in}}%
\pgfpathlineto{\pgfqpoint{0.000000in}{-0.048611in}}%
\pgfusepath{stroke,fill}%
}%
\begin{pgfscope}%
\pgfsys@transformshift{1.245077in}{0.576229in}%
\pgfsys@useobject{currentmarker}{}%
\end{pgfscope}%
\end{pgfscope}%
\begin{pgfscope}%
\definecolor{textcolor}{rgb}{0.000000,0.000000,0.000000}%
\pgfsetstrokecolor{textcolor}%
\pgfsetfillcolor{textcolor}%
\pgftext[x=1.245077in,y=0.479007in,,top]{\color{textcolor}{\rmfamily\fontsize{12.000000}{14.400000}\selectfont\catcode`\^=\active\def^{\ifmmode\sp\else\^{}\fi}\catcode`\%=\active\def%{\%}$\mathdefault{2.6}$}}%
\end{pgfscope}%
\begin{pgfscope}%
\pgfpathrectangle{\pgfqpoint{0.460969in}{0.576229in}}{\pgfqpoint{5.169031in}{3.027889in}}%
\pgfusepath{clip}%
\pgfsetbuttcap%
\pgfsetroundjoin%
\pgfsetlinewidth{0.501875pt}%
\definecolor{currentstroke}{rgb}{0.501961,0.501961,0.501961}%
\pgfsetstrokecolor{currentstroke}%
\pgfsetstrokeopacity{0.500000}%
\pgfsetdash{{1.850000pt}{0.800000pt}}{0.000000pt}%
\pgfpathmoveto{\pgfqpoint{2.343381in}{0.576229in}}%
\pgfpathlineto{\pgfqpoint{2.343381in}{3.604118in}}%
\pgfusepath{stroke}%
\end{pgfscope}%
\begin{pgfscope}%
\pgfsetbuttcap%
\pgfsetroundjoin%
\definecolor{currentfill}{rgb}{0.000000,0.000000,0.000000}%
\pgfsetfillcolor{currentfill}%
\pgfsetlinewidth{0.803000pt}%
\definecolor{currentstroke}{rgb}{0.000000,0.000000,0.000000}%
\pgfsetstrokecolor{currentstroke}%
\pgfsetdash{}{0pt}%
\pgfsys@defobject{currentmarker}{\pgfqpoint{0.000000in}{-0.048611in}}{\pgfqpoint{0.000000in}{0.000000in}}{%
\pgfpathmoveto{\pgfqpoint{0.000000in}{0.000000in}}%
\pgfpathlineto{\pgfqpoint{0.000000in}{-0.048611in}}%
\pgfusepath{stroke,fill}%
}%
\begin{pgfscope}%
\pgfsys@transformshift{2.343381in}{0.576229in}%
\pgfsys@useobject{currentmarker}{}%
\end{pgfscope}%
\end{pgfscope}%
\begin{pgfscope}%
\definecolor{textcolor}{rgb}{0.000000,0.000000,0.000000}%
\pgfsetstrokecolor{textcolor}%
\pgfsetfillcolor{textcolor}%
\pgftext[x=2.343381in,y=0.479007in,,top]{\color{textcolor}{\rmfamily\fontsize{12.000000}{14.400000}\selectfont\catcode`\^=\active\def^{\ifmmode\sp\else\^{}\fi}\catcode`\%=\active\def%{\%}$\mathdefault{2.8}$}}%
\end{pgfscope}%
\begin{pgfscope}%
\pgfpathrectangle{\pgfqpoint{0.460969in}{0.576229in}}{\pgfqpoint{5.169031in}{3.027889in}}%
\pgfusepath{clip}%
\pgfsetbuttcap%
\pgfsetroundjoin%
\pgfsetlinewidth{0.501875pt}%
\definecolor{currentstroke}{rgb}{0.501961,0.501961,0.501961}%
\pgfsetstrokecolor{currentstroke}%
\pgfsetstrokeopacity{0.500000}%
\pgfsetdash{{1.850000pt}{0.800000pt}}{0.000000pt}%
\pgfpathmoveto{\pgfqpoint{3.441685in}{0.576229in}}%
\pgfpathlineto{\pgfqpoint{3.441685in}{3.604118in}}%
\pgfusepath{stroke}%
\end{pgfscope}%
\begin{pgfscope}%
\pgfsetbuttcap%
\pgfsetroundjoin%
\definecolor{currentfill}{rgb}{0.000000,0.000000,0.000000}%
\pgfsetfillcolor{currentfill}%
\pgfsetlinewidth{0.803000pt}%
\definecolor{currentstroke}{rgb}{0.000000,0.000000,0.000000}%
\pgfsetstrokecolor{currentstroke}%
\pgfsetdash{}{0pt}%
\pgfsys@defobject{currentmarker}{\pgfqpoint{0.000000in}{-0.048611in}}{\pgfqpoint{0.000000in}{0.000000in}}{%
\pgfpathmoveto{\pgfqpoint{0.000000in}{0.000000in}}%
\pgfpathlineto{\pgfqpoint{0.000000in}{-0.048611in}}%
\pgfusepath{stroke,fill}%
}%
\begin{pgfscope}%
\pgfsys@transformshift{3.441685in}{0.576229in}%
\pgfsys@useobject{currentmarker}{}%
\end{pgfscope}%
\end{pgfscope}%
\begin{pgfscope}%
\definecolor{textcolor}{rgb}{0.000000,0.000000,0.000000}%
\pgfsetstrokecolor{textcolor}%
\pgfsetfillcolor{textcolor}%
\pgftext[x=3.441685in,y=0.479007in,,top]{\color{textcolor}{\rmfamily\fontsize{12.000000}{14.400000}\selectfont\catcode`\^=\active\def^{\ifmmode\sp\else\^{}\fi}\catcode`\%=\active\def%{\%}$\mathdefault{3.0}$}}%
\end{pgfscope}%
\begin{pgfscope}%
\pgfpathrectangle{\pgfqpoint{0.460969in}{0.576229in}}{\pgfqpoint{5.169031in}{3.027889in}}%
\pgfusepath{clip}%
\pgfsetbuttcap%
\pgfsetroundjoin%
\pgfsetlinewidth{0.501875pt}%
\definecolor{currentstroke}{rgb}{0.501961,0.501961,0.501961}%
\pgfsetstrokecolor{currentstroke}%
\pgfsetstrokeopacity{0.500000}%
\pgfsetdash{{1.850000pt}{0.800000pt}}{0.000000pt}%
\pgfpathmoveto{\pgfqpoint{4.539989in}{0.576229in}}%
\pgfpathlineto{\pgfqpoint{4.539989in}{3.604118in}}%
\pgfusepath{stroke}%
\end{pgfscope}%
\begin{pgfscope}%
\pgfsetbuttcap%
\pgfsetroundjoin%
\definecolor{currentfill}{rgb}{0.000000,0.000000,0.000000}%
\pgfsetfillcolor{currentfill}%
\pgfsetlinewidth{0.803000pt}%
\definecolor{currentstroke}{rgb}{0.000000,0.000000,0.000000}%
\pgfsetstrokecolor{currentstroke}%
\pgfsetdash{}{0pt}%
\pgfsys@defobject{currentmarker}{\pgfqpoint{0.000000in}{-0.048611in}}{\pgfqpoint{0.000000in}{0.000000in}}{%
\pgfpathmoveto{\pgfqpoint{0.000000in}{0.000000in}}%
\pgfpathlineto{\pgfqpoint{0.000000in}{-0.048611in}}%
\pgfusepath{stroke,fill}%
}%
\begin{pgfscope}%
\pgfsys@transformshift{4.539989in}{0.576229in}%
\pgfsys@useobject{currentmarker}{}%
\end{pgfscope}%
\end{pgfscope}%
\begin{pgfscope}%
\definecolor{textcolor}{rgb}{0.000000,0.000000,0.000000}%
\pgfsetstrokecolor{textcolor}%
\pgfsetfillcolor{textcolor}%
\pgftext[x=4.539989in,y=0.479007in,,top]{\color{textcolor}{\rmfamily\fontsize{12.000000}{14.400000}\selectfont\catcode`\^=\active\def^{\ifmmode\sp\else\^{}\fi}\catcode`\%=\active\def%{\%}$\mathdefault{3.2}$}}%
\end{pgfscope}%
\begin{pgfscope}%
\definecolor{textcolor}{rgb}{0.000000,0.000000,0.000000}%
\pgfsetstrokecolor{textcolor}%
\pgfsetfillcolor{textcolor}%
\pgftext[x=3.045485in,y=0.272084in,,top]{\color{textcolor}{\rmfamily\fontsize{14.000000}{16.800000}\selectfont\catcode`\^=\active\def^{\ifmmode\sp\else\^{}\fi}\catcode`\%=\active\def%{\%}$1000/T, \text{К}^{-1}$}}%
\end{pgfscope}%
\begin{pgfscope}%
\pgfpathrectangle{\pgfqpoint{0.460969in}{0.576229in}}{\pgfqpoint{5.169031in}{3.027889in}}%
\pgfusepath{clip}%
\pgfsetbuttcap%
\pgfsetroundjoin%
\pgfsetlinewidth{0.501875pt}%
\definecolor{currentstroke}{rgb}{0.501961,0.501961,0.501961}%
\pgfsetstrokecolor{currentstroke}%
\pgfsetstrokeopacity{0.500000}%
\pgfsetdash{{1.850000pt}{0.800000pt}}{0.000000pt}%
\pgfpathmoveto{\pgfqpoint{0.460969in}{0.870471in}}%
\pgfpathlineto{\pgfqpoint{5.630000in}{0.870471in}}%
\pgfusepath{stroke}%
\end{pgfscope}%
\begin{pgfscope}%
\pgfsetbuttcap%
\pgfsetroundjoin%
\definecolor{currentfill}{rgb}{0.000000,0.000000,0.000000}%
\pgfsetfillcolor{currentfill}%
\pgfsetlinewidth{0.803000pt}%
\definecolor{currentstroke}{rgb}{0.000000,0.000000,0.000000}%
\pgfsetstrokecolor{currentstroke}%
\pgfsetdash{}{0pt}%
\pgfsys@defobject{currentmarker}{\pgfqpoint{-0.048611in}{0.000000in}}{\pgfqpoint{-0.000000in}{0.000000in}}{%
\pgfpathmoveto{\pgfqpoint{-0.000000in}{0.000000in}}%
\pgfpathlineto{\pgfqpoint{-0.048611in}{0.000000in}}%
\pgfusepath{stroke,fill}%
}%
\begin{pgfscope}%
\pgfsys@transformshift{0.460969in}{0.870471in}%
\pgfsys@useobject{currentmarker}{}%
\end{pgfscope}%
\end{pgfscope}%
\begin{pgfscope}%
\definecolor{textcolor}{rgb}{0.000000,0.000000,0.000000}%
\pgfsetstrokecolor{textcolor}%
\pgfsetfillcolor{textcolor}%
\pgftext[x=0.282151in, y=0.812610in, left, base]{\color{textcolor}{\rmfamily\fontsize{12.000000}{14.400000}\selectfont\catcode`\^=\active\def^{\ifmmode\sp\else\^{}\fi}\catcode`\%=\active\def%{\%}$\mathdefault{1}$}}%
\end{pgfscope}%
\begin{pgfscope}%
\pgfpathrectangle{\pgfqpoint{0.460969in}{0.576229in}}{\pgfqpoint{5.169031in}{3.027889in}}%
\pgfusepath{clip}%
\pgfsetbuttcap%
\pgfsetroundjoin%
\pgfsetlinewidth{0.501875pt}%
\definecolor{currentstroke}{rgb}{0.501961,0.501961,0.501961}%
\pgfsetstrokecolor{currentstroke}%
\pgfsetstrokeopacity{0.500000}%
\pgfsetdash{{1.850000pt}{0.800000pt}}{0.000000pt}%
\pgfpathmoveto{\pgfqpoint{0.460969in}{1.208247in}}%
\pgfpathlineto{\pgfqpoint{5.630000in}{1.208247in}}%
\pgfusepath{stroke}%
\end{pgfscope}%
\begin{pgfscope}%
\pgfsetbuttcap%
\pgfsetroundjoin%
\definecolor{currentfill}{rgb}{0.000000,0.000000,0.000000}%
\pgfsetfillcolor{currentfill}%
\pgfsetlinewidth{0.803000pt}%
\definecolor{currentstroke}{rgb}{0.000000,0.000000,0.000000}%
\pgfsetstrokecolor{currentstroke}%
\pgfsetdash{}{0pt}%
\pgfsys@defobject{currentmarker}{\pgfqpoint{-0.048611in}{0.000000in}}{\pgfqpoint{-0.000000in}{0.000000in}}{%
\pgfpathmoveto{\pgfqpoint{-0.000000in}{0.000000in}}%
\pgfpathlineto{\pgfqpoint{-0.048611in}{0.000000in}}%
\pgfusepath{stroke,fill}%
}%
\begin{pgfscope}%
\pgfsys@transformshift{0.460969in}{1.208247in}%
\pgfsys@useobject{currentmarker}{}%
\end{pgfscope}%
\end{pgfscope}%
\begin{pgfscope}%
\definecolor{textcolor}{rgb}{0.000000,0.000000,0.000000}%
\pgfsetstrokecolor{textcolor}%
\pgfsetfillcolor{textcolor}%
\pgftext[x=0.282151in, y=1.150386in, left, base]{\color{textcolor}{\rmfamily\fontsize{12.000000}{14.400000}\selectfont\catcode`\^=\active\def^{\ifmmode\sp\else\^{}\fi}\catcode`\%=\active\def%{\%}$\mathdefault{2}$}}%
\end{pgfscope}%
\begin{pgfscope}%
\pgfpathrectangle{\pgfqpoint{0.460969in}{0.576229in}}{\pgfqpoint{5.169031in}{3.027889in}}%
\pgfusepath{clip}%
\pgfsetbuttcap%
\pgfsetroundjoin%
\pgfsetlinewidth{0.501875pt}%
\definecolor{currentstroke}{rgb}{0.501961,0.501961,0.501961}%
\pgfsetstrokecolor{currentstroke}%
\pgfsetstrokeopacity{0.500000}%
\pgfsetdash{{1.850000pt}{0.800000pt}}{0.000000pt}%
\pgfpathmoveto{\pgfqpoint{0.460969in}{1.546023in}}%
\pgfpathlineto{\pgfqpoint{5.630000in}{1.546023in}}%
\pgfusepath{stroke}%
\end{pgfscope}%
\begin{pgfscope}%
\pgfsetbuttcap%
\pgfsetroundjoin%
\definecolor{currentfill}{rgb}{0.000000,0.000000,0.000000}%
\pgfsetfillcolor{currentfill}%
\pgfsetlinewidth{0.803000pt}%
\definecolor{currentstroke}{rgb}{0.000000,0.000000,0.000000}%
\pgfsetstrokecolor{currentstroke}%
\pgfsetdash{}{0pt}%
\pgfsys@defobject{currentmarker}{\pgfqpoint{-0.048611in}{0.000000in}}{\pgfqpoint{-0.000000in}{0.000000in}}{%
\pgfpathmoveto{\pgfqpoint{-0.000000in}{0.000000in}}%
\pgfpathlineto{\pgfqpoint{-0.048611in}{0.000000in}}%
\pgfusepath{stroke,fill}%
}%
\begin{pgfscope}%
\pgfsys@transformshift{0.460969in}{1.546023in}%
\pgfsys@useobject{currentmarker}{}%
\end{pgfscope}%
\end{pgfscope}%
\begin{pgfscope}%
\definecolor{textcolor}{rgb}{0.000000,0.000000,0.000000}%
\pgfsetstrokecolor{textcolor}%
\pgfsetfillcolor{textcolor}%
\pgftext[x=0.282151in, y=1.488162in, left, base]{\color{textcolor}{\rmfamily\fontsize{12.000000}{14.400000}\selectfont\catcode`\^=\active\def^{\ifmmode\sp\else\^{}\fi}\catcode`\%=\active\def%{\%}$\mathdefault{3}$}}%
\end{pgfscope}%
\begin{pgfscope}%
\pgfpathrectangle{\pgfqpoint{0.460969in}{0.576229in}}{\pgfqpoint{5.169031in}{3.027889in}}%
\pgfusepath{clip}%
\pgfsetbuttcap%
\pgfsetroundjoin%
\pgfsetlinewidth{0.501875pt}%
\definecolor{currentstroke}{rgb}{0.501961,0.501961,0.501961}%
\pgfsetstrokecolor{currentstroke}%
\pgfsetstrokeopacity{0.500000}%
\pgfsetdash{{1.850000pt}{0.800000pt}}{0.000000pt}%
\pgfpathmoveto{\pgfqpoint{0.460969in}{1.883799in}}%
\pgfpathlineto{\pgfqpoint{5.630000in}{1.883799in}}%
\pgfusepath{stroke}%
\end{pgfscope}%
\begin{pgfscope}%
\pgfsetbuttcap%
\pgfsetroundjoin%
\definecolor{currentfill}{rgb}{0.000000,0.000000,0.000000}%
\pgfsetfillcolor{currentfill}%
\pgfsetlinewidth{0.803000pt}%
\definecolor{currentstroke}{rgb}{0.000000,0.000000,0.000000}%
\pgfsetstrokecolor{currentstroke}%
\pgfsetdash{}{0pt}%
\pgfsys@defobject{currentmarker}{\pgfqpoint{-0.048611in}{0.000000in}}{\pgfqpoint{-0.000000in}{0.000000in}}{%
\pgfpathmoveto{\pgfqpoint{-0.000000in}{0.000000in}}%
\pgfpathlineto{\pgfqpoint{-0.048611in}{0.000000in}}%
\pgfusepath{stroke,fill}%
}%
\begin{pgfscope}%
\pgfsys@transformshift{0.460969in}{1.883799in}%
\pgfsys@useobject{currentmarker}{}%
\end{pgfscope}%
\end{pgfscope}%
\begin{pgfscope}%
\definecolor{textcolor}{rgb}{0.000000,0.000000,0.000000}%
\pgfsetstrokecolor{textcolor}%
\pgfsetfillcolor{textcolor}%
\pgftext[x=0.282151in, y=1.825938in, left, base]{\color{textcolor}{\rmfamily\fontsize{12.000000}{14.400000}\selectfont\catcode`\^=\active\def^{\ifmmode\sp\else\^{}\fi}\catcode`\%=\active\def%{\%}$\mathdefault{4}$}}%
\end{pgfscope}%
\begin{pgfscope}%
\pgfpathrectangle{\pgfqpoint{0.460969in}{0.576229in}}{\pgfqpoint{5.169031in}{3.027889in}}%
\pgfusepath{clip}%
\pgfsetbuttcap%
\pgfsetroundjoin%
\pgfsetlinewidth{0.501875pt}%
\definecolor{currentstroke}{rgb}{0.501961,0.501961,0.501961}%
\pgfsetstrokecolor{currentstroke}%
\pgfsetstrokeopacity{0.500000}%
\pgfsetdash{{1.850000pt}{0.800000pt}}{0.000000pt}%
\pgfpathmoveto{\pgfqpoint{0.460969in}{2.221575in}}%
\pgfpathlineto{\pgfqpoint{5.630000in}{2.221575in}}%
\pgfusepath{stroke}%
\end{pgfscope}%
\begin{pgfscope}%
\pgfsetbuttcap%
\pgfsetroundjoin%
\definecolor{currentfill}{rgb}{0.000000,0.000000,0.000000}%
\pgfsetfillcolor{currentfill}%
\pgfsetlinewidth{0.803000pt}%
\definecolor{currentstroke}{rgb}{0.000000,0.000000,0.000000}%
\pgfsetstrokecolor{currentstroke}%
\pgfsetdash{}{0pt}%
\pgfsys@defobject{currentmarker}{\pgfqpoint{-0.048611in}{0.000000in}}{\pgfqpoint{-0.000000in}{0.000000in}}{%
\pgfpathmoveto{\pgfqpoint{-0.000000in}{0.000000in}}%
\pgfpathlineto{\pgfqpoint{-0.048611in}{0.000000in}}%
\pgfusepath{stroke,fill}%
}%
\begin{pgfscope}%
\pgfsys@transformshift{0.460969in}{2.221575in}%
\pgfsys@useobject{currentmarker}{}%
\end{pgfscope}%
\end{pgfscope}%
\begin{pgfscope}%
\definecolor{textcolor}{rgb}{0.000000,0.000000,0.000000}%
\pgfsetstrokecolor{textcolor}%
\pgfsetfillcolor{textcolor}%
\pgftext[x=0.282151in, y=2.163714in, left, base]{\color{textcolor}{\rmfamily\fontsize{12.000000}{14.400000}\selectfont\catcode`\^=\active\def^{\ifmmode\sp\else\^{}\fi}\catcode`\%=\active\def%{\%}$\mathdefault{5}$}}%
\end{pgfscope}%
\begin{pgfscope}%
\pgfpathrectangle{\pgfqpoint{0.460969in}{0.576229in}}{\pgfqpoint{5.169031in}{3.027889in}}%
\pgfusepath{clip}%
\pgfsetbuttcap%
\pgfsetroundjoin%
\pgfsetlinewidth{0.501875pt}%
\definecolor{currentstroke}{rgb}{0.501961,0.501961,0.501961}%
\pgfsetstrokecolor{currentstroke}%
\pgfsetstrokeopacity{0.500000}%
\pgfsetdash{{1.850000pt}{0.800000pt}}{0.000000pt}%
\pgfpathmoveto{\pgfqpoint{0.460969in}{2.559351in}}%
\pgfpathlineto{\pgfqpoint{5.630000in}{2.559351in}}%
\pgfusepath{stroke}%
\end{pgfscope}%
\begin{pgfscope}%
\pgfsetbuttcap%
\pgfsetroundjoin%
\definecolor{currentfill}{rgb}{0.000000,0.000000,0.000000}%
\pgfsetfillcolor{currentfill}%
\pgfsetlinewidth{0.803000pt}%
\definecolor{currentstroke}{rgb}{0.000000,0.000000,0.000000}%
\pgfsetstrokecolor{currentstroke}%
\pgfsetdash{}{0pt}%
\pgfsys@defobject{currentmarker}{\pgfqpoint{-0.048611in}{0.000000in}}{\pgfqpoint{-0.000000in}{0.000000in}}{%
\pgfpathmoveto{\pgfqpoint{-0.000000in}{0.000000in}}%
\pgfpathlineto{\pgfqpoint{-0.048611in}{0.000000in}}%
\pgfusepath{stroke,fill}%
}%
\begin{pgfscope}%
\pgfsys@transformshift{0.460969in}{2.559351in}%
\pgfsys@useobject{currentmarker}{}%
\end{pgfscope}%
\end{pgfscope}%
\begin{pgfscope}%
\definecolor{textcolor}{rgb}{0.000000,0.000000,0.000000}%
\pgfsetstrokecolor{textcolor}%
\pgfsetfillcolor{textcolor}%
\pgftext[x=0.282151in, y=2.501490in, left, base]{\color{textcolor}{\rmfamily\fontsize{12.000000}{14.400000}\selectfont\catcode`\^=\active\def^{\ifmmode\sp\else\^{}\fi}\catcode`\%=\active\def%{\%}$\mathdefault{6}$}}%
\end{pgfscope}%
\begin{pgfscope}%
\pgfpathrectangle{\pgfqpoint{0.460969in}{0.576229in}}{\pgfqpoint{5.169031in}{3.027889in}}%
\pgfusepath{clip}%
\pgfsetbuttcap%
\pgfsetroundjoin%
\pgfsetlinewidth{0.501875pt}%
\definecolor{currentstroke}{rgb}{0.501961,0.501961,0.501961}%
\pgfsetstrokecolor{currentstroke}%
\pgfsetstrokeopacity{0.500000}%
\pgfsetdash{{1.850000pt}{0.800000pt}}{0.000000pt}%
\pgfpathmoveto{\pgfqpoint{0.460969in}{2.897128in}}%
\pgfpathlineto{\pgfqpoint{5.630000in}{2.897128in}}%
\pgfusepath{stroke}%
\end{pgfscope}%
\begin{pgfscope}%
\pgfsetbuttcap%
\pgfsetroundjoin%
\definecolor{currentfill}{rgb}{0.000000,0.000000,0.000000}%
\pgfsetfillcolor{currentfill}%
\pgfsetlinewidth{0.803000pt}%
\definecolor{currentstroke}{rgb}{0.000000,0.000000,0.000000}%
\pgfsetstrokecolor{currentstroke}%
\pgfsetdash{}{0pt}%
\pgfsys@defobject{currentmarker}{\pgfqpoint{-0.048611in}{0.000000in}}{\pgfqpoint{-0.000000in}{0.000000in}}{%
\pgfpathmoveto{\pgfqpoint{-0.000000in}{0.000000in}}%
\pgfpathlineto{\pgfqpoint{-0.048611in}{0.000000in}}%
\pgfusepath{stroke,fill}%
}%
\begin{pgfscope}%
\pgfsys@transformshift{0.460969in}{2.897128in}%
\pgfsys@useobject{currentmarker}{}%
\end{pgfscope}%
\end{pgfscope}%
\begin{pgfscope}%
\definecolor{textcolor}{rgb}{0.000000,0.000000,0.000000}%
\pgfsetstrokecolor{textcolor}%
\pgfsetfillcolor{textcolor}%
\pgftext[x=0.282151in, y=2.839266in, left, base]{\color{textcolor}{\rmfamily\fontsize{12.000000}{14.400000}\selectfont\catcode`\^=\active\def^{\ifmmode\sp\else\^{}\fi}\catcode`\%=\active\def%{\%}$\mathdefault{7}$}}%
\end{pgfscope}%
\begin{pgfscope}%
\pgfpathrectangle{\pgfqpoint{0.460969in}{0.576229in}}{\pgfqpoint{5.169031in}{3.027889in}}%
\pgfusepath{clip}%
\pgfsetbuttcap%
\pgfsetroundjoin%
\pgfsetlinewidth{0.501875pt}%
\definecolor{currentstroke}{rgb}{0.501961,0.501961,0.501961}%
\pgfsetstrokecolor{currentstroke}%
\pgfsetstrokeopacity{0.500000}%
\pgfsetdash{{1.850000pt}{0.800000pt}}{0.000000pt}%
\pgfpathmoveto{\pgfqpoint{0.460969in}{3.234904in}}%
\pgfpathlineto{\pgfqpoint{5.630000in}{3.234904in}}%
\pgfusepath{stroke}%
\end{pgfscope}%
\begin{pgfscope}%
\pgfsetbuttcap%
\pgfsetroundjoin%
\definecolor{currentfill}{rgb}{0.000000,0.000000,0.000000}%
\pgfsetfillcolor{currentfill}%
\pgfsetlinewidth{0.803000pt}%
\definecolor{currentstroke}{rgb}{0.000000,0.000000,0.000000}%
\pgfsetstrokecolor{currentstroke}%
\pgfsetdash{}{0pt}%
\pgfsys@defobject{currentmarker}{\pgfqpoint{-0.048611in}{0.000000in}}{\pgfqpoint{-0.000000in}{0.000000in}}{%
\pgfpathmoveto{\pgfqpoint{-0.000000in}{0.000000in}}%
\pgfpathlineto{\pgfqpoint{-0.048611in}{0.000000in}}%
\pgfusepath{stroke,fill}%
}%
\begin{pgfscope}%
\pgfsys@transformshift{0.460969in}{3.234904in}%
\pgfsys@useobject{currentmarker}{}%
\end{pgfscope}%
\end{pgfscope}%
\begin{pgfscope}%
\definecolor{textcolor}{rgb}{0.000000,0.000000,0.000000}%
\pgfsetstrokecolor{textcolor}%
\pgfsetfillcolor{textcolor}%
\pgftext[x=0.282151in, y=3.177042in, left, base]{\color{textcolor}{\rmfamily\fontsize{12.000000}{14.400000}\selectfont\catcode`\^=\active\def^{\ifmmode\sp\else\^{}\fi}\catcode`\%=\active\def%{\%}$\mathdefault{8}$}}%
\end{pgfscope}%
\begin{pgfscope}%
\pgfpathrectangle{\pgfqpoint{0.460969in}{0.576229in}}{\pgfqpoint{5.169031in}{3.027889in}}%
\pgfusepath{clip}%
\pgfsetbuttcap%
\pgfsetroundjoin%
\pgfsetlinewidth{0.501875pt}%
\definecolor{currentstroke}{rgb}{0.501961,0.501961,0.501961}%
\pgfsetstrokecolor{currentstroke}%
\pgfsetstrokeopacity{0.500000}%
\pgfsetdash{{1.850000pt}{0.800000pt}}{0.000000pt}%
\pgfpathmoveto{\pgfqpoint{0.460969in}{3.572680in}}%
\pgfpathlineto{\pgfqpoint{5.630000in}{3.572680in}}%
\pgfusepath{stroke}%
\end{pgfscope}%
\begin{pgfscope}%
\pgfsetbuttcap%
\pgfsetroundjoin%
\definecolor{currentfill}{rgb}{0.000000,0.000000,0.000000}%
\pgfsetfillcolor{currentfill}%
\pgfsetlinewidth{0.803000pt}%
\definecolor{currentstroke}{rgb}{0.000000,0.000000,0.000000}%
\pgfsetstrokecolor{currentstroke}%
\pgfsetdash{}{0pt}%
\pgfsys@defobject{currentmarker}{\pgfqpoint{-0.048611in}{0.000000in}}{\pgfqpoint{-0.000000in}{0.000000in}}{%
\pgfpathmoveto{\pgfqpoint{-0.000000in}{0.000000in}}%
\pgfpathlineto{\pgfqpoint{-0.048611in}{0.000000in}}%
\pgfusepath{stroke,fill}%
}%
\begin{pgfscope}%
\pgfsys@transformshift{0.460969in}{3.572680in}%
\pgfsys@useobject{currentmarker}{}%
\end{pgfscope}%
\end{pgfscope}%
\begin{pgfscope}%
\definecolor{textcolor}{rgb}{0.000000,0.000000,0.000000}%
\pgfsetstrokecolor{textcolor}%
\pgfsetfillcolor{textcolor}%
\pgftext[x=0.282151in, y=3.514818in, left, base]{\color{textcolor}{\rmfamily\fontsize{12.000000}{14.400000}\selectfont\catcode`\^=\active\def^{\ifmmode\sp\else\^{}\fi}\catcode`\%=\active\def%{\%}$\mathdefault{9}$}}%
\end{pgfscope}%
\begin{pgfscope}%
\definecolor{textcolor}{rgb}{0.000000,0.000000,0.000000}%
\pgfsetstrokecolor{textcolor}%
\pgfsetfillcolor{textcolor}%
\pgftext[x=0.226595in,y=2.090174in,,bottom,rotate=90.000000]{\color{textcolor}{\rmfamily\fontsize{14.000000}{16.800000}\selectfont\catcode`\^=\active\def^{\ifmmode\sp\else\^{}\fi}\catcode`\%=\active\def%{\%}$\ln \gamma_{\text{эксп}}$}}%
\end{pgfscope}%
\begin{pgfscope}%
\pgfpathrectangle{\pgfqpoint{0.460969in}{0.576229in}}{\pgfqpoint{5.169031in}{3.027889in}}%
\pgfusepath{clip}%
\pgfsetrectcap%
\pgfsetroundjoin%
\pgfsetlinewidth{2.007500pt}%
\definecolor{currentstroke}{rgb}{0.121569,0.466667,0.705882}%
\pgfsetstrokecolor{currentstroke}%
\pgfsetdash{}{0pt}%
\pgfpathmoveto{\pgfqpoint{5.395044in}{2.968590in}}%
\pgfpathlineto{\pgfqpoint{4.796735in}{2.957485in}}%
\pgfpathlineto{\pgfqpoint{4.236056in}{2.948453in}}%
\pgfpathlineto{\pgfqpoint{3.709564in}{2.939377in}}%
\pgfpathlineto{\pgfqpoint{3.214225in}{2.930812in}}%
\pgfpathlineto{\pgfqpoint{2.747355in}{2.919806in}}%
\pgfpathlineto{\pgfqpoint{2.306566in}{2.911717in}}%
\pgfpathlineto{\pgfqpoint{1.889734in}{2.903314in}}%
\pgfpathlineto{\pgfqpoint{1.494956in}{2.893889in}}%
\pgfpathlineto{\pgfqpoint{1.120527in}{2.884958in}}%
\pgfpathlineto{\pgfqpoint{0.695925in}{2.877421in}}%
\pgfusepath{stroke}%
\end{pgfscope}%
\begin{pgfscope}%
\pgfpathrectangle{\pgfqpoint{0.460969in}{0.576229in}}{\pgfqpoint{5.169031in}{3.027889in}}%
\pgfusepath{clip}%
\pgfsetbuttcap%
\pgfsetroundjoin%
\definecolor{currentfill}{rgb}{0.121569,0.466667,0.705882}%
\pgfsetfillcolor{currentfill}%
\pgfsetlinewidth{1.003750pt}%
\definecolor{currentstroke}{rgb}{0.121569,0.466667,0.705882}%
\pgfsetstrokecolor{currentstroke}%
\pgfsetdash{}{0pt}%
\pgfsys@defobject{currentmarker}{\pgfqpoint{-0.041667in}{-0.041667in}}{\pgfqpoint{0.041667in}{0.041667in}}{%
\pgfpathmoveto{\pgfqpoint{0.000000in}{-0.041667in}}%
\pgfpathcurveto{\pgfqpoint{0.011050in}{-0.041667in}}{\pgfqpoint{0.021649in}{-0.037276in}}{\pgfqpoint{0.029463in}{-0.029463in}}%
\pgfpathcurveto{\pgfqpoint{0.037276in}{-0.021649in}}{\pgfqpoint{0.041667in}{-0.011050in}}{\pgfqpoint{0.041667in}{0.000000in}}%
\pgfpathcurveto{\pgfqpoint{0.041667in}{0.011050in}}{\pgfqpoint{0.037276in}{0.021649in}}{\pgfqpoint{0.029463in}{0.029463in}}%
\pgfpathcurveto{\pgfqpoint{0.021649in}{0.037276in}}{\pgfqpoint{0.011050in}{0.041667in}}{\pgfqpoint{0.000000in}{0.041667in}}%
\pgfpathcurveto{\pgfqpoint{-0.011050in}{0.041667in}}{\pgfqpoint{-0.021649in}{0.037276in}}{\pgfqpoint{-0.029463in}{0.029463in}}%
\pgfpathcurveto{\pgfqpoint{-0.037276in}{0.021649in}}{\pgfqpoint{-0.041667in}{0.011050in}}{\pgfqpoint{-0.041667in}{0.000000in}}%
\pgfpathcurveto{\pgfqpoint{-0.041667in}{-0.011050in}}{\pgfqpoint{-0.037276in}{-0.021649in}}{\pgfqpoint{-0.029463in}{-0.029463in}}%
\pgfpathcurveto{\pgfqpoint{-0.021649in}{-0.037276in}}{\pgfqpoint{-0.011050in}{-0.041667in}}{\pgfqpoint{0.000000in}{-0.041667in}}%
\pgfpathlineto{\pgfqpoint{0.000000in}{-0.041667in}}%
\pgfpathclose%
\pgfusepath{stroke,fill}%
}%
\begin{pgfscope}%
\pgfsys@transformshift{5.395044in}{2.968590in}%
\pgfsys@useobject{currentmarker}{}%
\end{pgfscope}%
\begin{pgfscope}%
\pgfsys@transformshift{4.796735in}{2.957485in}%
\pgfsys@useobject{currentmarker}{}%
\end{pgfscope}%
\begin{pgfscope}%
\pgfsys@transformshift{4.236056in}{2.948453in}%
\pgfsys@useobject{currentmarker}{}%
\end{pgfscope}%
\begin{pgfscope}%
\pgfsys@transformshift{3.709564in}{2.939377in}%
\pgfsys@useobject{currentmarker}{}%
\end{pgfscope}%
\begin{pgfscope}%
\pgfsys@transformshift{3.214225in}{2.930812in}%
\pgfsys@useobject{currentmarker}{}%
\end{pgfscope}%
\begin{pgfscope}%
\pgfsys@transformshift{2.747355in}{2.919806in}%
\pgfsys@useobject{currentmarker}{}%
\end{pgfscope}%
\begin{pgfscope}%
\pgfsys@transformshift{2.306566in}{2.911717in}%
\pgfsys@useobject{currentmarker}{}%
\end{pgfscope}%
\begin{pgfscope}%
\pgfsys@transformshift{1.889734in}{2.903314in}%
\pgfsys@useobject{currentmarker}{}%
\end{pgfscope}%
\begin{pgfscope}%
\pgfsys@transformshift{1.494956in}{2.893889in}%
\pgfsys@useobject{currentmarker}{}%
\end{pgfscope}%
\begin{pgfscope}%
\pgfsys@transformshift{1.120527in}{2.884958in}%
\pgfsys@useobject{currentmarker}{}%
\end{pgfscope}%
\begin{pgfscope}%
\pgfsys@transformshift{0.695925in}{2.877421in}%
\pgfsys@useobject{currentmarker}{}%
\end{pgfscope}%
\end{pgfscope}%
\begin{pgfscope}%
\pgfpathrectangle{\pgfqpoint{0.460969in}{0.576229in}}{\pgfqpoint{5.169031in}{3.027889in}}%
\pgfusepath{clip}%
\pgfsetrectcap%
\pgfsetroundjoin%
\pgfsetlinewidth{2.007500pt}%
\definecolor{currentstroke}{rgb}{1.000000,0.498039,0.054902}%
\pgfsetstrokecolor{currentstroke}%
\pgfsetdash{}{0pt}%
\pgfpathmoveto{\pgfqpoint{5.395044in}{2.640971in}}%
\pgfpathlineto{\pgfqpoint{4.796735in}{2.628480in}}%
\pgfpathlineto{\pgfqpoint{4.236056in}{2.611099in}}%
\pgfpathlineto{\pgfqpoint{3.709564in}{2.603767in}}%
\pgfpathlineto{\pgfqpoint{3.214225in}{2.590559in}}%
\pgfpathlineto{\pgfqpoint{2.747355in}{2.601701in}}%
\pgfpathlineto{\pgfqpoint{2.306566in}{2.605845in}}%
\pgfpathlineto{\pgfqpoint{1.889734in}{2.618593in}}%
\pgfpathlineto{\pgfqpoint{1.494956in}{2.649167in}}%
\pgfpathlineto{\pgfqpoint{1.120527in}{2.677331in}}%
\pgfpathlineto{\pgfqpoint{0.695925in}{2.750713in}}%
\pgfusepath{stroke}%
\end{pgfscope}%
\begin{pgfscope}%
\pgfpathrectangle{\pgfqpoint{0.460969in}{0.576229in}}{\pgfqpoint{5.169031in}{3.027889in}}%
\pgfusepath{clip}%
\pgfsetbuttcap%
\pgfsetmiterjoin%
\definecolor{currentfill}{rgb}{1.000000,0.498039,0.054902}%
\pgfsetfillcolor{currentfill}%
\pgfsetlinewidth{1.003750pt}%
\definecolor{currentstroke}{rgb}{1.000000,0.498039,0.054902}%
\pgfsetstrokecolor{currentstroke}%
\pgfsetdash{}{0pt}%
\pgfsys@defobject{currentmarker}{\pgfqpoint{-0.041667in}{-0.041667in}}{\pgfqpoint{0.041667in}{0.041667in}}{%
\pgfpathmoveto{\pgfqpoint{-0.041667in}{-0.041667in}}%
\pgfpathlineto{\pgfqpoint{0.041667in}{-0.041667in}}%
\pgfpathlineto{\pgfqpoint{0.041667in}{0.041667in}}%
\pgfpathlineto{\pgfqpoint{-0.041667in}{0.041667in}}%
\pgfpathlineto{\pgfqpoint{-0.041667in}{-0.041667in}}%
\pgfpathclose%
\pgfusepath{stroke,fill}%
}%
\begin{pgfscope}%
\pgfsys@transformshift{5.395044in}{2.640971in}%
\pgfsys@useobject{currentmarker}{}%
\end{pgfscope}%
\begin{pgfscope}%
\pgfsys@transformshift{4.796735in}{2.628480in}%
\pgfsys@useobject{currentmarker}{}%
\end{pgfscope}%
\begin{pgfscope}%
\pgfsys@transformshift{4.236056in}{2.611099in}%
\pgfsys@useobject{currentmarker}{}%
\end{pgfscope}%
\begin{pgfscope}%
\pgfsys@transformshift{3.709564in}{2.603767in}%
\pgfsys@useobject{currentmarker}{}%
\end{pgfscope}%
\begin{pgfscope}%
\pgfsys@transformshift{3.214225in}{2.590559in}%
\pgfsys@useobject{currentmarker}{}%
\end{pgfscope}%
\begin{pgfscope}%
\pgfsys@transformshift{2.747355in}{2.601701in}%
\pgfsys@useobject{currentmarker}{}%
\end{pgfscope}%
\begin{pgfscope}%
\pgfsys@transformshift{2.306566in}{2.605845in}%
\pgfsys@useobject{currentmarker}{}%
\end{pgfscope}%
\begin{pgfscope}%
\pgfsys@transformshift{1.889734in}{2.618593in}%
\pgfsys@useobject{currentmarker}{}%
\end{pgfscope}%
\begin{pgfscope}%
\pgfsys@transformshift{1.494956in}{2.649167in}%
\pgfsys@useobject{currentmarker}{}%
\end{pgfscope}%
\begin{pgfscope}%
\pgfsys@transformshift{1.120527in}{2.677331in}%
\pgfsys@useobject{currentmarker}{}%
\end{pgfscope}%
\begin{pgfscope}%
\pgfsys@transformshift{0.695925in}{2.750713in}%
\pgfsys@useobject{currentmarker}{}%
\end{pgfscope}%
\end{pgfscope}%
\begin{pgfscope}%
\pgfpathrectangle{\pgfqpoint{0.460969in}{0.576229in}}{\pgfqpoint{5.169031in}{3.027889in}}%
\pgfusepath{clip}%
\pgfsetrectcap%
\pgfsetroundjoin%
\pgfsetlinewidth{2.007500pt}%
\definecolor{currentstroke}{rgb}{0.172549,0.627451,0.172549}%
\pgfsetstrokecolor{currentstroke}%
\pgfsetdash{}{0pt}%
\pgfpathmoveto{\pgfqpoint{5.395044in}{0.713860in}}%
\pgfpathlineto{\pgfqpoint{4.796735in}{0.736650in}}%
\pgfpathlineto{\pgfqpoint{4.236056in}{0.763730in}}%
\pgfpathlineto{\pgfqpoint{3.709564in}{0.804943in}}%
\pgfpathlineto{\pgfqpoint{3.214225in}{0.867800in}}%
\pgfpathlineto{\pgfqpoint{2.747355in}{0.926120in}}%
\pgfpathlineto{\pgfqpoint{2.306566in}{0.978730in}}%
\pgfpathlineto{\pgfqpoint{1.889734in}{1.020704in}}%
\pgfpathlineto{\pgfqpoint{1.494956in}{1.078085in}}%
\pgfpathlineto{\pgfqpoint{1.120527in}{1.120580in}}%
\pgfpathlineto{\pgfqpoint{0.695925in}{1.204019in}}%
\pgfusepath{stroke}%
\end{pgfscope}%
\begin{pgfscope}%
\pgfpathrectangle{\pgfqpoint{0.460969in}{0.576229in}}{\pgfqpoint{5.169031in}{3.027889in}}%
\pgfusepath{clip}%
\pgfsetbuttcap%
\pgfsetmiterjoin%
\definecolor{currentfill}{rgb}{0.172549,0.627451,0.172549}%
\pgfsetfillcolor{currentfill}%
\pgfsetlinewidth{1.003750pt}%
\definecolor{currentstroke}{rgb}{0.172549,0.627451,0.172549}%
\pgfsetstrokecolor{currentstroke}%
\pgfsetdash{}{0pt}%
\pgfsys@defobject{currentmarker}{\pgfqpoint{-0.041667in}{-0.041667in}}{\pgfqpoint{0.041667in}{0.041667in}}{%
\pgfpathmoveto{\pgfqpoint{0.000000in}{0.041667in}}%
\pgfpathlineto{\pgfqpoint{-0.041667in}{-0.041667in}}%
\pgfpathlineto{\pgfqpoint{0.041667in}{-0.041667in}}%
\pgfpathlineto{\pgfqpoint{0.000000in}{0.041667in}}%
\pgfpathclose%
\pgfusepath{stroke,fill}%
}%
\begin{pgfscope}%
\pgfsys@transformshift{5.395044in}{0.713860in}%
\pgfsys@useobject{currentmarker}{}%
\end{pgfscope}%
\begin{pgfscope}%
\pgfsys@transformshift{4.796735in}{0.736650in}%
\pgfsys@useobject{currentmarker}{}%
\end{pgfscope}%
\begin{pgfscope}%
\pgfsys@transformshift{4.236056in}{0.763730in}%
\pgfsys@useobject{currentmarker}{}%
\end{pgfscope}%
\begin{pgfscope}%
\pgfsys@transformshift{3.709564in}{0.804943in}%
\pgfsys@useobject{currentmarker}{}%
\end{pgfscope}%
\begin{pgfscope}%
\pgfsys@transformshift{3.214225in}{0.867800in}%
\pgfsys@useobject{currentmarker}{}%
\end{pgfscope}%
\begin{pgfscope}%
\pgfsys@transformshift{2.747355in}{0.926120in}%
\pgfsys@useobject{currentmarker}{}%
\end{pgfscope}%
\begin{pgfscope}%
\pgfsys@transformshift{2.306566in}{0.978730in}%
\pgfsys@useobject{currentmarker}{}%
\end{pgfscope}%
\begin{pgfscope}%
\pgfsys@transformshift{1.889734in}{1.020704in}%
\pgfsys@useobject{currentmarker}{}%
\end{pgfscope}%
\begin{pgfscope}%
\pgfsys@transformshift{1.494956in}{1.078085in}%
\pgfsys@useobject{currentmarker}{}%
\end{pgfscope}%
\begin{pgfscope}%
\pgfsys@transformshift{1.120527in}{1.120580in}%
\pgfsys@useobject{currentmarker}{}%
\end{pgfscope}%
\begin{pgfscope}%
\pgfsys@transformshift{0.695925in}{1.204019in}%
\pgfsys@useobject{currentmarker}{}%
\end{pgfscope}%
\end{pgfscope}%
\begin{pgfscope}%
\pgfpathrectangle{\pgfqpoint{0.460969in}{0.576229in}}{\pgfqpoint{5.169031in}{3.027889in}}%
\pgfusepath{clip}%
\pgfsetrectcap%
\pgfsetroundjoin%
\pgfsetlinewidth{2.007500pt}%
\definecolor{currentstroke}{rgb}{0.839216,0.152941,0.156863}%
\pgfsetstrokecolor{currentstroke}%
\pgfsetdash{}{0pt}%
\pgfpathmoveto{\pgfqpoint{5.395044in}{3.277177in}}%
\pgfpathlineto{\pgfqpoint{4.236056in}{3.309484in}}%
\pgfpathlineto{\pgfqpoint{3.709564in}{3.329531in}}%
\pgfpathlineto{\pgfqpoint{3.214225in}{3.349426in}}%
\pgfpathlineto{\pgfqpoint{2.747355in}{3.375110in}}%
\pgfpathlineto{\pgfqpoint{2.306566in}{3.426925in}}%
\pgfpathlineto{\pgfqpoint{1.889734in}{3.449912in}}%
\pgfpathlineto{\pgfqpoint{1.494956in}{3.466487in}}%
\pgfpathlineto{\pgfqpoint{1.120527in}{3.455669in}}%
\pgfpathlineto{\pgfqpoint{0.695925in}{3.466487in}}%
\pgfusepath{stroke}%
\end{pgfscope}%
\begin{pgfscope}%
\pgfpathrectangle{\pgfqpoint{0.460969in}{0.576229in}}{\pgfqpoint{5.169031in}{3.027889in}}%
\pgfusepath{clip}%
\pgfsetbuttcap%
\pgfsetmiterjoin%
\definecolor{currentfill}{rgb}{0.839216,0.152941,0.156863}%
\pgfsetfillcolor{currentfill}%
\pgfsetlinewidth{1.003750pt}%
\definecolor{currentstroke}{rgb}{0.839216,0.152941,0.156863}%
\pgfsetstrokecolor{currentstroke}%
\pgfsetdash{}{0pt}%
\pgfsys@defobject{currentmarker}{\pgfqpoint{-0.035355in}{-0.058926in}}{\pgfqpoint{0.035355in}{0.058926in}}{%
\pgfpathmoveto{\pgfqpoint{-0.000000in}{-0.058926in}}%
\pgfpathlineto{\pgfqpoint{0.035355in}{0.000000in}}%
\pgfpathlineto{\pgfqpoint{0.000000in}{0.058926in}}%
\pgfpathlineto{\pgfqpoint{-0.035355in}{0.000000in}}%
\pgfpathlineto{\pgfqpoint{-0.000000in}{-0.058926in}}%
\pgfpathclose%
\pgfusepath{stroke,fill}%
}%
\begin{pgfscope}%
\pgfsys@transformshift{5.395044in}{3.277177in}%
\pgfsys@useobject{currentmarker}{}%
\end{pgfscope}%
\begin{pgfscope}%
\pgfsys@transformshift{4.236056in}{3.309484in}%
\pgfsys@useobject{currentmarker}{}%
\end{pgfscope}%
\begin{pgfscope}%
\pgfsys@transformshift{3.709564in}{3.329531in}%
\pgfsys@useobject{currentmarker}{}%
\end{pgfscope}%
\begin{pgfscope}%
\pgfsys@transformshift{3.214225in}{3.349426in}%
\pgfsys@useobject{currentmarker}{}%
\end{pgfscope}%
\begin{pgfscope}%
\pgfsys@transformshift{2.747355in}{3.375110in}%
\pgfsys@useobject{currentmarker}{}%
\end{pgfscope}%
\begin{pgfscope}%
\pgfsys@transformshift{2.306566in}{3.426925in}%
\pgfsys@useobject{currentmarker}{}%
\end{pgfscope}%
\begin{pgfscope}%
\pgfsys@transformshift{1.889734in}{3.449912in}%
\pgfsys@useobject{currentmarker}{}%
\end{pgfscope}%
\begin{pgfscope}%
\pgfsys@transformshift{1.494956in}{3.466487in}%
\pgfsys@useobject{currentmarker}{}%
\end{pgfscope}%
\begin{pgfscope}%
\pgfsys@transformshift{1.120527in}{3.455669in}%
\pgfsys@useobject{currentmarker}{}%
\end{pgfscope}%
\begin{pgfscope}%
\pgfsys@transformshift{0.695925in}{3.466487in}%
\pgfsys@useobject{currentmarker}{}%
\end{pgfscope}%
\end{pgfscope}%
\begin{pgfscope}%
\pgfsetrectcap%
\pgfsetmiterjoin%
\pgfsetlinewidth{1.003750pt}%
\definecolor{currentstroke}{rgb}{0.000000,0.000000,0.000000}%
\pgfsetstrokecolor{currentstroke}%
\pgfsetdash{}{0pt}%
\pgfpathmoveto{\pgfqpoint{0.460969in}{0.576229in}}%
\pgfpathlineto{\pgfqpoint{0.460969in}{3.604118in}}%
\pgfusepath{stroke}%
\end{pgfscope}%
\begin{pgfscope}%
\pgfsetrectcap%
\pgfsetmiterjoin%
\pgfsetlinewidth{1.003750pt}%
\definecolor{currentstroke}{rgb}{0.000000,0.000000,0.000000}%
\pgfsetstrokecolor{currentstroke}%
\pgfsetdash{}{0pt}%
\pgfpathmoveto{\pgfqpoint{0.460969in}{0.576229in}}%
\pgfpathlineto{\pgfqpoint{5.630000in}{0.576229in}}%
\pgfusepath{stroke}%
\end{pgfscope}%
\begin{pgfscope}%
\pgfsetbuttcap%
\pgfsetmiterjoin%
\definecolor{currentfill}{rgb}{1.000000,1.000000,1.000000}%
\pgfsetfillcolor{currentfill}%
\pgfsetfillopacity{0.800000}%
\pgfsetlinewidth{1.003750pt}%
\definecolor{currentstroke}{rgb}{0.800000,0.800000,0.800000}%
\pgfsetstrokecolor{currentstroke}%
\pgfsetstrokeopacity{0.800000}%
\pgfsetdash{}{0pt}%
\pgfpathmoveto{\pgfqpoint{4.665140in}{1.593885in}}%
\pgfpathlineto{\pgfqpoint{5.513333in}{1.593885in}}%
\pgfpathquadraticcurveto{\pgfqpoint{5.546667in}{1.593885in}}{\pgfqpoint{5.546667in}{1.627218in}}%
\pgfpathlineto{\pgfqpoint{5.546667in}{2.553129in}}%
\pgfpathquadraticcurveto{\pgfqpoint{5.546667in}{2.586463in}}{\pgfqpoint{5.513333in}{2.586463in}}%
\pgfpathlineto{\pgfqpoint{4.665140in}{2.586463in}}%
\pgfpathquadraticcurveto{\pgfqpoint{4.631807in}{2.586463in}}{\pgfqpoint{4.631807in}{2.553129in}}%
\pgfpathlineto{\pgfqpoint{4.631807in}{1.627218in}}%
\pgfpathquadraticcurveto{\pgfqpoint{4.631807in}{1.593885in}}{\pgfqpoint{4.665140in}{1.593885in}}%
\pgfpathlineto{\pgfqpoint{4.665140in}{1.593885in}}%
\pgfpathclose%
\pgfusepath{stroke,fill}%
\end{pgfscope}%
\begin{pgfscope}%
\pgfsetrectcap%
\pgfsetroundjoin%
\pgfsetlinewidth{2.007500pt}%
\definecolor{currentstroke}{rgb}{0.121569,0.466667,0.705882}%
\pgfsetstrokecolor{currentstroke}%
\pgfsetdash{}{0pt}%
\pgfpathmoveto{\pgfqpoint{4.698473in}{2.461463in}}%
\pgfpathlineto{\pgfqpoint{4.865140in}{2.461463in}}%
\pgfpathlineto{\pgfqpoint{5.031807in}{2.461463in}}%
\pgfusepath{stroke}%
\end{pgfscope}%
\begin{pgfscope}%
\pgfsetbuttcap%
\pgfsetroundjoin%
\definecolor{currentfill}{rgb}{0.121569,0.466667,0.705882}%
\pgfsetfillcolor{currentfill}%
\pgfsetlinewidth{1.003750pt}%
\definecolor{currentstroke}{rgb}{0.121569,0.466667,0.705882}%
\pgfsetstrokecolor{currentstroke}%
\pgfsetdash{}{0pt}%
\pgfsys@defobject{currentmarker}{\pgfqpoint{-0.041667in}{-0.041667in}}{\pgfqpoint{0.041667in}{0.041667in}}{%
\pgfpathmoveto{\pgfqpoint{0.000000in}{-0.041667in}}%
\pgfpathcurveto{\pgfqpoint{0.011050in}{-0.041667in}}{\pgfqpoint{0.021649in}{-0.037276in}}{\pgfqpoint{0.029463in}{-0.029463in}}%
\pgfpathcurveto{\pgfqpoint{0.037276in}{-0.021649in}}{\pgfqpoint{0.041667in}{-0.011050in}}{\pgfqpoint{0.041667in}{0.000000in}}%
\pgfpathcurveto{\pgfqpoint{0.041667in}{0.011050in}}{\pgfqpoint{0.037276in}{0.021649in}}{\pgfqpoint{0.029463in}{0.029463in}}%
\pgfpathcurveto{\pgfqpoint{0.021649in}{0.037276in}}{\pgfqpoint{0.011050in}{0.041667in}}{\pgfqpoint{0.000000in}{0.041667in}}%
\pgfpathcurveto{\pgfqpoint{-0.011050in}{0.041667in}}{\pgfqpoint{-0.021649in}{0.037276in}}{\pgfqpoint{-0.029463in}{0.029463in}}%
\pgfpathcurveto{\pgfqpoint{-0.037276in}{0.021649in}}{\pgfqpoint{-0.041667in}{0.011050in}}{\pgfqpoint{-0.041667in}{0.000000in}}%
\pgfpathcurveto{\pgfqpoint{-0.041667in}{-0.011050in}}{\pgfqpoint{-0.037276in}{-0.021649in}}{\pgfqpoint{-0.029463in}{-0.029463in}}%
\pgfpathcurveto{\pgfqpoint{-0.021649in}{-0.037276in}}{\pgfqpoint{-0.011050in}{-0.041667in}}{\pgfqpoint{0.000000in}{-0.041667in}}%
\pgfpathlineto{\pgfqpoint{0.000000in}{-0.041667in}}%
\pgfpathclose%
\pgfusepath{stroke,fill}%
}%
\begin{pgfscope}%
\pgfsys@transformshift{4.865140in}{2.461463in}%
\pgfsys@useobject{currentmarker}{}%
\end{pgfscope}%
\end{pgfscope}%
\begin{pgfscope}%
\definecolor{textcolor}{rgb}{0.000000,0.000000,0.000000}%
\pgfsetstrokecolor{textcolor}%
\pgfsetfillcolor{textcolor}%
\pgftext[x=5.165140in,y=2.403129in,left,base]{\color{textcolor}{\rmfamily\fontsize{12.000000}{14.400000}\selectfont\catcode`\^=\active\def^{\ifmmode\sp\else\^{}\fi}\catcode`\%=\active\def%{\%}Si}}%
\end{pgfscope}%
\begin{pgfscope}%
\pgfsetrectcap%
\pgfsetroundjoin%
\pgfsetlinewidth{2.007500pt}%
\definecolor{currentstroke}{rgb}{1.000000,0.498039,0.054902}%
\pgfsetstrokecolor{currentstroke}%
\pgfsetdash{}{0pt}%
\pgfpathmoveto{\pgfqpoint{4.698473in}{2.225818in}}%
\pgfpathlineto{\pgfqpoint{4.865140in}{2.225818in}}%
\pgfpathlineto{\pgfqpoint{5.031807in}{2.225818in}}%
\pgfusepath{stroke}%
\end{pgfscope}%
\begin{pgfscope}%
\pgfsetbuttcap%
\pgfsetmiterjoin%
\definecolor{currentfill}{rgb}{1.000000,0.498039,0.054902}%
\pgfsetfillcolor{currentfill}%
\pgfsetlinewidth{1.003750pt}%
\definecolor{currentstroke}{rgb}{1.000000,0.498039,0.054902}%
\pgfsetstrokecolor{currentstroke}%
\pgfsetdash{}{0pt}%
\pgfsys@defobject{currentmarker}{\pgfqpoint{-0.041667in}{-0.041667in}}{\pgfqpoint{0.041667in}{0.041667in}}{%
\pgfpathmoveto{\pgfqpoint{-0.041667in}{-0.041667in}}%
\pgfpathlineto{\pgfqpoint{0.041667in}{-0.041667in}}%
\pgfpathlineto{\pgfqpoint{0.041667in}{0.041667in}}%
\pgfpathlineto{\pgfqpoint{-0.041667in}{0.041667in}}%
\pgfpathlineto{\pgfqpoint{-0.041667in}{-0.041667in}}%
\pgfpathclose%
\pgfusepath{stroke,fill}%
}%
\begin{pgfscope}%
\pgfsys@transformshift{4.865140in}{2.225818in}%
\pgfsys@useobject{currentmarker}{}%
\end{pgfscope}%
\end{pgfscope}%
\begin{pgfscope}%
\definecolor{textcolor}{rgb}{0.000000,0.000000,0.000000}%
\pgfsetstrokecolor{textcolor}%
\pgfsetfillcolor{textcolor}%
\pgftext[x=5.165140in,y=2.167485in,left,base]{\color{textcolor}{\rmfamily\fontsize{12.000000}{14.400000}\selectfont\catcode`\^=\active\def^{\ifmmode\sp\else\^{}\fi}\catcode`\%=\active\def%{\%}Ge}}%
\end{pgfscope}%
\begin{pgfscope}%
\pgfsetrectcap%
\pgfsetroundjoin%
\pgfsetlinewidth{2.007500pt}%
\definecolor{currentstroke}{rgb}{0.172549,0.627451,0.172549}%
\pgfsetstrokecolor{currentstroke}%
\pgfsetdash{}{0pt}%
\pgfpathmoveto{\pgfqpoint{4.698473in}{1.990174in}}%
\pgfpathlineto{\pgfqpoint{4.865140in}{1.990174in}}%
\pgfpathlineto{\pgfqpoint{5.031807in}{1.990174in}}%
\pgfusepath{stroke}%
\end{pgfscope}%
\begin{pgfscope}%
\pgfsetbuttcap%
\pgfsetmiterjoin%
\definecolor{currentfill}{rgb}{0.172549,0.627451,0.172549}%
\pgfsetfillcolor{currentfill}%
\pgfsetlinewidth{1.003750pt}%
\definecolor{currentstroke}{rgb}{0.172549,0.627451,0.172549}%
\pgfsetstrokecolor{currentstroke}%
\pgfsetdash{}{0pt}%
\pgfsys@defobject{currentmarker}{\pgfqpoint{-0.041667in}{-0.041667in}}{\pgfqpoint{0.041667in}{0.041667in}}{%
\pgfpathmoveto{\pgfqpoint{0.000000in}{0.041667in}}%
\pgfpathlineto{\pgfqpoint{-0.041667in}{-0.041667in}}%
\pgfpathlineto{\pgfqpoint{0.041667in}{-0.041667in}}%
\pgfpathlineto{\pgfqpoint{0.000000in}{0.041667in}}%
\pgfpathclose%
\pgfusepath{stroke,fill}%
}%
\begin{pgfscope}%
\pgfsys@transformshift{4.865140in}{1.990174in}%
\pgfsys@useobject{currentmarker}{}%
\end{pgfscope}%
\end{pgfscope}%
\begin{pgfscope}%
\definecolor{textcolor}{rgb}{0.000000,0.000000,0.000000}%
\pgfsetstrokecolor{textcolor}%
\pgfsetfillcolor{textcolor}%
\pgftext[x=5.165140in,y=1.931840in,left,base]{\color{textcolor}{\rmfamily\fontsize{12.000000}{14.400000}\selectfont\catcode`\^=\active\def^{\ifmmode\sp\else\^{}\fi}\catcode`\%=\active\def%{\%}SiC}}%
\end{pgfscope}%
\begin{pgfscope}%
\pgfsetrectcap%
\pgfsetroundjoin%
\pgfsetlinewidth{2.007500pt}%
\definecolor{currentstroke}{rgb}{0.839216,0.152941,0.156863}%
\pgfsetstrokecolor{currentstroke}%
\pgfsetdash{}{0pt}%
\pgfpathmoveto{\pgfqpoint{4.698473in}{1.754529in}}%
\pgfpathlineto{\pgfqpoint{4.865140in}{1.754529in}}%
\pgfpathlineto{\pgfqpoint{5.031807in}{1.754529in}}%
\pgfusepath{stroke}%
\end{pgfscope}%
\begin{pgfscope}%
\pgfsetbuttcap%
\pgfsetmiterjoin%
\definecolor{currentfill}{rgb}{0.839216,0.152941,0.156863}%
\pgfsetfillcolor{currentfill}%
\pgfsetlinewidth{1.003750pt}%
\definecolor{currentstroke}{rgb}{0.839216,0.152941,0.156863}%
\pgfsetstrokecolor{currentstroke}%
\pgfsetdash{}{0pt}%
\pgfsys@defobject{currentmarker}{\pgfqpoint{-0.035355in}{-0.058926in}}{\pgfqpoint{0.035355in}{0.058926in}}{%
\pgfpathmoveto{\pgfqpoint{-0.000000in}{-0.058926in}}%
\pgfpathlineto{\pgfqpoint{0.035355in}{0.000000in}}%
\pgfpathlineto{\pgfqpoint{0.000000in}{0.058926in}}%
\pgfpathlineto{\pgfqpoint{-0.035355in}{0.000000in}}%
\pgfpathlineto{\pgfqpoint{-0.000000in}{-0.058926in}}%
\pgfpathclose%
\pgfusepath{stroke,fill}%
}%
\begin{pgfscope}%
\pgfsys@transformshift{4.865140in}{1.754529in}%
\pgfsys@useobject{currentmarker}{}%
\end{pgfscope}%
\end{pgfscope}%
\begin{pgfscope}%
\definecolor{textcolor}{rgb}{0.000000,0.000000,0.000000}%
\pgfsetstrokecolor{textcolor}%
\pgfsetfillcolor{textcolor}%
\pgftext[x=5.165140in,y=1.696196in,left,base]{\color{textcolor}{\rmfamily\fontsize{12.000000}{14.400000}\selectfont\catcode`\^=\active\def^{\ifmmode\sp\else\^{}\fi}\catcode`\%=\active\def%{\%}InSb}}%
\end{pgfscope}%
\end{pgfpicture}%
\makeatother%
\endgroup%

    \caption{Температурные зависимости удельной проводимости полупроводников}
    \label{fig:ex2-graph}
\end{figure}

\subsection*{ \boxed{\text{ Задание 3. }} }
\begin{quote}
    По данным \cref{tab:semiconductors} рассчитать концентрации собственных носителей заряда в полупроводниках $Si$, $Ge$, $InSb$ и $SiC$ при $T = 300$ К по формуле $n_i=p_i = \sqrt{N_cN_\nu}\exp\left({-\frac{\Delta\text{Э}}{2kT}}\right)$.
\end{quote}

Сперва вычислим $2kT$ в эВ:
\begin{equation}
    \label{eq:2kt_si_to_ev}
    2kT = \frac{2 \cdot 1,38 \cdot 10^{-23} \frac{\text{Дж}}{\text{К}} \cdot 300 \text{К}}{1,602 \cdot 10^{-19} \frac{\text{Дж}}{\text{эВ}}} = \frac{8,28 \cdot 10^{-21}}{1,602 \cdot 10^{-19}} \text{ эВ} \approx 0,0517 \text{ эВ}
\end{equation}

1. $Si$:
\begin{equation}
    \label{eq:ni_si}
    n_{i, \text{Si}} = \sqrt{2,74 \cdot 10^{25} \cdot 1,05 \cdot 10^{25}} \cdot \exp\left(-\frac{1,12}{0,0517}\right) \approx 6,62 \cdot 10^{15} \text{ м}^{-3}
\end{equation}

2. $Ge$:
\begin{equation}
    \label{eq:ni_ge}
    n_{i, \text{Ge}} = \sqrt{1,02 \cdot 10^{25} \cdot 0,61 \cdot 10^{25}} \cdot \exp\left(-\frac{0,66}{0,0517}\right) \approx 2,25 \cdot 10^{19} \text{ м}^{-3}
\end{equation}

3. $InSb$:
\begin{equation}
    \label{eq:ni_insb}
    n_{i, \text{InSb}} = \sqrt{3,7 \cdot 10^{22} \cdot 6,3 \cdot 10^{24}} \cdot \exp\left(-\frac{0,18}{0,0517}\right) \approx 1,48 \cdot 10^{22} \text{ м}^{-3}
\end{equation}

4. $SiC$:
\begin{equation}
    \label{eq:ni_sic}
    n_{i, \text{SiC}} = \sqrt{1,44 \cdot 10^{25} \cdot 1,93 \cdot 10^{25}} \cdot \exp\left(-\frac{2,90}{0,0517}\right) \approx 7,26 \cdot 10^{0} \text{ м}^{-3}
\end{equation}

\subsection*{ \boxed{\text{ Задание 4. }} }
\begin{quote}
    Оценить значения собственной удельной проводимости в этих полупроводниках при $300$ К формулой $\gamma_i=qn(\mu_n+\mu_p)$.
\end{quote}

1. $Si$:
\begin{equation}
    \label{eq:gamma_si}
    \gamma_{i, \text{Si}} = 1,602 \cdot 10^{-19} \cdot 6,63 \cdot 10^{15} \cdot (0,13 + 0,05) \approx 1,91 \cdot 10^{-4} \text{ См/м}
\end{equation}

2. $Ge$:
\begin{equation}
    \label{eq:gamma_ge}
    \gamma_{i, \text{Ge}} = 1,602 \cdot 10^{-19} \cdot 2,25 \cdot 10^{19} \cdot (0,39 + 0,19) \approx 2,09 \text{ См/м}
\end{equation}

3. $InSb$:
\begin{equation}
    \label{eq:gamma_insb}
    \gamma_{i, \text{InSb}} = 1,602 \cdot 10^{-19} \cdot 1,48 \cdot 10^{22} \cdot (7,8 + 0,075) \approx 1,87 \cdot 10^{4} \text{ См/м}
\end{equation}

4. $SiC$:
\begin{equation}
    \label{eq:gamma_sic}
    \gamma_{i, \text{SiC}} = 1,602 \cdot 10^{-19} \cdot 7,31 \cdot (0,04 + 0,006) \approx 5,35 \cdot 10^{-20} \text{ См/м}
\end{equation}

\subsection*{ \boxed{\text{ Задание 5. }} }
\begin{quote}
    Сравнить полученные в результате расчетов значения $\gamma_i$ со своими экспериментальными данными $\gamma_\text{эксп}$ (\cref{tab:semiconductors-results-si,tab:semiconductors-results-ge,tab:semiconductors-results-sic,tab:semiconductors-results-insb}) и решить, какие же носители (собственные или примесные) определяют электрическую проводимость исследуемых образцов.

    Если, согласно проведенному анализу, в полупроводнике наблюдается только примесная электропроводность ($\gamma_{\text{эксп}} \gg \gamma_i$), следует оценить, все ли примеси ионизированы в исследованном температурном интервале или нет. 
\end{quote}

Сравним расчетные значения собственной проводимости $\gamma_i$ с экспериментальными данными $\gamma_{\text{эксп}}$ при $T \approx 300$~К:

\begin{table}[H]
    \centering
    \caption{Сравнение собственной и экспериментальной проводимости при 300 К}
    \label{tab:sravnenie}
    \begin{tabularx}{\linewidth}{ L C C C }
        Материал & $\gamma_i$, См/м & $\gamma_{\text{эксп}}$ (300 К), См/м & Характер проводимости \\
        \midrule
        Si   & $1,91 \cdot 10^{-4}$ & 1355,0 & $\gamma_{\text{эксп}} \gg \gamma_i$ (примесная) \\
        Ge   & 2,09                 & 513,7  & $\gamma_{\text{эксп}} \gg \gamma_i$ (примесная) \\
        InSb & $1,87 \cdot 10^{4}$  & 3378,4 & $\gamma_{\text{эксп}} \approx \gamma_i$ (собственная) \\
        SiC  & $5,35 \cdot 10^{-20}$ & 1,71   & $\gamma_{\text{эксп}} \gg \gamma_i$ (примесная) \\
    \end{tabularx}
\end{table}

На основании \cref{tab:sravnenie}, установлено, что примесной проводимостью обладают $Si$, $Ge$ и $SiC$. Проведем для них оценку степени ионизации примесей в исследованном интервале температур. Для анализа сравним энергию ионизации примеси $\Delta\text{Э}_{\text{пр}}$ с энергией тепловой генерации при максимальной температуре $kT_{\max} = 400$ К.

\begin{enumerate}
    \item \textit{Кремний ($Si$) и Германий ($Ge$)}. С ростом температуры проводимость изменяется незначительно, следовательно концентрация носителей постоянна. Тепловой энергии $kT_{\max}$ достаточно для отрыва всех электронов от примесных центров. Условие $\Delta \text{Э}_{\text{пр}} \ll kT_{\max}$ выполняется, примеси полностью ионизированы.
    
    \item \textit{Карбид кремния (SiC)}. Проводимость возрастает во всем диапазоне температур. Тепловой энергии недостаточно для полной ионизации. Условие $\Delta \text{Э}_{\text{пр}} \ll kT_{\max}$ не выполняется, следовательно примеси ионизированы лишь частично.
\end{enumerate}

Так как для кремния и германия условие полной ионизации выполняется, $N_{\text{пр}} \approx n_\text{пр}$ при 300 К. Рассчитаем её, полагая проводимость электронной ($n$-тип, $\mu_n > \mu_p$):

$$ N_{\text{пр}} \approx n = \frac{\gamma_{\text{эксп}}}{q \mu_n}. $$

1. Для кремния ($Si$):
$$ N_{\text{пр, Si}} = \frac{1355,0}{1,602 \cdot 10^{-19} \cdot 0,13} \approx 6,51 \cdot 10^{22} \, \text{м}^{-3}. $$

2. Для германия ($Ge$):
$$ N_{\text{пр, Ge}} = \frac{513,7}{1,602 \cdot 10^{-19} \cdot 0,39} \approx 8,22 \cdot 10^{21} \, \text{м}^{-3}. $$

    
\subsection*{ \boxed{\text{ Задание 6. }} }
\begin{quote}
    Если в полупроводнике не все примеси ионизированы, то по наклону кривой $\ln\gamma_{\text{эксп}}(1/T)$ можно найти $\Delta\text{Э}_{\text{пр}}$ по \cref{eq:delta_E_pr}. Рассчитать значения $n_\text{эксп}$ по \cref{eq:n_exp}.

    Рассчитывая $n(T_1)$ и $n(T_2)$ по значению $\gamma_{\text{эксп}}(T_1)$ и $\gamma_{\text{эксп}}(T_2)$, будем полагать, что изменениями подвижности носителей заряда при изменении температуры при неполной ионизации примесей можно пренебречь. 
\end{quote}

Согласно анализу из Задания 5, неполная ионизация примесей наблюдается только в карбиде кремния ($SiC$). Рассчитаем концентрацию носителей заряда $n_{\text{эксп}}$ для крайних точек температурного диапазона по формуле $n = \frac{\gamma}{q(\mu_n + \mu_p)}$.

1. При $T_1 = 298$ К ($\gamma_1 = 1,71$ См/м):
$$ n(T_1) = \frac{1,71}{1,602 \cdot 10^{-19} \cdot (0,04 + 0,006)} = \frac{1,71}{7,369 \cdot 10^{-21}} \approx 2,32 \cdot 10^{20} \, \text{м}^{-3}. $$

2. При $T_2 = 400$ К ($\gamma_2 = 7,30$ См/м):
$$ n(T_2) = \frac{7,30}{7,369 \cdot 10^{-21}} \approx 9,91 \cdot 10^{20} \, \text{м}^{-3}. $$


Рассчитаем энергию ионизации примеси $\Delta \text{Э}_{\text{пр}}$ по двум точкам:

$$ \Delta \text{Э}_{\text{пр}} = 2k \frac{T_2 T_1}{T_2 - T_1} \ln \frac{n(T_2)}{n(T_1)}. $$

Подставим значения (пост. Больцмана $k \approx 8,617 \cdot 10^{-5}$ эВ/К):

$$ \frac{T_2 T_1}{T_2 - T_1} = \frac{400 \cdot 298}{400 - 298} \approx 1168,6 \, \text{К}. $$

$$ \ln \frac{n(T_2)}{n(T_1)} = \ln \frac{9,91 \cdot 10^{20}}{2,32 \cdot 10^{20}} = \ln(4,27) \approx 1,45. $$

$$ \Delta \text{Э}_{\text{пр}} = 2 \cdot (8,617 \cdot 10^{-5}) \cdot 1168,6 \cdot 1,45 \approx \boxed{0,29 \text{ эВ}}. $$

\subsection*{ \boxed{\text{ Задание 7. }} }
\begin{quote}
    Для полупроводников, у которых $\gamma_\text{эксп}\approx\gamma_i$, определить $\Delta\text{Э}$ по форм. (9).

    Значения температур $T_2$ и $T_1$ выбираются таким образом, чтобы соответствующие значения $\gamma_\text{эксп}$ располагались на прямолинейном участке построенной зависимости $\ln\gamma_{\text{эксп}}(1/T)$. 
\end{quote}

Согласно анализу, проведенному в Задании 5, условие $\gamma_{\text{эксп}} \approx \gamma_i$ выполняется для $InSb$. Для определения ширины запрещенной зоны $\Delta \text{Э}$ выберем две точки в диапазоне температур от 298 К до 400 К.

1. Расчет концентраций носителей $n_{\text{эксп}}$ по формуле (7) с учетом температурной зависимости подвижности $\mu(T)$ из табл. 2:
$$ \mu_n(T) = 7,8 \cdot (T/300)^{-1,6} \quad \text{и} \quad \mu_p(T) = 0,075 \cdot (T/300)^{-2,1} $$

Для $T_1 = 298$ К:
$$ \begin{aligned} \mu_n(298) &= 7,8 \cdot (298/300)^{-1,6} \approx 7,884 \text{ м}^2/(\text{В}\cdot\text{с}) \\ \mu_p(298) &= 0,075 \cdot (298/300)^{-2,1} \approx 0,076 \text{ м}^2/(\text{В}\cdot\text{с}) \end{aligned} $$
Суммарная подвижность: $\mu_{sum}(298) \approx 7,960 \text{ м}^2/(\text{В}\cdot\text{с})$.
$$ n(T_1) = \frac{3378,4}{1,602 \cdot 10^{-19} \cdot 7,960} \approx 2,65 \cdot 10^{21} \text{ м}^{-3} $$

Для $T_2 = 400$ К:
$$ \begin{aligned} \mu_n(400) &= 7,8 \cdot (400/300)^{-1,6} \approx 4,923 \text{ м}^2/(\text{В}\cdot\text{с}) \\ \mu_p(400) &= 0,075 \cdot (400/300)^{-2,1} \approx 0,041 \text{ м}^2/(\text{В}\cdot\text{с}) \end{aligned} $$
Суммарная подвижность: $\mu_{sum}(400) \approx 4,964 \text{ м}^2/(\text{В}\cdot\text{с})$.
$$ n(T_2) = \frac{5917,2}{1,602 \cdot 10^{-19} \cdot 4,964} \approx 7,44 \cdot 10^{21} \text{ м}^{-3} $$

2. Определение ширины запрещенной зоны по формуле (9):
$$ \begin{aligned} \Delta \text{Э} &= 2k \frac{T_2 T_1}{T_2 - T_1} \left[ \ln \frac{n(T_2)}{n(T_1)} - \frac{3}{2} \ln \frac{T_2}{T_1} \right] \\ \frac{T_2 T_1}{T_2 - T_1} &= \frac{400 \cdot 298}{400 - 298} \approx 1168,6 \text{ К} \\ \ln \frac{n(T_2)}{n(T_1)} &= \ln \frac{7,44 \cdot 10^{21}}{2,65 \cdot 10^{21}} \approx 1,032 \\ \frac{3}{2} \ln \frac{T_2}{T_1} &= 1,5 \cdot \ln \frac{400}{298} \approx 0,442 \end{aligned} $$

Подставим вычисленные значения в итоговую формулу:
\begin{equation}
    \Delta \text{Э} = 2 \cdot (8,617 \cdot 10^{-5}) \cdot 1168,6 \cdot (1,032 - 0,442) \approx \boxed{0,119 \text{ эВ}}
    \label{eq:delta-e-exp}
\end{equation}


\subsection*{ \boxed{\text{ Задание 8. }} }
\begin{quote}
    Значение $n_\text{эксп}$ рассчитать по формуле $n_\text{эксп} = \frac{\gamma_\text{эксп}}{q(\mu_n+\mu_p)}$.
\end{quote}

Для вычислений используются функции температурной зависимости подвижности $\mu(T)$ из табл. 2. Для примесных полупроводников ($Si, Ge, SiC$) проводимость определяется основными носителями, поэтому используется только $\mu_n(T)$. Для собственного полупроводника ($InSb$) учитывается сумма подвижностей электронов и дырок $\mu_n(T) + \mu_p(T)$.

\begin{itemize}
    \item Для $Si$: $\mu(T) = 0,15 \cdot (T/300)^{-2,5} \text{ м}^2/(\text{В}\cdot\text{с})$;
    \item Для $Ge$: $\mu(T) = 0,39 \cdot (T/300)^{-1,66} \text{ м}^2/(\text{В}\cdot\text{с})$;
    \item Для $SiC$: $\mu(T) = 0,01 \cdot (T/300)^{-1} \text{ м}^2/(\text{В}\cdot\text{с})$;
    \item Для $InSb$: $\mu(T) = 7,8 \cdot (T/300)^{-1,6} + 0,075 \cdot (T/300)^{-2,1} \text{ м}^2/(\text{В}\cdot\text{с})$.
\end{itemize}

Приведем пример расчета для кремния ($Si$) при $T = 298$ К:
$$ \mu_{Si}(298) = 0,15 \cdot \left(\frac{298}{300}\right)^{-2,5} \approx 0,1525 \text{ м}^2/(\text{В}\cdot\text{с}) $$
$$ n_{\text{эксп, Si}} = \frac{1355,0}{1,602 \cdot 10^{-19} \cdot 0,1525} \approx 5,55 \cdot 10^{22} \text{ м}^{-3} $$

Результаты расчетов концентраций $n_{\text{эксп}}$ для всех образцов при каждой температурной точке представлены в \cref{tab:n_exp_summary}.

\begin{table}[H]
    \centering
    \caption{Экспериментальные значения концентрации носителей заряда $n_{\text{эксп}}, \text{м}^{-3}$}
    \label{tab:n_exp_summary}
    \begin{tblr}{
        colspec = { Q[c,m] *{4}{X[c,m]} },
    }
        {$T$, К} & {$n_{\text{эксп, Si}} \cdot 10^{22}$} & {$n_{\text{эксп, Ge}} \cdot 10^{21}$} & {$n_{\text{эксп, SiC}} \cdot 10^{20}$} & {$n_{\text{эксп, InSb}} \cdot 10^{21}$}\\
        \midrule
        298 & 5,55 & 8,14 & 10,60 & 2,65 \\
        308 & 5,83 & 8,28 & 11,73 & ---  \\
        318 & 6,14 & 8,29 & 13,10 & 3,23 \\
        328 & 6,47 & 8,55 & 15,29 & 3,60 \\
        338 & 6,80 & 8,63 & 18,99 & 4,01 \\
        348 & 7,07 & 9,36 & 23,24 & 4,53 \\
        358 & 7,41 & 9,93 & 27,93 & 5,53 \\
        368 & 7,75 & 10,80 & 32,47 & 6,19 \\
        378 & 8,05 & 12,35 & 39,56 & 6,78 \\
        388 & 8,37 & 14,00 & 46,02 & 6,85 \\
        400 & 8,83 & 18,34 & 60,76 & 7,44 \\
    \end{tblr}
\end{table}

\subsection*{ \boxed{\text{ Задание 9. }} }
\begin{quote}
    Для каждого из материалов на построенных зависимостях $\ln\gamma_{\text{эксп}}(1/T)$ определить температурные диапазоны реализации участков: ионизации примеси; истощения примеси; собственной электропроводности. 
\end{quote}

Температурный ход удельной проводимости определяется соотношением $\gamma = q n \mu$. Вид графика зависит от того, какой фактор доминирует: экспоненциальный рост концентрации $n$ (\cref{eq:ni_300}) (ионизация, собственная проводимость) или степенное падение подвижности $\mu$ (\cref{tab:setup_parameters}) (истощение примеси).

В широкозонном карбиде кремния ($SiC$, $\Delta \text{Э} = 2,90$ эВ) тепловой энергии недостаточно для полной ионизации. Наблюдаемый на всем графике экспоненциальный рост проводимости соответствует области ионизации примеси.

В кремнии ($Si$) примеси уже полностью истощены, а энергии для перехода в собственную проводимость еще недостаточно. Монотонный пологий спад $\gamma$ обусловлен исключительно падением подвижности носителей при растущих колебаниях кристаллической решетки. Весь интервал относится к области истощения примеси.

В германии ($Ge$) наблюдается смена механизмов. В низкотемпературной части проводимость слабо падает (или не изменяется), что характерно для истощения примеси. При $10^3/T < 2,96$ начинается резкий экспоненциальный рост, указывающий на генерацию собственных носителей заряда (переход к собственной электропроводности).

Для узкозонного антимонида индия ($InSb$, $\Delta \text{Э} = 0,18$ эВ) в Задании 5 установлено, что уже при $300$ К выполняется условие $\gamma_{\text{эксп}} \approx \gamma_i$. Следовательно, весь исследуемый температурный интервал соответствует области собственной электропроводности.

Систематизированные данные по температурным диапазонам приведены в \cref{table:diapozons}.

\begin{table}[H]
    \centering
    \caption{Температурные диапазоны реализации различных механизмов проводимости}
    \label{table:diapozons}
    \begin{tblr}{
        colspec = {l c c c},
        rows = {m},
    }
        Материал & {Ионизация прим.\\ $10^3/T$, К$^{-1}$} & {Истощение прим. \\ $10^3/T$, К$^{-1}$} & {Собств. проводимость \\ $10^3/T$, К$^{-1}$} \\
        \midrule
        Si & --- & $2,50 \dots 3,36$ & --- \\
        Ge & --- & $2,96 \dots 3,36$ & $2,50 \dots 2,96$ \\
        SiC & $2,50 \dots 3,36$ & --- & --- \\
        InSb & --- & --- & $2,50 \dots 3,36$ \\
    \end{tblr}
\end{table}

\section*{Выводы}

В ходе выполнения лабораторной работы получены следующие результаты:
\begin{enumerate}
    \item Рассчитаны температурные зависимости удельного сопротивления $\rho(T)$, удельной проводимости $\gamma(T)$ и концентрации носителей заряда $n(T)$ для полупроводников $Si$, $Ge$, $SiC$ и $InSb$ в диапазоне температур $298 \dots 400 \text{ К}$.
    \item Установлен преобладающий тип проводимости материалов при $T = 300 \text{ К}$. Для $Si$, $Ge$ и $SiC$ зафиксирована примесная проводимость ($\gamma_{\text{эксп}} \gg \gamma_i$). Для $InSb$ проводимость является собственной ($\gamma_{\text{эксп}} \approx \gamma_i$).
    \item Оценена степень ионизации примесей в исследованном температурном интервале. Установлено, что в $Si$ и $Ge$ примеси ионизированы полностью, в то время как в $SiC$ наблюдается частичная ионизация.
    \item На основе экспериментальных данных вычислена энергия ионизации примеси для карбида кремния: $\Delta\text{Э}_{\text{пр}} \approx 0,29 \text{ эВ}$.
    \item Рассчитана ширина запрещенной зоны для антимонида индия с учетом температурной зависимости подвижности носителей заряда: $\Delta\text{Э} \approx 0,119 \text{ эВ}$.
    \item Построены графики зависимости $\ln\gamma_{\text{эксп}} = f(10^3/T)$ и определены температурные границы участков реализации различных механизмов электропроводности (ионизация примеси, истощение примеси, собственная проводимость) для каждого исследованного образца.
\end{enumerate}