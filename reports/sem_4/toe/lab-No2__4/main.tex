
\section*{Цель работы}
Экспериментальное исследование переходных процессов в линейных цепях при мгновенном изменении сопротивления резистора одной из ветвей и при действии источника ступенчатого напряжения.

\section*{Теоретические сведения}

Переходные процессы в цепях с источником постоянного напряжения возникают при замыкании и размыкании ключа, что вызывает мгновенное изменение параметров цепи (например, сопротивления). При воздействии источника ступенчатого напряжения переходные процессы инициируются скачкообразным изменением входного сигнала.

В исследуемых линейных цепях процессы описываются системами линейных дифференциальных уравнений с постоянными коэффициентами. Любая реакция цепи (напряжение или ток) представляется в виде суммы вынужденной и свободной составляющих:
\begin{equation}
    u(t) = u_{\text{вын}} + u_{\text{св}}(t)
    \label{eq:reaction_sum}
\end{equation}

\begin{figure}[hbt]
    \centering
    \includegraphics[width=0.35\textwidth]{figs/scheme_c.pdf}
    \caption{Цепь первого порядка}
    \label{fig:scheme_c}
\end{figure}

\textit{Вынужденная составляющая} ($u_{\text{вын}}$) в цепях постоянного тока является постоянной величиной. Она рассчитывается по эквивалентной схеме установившегося режима, где индуктивности заменяются короткими замыканиями, а емкости --- разрывами.

\textit{Свободная составляющая} ($u_{\text{св}}$) определяется параметрами цепи и начальными условиями:
\begin{equation}
    u_{\text{св}}(t) = \sum_{k=1}^{n} A_k e^{p_k t},
    \label{eq:free_comp}
\end{equation}
где $n$ --- порядок цепи; $A_k$ --- постоянные интегрирования; $p_k$ --- частоты собственных колебаний (корни характеристического полинома).

Характер переходного процесса зависит от корней $p_k$:
\begin{itemize}
    \item Отрицательные вещественные корни ($p_k = -\alpha_k$) соответствуют апериодическому (затухающему по экспоненте) процессу.
    \item Комплексно-сопряженные корни ($p_{k, k+1} = -\alpha_k \pm j\omega_k$) соответствуют колебательному процессу (затухающей синусоиде).
\end{itemize}

Для $RC$-цепи первого порядка корень характеристического уравнения определяется как (\cref{fig:scheme_c}):
\begin{equation}
    p_1 = -\frac{R_1 + R_2 + R_3}{C_2 R_3 (R_1 + R_2)}
    \label{eq:p1_rc}
\end{equation}
При замыкании ключа сопротивление $R_1 = 0$.

Для $RLC$-цепи второго порядка корни определяются выражением (\cref{fig:scheme_rlc2}):
\begin{equation}
    \begin{split}
        p_{1,2} = -\alpha \pm j\omega = -\frac{1}{2} \left( \frac{R_1 + R_2 + R_3}{C_2 R_3 (R_1 + R_2)} + \frac{R_4}{L} \right) \pm \\
        \pm j \sqrt{ \frac{(R_1 + R_2)(R_3 + R_4) + R_3 R_4}{L C_2 R_3 (R_1 + R_2)} - \frac{1}{4} \left( \frac{R_1 + R_2 + R_3}{C_2 R_3 (R_1 + R_2)} + \frac{R_4}{L} \right)^2 }
    \end{split}
    \label{eq:p12_rlc}
\end{equation}

\section*{Инструкция по выполнению измерений}

\subsection*{Задание 1. Исследование $\bm{RC}$-цепи первого порядка}

\begin{figure}[H]
    \centering
    \includegraphics[width=0.5\linewidth]{figs/scheme_rc1.pdf}
    \caption{Схема для исследования $RC$-цепи первого порядка}
    \label{fig:scheme_rc1}
\end{figure}

\begin{enumerate}
    \item \textit{Сборка схемы.} Соберите цепь согласно \cref{fig:scheme_rc1}. Установите параметры: $C_2 = 0.05$ мкФ; $R_1 = 2$ кОм; $R_2 = 1$ кОм; $R_3 = 4$ кОм.
    \item \textit{Подключение приборов.} 
    \begin{itemize}
        \item Подключите вывод 4 электронного ключа к источнику постоянного напряжения.
        \item К клеммам 1 и 3 ключа подключите генератор сигналов (ГС), соединив его заземленный вывод с клеммой 3.
        \item Канал II осциллографа подключите к конденсатору $C_2$, соединив земляную клемму с «минусом» источника питания.
    \end{itemize}
    \item \textit{Настройка сигналов.} На источнике постоянного напряжения установите $4$ В. На ГС задайте прямоугольные импульсы: частота $1$ кГц (период $T = 1$ мс), амплитуда $4$ В.
    \item \textit{Настройка осциллографа.} 
    \begin{itemize}
        \item Установите синхронизацию в режим «Авт.».
        \item Переведите входы каналов в положение «$\perp$» и совместите линии развертки с нижним уровнем сетки экрана (нулевой уровень).
        \item Переведите канал II в режим «$\approx$». Настройте развертку так, чтобы на экране отображался один полный период коммутации.
    \end{itemize}
    \item \textit{Снятие данных.} Измерьте по осциллограмме установившиеся значения напряжения $u_{1\text{вын}}$ (при замкнутом ключе) и $u_{2\text{вын}}$ (при разомкнутом ключе). Запишите данные в \cref{tab:rc1_data}.
    \item \textit{Определение постоянных времени.} Используя метод касательной к экспоненте на экране осциллографа, определите $\tau_1$ (для процесса при замыкании) и $\tau_2$ (для процесса при размыкании). Занесите в \cref{tab:rc1_data}.
\end{enumerate}

\subsection*{Задание 2. Исследование $\bm{RLC}$-цепи второго порядка}

\begin{figure}[H]
    \centering
    \includegraphics[width=0.5\linewidth]{figs/scheme_rlc2.pdf}
    \caption{Схема для исследования $RLC$-цепи второго порядка}
    \label{fig:scheme_rlc2}
\end{figure}

\begin{enumerate}
    \item \textit{Модификация схемы.} В схеме из Задания 1 параллельно конденсатору $C_2$ подключите последовательно соединенные катушку индуктивности $L = 10$ мГн и резистор $R_4 = 0.2$ кОм (\cref{fig:scheme_rlc2}).
    \item \textit{Снятие данных.} Зафиксируйте осциллограмму напряжения на конденсаторе. Измерьте установившиеся значения $u_{1\text{вын}}$ и $u_{2\text{вын}}$. Запишите в \cref{tab:rlc2_data}.
    \item \textit{Параметры колебаний.} Измерьте амплитуды двух соседних максимумов затухающих колебаний $U_{1m}$ и $U_{2m}$, а также моменты времени их наступления $t_1$ и $t_2$. Данные занесите в \cref{tab:rlc2_data}.
\end{enumerate}

\subsection*{Задание 3. Исследование $\bm{RC}$-цепи второго порядка при ступенчатом воздействии}

\begin{figure}[H]
    \centering
    \includegraphics[width=0.5\linewidth]{figs/scheme_rc2_step.pdf}
    \caption{Схема $RC$-цепи второго порядка}
    \label{fig:scheme_rc2_step}
\end{figure}

\begin{enumerate}
    \item \textit{Сборка схемы.} Соберите цепь согласно \cref{fig:scheme_rc2_step}. Параметры: $R_1 = 2$ кОм; $R_2 = 1$ кОм; $R_3 = 4$ кОм; $C_1 = C_2 = 0.05$ мкФ.
    \item \textit{Настройка ГС.} Установите генерацию прямоугольных импульсов: частота $0.5$ кГц ($T = 2$ мс), амплитуда $4$ В.
    \item \textit{Подключение.} Канал I осциллографа подключите ко входу схемы, канал II --- к конденсатору $C_2$.
    \item \textit{Наблюдение.} Зафиксируйте осциллограммы входного напряжения и напряжения на $C_2$. Качественно выделите экспоненциальные составляющие. Зарисуйте форму сигнала.
\end{enumerate}

\subsection*{Задание 4. Исследование $\bm{RLC}$-цепи третьего порядка при ступенчатом воздействии}

\begin{figure}[H]
    \centering
    \includegraphics[width=0.5\linewidth]{figs/scheme_rlc3_step.pdf}
    \caption{Схема $RLC$-цепи третьего порядка}
    \label{fig:scheme_rlc3_step}
\end{figure}

\begin{enumerate}
    \item \textit{Сборка схемы.} Соберите цепь согласно \cref{fig:scheme_rlc3_step}. Параметры: $R_2 = 1$ кОм; $R_4 = 0.2$ кОм; $L = 10$ мГн; $C_1 = C_2 = 0.05$ мкФ.
    \item \textit{Подключение.} Канал II осциллографа подключите к резистору $R_2$.
    \item \textit{Снятие данных.} Зафиксируйте осциллограмму напряжения $u_2(t)$. Измерьте начальное $u_{2\text{нач}}$ и вынужденное $u_{2\text{вын}}$ значения напряжения. Занесите в \cref{tab:rlc3_data}.
\end{enumerate}

\section*{Инструкция по обработке результатов}

\begin{enumerate}
    \item По данным \cref{tab:rc1_data} рассчитайте экспериментальные значения частот собственных колебаний $p_1$ и $p_2$ через обратные величины измеренных постоянных времени $\tau_1$ и $\tau_2$.
    \item Вычислите теоретические значения корней по \cref{eq:p1_rc} для замкнутого и разомкнутого состояний ключа. Сравните с экспериментом.
    \item По данным \cref{tab:rlc2_data} рассчитайте коэффициент затухания $\alpha$ и частоту затухающих колебаний $\omega$ по формулам:
    \begin{equation}
        \alpha = \frac{1}{t_2 - t_1} \ln\left(\frac{U_{1m}}{U_{2m}}\right); \quad \omega = \frac{2\pi}{t_2 - t_1}
        \label{eq:alpha_omega}
    \end{equation}
    \item Вычислите теоретические значения корней по \cref{eq:p12_rlc} и сопоставьте их с полученными из \cref{eq:alpha_omega}.
    \item Рассчитайте теоретические установившиеся значения напряжений для всех исследованных схем методом эквивалентных схем постоянного тока и сравните их с экспериментальными значениями из протокола.
\end{enumerate}
