%&etulab_fmt
\ifdefined\Department
\else
    \documentclass{etulab}
\fi
% <- ПОДКЛЮЧЕНИЕ СТИЛЯ

% === КОНФИГУРАЦИЯ ДАННЫХ ДЛЯ ТИТУЛЬНОГО ЛИСТА ===
\Department{ТОЭ}
\WorkType{лабораторной работе №4}
\Discipline{теоретические основы электротехники}
\WorkTitle{ИССЛЕДОВАНИЕ ПЕРЕХОДНЫХ ПРОЦЕССОВ В ЛИНЕЙНЫХ ЦЕПЯХ}
\Group{4494}
\Variant{---}
\StudentName{\parbox[t]{4cm}{\raggedleft 
        Рахметов А. Р. \\
        Фролов А. А. 
    }}
\TeacherName{Балданова Ю. А.}
\Year{2026}

% Включение библиографии
% bibtex автоматически будет включен в процесс сборки документа
% при обнаружении файла references.bib в директории с основным _report.tex файлом
% \addbibresource{references.bib}  % Имя файла references.bib не следует менять! Используется оптимизация скрипта сборки документа

% === КОНФИГУРАЦИЯ ДОКУМЕНТА
% --- Если нужна нумерация формул НЕ по разделам.
% --- Было: 0.1, 0.2, ..., 1.1, 1.2, ...
% --- Будет: 1, 2, 3, ...
\renewcommand{\theequation}{\arabic{equation}}

% === СБОРКА ДОКУМЕНТА ===
\begin{document}

 

% --- ТИТУЛЬНЫЙ ЛИСТ ---
\maketitle



% --- ОГЛАВЛЕНИЕ ---
% \tableofcontents
% \thispagestyle{empty}
% \newpage



% --- ВВЕДЕНИЕ ---
% \newpage
% \addcontentsline{toc}{section}{Введение}
% \phantomsection
% \centeredsection{\uppercase{Введение}}
% В современном мире задачи транспортной логистики и автоматизации навигации играют ключевую роль во многих отраслях, от грузоперевозок до разработки беспилотных транспортных средств. Одной из фундаментальных проблем в этой области является построение оптимального маршрута с учетом различных ограничений, таких как запас хода транспортного средства. Эффективное решение этой задачи позволяет сократить временные и энергетические затраты, что обуславливает \textit{актуальность} данной курсовой работы. Разработка алгоритмов, способных находить кратчайший путь в графе с весами, является классической задачей теории графов, имеющей широкое практическое применение.

% Комментарий к абзацу об актуальности:
% Здесь мы идем от общего к частному. Начинаем с широкой области («транспортная логистика», «автоматизация навигации»), затем сужаем ее до конкретной проблемы («построение оптимального маршрута с учетом ограничений»), и, наконец, связываем это с методами решения («задачи теории графов»). Это показывает, что ваша работа вписана в более широкий научный и практический контекст. Ключевое слово *«актуальность»* выделено курсивом для акцента.

Целью данной курсовой работы является разработка программы на математическом языке Matlab для поиска и визуализации оптимального маршрута движения объекта с ограниченным запасом хода между двумя заданными точками на местности с набором пунктов дозаправки.

% Комментарий к цели работы:
% Цель — это одно, максимум два предложения, которые четко и однозначно формулируют конечный результат вашей работы. Используется глагол в неопределенной форме («разработка», «исследование», «создание»). Формулировка цели должна точно соответствовать теме вашей курсовой работы.

Для достижения поставленной цели необходимо было решить следующие задачи:
\begin{enumerate}
    \item Проанализировать исходные данные: характеристики подвижного объекта, координаты начальной, конечной и промежуточных точек (пунктов дозаправки).
    \item Представить карту местности в виде неориентированного взвешенного графа, где вершины соответствуют точкам на местности, а ребра — возможным перемещениям между ними.
    \item Реализовать алгоритм построения графа с учетом ограничения на максимальное расстояние, проходимое объектом без дозаправки.
    \item Реализовать алгоритм поиска кратчайшего пути в графе. В соответствии с заданием (см. \cref{table:source_data}), следует использовать алгоритм Форда-Беллмана.
    \item Разработать функцию для формирования NMEA-подобных сообщений, описывающих движение по оптимальному маршруту.
    \item Создать модуль для графической визуализации исходного графа, всех возможных путей и найденного оптимального маршрута.
    \item Обеспечить сохранение результатов расчетов (длина пути, NMEA-сообщения) в текстовый файл.
\end{enumerate}

% \paragraph{Комментарий к задачам:}
% Задачи — это конкретные шаги, которые вы предприняли для достижения цели. Они должны быть представлены в виде нумерованного списка и отражать логику вашей работы и структуру основной части отчета. По сути, каждый пункт списка задач может стать основой для подраздела в основной главе. Формулировки задач также начинаются с глагола («проанализировать», «представить», «реализовать»).

Объектом исследования является процесс нахождения оптимального пути в дискретной среде с ограничениями.

Предметом исследования являются алгоритмы на графах, в частности алгоритм Форда-Беллмана, и методы их программной реализации в среде Matlab для решения прикладных навигационных задач.

% \paragraph{Комментарий к объекту и предмету:}
% Это формальный, но важный элемент введения.
% *   *Объект* — это более широкое явление или процесс, который вы изучаете. Это ответ на вопрос «что исследуется?».
% *   *Предмет* — это конкретная часть объекта, его свойства или методы его изучения, которые рассматриваются в вашей работе. Это ответ на вопрос «какие аспекты объекта исследуются?».

% Курсовая работа состоит из введения, основной части, заключения, списка использованных источников и приложений.
% Во введении обосновывается актуальность темы, ставятся цель и задачи исследования.
% Основная часть содержит постановку задачи, описание математической модели, описание алгоритмов и программной реализации.
% В заключении приводятся основные выводы по проделанной работе.
% В приложениях содержится листинг кода разработанных программных модулей.

% \paragraph{Комментарий к структуре работы:}
% Этот абзац кратко описывает, из каких частей состоит ваш отчет. Он служит своего рода «содержанием в прозе» и помогает проверяющему быстро сориентироваться в документе.




% --- ПРОТОКОЛ ---
\newpage
\centeredsection{ПРОТОКОЛ НАБЛЮДЕНИЙ}

\begin{enumerate}
    \item \textbf{Исследование свободных процессов в цепи первого порядка (\cref{fig:circuit_a})} \\
    $k_t = 0.2 \text{ мс/дел}$, $k_V = 2 \text{ В/дел}$

    \item \textbf{Исследование свободных процессов в цепи второго порядка (\cref{fig:circuit_b})}
    \begin{itemize}
        \item[а)] Колебательный затухающий: $k_t = 50 \text{ мкс/дел}$, $k_V = 0.1 \text{ В/дел}$, $R_1 = 0.5 \text{ кОм}$
        \item[б)] Апериодический: $k_t = 50 \text{ мкс/дел}$, $k_V = 0.1 \text{ В/дел}$, $R_1 = 3 \text{ кОм}$
        \item[в)] Критический: $k_t = 0.1 \text{ мс/дел}$, $k_V = 0.2 \text{ В/дел}$, $R_1 = 1.5 \text{ кОм}$
        \item[г)] Колебательный незатухающий: $k_t = 20 \text{ мс/дел}$, $k_V = 2 \text{ В/дел}$, $R_1 = 0 \text{ кОм}$
    \end{itemize}

    \item \textbf{Исследование свободных процессов в цепи третьего порядка (\cref{fig:circuit_c})} \\
    $k_t = 50 \text{ мкс/дел}$, $k_V = 0.1 \text{ В/дел}$, $R_1 = 1 \text{ кОм}$
\end{enumerate}

\noindent $U_m = 8$ В; $T_c = 1.2$ мс.

\vfill

\begin{flushright}
    Выполнили студенты гр. 4494 \\
    Рахметов А. Р. \\
    Фролов А. А. \\
    Муравьёв С. И.
\end{flushright}

\vfill

\begin{flushright}
    Преподаватель: Балданова Ю. А. \\
    12.02.26
\end{flushright}

\vfill
\noindent
\makeatletter
    \@StudentName, гр. \@Group ~~\hrulefill~~ «\rule{1cm}{0.4pt}» \rule{3cm}{0.4pt} 20\rule{0.75cm}{0.4pt} г.
\makeatother
\newpage
% --- ПРОТОКОЛ ---



% --- ОСНОВНОЕ СОДЕРЖАНИЕ ---
\section*{Цель работы}
Исследовать эффективность работы фотоэлектрической панели в зависимости от угла падения светового потока, построить семейство вольт-амперных характеристик (ВАХ) и определить зависимость выходной мощности и коэффициента заполнения от угла наклона.

\section*{Схема эксперимента}
Схема подключения фотоэлектрического модуля (ФЭМ) аналогична использованной в работе №1 и включает в себя панель, нагрузочный потенциометр, амперметр и вольтметр. Измерения проводились при фиксированной мощности источника света 50 Вт для различных углов наклона панели относительно горизонта ($0^\circ \dots 90^\circ$).

\section*{Протокол измерений}

В таблицах \ref{tab:pmax}, \ref{tab:sc} и \ref{tab:oc} представлены данные измерений для режимов максимальной мощности, короткого замыкания и холостого хода соответственно.

% --- РЕЖИМ МАКСИМАЛЬНОЙ МОЩНОСТИ (Pmax) ---
\begin{table}[H]
    \caption{Режим максимальной мощности ($P_{max}$), 50 Вт}
    \label{tab:pmax}
    \centering
    
    % Моно Pmax
    \begin{subtable}[t]{0.48\textwidth}
        \centering
        \caption{Монокристаллическая панель}
        \begin{tabular}{|c|c|c|c|}
            \hline
            $\alpha, ^\circ$ & $I, \text{мА}$ & $U, \text{В}$ & $P, \text{мВт}$ \\
            \hline
            90 & 27.4 & 12.14 & 332.64 \\
            75 & 27.1 & 12.02 & 325.74 \\
            60 & 25.5 & 11.31 & 288.41 \\
            45 & 22.4 & 9.98  & 223.55 \\
            30 & 17.3 & 7.74  & 133.90 \\
            15 & 12.3 & 5.53  & 68.02 \\
            0  & 9.3  & 4.12  & 38.32 \\
            \hline
        \end{tabular}
    \end{subtable}
    \hfill
    % Поли Pmax
    \begin{subtable}[t]{0.48\textwidth}
        \centering
        \caption{Поликристаллическая панель}
        \begin{tabular}{|c|c|c|c|}
            \hline
            $\alpha, ^\circ$ & $I, \text{мА}$ & $U, \text{В}$ & $P, \text{мВт}$ \\
            \hline
            90 & 42.7 & 14.93 & 637.51 \\
            75 & 38.7 & 13.53 & 523.61 \\
            60 & 29.8 & 10.41 & 310.22 \\
            45 & 21.7 & 7.58  & 164.49 \\
            30 & 14.9 & 5.22  & 77.78 \\
            15 & 10.2 & 3.55  & 36.21 \\
            0  & 7.1  & 2.66  & 18.89 \\
            \hline
        \end{tabular}
    \end{subtable}
\end{table}

% --- РЕЖИМ КОРОТКОГО ЗАМЫКАНИЯ (КЗ) ---
\begin{table}[H]
    \caption{Режим короткого замыкания (КЗ), 50 Вт}
    \label{tab:sc}
    \centering
    
    % Моно КЗ
    \begin{subtable}[t]{0.48\textwidth}
        \centering
        \caption{Монокристаллическая панель}
        \begin{tabular}{|c|c|c|c|}
            \hline
            $\alpha, ^\circ$ & $I, \text{мА}$ & $U, \text{В}$ & $P, \text{мВт}$ \\
            \hline
            90 & 42.8 & 0 & 0 \\
            75 & 42.0 & 0 & 0 \\
            60 & 35.9 & 0 & 0 \\
            45 & 27.1 & 0 & 0 \\
            30 & 19.1 & 0 & 0 \\
            15 & 14.0 & 0 & 0 \\
            0  & 9.6  & 0 & 0 \\
            \hline
        \end{tabular}
    \end{subtable}
    \hfill
    % Поли КЗ
    \begin{subtable}[t]{0.48\textwidth}
        \centering
        \caption{Поликристаллическая панель}
        \begin{tabular}{|c|c|c|c|}
            \hline
            $\alpha, ^\circ$ & $I, \text{мА}$ & $U, \text{В}$ & $P, \text{мВт}$ \\
            \hline
            90 & 45.4 & 0 & 0 \\
            75 & 42.0 & 0 & 0 \\
            60 & 39.3 & 0 & 0 \\
            45 & 23.6 & 0 & 0 \\
            30 & 15.6 & 0 & 0 \\
            15 & 11.0 & 0 & 0 \\
            0  & 7.5  & 0 & 0 \\
            \hline
        \end{tabular}
    \end{subtable}
\end{table}

% --- РЕЖИМ ХОЛОСТОГО ХОДА (ХХ) ---
\begin{table}[H]
    \caption{Режим холостого хода (ХХ), 50 Вт}
    \label{tab:oc}
    \centering
    
    % Моно ХХ
    \begin{subtable}[t]{0.48\textwidth}
        \centering
        \caption{Монокристаллическая панель}
        \begin{tabular}{|c|c|c|c|}
            \hline
            $\alpha, ^\circ$ & $I, \text{мА}$ & $U, \text{В}$ & $P, \text{мВт}$ \\
            \hline
            90 & 0 & 16.05 & 0 \\
            75 & 0 & 15.75 & 0 \\
            60 & 0 & 15.50 & 0 \\
            45 & 0 & 15.26 & 0 \\
            30 & 0 & 14.95 & 0 \\
            15 & 0 & 14.60 & 0 \\
            0  & 0 & 14.25 & 0 \\
            \hline
        \end{tabular}
    \end{subtable}
    \hfill
    % Поли ХХ
    \begin{subtable}[t]{0.48\textwidth}
        \centering
        \caption{Поликристаллическая панель}
        \begin{tabular}{|c|c|c|c|}
            \hline
            $\alpha, ^\circ$ & $I, \text{мА}$ & $U, \text{В}$ & $P, \text{мВт}$ \\
            \hline
            90 & 0 & 19.54 & 0 \\
            75 & 0 & 19.18 & 0 \\
            60 & 0 & 18.69 & 0 \\
            45 & 0 & 18.19 & 0 \\
            30 & 0 & 17.50 & 0 \\
            15 & 0 & 16.72 & 0 \\
            0  & 0 & 15.93 & 0 \\
            \hline
        \end{tabular}
    \end{subtable}
\end{table}

\section*{Графические зависимости}

\subsection*{Семейство ВАХ для различных углов наклона}
На \cref{fig:iv_family} представлено семейство вольт-амперных характеристик, построенных по трем характеристическим точкам (КЗ, ММ, ХХ) для каждого угла.

\begin{figure}[hab]
    \centering
    \includegraphics[width=\linewidth]{code/results/iv_family.png}
    \caption{Семейство ВАХ для моно- и поликристаллических панелей}
    \label{fig:iv_family}
\end{figure}

\subsection*{Зависимость мощности от угла наклона}
График зависимости $P = f(\alpha)$ представлен на \cref{fig:power_angle}.

\begin{figure}[hab]
    \centering
    \includegraphics[width=0.8\linewidth]{code/results/power_vs_angle.png}
    \caption{Зависимость максимальной мощности от угла наклона панели}
    \label{fig:power_angle}
\end{figure}

\subsection*{Зависимость коэффициента заполнения от угла наклона}
График зависимости $K = f(\alpha)$ представлен на \cref{fig:fill_factor}, где
$$
K(\alpha) = \frac{P_\alpha}{U_{\text{ХХ},\alpha} \cdot I_{\text{КЗ},\alpha}}
$$

\begin{figure}[hab]
    \centering
    \includegraphics[width=0.8\linewidth]{code/results/fill_factor.png}
    \caption{Изменение коэффициента заполнения ВАХ при изменении угла}
    \label{fig:fill_factor}
\end{figure}

\section*{Выводы}

В ходе работы исследовано влияние угла падения света на энергетические характеристики солнечных панелей.
\begin{itemize}
    \item \textit{Максимальная мощность:} Для обоих типов панелей максимальная мощность достигается при угле наклона $90^\circ$. При уменьшении угла до $0^\circ$ мощность падает нелинейно. Поликристаллическая панель показала более высокую пиковую мощность ($P_{max} \approx 637$ мВт) по сравнению с монокристаллической ($P_{max} \approx 332$ мВт).
    
    \item \textit{Коэффициент заполнения ($K$):} Расчет показал, что коэффициент заполнения поликристаллической панели ($K \approx 0.72$) выше, чем у монокристаллической ($K \approx 0.48$). Хотя теоретически монокристаллические модули обладают более высоким $K$, полученный результат может быть связан с износом конкретного образца монокристаллической панели, используемого в лабораторной установке.

    \item \textit{Влияние типа панели:} Поликристаллическая панель в данном эксперименте продемонстрировала большую чувствительность к углу наклона: падение мощности при переходе от $90^\circ$ к $60^\circ$ у неё более выражено, чем у монокристаллической.}
\end{itemize}
% --- ОСНОВНОЕ СОДЕРЖАНИЕ ---



% --- ОБРАБОТКА РЕЗУЛЬТАТОВ
\clearpage
\centeredsection{\MakeUppercase{Обработка результатов измерений}}
\subsection*{ \boxed{\text{ Задание 1. }} }
\begin{quote}
    Определение основной погрешности коэффициента отклонения.
\end{quote}

По результатам измерений в \cref{tab:general-src} рассчитаем действительный коэффициент
\begin{equation}
    k_o^*=\frac{2\cdot U_{1A} }{L_{2A}} = \frac{2 \cdot 2.8}{6} = \frac{14}{15},
\end{equation}
а затем относительную погрешность (где номинальное значение $k_0 = 1$):
\begin{equation}
    \delta_{k_o} = \frac{k_o - k_o^*}{k_o^*} \cdot 100\% = \frac{1 - \frac{14}{15}}{\frac{14}{15}} \cdot 100\% = 
    \frac{1}{14} \cdot 100\% \approx \boxed{7.14 \%} 
\end{equation}

\subsection*{ \boxed{\text{ Задание 2. }} }
\begin{quote}
    Определение основной погрешности коэффициента развертки.
\end{quote}

Номинальное значение коэффициента развертки $k_p = 0.5$ мс/дел.
Расчет действительных значений $k_p^*$ для трех измерений:

\begin{enumerate}
    \item Для $n=1$, $L_{1T} = 6.8$ дел, $f_1 = 300$ Гц:
    $$ k_{p1}^* = \frac{n}{f_1 \cdot L_{nT}} = \frac{1}{300 \cdot 6.8 \cdot 10^{-3}} \approx 0.490 \text{ мс/дел} $$
    $$ \delta_{k_{p1}} = \frac{0.5 - 0.490}{0.490} \cdot 100\% \approx 2.04\% $$

    \item Для $n=2$, $L_{2T} = 8.0$ дел, $f_2 = 500$ Гц:
    $$ k_{p2}^* = \frac{2}{500 \cdot 8.0 \cdot 10^{-3}} = 0.500 \text{ мс/дел} $$
    $$ \delta_{k_{p2}} = 0\% $$

    \item Для $n=3$, $L_{3T} = 8.6$ дел, $f_3 = 700$ Гц:
    $$ k_{p3}^* = \frac{3}{700 \cdot 8.6 \cdot 10^{-3}} \approx 0.498 \text{ мс/дел} $$
    $$ \delta_{k_{p3}} = \frac{0.5 - 0.498}{0.498} \cdot 100\% \approx 0.40\% $$
\end{enumerate}

Среднее значение погрешности развертки составляет $\approx 0.81\%$.

\subsection*{ \boxed{\text{ Задание 3. }} }
\begin{quote}
    Определить характеристики нелинейных искажений изображения по осям Y и X.
\end{quote}

На основе данных протокола ($L_{Y1}=3.0$, $L_Y=2.8$, $L_{X1}=1.1$, $L_X=1.0$):

\begin{itemize}
    \item {Амплитудная нелинейность (по оси Y):}
    $$ \delta_{\text{н.а.}} = \frac{|L_{Y1} - L_Y|}{L_Y} \cdot 100\% = \frac{|3.0 - 2.8|}{2.8} \cdot 100\% \approx 7.14\% $$

    \item {Нелинейность развертки (по оси X):}
    $$ \delta_{\text{н.р.}} = \frac{|L_{X1} - L_X|}{L_X} \cdot 100\% = \frac{|1.1 - 1.0|}{1.0} \cdot 100\% = 10.0\% $$
\end{itemize}

\subsection*{ \boxed{\text{ Задание 4. }} }
\begin{quote}
    Определить амплитудно-частотную характеристику (АЧХ) канала вертикального отклонения.
\end{quote}

Расчет нормированного коэффициента передачи производится относительно опорного значения $L_{2A}(f_0) = 6$ дел. на частоте $f_0 = 1$ кГц:
$$K(f) = \frac{L_{2A}(f)}{6}$$

\subsubsection*{1. АЧХ в области верхних частот}
Результаты расчета $K(f)$ для области ВЧ сведены в \cref{tab:afc_high_calc}.

\begin{table}[H]
    \centering
    \caption{Расчет АЧХ в области ВЧ}
    \label{tab:afc_high_calc}
    \begin{tblr}{
        colspec = {l *{11}{X[c]}},
        hlines, vlines,
        rows = {m, font=\small},
    }
        $f$, МГц & 0.001 & 2 & 4 & 6 & 8 & 10 & 12 & 14 & 16 & 18 & 20 \\
        $L_{2A}$, дел. & 6 & 5.8 & 5.6 & 5.2 & 5.0 & 4.8 & 4.4 & 3.8 & 3.2 & 2.8 & 2.6 \\
        $K(f)$ & 1.00 & 0.97 & 0.93 & 0.87 & 0.83 & 0.80 & 0.73 & 0.63 & 0.53 & 0.47 & 0.43  \\
    \end{tblr}
\end{table}

Определение верхней граничной частоты $f_{\text{в}}$ по уровню $K(f_\text{в}) = 0.707$ методом линейной интерполяции между точками 12 МГц ($K=0.73$) и 14 МГц ($K=0.63$):
\begin{equation*}
    f_x = f_{12} + (f_{14} - f_{12}) \cdot \frac{K_{12} - K_{\text{целевой}}}{K_{12} - K_{14}}
\end{equation*}
\begin{equation*}
    f_{\text{в}} = 12 + (14 - 12) \cdot \frac{0.73 - 0.707}{0.73 - 0.63} = 12 + 2 \cdot \frac{0.023}{0.1} = 12.46 \text{ МГц}
\end{equation*}

\subsubsection*{2. АЧХ в области нижних частот (закрытый вход AC)}
Результаты расчета $K(f)$ для области НЧ при закрытом входе сведены в \cref{tab:afc_low_calc}.

\begin{table}[H]
    \centering
    \caption{Расчет АЧХ в области НЧ (режим AC)}
    \label{tab:afc_low_calc}
    \begin{tblr}{
        colspec = {l *{10}{X[c]}},
        hlines, vlines,
        rows = {m, font=\small},
    }
        $f$, Гц & 1000 & 800 & 100 & 50 & 40 & 10 & 8 & 6 & 4 & 2 \\
        $L_{2A}$, дел. & 6 & 6 & 6 & 6 & 6 & 5.75 & 5.6 & 5.2 & 4.8 & 3.8 \\
        $K(f)$ & 1.00 & 1.00 & 1.00 & 1.00 & 1.00 & 0.96 & 0.93 & 0.87 & 0.80 & 0.63 \\
    \end{tblr}
\end{table}

Определение нижней граничной частоты $f_{\text{н}}$ по уровню $K(f) = 0.707$ между точками 2 Гц ($K=0.63$) и 4 Гц ($K=0.80$):
$$f_{\text{н}} = 2 + (4 - 2) \cdot \frac{0.707 - 0.63}{0.80 - 0.63} = 2 + 2 \cdot \frac{0.077}{0.17} \approx 2.91 \text{ Гц}$$

\subsubsection*{3. АЧХ в области нижних частот (открытый вход DC)}
Для открытого входа во всем исследованном диапазоне частот (от 2 Гц до 1000 Гц) размер изображения остается неизменным: $L_{2A} = 6$ дел., следовательно, $K(f) = 2.00$.
Нижняя граничная частота в данном режиме $f_{\text{н}} = 0$ Гц.

\subsubsection*{4. Полоса пропускания}
Рабочая полоса пропускания канала вертикального отклонения для режима AC:
$$\Delta f = f_{\text{в}} - f_{\text{н}} = 12.46 \text{ МГц} - 2.91 \text{ Гц} \approx 12.46 \text{ МГц}$$

\begin{figure}[hbt]
    \centering
    \begin{tikzpicture}
        \begin{semilogxaxis}[
            width=\linewidth,
            height=8cm,
            grid=major,
            xlabel={Частота $f$, Гц},
            ylabel={Размах $2A$, дел.},
            xmin=1, xmax=20000000,
            ymin=0, ymax=7,
            legend pos=south west
        ]
        % НЧ часть (AC)
        \addplot[color=blue, mark=*, mark size=1.5pt] coordinates {
            (2, 3.8) (4, 4.8) (6, 5.2) (8, 5.6) (10, 5.75)
            (40, 6) (50, 6) (100, 6) (800, 6) (1000, 6)
        };
        \addlegendentry{НЧ (AC)}

        % ВЧ часть
        \addplot[color=red, mark=square*, mark size=1.5pt] coordinates {
            (1000, 6) (2000000, 5.8) (4000000, 5.6) (6000000, 5.2)
            (8000000, 5.0) (10000000, 4.8) (12000000, 4.4) (14000000, 3.8)
            (16000000, 3.2) (18000000, 2.8) (20000000, 2.6)
        };
        \addlegendentry{ВЧ (DC)}

        % Уровень 0.707
        \addplot[color=black, dashed, domain=1:20000000] {4.24};
        \node at (axis cs: 100, 4.4) [anchor=south] {Уровень 0.707};

        \end{semilogxaxis}
    \end{tikzpicture}
    \caption{График экспериментальной АЧХ осциллографа}
    \label{fig:afc_graph}
\end{figure}

\subsection*{ \boxed{\text{ Задание 6. }} }
\begin{quote}
    Оценить погрешности измерений.
\end{quote}

\subsubsection*{1. Измерение параметров напряжения}
Расчет значения напряжения:
$$ U = k_o \cdot L_A = 1 \cdot 3 = 3.0 \text{ В} $$

Визуальная погрешность измерения амплитуды:
$$ \delta_{\text{в.а.}} = \frac{b}{L_A} \cdot 100\% = \frac{0.05}{3} \cdot 100\% \approx 1.67\% $$

Суммарная относительная погрешность измерения амплитуды:
$$ \delta_A = \delta_{k_o} + \delta_{\text{н.а.}} + \delta_{\text{в.а.}} = 7.14 + 7.14 + 1.67 = 15.95\% $$

Абсолютная погрешность измерения напряжения:
$$ \Delta U = U \cdot \frac{\delta_A}{100} = 3.0 \cdot 0.1595 \approx 0.48 \text{ В} $$

\subsubsection*{2. Измерение временных интервалов}
Расчет временного интервала (периода):
$$ t_T = k_p \cdot L_T = 0.5 \cdot 6.8 = 3.4 \text{ мс} $$

Визуальная погрешность измерения длительности:
$$ \delta_{\text{в.д.}} = \frac{b}{L_T} \cdot 100\% = \frac{0.05}{6.8} \cdot 100\% \approx 0.74\% $$

Суммарная относительная погрешность измерения временного интервала:
$$ \delta_t = \delta_{k_p} + \delta_{\text{н.р.}} + \delta_{\text{в.д.}} = 0.81 + 10.0 + 0.74 = 11.55\% $$

Абсолютная погрешность измерения времени:
$$ \Delta t = t_T \cdot \frac{\delta_t}{100} = 3.4 \cdot 0.1155 \approx 0.39 \text{ мс} $$

\subsubsection*{Результат измерений:}
\begin{equation*}
   \boxed{
        \begin{aligned}
            U &= (3.0 \pm 0.5) \text{ В} \\
            t_T &= (3.4 \pm 0.4) \text{ мс}
        \end{aligned}
    } 
\end{equation*}














\section*{Выводы}
В ходе выполнения лабораторной работы были исследованы метрологические характеристики электронно-лучевого осциллографа. На основании полученных данных можно сделать следующие выводы:

\begin{enumerate}
    \item \textit{Коэффициенты отклонения и развертки:} Установлено, что относительная погрешность коэффициента отклонения составляет $7.14\%$, что превышает погрешность коэффициента развертки ($0.81\%$). Это указывает на более высокую точность калибровки канала горизонтального отклонения в сравнении с каналом вертикального отклонения.
    
    \item \textit{Нелинейность изображения:} Выявлена значительная нелинейность развертки ($10.0\%$) и амплитудная нелинейность ($7.14\%$). Такие показатели свидетельствуют о существенных искажениях геометрии сигнала при его смещении от центра экрана к краям, что необходимо учитывать при проведении точных измерений.
    
    \item \textit{Амплитудно-частотная характеристика:} Построена АЧХ и определена рабочая полоса пропускания. Верхняя граничная частота составила $f_{\text{в}} \approx 12.46$ МГц. Нижняя граничная частота в режиме закрытого входа ($AC$) составила $f_{\text{н}} \approx 2.91$ Гц, а в режиме открытого входа ($DC$) --- $f_{\text{н}} = 0$ Гц.
    
    \item \textit{Точность измерений:} Суммарные погрешности измерения амплитуды и временных интервалов равны в худшем случае $\delta_A = 15.95\%$ и $\delta_t = 11.55\%$ соответственно. Такие погрешности обусловлены в первую очередь инструментальными недостатками прибора (нелинейность и погрешность калибровки), в то время как визуальные составляющие погрешностей при выбранном масштабе минимальны ($\delta_{\text{в.а.}} = 1.67$ и $\delta_{\text{в.д.}} = 0.74$).
\end{enumerate}

% --- ОБРАБОТКА РЕЗУЛЬТАТОВ


% --- ЗАКЛЮЧЕНИЕ ---
% \newpage
% \addcontentsline{toc}{section}{Заключение}
% \phantomsection
% \centeredsection{\uppercase{Заключение}}
% В ходе выполнения данной курсовой работы была успешно решена задача разработки и реализации программного обеспечения на языке Matlab для поиска и визуализации оптимального маршрута движения объекта с ограниченным запасом хода.

Для решения поставленной задачи была разработана математическая модель, представляющая территорию в виде неориентированного взвешенного графа. На основе этой модели был реализован программный код, включающий модули для построения графа с учетом ограничений, поиска кратчайшего пути с помощью алгоритма Форда-Беллмана, генерации навигационных сообщений и графического отображения результатов.

В результате проделанной работы были получены следующие основные результаты:
\begin{itemize}
    \item Разработана \textit{программа} в среде Matlab, позволяющая автоматизировать процесс поиска оптимального маршрута.
    \item Сформирована \textit{математическая модель} задачи на основе теории графов, формализующая условия и ограничения.
    \item Реализован программный модуль для поиска кратчайшего пути на графе с использованием алгоритма Форда-Беллмана.
    \item Разработан алгоритм и реализующая его функция для формирования NMEA-подобных сообщений, описывающих движение по найденному маршруту.
    \item Создан модуль визуализации, который наглядно представляет построенный граф, все возможные пути и итоговый оптимальный маршрут, что упрощает анализ результатов.
    \item Проведено тестирование программы на конкретном примере, подтвердившее корректность работы реализованных алгоритмов и всей программы в целом.
\end{itemize}

В результате, все поставленные в работе задачи были выполнены в полном объеме, а основная цель курсовой работы — достигнута. Разработанный программный продукт является законченным решением, готовым к использованию для решения аналогичных навигационно-логистических задач.




% --- СПИСОК ИСПОЛЬЗОВАННЫХ ИСТОЧНИКОВ ---
% \newpage
% \addcontentsline{toc}{section}{Список использованных источников}
% \phantomsection
% \centeredsection{\uppercase{Список использованных источников}}
% \printbibliography


% --- ПРИЛОЖЕНИЯ ---
% \begin{appendices}
%     \section{Блок-схема алгоритма Форда-Беллмана}
\label{app:block_diagram_FB}
\begin{figure}[H]
    \centering
    \includegraphics[keepaspectratio, height=\freeht, width=\linewidth]{block_diagram/FB.drawio.pdf}
    % \caption{caption}
    % \label{fig:image}
\end{figure}

\section{Блок-схема buildGraph}
\label{app:block_diagram_BG}
\begin{figure}[H]
    \centering
    \includegraphics[keepaspectratio, height=\freeht, width=\linewidth]{block_diagram/BG.drawio.pdf}
    % \caption{caption}
    % \label{fig:image}
\end{figure}

\section{main.m}
\label{app:main.m}

\begin{codemultipage}
    % \captionof{listing}{caption\label{lst:label}}
    \inputminted{matlab}{code/main.m}
\end{codemultipage}



\section{buildGraph.m}
\label{app:buildGraph.m}

\begin{codemultipage}
    % \captionof{listing}{caption\label{lst:label}}
    \inputminted{matlab}{code/functions/buildGraph.m}
\end{codemultipage}



\section{findOptimalPathByFordBellman.m}
\label{app:findOptimalPathByFordBellman.m}

\begin{codemultipage}
    % \captionof{listing}{caption\label{lst:label}}
    \inputminted{matlab}{code/functions/findOptimalPathByFordBellman.m}
\end{codemultipage}



\section{generateNmeaMessages.m}
\label{app:generateNmeaMessages.m}

\begin{codemultipage}
    % \captionof{listing}{caption\label{lst:label}}
    \inputminted{matlab}{code/functions/generateNmeaMessages.m}
\end{codemultipage}



\section{plotRoute.m}
\label{app:plotRoute.m}

\begin{codemultipage}
    % \captionof{listing}{caption\label{lst:label}}
    \inputminted{matlab}{code/functions/plotRoute.m}
\end{codemultipage}



\section{saveResults.m}
\label{app:saveResults.m}

\begin{codemultipage}
    % \captionof{listing}{caption\label{lst:label}}
    \inputminted{matlab}{code/functions/saveResults.m}
\end{codemultipage}



\section{refueling\_points.txt}
\label{app:refueling_points.txt}

\begin{codemultipage}
    % \captionof{listing}{caption\label{lst:label}}
    \inputminted{matlab}{code/data/refueling_points.txt}
\end{codemultipage}


\section{graph\_distances.csv}
\label{app:graph_distances.csv}

\begin{codemultipage}
    % \captionof{listing}{caption\label{lst:label}}
    \inputminted{text}{code/results/graph_distances.csv}
\end{codemultipage}
% \end{appendices}



\end{document}