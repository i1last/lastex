\begin{enumerate}
    \item \textbf{Исследование свободных процессов в цепи первого порядка (\cref{fig:circuit_a})} \\
    $k_t = 0.2 \text{ мс/дел}$, $k_V = 2 \text{ В/дел}$

    \item \textbf{Исследование свободных процессов в цепи второго порядка (\cref{fig:circuit_b})}
    \begin{itemize}
        \item[а)] Колебательный затухающий: $k_t = 50 \text{ мкс/дел}$, $k_V = 0.1 \text{ В/дел}$, $R_1 = 0.5 \text{ кОм}$
        \item[б)] Апериодический: $k_t = 50 \text{ мкс/дел}$, $k_V = 0.1 \text{ В/дел}$, $R_1 = 3 \text{ кОм}$
        \item[в)] Критический: $k_t = 0.1 \text{ мс/дел}$, $k_V = 0.2 \text{ В/дел}$, $R_1 = 1.5 \text{ кОм}$
        \item[г)] Колебательный незатухающий: $k_t = 20 \text{ мс/дел}$, $k_V = 2 \text{ В/дел}$, $R_1 = 0 \text{ кОм}$
    \end{itemize}

    \item \textbf{Исследование свободных процессов в цепи третьего порядка (\cref{fig:circuit_c})} \\
    $k_t = 50 \text{ мкс/дел}$, $k_V = 0.1 \text{ В/дел}$, $R_1 = 1 \text{ кОм}$
\end{enumerate}

\noindent $U_m = 8$ В; $T_c = 1.2$ мс.

\vfill

\begin{flushright}
    Выполнили студенты гр. 4494 \\
    Рахметов А. Р. \\
    Фролов А. А. \\
    Муравьёв С. И.
\end{flushright}

\vfill

\begin{flushright}
    Преподаватель: Балданова Ю. А. \\
    12.02.26
\end{flushright}