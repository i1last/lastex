\subsubsection*{Цель работы}
Экспериментальное определение ВАХ линейных и нелинейных резисторов и источников электромагнитной энергии; изучение временных реакций линейных и нелинейных резисторов на заданные воздействия.

\subsubsection*{Теоретическая основа}
Зависимость между напряжением $u$ и током $i$ элемента электрической цепи называется его ВАХ.

У линейного резистора ВАХ описывается уравнением прямой, проходящей через начало координат: $u = R \cdot i$, где $R$ --- сопротивление.
Угол наклона этой прямой к оси тока определяет величину сопротивления $R$. Форма тока при синусоидальном напряжении будет синусоидальной.

У нелинейного резистора ВАХ соответствует нелинейному уравнению: $u = f(i)$, где функция $f(i)$ не является линейной.
В этом случае ток может не совпадать по форме с приложенным напряжением, например, при синусоидальном напряжении ток будет несинусоидальным, что и является проявлением нелинейности элемента.

Идеальные источники напряжения (ИН) и тока (ИТ) имеют ВАХ, которые являются прямыми линиями.
\begin{itemize}
    \item Идеальный ИН: \(u = u_0\) --- напряжение на его клеммах не зависит от тока.
    \item Идеальный ИТ: \(i = i_0\) --- ток не зависит от напряжения на его клеммах.
\end{itemize}

Характеристики реальных источников приближаются к этим идеальным, но имеют некоторый внутренний наклон, обусловленный внутренним сопротивлением \( R_0 \).
\[u = u_0 - i \cdot R_0\]
\[i = i_0 - \frac{u}{R_0}\]