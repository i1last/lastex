\subsection*{ \boxed{\text{ Задание 1. }} }
\begin{quote}
    Определение основной погрешности коэффициента отклонения.
\end{quote}

По результатам измерений в \cref{tab:general-src} рассчитаем действительный коэффициент
\begin{equation}
    k_o^*=\frac{2\cdot U_{1A} }{L_{2A}} = \frac{2 \cdot 2.8}{6} = \frac{14}{15},
\end{equation}
а затем относительную погрешность (где номинальное значение $k_0 = 1$):
\begin{equation}
    \delta_{k_o} = \frac{k_o - k_o^*}{k_o^*} \cdot 100\% = \frac{1 - \frac{14}{15}}{\frac{14}{15}} \cdot 100\% = 
    \frac{1}{14} \cdot 100\% \approx \boxed{7.14 \%} 
\end{equation}

\subsection*{ \boxed{\text{ Задание 2. }} }
\begin{quote}
    Определение основной погрешности коэффициента развертки.
\end{quote}

Номинальное значение коэффициента развертки $k_p = 0.5$ мс/дел.
Расчет действительных значений $k_p^*$ для трех измерений:

\begin{enumerate}
    \item Для $n=1$, $L_{1T} = 6.8$ дел, $f_1 = 300$ Гц:
    $$ k_{p1}^* = \frac{n}{f_1 \cdot L_{nT}} = \frac{1}{300 \cdot 6.8 \cdot 10^{-3}} \approx 0.490 \text{ мс/дел} $$
    $$ \delta_{k_{p1}} = \frac{0.5 - 0.490}{0.490} \cdot 100\% \approx 2.04\% $$

    \item Для $n=2$, $L_{2T} = 8.0$ дел, $f_2 = 500$ Гц:
    $$ k_{p2}^* = \frac{2}{500 \cdot 8.0 \cdot 10^{-3}} = 0.500 \text{ мс/дел} $$
    $$ \delta_{k_{p2}} = 0\% $$

    \item Для $n=3$, $L_{3T} = 8.6$ дел, $f_3 = 700$ Гц:
    $$ k_{p3}^* = \frac{3}{700 \cdot 8.6 \cdot 10^{-3}} \approx 0.498 \text{ мс/дел} $$
    $$ \delta_{k_{p3}} = \frac{0.5 - 0.498}{0.498} \cdot 100\% \approx 0.40\% $$
\end{enumerate}

Среднее значение погрешности развертки составляет $\approx 0.81\%$.

\subsection*{ \boxed{\text{ Задание 3. }} }
\begin{quote}
    Определить характеристики нелинейных искажений изображения по осям Y и X.
\end{quote}

На основе данных протокола ($L_{Y1}=3.0$, $L_Y=2.8$, $L_{X1}=1.1$, $L_X=1.0$):

\begin{itemize}
    \item {Амплитудная нелинейность (по оси Y):}
    $$ \delta_{\text{н.а.}} = \frac{|L_{Y1} - L_Y|}{L_Y} \cdot 100\% = \frac{|3.0 - 2.8|}{2.8} \cdot 100\% \approx 7.14\% $$

    \item {Нелинейность развертки (по оси X):}
    $$ \delta_{\text{н.р.}} = \frac{|L_{X1} - L_X|}{L_X} \cdot 100\% = \frac{|1.1 - 1.0|}{1.0} \cdot 100\% = 10.0\% $$
\end{itemize}

\subsection*{ \boxed{\text{ Задание 4. }} }
\begin{quote}
    Определить амплитудно-частотную характеристику (АЧХ) канала вертикального отклонения.
\end{quote}

Расчет нормированного коэффициента передачи производится относительно опорного значения $L_{2A}(f_0) = 6$ дел. на частоте $f_0 = 1$ кГц:
$$K(f) = \frac{L_{2A}(f)}{6}$$

\subsubsection*{1. АЧХ в области верхних частот}
Результаты расчета $K(f)$ для области ВЧ сведены в \cref{tab:afc_high_calc}.

\begin{table}[H]
    \centering
    \caption{Расчет АЧХ в области ВЧ}
    \label{tab:afc_high_calc}
    \begin{tblr}{
        colspec = {l *{11}{X[c]}},
        hlines, vlines,
        rows = {m, font=\small},
    }
        $f$, МГц & 0.001 & 2 & 4 & 6 & 8 & 10 & 12 & 14 & 16 & 18 & 20 \\
        $L_{2A}$, дел. & 6 & 5.8 & 5.6 & 5.2 & 5.0 & 4.8 & 4.4 & 3.8 & 3.2 & 2.8 & 2.6 \\
        $K(f)$ & 1.00 & 0.97 & 0.93 & 0.87 & 0.83 & 0.80 & 0.73 & 0.63 & 0.53 & 0.47 & 0.43  \\
    \end{tblr}
\end{table}

Определение верхней граничной частоты $f_{\text{в}}$ по уровню $K(f_\text{в}) = 0.707$ методом линейной интерполяции между точками 12 МГц ($K=0.73$) и 14 МГц ($K=0.63$):
\begin{equation*}
    f_x = f_{12} + (f_{14} - f_{12}) \cdot \frac{K_{12} - K_{\text{целевой}}}{K_{12} - K_{14}}
\end{equation*}
\begin{equation*}
    f_{\text{в}} = 12 + (14 - 12) \cdot \frac{0.73 - 0.707}{0.73 - 0.63} = 12 + 2 \cdot \frac{0.023}{0.1} = 12.46 \text{ МГц}
\end{equation*}

\subsubsection*{2. АЧХ в области нижних частот (закрытый вход AC)}
Результаты расчета $K(f)$ для области НЧ при закрытом входе сведены в \cref{tab:afc_low_calc}.

\begin{table}[H]
    \centering
    \caption{Расчет АЧХ в области НЧ (режим AC)}
    \label{tab:afc_low_calc}
    \begin{tblr}{
        colspec = {l *{10}{X[c]}},
        hlines, vlines,
        rows = {m, font=\small},
    }
        $f$, Гц & 1000 & 800 & 100 & 50 & 40 & 10 & 8 & 6 & 4 & 2 \\
        $L_{2A}$, дел. & 6 & 6 & 6 & 6 & 6 & 5.75 & 5.6 & 5.2 & 4.8 & 3.8 \\
        $K(f)$ & 1.00 & 1.00 & 1.00 & 1.00 & 1.00 & 0.96 & 0.93 & 0.87 & 0.80 & 0.63 \\
    \end{tblr}
\end{table}

Определение нижней граничной частоты $f_{\text{н}}$ по уровню $K(f) = 0.707$ между точками 2 Гц ($K=0.63$) и 4 Гц ($K=0.80$):
$$f_{\text{н}} = 2 + (4 - 2) \cdot \frac{0.707 - 0.63}{0.80 - 0.63} = 2 + 2 \cdot \frac{0.077}{0.17} \approx 2.91 \text{ Гц}$$

\subsubsection*{3. АЧХ в области нижних частот (открытый вход DC)}
Для открытого входа во всем исследованном диапазоне частот (от 2 Гц до 1000 Гц) размер изображения остается неизменным: $L_{2A} = 6$ дел., следовательно, $K(f) = 2.00$.
Нижняя граничная частота в данном режиме $f_{\text{н}} = 0$ Гц.

\subsubsection*{4. Полоса пропускания}
Рабочая полоса пропускания канала вертикального отклонения для режима AC:
$$\Delta f = f_{\text{в}} - f_{\text{н}} = 12.46 \text{ МГц} - 2.91 \text{ Гц} \approx 12.46 \text{ МГц}$$

\begin{figure}[hbt]
    \centering
    \begin{tikzpicture}
        \begin{semilogxaxis}[
            width=\linewidth,
            height=8cm,
            grid=major,
            xlabel={Частота $f$, Гц},
            ylabel={Размах $2A$, дел.},
            xmin=1, xmax=20000000,
            ymin=0, ymax=7,
            legend pos=south west
        ]
        % НЧ часть (AC)
        \addplot[color=blue, mark=*, mark size=1.5pt] coordinates {
            (2, 3.8) (4, 4.8) (6, 5.2) (8, 5.6) (10, 5.75)
            (40, 6) (50, 6) (100, 6) (800, 6) (1000, 6)
        };
        \addlegendentry{НЧ (AC)}

        % ВЧ часть
        \addplot[color=red, mark=square*, mark size=1.5pt] coordinates {
            (1000, 6) (2000000, 5.8) (4000000, 5.6) (6000000, 5.2)
            (8000000, 5.0) (10000000, 4.8) (12000000, 4.4) (14000000, 3.8)
            (16000000, 3.2) (18000000, 2.8) (20000000, 2.6)
        };
        \addlegendentry{ВЧ (DC)}

        % Уровень 0.707
        \addplot[color=black, dashed, domain=1:20000000] {4.24};
        \node at (axis cs: 100, 4.4) [anchor=south] {Уровень 0.707};

        \end{semilogxaxis}
    \end{tikzpicture}
    \caption{График экспериментальной АЧХ осциллографа}
    \label{fig:afc_graph}
\end{figure}

\subsection*{ \boxed{\text{ Задание 6. }} }
\begin{quote}
    Оценить погрешности измерений.
\end{quote}

\subsubsection*{1. Измерение параметров напряжения}
Расчет значения напряжения:
$$ U = k_o \cdot L_A = 1 \cdot 3 = 3.0 \text{ В} $$

Визуальная погрешность измерения амплитуды:
$$ \delta_{\text{в.а.}} = \frac{b}{L_A} \cdot 100\% = \frac{0.05}{3} \cdot 100\% \approx 1.67\% $$

Суммарная относительная погрешность измерения амплитуды:
$$ \delta_A = \delta_{k_o} + \delta_{\text{н.а.}} + \delta_{\text{в.а.}} = 7.14 + 7.14 + 1.67 = 15.95\% $$

Абсолютная погрешность измерения напряжения:
$$ \Delta U = U \cdot \frac{\delta_A}{100} = 3.0 \cdot 0.1595 \approx 0.48 \text{ В} $$

\subsubsection*{2. Измерение временных интервалов}
Расчет временного интервала (периода):
$$ t_T = k_p \cdot L_T = 0.5 \cdot 6.8 = 3.4 \text{ мс} $$

Визуальная погрешность измерения длительности:
$$ \delta_{\text{в.д.}} = \frac{b}{L_T} \cdot 100\% = \frac{0.05}{6.8} \cdot 100\% \approx 0.74\% $$

Суммарная относительная погрешность измерения временного интервала:
$$ \delta_t = \delta_{k_p} + \delta_{\text{н.р.}} + \delta_{\text{в.д.}} = 0.81 + 10.0 + 0.74 = 11.55\% $$

Абсолютная погрешность измерения времени:
$$ \Delta t = t_T \cdot \frac{\delta_t}{100} = 3.4 \cdot 0.1155 \approx 0.39 \text{ мс} $$

\subsubsection*{Результат измерений:}
\begin{equation*}
   \boxed{
        \begin{aligned}
            U &= (3.0 \pm 0.5) \text{ В} \\
            t_T &= (3.4 \pm 0.4) \text{ мс}
        \end{aligned}
    } 
\end{equation*}














\section*{Выводы}
В ходе выполнения лабораторной работы были исследованы метрологические характеристики электронно-лучевого осциллографа. На основании полученных данных можно сделать следующие выводы:

\begin{enumerate}
    \item \textit{Коэффициенты отклонения и развертки:} Установлено, что относительная погрешность коэффициента отклонения составляет $7.14\%$, что превышает погрешность коэффициента развертки ($0.81\%$). Это указывает на более высокую точность калибровки канала горизонтального отклонения в сравнении с каналом вертикального отклонения.
    
    \item \textit{Нелинейность изображения:} Выявлена значительная нелинейность развертки ($10.0\%$) и амплитудная нелинейность ($7.14\%$). Такие показатели свидетельствуют о существенных искажениях геометрии сигнала при его смещении от центра экрана к краям, что необходимо учитывать при проведении точных измерений.
    
    \item \textit{Амплитудно-частотная характеристика:} Построена АЧХ и определена рабочая полоса пропускания. Верхняя граничная частота составила $f_{\text{в}} \approx 12.46$ МГц. Нижняя граничная частота в режиме закрытого входа ($AC$) составила $f_{\text{н}} \approx 2.91$ Гц, а в режиме открытого входа ($DC$) --- $f_{\text{н}} = 0$ Гц.
    
    \item \textit{Точность измерений:} Суммарные погрешности измерения амплитуды и временных интервалов равны в худшем случае $\delta_A = 15.95\%$ и $\delta_t = 11.55\%$ соответственно. Такие погрешности обусловлены в первую очередь инструментальными недостатками прибора (нелинейность и погрешность калибровки), в то время как визуальные составляющие погрешностей при выбранном масштабе минимальны ($\delta_{\text{в.а.}} = 1.67$ и $\delta_{\text{в.д.}} = 0.74$).
\end{enumerate}
