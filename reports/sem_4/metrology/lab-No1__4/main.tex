\section*{Цель}
Исследование метрологических характеристик осциллографа и измерение амплитудных и временных параметров электрических сигналов различной формы.

\section*{Задание}
\begin{enumerate}
    \item Ознакомиться с органами управления осциллографа и аппаратурой, применяемой для его исследования.
    \item Определить основные погрешности коэффициентов отклонения и коэффициентов развертки.
    \item Определить характеристики нелинейных искажений изображения по осям Y и X.
    \item Определить амплитудно-частотную характеристику (АЧХ) канала вертикального отклонения.
    \item Измерить амплитудные и временные параметры сигналов по указанию преподавателя.
    \item Оценить погрешности измерений, используя результаты исследования осциллографа и его метрологические характеристики, указанные в описании.
\end{enumerate}

\section{Порядок выполнения работы}

\subsection{Подготовка прибора к работе}
\begin{enumerate}
    \item Установить переключатели входов в положение \textit{GND}.
    \item Включить осциллограф и с помощью регуляторов \textit{INTEN} (яркость) и \textit{FOCUS} (фокус) добиться четкого изображения линии развертки.
    \item Перевести режим развертки (\textit{MODE}) в положение \textit{AUTO}.
    \item Совместить линию луча с центральной горизонтальной осью сетки регулятором \textit{POSITION} $\updownarrow$.
    \item Начало развертки установить у левого края экрана (\textit{POSITION} $\leftrightarrow$).
    \item Перевести вход в режим \textit{AC} или \textit{DC} для подачи сигнала.
    \item Собрать схему испытаний \cref{fig:scheme.pdf}.
\end{enumerate}

\begin{figure}[hbt]
    \centering
    \includegraphics[width=0.5\textwidth]{figs/scheme.pdf}
    \caption{Схема испытания осциллографа}
    \label{fig:scheme.pdf}
\end{figure}

\subsection{Определение погрешности коэффициентов отклонения $k_o$}
Для экспериментального определения действительного значения $ k_o^* $ необходимо:
\begin{enumerate}
    \item Подать на вход синусоидальный сигнал частотой $ f = 1 $~кГц.
    \item Установить амплитуду генератора так, чтобы размер изображения по вертикали $ L_{2A} $ составлял 6 делений.
    \item Измерить напряжение внешним вольтметром и вычислить двойную амплитуду $ U_{2A} $.
    \item Рассчитать действительный коэффициент:
          $$k_o^* = \frac{U_{2A}}{L_{2A}}$$
    \item Вычислить относительную погрешность:
          $$\delta_{k_o} =\frac{k_o - k_o^*}{k_o^*} \cdot 100\%$$
\end{enumerate}

\subsection{Определение погрешности коэффициентов развертки $k_p$}
\begin{enumerate}
    \item Подать сигнал (синус или меандр) с известным периодом $ T $.
    \item Изменением частоты генератора $ f $ добиться отображения $ n = 1 \dots 3 $ полных периодов на экране.
    \item Измерить длину этих периодов в делениях ($ L_{nT} $).
    \item Рассчитать действительный коэффициент:
          $$k_p^* = \frac{n}{f L_{nT}}$$
    \item Вычислить относительную погрешность:
          $$\delta_{k_p} = \frac{k_p - k_p^*}{k_p^*} \cdot 100\%$$
\end{enumerate}

\subsection{Оценка нелинейности изображения}
Для оценки используется сигнал прямоугольной формы со скважностью $ q = 0.5 $. Для этого следует установить такое значение амплитуды сигнала, чтобы размер изображения по оси Y в центре экрана занимал 6 делений, а также частоту генератора, при которой по оси Х полностью разместились бы 5 полупериодов сигнала. Вариант наблюдаемой осциллограммы представлен на \cref{fig:ocil.pdf}.

\begin{figure}[htb]
    \centering
    \includegraphics[width=0.5\textwidth]{figs/ocil.pdf}
    \caption{Вариант наблюдаемой осциллограммы}
    \label{fig:ocil.pdf}
\end{figure}

\begin{itemize}
    \item \textit{Амплитудная нелинейность:} Измеряется разность размеров изображения $ |L_{Y1} - L_Y| $ в центре и у краев экрана по вертикали.
          $$\delta_{\text{н.а.}} = \frac{|L_{Y1} - L_Y|}{L_Y}\cdot 100\%$$
    \item \textit{Нелинейность развертки:} Измеряется разность размеров полупериодов $ |L_{X1} - L_X| $ в центре и по краям по горизонтали.
          $$\delta_{\text{н.р.}} = \frac{|L_{X1} - L_X|}{L_X}\cdot 100\%$$
\end{itemize}

\subsection{Снятие амплитудно-частотной характеристики (АЧХ)}
Необходимо определить зависимость размера изображения $ L_Y $ от частоты $ f $ при неизменном входном напряжении по схеме на \cref{fig:scheme.pdf}:
\begin{enumerate}
    \item Установить опорную частоту $ f_0 = 1 $~кГц и размер $ L_{2A} = 6 $ делений на генераторе.
    \item \textit{Область ВЧ:} Увеличивать частоту (2 МГц, 4 МГц и т.д.) до спада амплитуды до уровня $ 0.707 $ от начальной.
    \item \textit{Область НЧ (для режима AC):} Уменьшать частоту от 1000 Гц вниз до момента спада амплитуды.
    \item Результаты заносятся в \cref{table:vc,table:nc}.
\end{enumerate}

\subsection{Измерение параметров сигналов}
Расчет искомых величин производится по формулам:
\begin{itemize}
    \item Напряжение: $$ U = k_o L_A $$
    \item Временной интервал: $$ t_T = k_p L_T $$
\end{itemize}

Суммарная погрешность измерения амплитуды $ \delta_A $ складывается из погрешности коэффициента $ \delta_{k_o} $, нелинейности $ \delta_{\text{н.а.}} $ и визуальной погрешности $ \delta_{\text{в.а.}} = \frac{b}{L_A} \cdot 100\%$ (где $ b $ — толщина луча).