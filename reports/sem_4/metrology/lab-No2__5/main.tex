\section*{Цель работы}
Ознакомление с методами обработки результатов прямых и косвенных измерений при однократных и многократных измерениях.

\section*{Задачи}
\begin{enumerate}
    \item Ознакомиться с лабораторным стендом и сменным модулем \textit{«Прямые, косвенные и совместные измерения»}.
    \item Выполнить прямые однократные измерения напряжения на выходе резистивного делителя.
    \item Выполнить косвенные однократные измерения тока и мощности, выделяемой на участке резистивного делителя.
    \item Выполнить прямые многократные измерения напряжения при наличии случайных погрешностей.
    \item Выполнить косвенные многократные измерения тока и мощности при наличии случайных погрешностей.
\end{enumerate}

\section*{Теоретические сведения}
Объектом испытаний для прямых измерений служит резистивный делитель напряжения, состоящий из нескольких последовательно соединенных резисторов ($R_1$, $R_2$, $R_0$). Резистор $R_0$ является образцовым сопротивлением. 

\subsection*{Однократные измерения}
При \textit{прямом однократном измерении} результат и его абсолютная погрешность определяются инструментальной точностью прибора (вольтметра). Результат записывается в виде:
$$U_x = U \pm \Delta U$$
где $\Delta U$ — инструментальная погрешность.

При \textit{косвенном однократном измерении} искомая величина вычисляется на основе прямых измерений других величин. Ток, протекающий через резисторы, определяется по падению напряжения $U_0$ на образцовом сопротивлении $R_0$:
$$I = \frac{U_0}{R_0}$$
Относительная и абсолютная погрешности измерения тока:
$$\delta_I = \delta_{U_0} + \delta_{R_0}$$
$$\Delta_I = \frac{I \cdot \delta_I}{100}$$
Мощность, выделяемая на резисторах $R_2$ и $R_0$, рассчитывается как $P_M = U \cdot I$. Погрешности измерения мощности:
$$\delta_{P_M} = \delta_U + \delta_I$$
$$\Delta_{P_M} = \frac{P_M \cdot \delta_{P_M}}{100}$$

\begin{figure}[H]
    \centering
    \includegraphics[width=0.5\linewidth]{figs/scheme_single.pdf}
    \caption{Схема для проведения однократных прямых и косвенных измерений}
    \label{fig:scheme_single}
\end{figure}

\subsection*{Многократные измерения}
Для исследования влияния случайных погрешностей в схему вводится генератор случайных сигналов (ГСС), выход которого суммируется с измеряемым напряжением. 

\begin{figure}[H]
    \centering
    \includegraphics[width=1\linewidth]{figs/scheme_multiple.pdf}
    \caption{Схема для проведения многократных измерений со случайной погрешностью}
    \label{fig:scheme_multiple}
\end{figure}

Оценка математического ожидания (среднее арифметическое) для ряда из $n$ наблюдений $U_i$:
$$\overline{U} = \frac{1}{n} \sum_{i=1}^{n} U_i$$
Оценка среднеквадратического отклонения (СКО) случайной погрешности измерения:
$$S[U] = \sqrt{\frac{1}{n-1} \sum_{i=1}^{n} (U_i - \overline{U})^2}$$
Оценка СКО среднего арифметического значения:
$$S[\overline{U}] = \frac{S[U]}{\sqrt{n}}$$
Доверительный интервал погрешности при нормальном законе распределения:
$$\Delta U = t_p(f) S[\overline{U}]$$
где $t_p(f)$ — коэффициент Стьюдента, зависящий от доверительной вероятности $P$ и числа степеней свободы $f = n - 1$ (см. \cref{tab:student_coef}).

\begin{table}[H]
    \centering
    \caption{Коэффициент Стьюдента при числе степеней свободы $f$}
    \label{tab:student_coef}
    \begin{tblr}{
        colspec = { Q[c,m] *{9}{X[c,m]} },
        hlines, vlines,
        rows = {m, font=\small}
    }
        {Доверительная \\ вероятность $P$} & $4$ & $5$ & $6$ & $7$ & $8$ & $10$ & $15$ & $20$ & $\infty$ \\
        $0.90$ & $2.13$ & $2.02$ & $1.94$ & $1.86$ & $1.81$ & $1.75$ & $1.72$ & $1.70$ & $1.65$ \\
        $0.95$ & $2.77$ & $2.57$ & $2.45$ & $2.31$ & $2.23$ & $2.13$ & $2.09$ & $2.04$ & $1.96$ \\
        $0.98$ & $3.75$ & $3.36$ & $3.14$ & $2.90$ & $2.76$ & $2.60$ & $2.53$ & $2.46$ & $2.33$ \\
    \end{tblr}
\end{table}

При косвенных многократных измерениях мощности ($\overline{P_M} = \overline{U} \cdot \overline{I}$) оценка СКО вычисляется по формуле:
$$S[\overline{P_M}] = \sqrt{\overline{U}^2 S^2[\overline{I}] + \overline{I}^2 S^2[\overline{U}]}$$
Эффективное число степеней свободы $f_{\text{эф}}$ для определения коэффициента Стьюдента $k_p(f_{\text{эф}})$ находится как:
$$f_{\text{эф}} = (n+1) \frac{\left(\overline{U}^2 S^2[\overline{I}] + \overline{I}^2 S^2[\overline{U}]\right)^2}{\overline{U}^4 S^4[\overline{I}] + \overline{I}^4 S^4[\overline{U}]} - 2$$

\section*{Инструкция по выполнению работы}

\subsection*{Часть 1. Однократные измерения}
\begin{enumerate}
    \item Соберите электрическую цепь в соответствии со схемой, представленной на \cref{fig:scheme_single}.
    \item Подайте на вход делителя постоянное напряжение, контролируя его значение с помощью цифрового вольтметра $V_k$.
    \item \textit{Прямое измерение напряжения:} Установите двухпозиционный переключатель $\Pi$ в положение \textit{1}.
    \item Снимите показание с цифрового вольтметра $V$. Это значение является напряжением $U$ на сумме сопротивлений $R_2$ и $R_0$. Занесите результат в \cref{tab:single_measurements} протокола.
    \item \textit{Косвенное измерение тока:} Переведите переключатель $\Pi$ в положение \textit{2}.
    \item Снимите показание с вольтметра $V$. Это значение является падением напряжения $U_0$ на образцовом сопротивлении $R_0$. Занесите результат в \cref{tab:single_measurements}.
\end{enumerate}

\subsection*{Часть 2. Многократные измерения}
\begin{enumerate}
    \item Соберите электрическую цепь в соответствии со схемой на \cref{fig:scheme_multiple}, включив в цепь сумматор $\Sigma$, генератор случайных сигналов (ГСС) и устройство выборки и хранения (УВХ).
    \item Включите ГСС. Установите заданный преподавателем уровень дисперсии случайной погрешности (внутренним переключателем).
    \item \textit{Снятие первого ряда значений ($U_{1i}$):} Установите переключатель $\Pi$ в положение \textit{1}.
    \item Нажмите кнопку \textit{«Выборка»} на блоке УВХ. Зафиксируйте дискретное значение напряжения $U_{1i}$ по вольтметру $V$ и занесите его в \cref{tab:multiple_measurements}.
    \item Повторите процедуру выборки $n$ раз (число $n$ задается преподавателем).
    \item \textit{Снятие второго ряда значений ($U_{2i}$):} Переведите переключатель $\Pi$ в положение \textit{2}.
    \item Аналогичным образом $n$ раз нажмите кнопку \textit{«Выборка»}, фиксируя значения напряжения $U_{2i}$ на образцовом резисторе. Занесите данные в \cref{tab:multiple_measurements}.
\end{enumerate}

\section*{Инструкция по обработке данных}
\begin{enumerate}
    \item \textit{Для однократных измерений:} 
    \begin{itemize}
        \item Определите инструментальную погрешность $\Delta U$ по спецификации вольтметра. Запишите результат $U_x$.
        \item Рассчитайте ток $I$ и его погрешность $\Delta_I$. Запишите результат $I_x$.
        \item Вычислите мощность $P_M$ и её погрешность $\Delta_{P_M}$. Запишите результат $P_{Mx}$.
    \end{itemize}
    \item \textit{Для многократных прямых измерений ($U_{1i}$):}
    \begin{itemize}
        \item Рассчитайте среднее значение $\overline{U}$.
        \item Вычислите отклонения $(U_{1i} - \overline{U})$ и их квадраты для каждого измерения.
        \item Найдите СКО единичного измерения $S[U]$ и СКО среднего $S[\overline{U}]$.
        \item Определите доверительный интервал $\Delta U$ с помощью коэффициента Стьюдента.
    \end{itemize}
    \item \textit{Для многократных косвенных измерений ($I_i$, $P_M$):}
    \begin{itemize}
        \item Для каждого $i$-го измерения рассчитайте ток $I_i = U_{2i} / R_0$.
        \item Найдите средний ток $\overline{I}$, СКО тока $S[I]$ и СКО среднего тока $S[\overline{I}]$.
        \item Вычислите среднюю мощность $\overline{P_M} = \overline{U} \cdot \overline{I}$.
        \item Рассчитайте СКО мощности $S[\overline{P_M}]$ и эффективное число степеней свободы $f_{\text{эф}}$.
        \item Найдите доверительный интервал и запишите итоговый результат $P_{Mx}$.
    \end{itemize}
\end{enumerate}