
\section*{Исходные данные}
\begin{table}[H]
    \centering
    \caption{Параметры установки и условия эксперимента}
    \label{tab:initial_data}
    \begin{tblr}{
        colspec = { X[c,m] X[c,m] X[c,m] X[c,m] X[c,m] },
        hlines, vlines,
    }
        Образцовое сопротивление $R_0$, Ом & Относит. погрешность $\delta_{R_0}$, \% & Относит. погрешность вольтметра $\delta_V$, \% & Число измерений $n$ & Доверительная вероятность $P$ \\
        & & & & \\
    \end{tblr}
\end{table}

\section*{Таблицы снятия данных}

\begin{table}[H]
    \centering
    \caption{Результаты однократных измерений}
    \label{tab:single_measurements}
    \begin{tblr}{
        colspec = { X[c,m] X[c,m] },
        hlines, vlines,
    }
        Напряжение на делителе $U$ (поз. 1), В & Напряжение на $R_0$ $U_0$ (поз. 2), В \\
        & \\
    \end{tblr}
\end{table}

\begin{longtblr}[
    caption = {Результаты многократных измерений со случайной погрешностью},
    label = {tab:multiple_measurements},
]{
    width = 0.8\linewidth,
    colspec = { Q[c,m, 3em] X[c,m] X[c,m] },
    hlines, vlines,
    rowhead = 1
}
    Номер $i$ & Напряжение $U_{1i}$ (поз. 1), В & Напряжение $U_{2i}$ (поз. 2), В \\
    $1$ & & \\
    $2$ & & \\
    $3$ & & \\
    $4$ & & \\
    $5$ & & \\
    $6$ & & \\
    $7$ & & \\
    $8$ & & \\
    $9$ & & \\
    $10$ & & \\
\end{longtblr}