\section*{Цель}
Познакомиться с программой Quartus II и освоить методику построения схемы, заданной логическим выражением.

\section*{Логическое выражение}
Дано логическое выражение:

\begin{equation}
    f = x_3 \land \overline{x_2 \lor x_1}
    \label{формула:исходное-логическое-выражение}
\end{equation}

\section*{Таблица истинности заданного логического выражения}
\begin{table}[H]
    \centering
    \caption{Таблица истинности для \cref{формула:исходное-логическое-выражение}}
    \label{таблица:таблица-истинности}
    \begin{tabular}{ccc|c}
            $x_1$ & $x_2$ & $x_3$ & $f$ \\
        \hline
            0 & 0 & 0 & 0 \\
            0 & 0 & 1 & 1 \\
            0 & 1 & 0 & 0 \\
            0 & 1 & 1 & 0 \\
            1 & 0 & 0 & 0 \\
            1 & 0 & 1 & 0 \\
            1 & 1 & 0 & 0 \\
            1 & 1 & 1 & 0 \\ 
    \end{tabular}
\end{table}

\section*{Работа в программе Quartus II}
Заданная логическая цепь была собрана (\cref{fig:core.png}) и протестирована на наличие ошибок. Quartus II не выявил критических ошибок. Результат анализа схемы представлен на \cref{fig:rtl-viewer.png}.

\begin{figure}[hbt]
    \centering
    \includegraphics[width=\linewidth]{figs/core.png}
    \caption{Логическая цепь}
    \label{fig:core.png}
\end{figure}

\begin{figure}[hbt]
    \centering
    \includegraphics[width=\linewidth]{figs/rtl-viewer.png}
    \caption{RTL Viewer}
    \label{fig:rtl-viewer.png}
\end{figure}

Была построена временная диаграмма при следующих параметрах:
\begin{itemize}
    \item $\text{Grid size} = 20 \text{ns}$
    \item $\text{End time} = 160 \text{ns}$
\end{itemize}

Результат симуляции без задержек представлен на \cref{изображение:вд-без-задержки}, с задержкой --- на \cref{изображение:вд-с-задержкой}

\begin{figure}[hbt]
    \centering
    
    \begin{subfigure}[a]{\linewidth}
        \centering
        \includegraphics[width=\textwidth]{figs/waveform.png}
        \caption{Без задержки}
        \label{изображение:вд-без-задержки}
    \end{subfigure}
    \hfill
    \vspace{0.5cm}
    \hfill
    \begin{subfigure}[a]{\linewidth}
        \centering
        \includegraphics[width=\textwidth]{figs/waveform-delay.png}
        \caption{С задержкой}
        \label{изображение:вд-с-задержкой}
    \end{subfigure}

    \caption{Временные диаграммы}
    \label{изображение:временные-диаграммы}
\end{figure}

Сравнительный анализ временной диаграммы без задержки (\cref{изображение:вд-без-задержки}) и таблицы истинности (\cref{таблица:таблица-истинности}) показывает, что таблица истинности соответствует составленной схеме.

Действительно, если упростить \cref{формула:исходное-логическое-выражение} по закону де Моргана
\begin{equation}
    x_3 \land \overline{x_2 \lor x_1} = 
    x_3 \land \overline{x_2} \land \overline{x_1} = 
    \overline{x_1} \land \overline{x_2} \land x_3,
    \label{формула:упрощенное-логическое-выражение}
\end{equation}
то можно заметить, что единственный возможный случай, при котором $f$ примет значение Истина, --- это комбинация $(x_1,x_2,x_3)=(0,0,1)$.

\section*{Вывод}
В ходе данной лабораторной работы были освоены основы работы в программе Quartus II для моделирования логических схем.

Корректность составленной схемы в Quartus II (\cref{fig:core.png,fig:rtl-viewer.png}) и результаты таблицы истинности (\cref{таблица:таблица-истинности}) были проверены путем упрощения исходного логического выражения (\cref{формула:упрощенное-логическое-выражение}).
