\section*{Цель работы}
Познакомиться с алгоритмом экологического контроля и мониторинга окружающей среды на примере расчета интегральных показателей индекса загрязнения атмосферы (ИЗА).

\section*{Теоретические сведения}

\subsection*{Основные понятия}

\begin{itemize}
    \item \textit{Атмосферный воздух} --- единое целое; жизненно важный компонент окружающей среды и неотъемлемая часть среды обитания человека, растений и животных.
    \item \textit{Факторы, влияющие на состояние атмосферы в городе:}
          \begin{itemize}
              \item Работа автотранспорта.
              \item Отопление зданий.
              \item Функционирование предприятий.
          \end{itemize}
    \item \textit{«Остров тепла»} --- область повышенной температуры воздуха над городом в виде купола, образующаяся из-за потоков тепла и выбросов.
    \item \textit{Озеленение} --- улучшает качество воздуха, так как растения поглощают углекислый газ и выделяют кислород.
    \item \textit{Инверсия температуры} --- атмосферное явление, при котором температура воздуха растёт с увеличением высоты (а не наоборот), что способствует застою загрязняющих веществ у поверхности.
\end{itemize}

Уровень загрязнения атмосферы (ЗА) создается в результате поступления выбросов вредных веществ от всех источников на территории города и атмосферных процессов, влияющих на перенос и рассеивание этих веществ от источников загрязнения.

\subsection*{Нормирование качества воздуха}

Предельно допустимые концентрации (ПДК) --- нормативы содержания вредных веществ в воздухе для 10 веществ (пыль, сернистый ангидрид, оксид углерода и т.д.) Все ПДК измеряются массой загрязняющего вещества в единице объема воздуха (мг/м${^3}$) при нормальных условиях (давление 1 атм, температура $0^\circ$С).

Виды ПДК для населенных мест:
\begin{itemize}
    \item \textit{ПДК м.р. (максимальная разовая)} --- концентрация, не вызывающая рефлекторных реакций в организме человека (запах, раздражение) при кратковременном воздействии (до 20 мин).
    \item \textit{ПДК с.с. (среднесуточная)} --- среднесуточная концентрация, не оказывающая вредного воздействия при длительном вдыхании.
    \item \textit{ПДК р.з.} --- ПДК вредных веществ в воздухе рабочей зоны.
\end{itemize}

Предельно допустимый выброс (ПДВ) --- максимальная масса выбросов загрязняющих веществ от конкретного источника за единицу времени, которая не приведёт к превышению ПДК в жилых районах.

\subsection*{Показатели качества воздуха}
\begin{enumerate}
    \item \textit{ИЗА (Комплексный индекс загрязнения атмосферы)} --- основной показатель, характеризующий уровень хронического, длительного загрязнения воздуха. Рассчитывается по среднегодовым концентрациям нескольких примесей.
    \item \textit{СИ (Стандартный индекс)} --- наибольшая измеренная разовая концентрация примеси, деленная на её ПДК. Используется для оценки кратковременных загрязнений.
    \item \textit{ИИЗА (Интегральный индекс загрязнения атмосферы)} --- также является основным показателем для оценки степени загрязнения воздуха в городе.
\end{enumerate}


Расчет суммарного индекса загрязнения атмсферы (ИЗА) основан на следующих положениях: а) опасность вещества зависит от его класса опасности; б) опасность возрастает при превышении ПДК.

\textbf{Шаг 1}. Расчет единичного (парциального) индекса (ИЗА\(_i\)). Индекс рассчитывается для каждого отдельного вещества по формуле:
\[
    \text{ИЗА}_i = (C_i / \text{ПДК}_i)^{K_i}
\]
где:
\begin{itemize}
    \item \textit{C\(_i\)} --- средняя концентрация i-го вещества.
    \item \textit{ПДК\(_i\)} --- среднесуточная ПДК для i-го вещества.
    \item \textit{K\(_i\)} --- безразмерная константа, приводящая степень вредности вещества к вредности диоксида серы.
\end{itemize}

\begin{table}[H]
    \caption{Коэффициенты приведения для различных классов опасности ($K_i$).}
    \label{table:table-1-teor}

    \noindent
    \begin{tabularx}{\linewidth}{|L|C|C|C|C|}
        \hline
                                   & \multicolumn{4}{c|}{\textbf{Класс опасности}}                                                                                                                                     \\
        \cline{2-5}
                                   & \textbf{I}                                     & \textbf{II}                               & \textbf{III}                                & \textbf{IV}                            \\
        \hline
        \textbf{Характеристика}    & \parbox{2.5cm}{\centering чрезвычайно опасные} & \parbox{2.5cm}{\centering высоко опасные} & \parbox{2.5cm}{\centering умеренно опасные} & \parbox{2.5cm}{\centering малоопасные} \\
        \hline
        \textbf{Коэффициент $K_i$} & {1.7}                                          & {1.3}                                     & {1.0}                                       & {0.9}                                  \\
        \hline
    \end{tabularx}
\end{table}

\textbf{Шаг 2.} Расчет итогового комплексного индекса (ИЗА\(_5\)).

Для итоговой оценки используется сумма \textit{пяти наибольших} единичных индексов ИЗА\(_i\).
\[
    \text{ИЗА}_5 = \sum_i \text{ИЗА}_i \quad (\text{для 5 веществ с наибольшими индексами})
\]

\textit{Пример: Из девяти веществ для расчета берутся пять с максимальными значениями ИЗА\(_i\): бенз(а)пирен (2,5), пыль (2,3), фенол (2,1), свинец (1,9) и двуокись серы (1,5). Суммарный ИЗА\(_5\) = 10,3.}


\begin{table}[H]
    \caption{Классификация уровней загрязнения по ИЗА\(_5\).}
    \label{table:table-2-teor}

    \noindent
    \begin{tabularx}{\linewidth}{|L|C|C|C|C|C|C|}
        \hline
        \textbf{Показатель} & \multicolumn{6}{c|}{\textbf{Уровень загрязнения атмосферы (по ИЗА\(_5\))}}                                                                                                        \\
        \hline
        Состояние атмосферы & Чистая                                                                     & Слабо-загр.  & Загрязненная  & Сильно загрязненная & Высоко загрязненная & Экстремально загрязненная \\
        \hline
        Значение ИЗА\(_5\)  & $<2,5$                                                                     & $2,5\div7,5$ & $7,5\div12,5$ & $12,5\div22,5$      & $22,5\div52,5$      & $>52,5$                   \\
        \hline
    \end{tabularx}
\end{table}




\newpage
\centeredsection{РАСЧЕТЫ}

\begin{table}[H]
    \centering
    \caption{Исходные данные}
    \label{table:src-data}
    \begin{tabularx}{\linewidth}{|L|C|C|C|C|C|C|}
        \hline
        \multirow{2}{*}{\textbf{\shortstack{Загр.                                                                                           \\в-ва}}} &
        \multirow{2}{*}{\textbf{\shortstack{Ср.-сут.                                                                                        \\ ПДК, $\frac{\text{мг}}{\text{м}^3}$}}} &
        \multirow{2}{*}{\textbf{\shortstack{Класс                                                                                           \\ опасн.}}} &
        \multicolumn{4}{c|}{\textbf{Ср. концентр. примесей в г-х}}                                                                          \\ \cline{4-7}
                              &                   &   & \textbf{Таганрог}   & \textbf{Азов}     & \textbf{Шахты}      & \textbf{Ростов}     \\ \hline
        Пыль                  & 0,15              & 3 & 0,4                 & 0,12              & 0,4                 & 0,7                 \\ \hline
        Двуокись серы         & 0,05              & 2 & 0,04                & 0,11              & 0,05                & 0,04                \\ \hline
        Двуокись азота        & 0,04              & 2 & 0,1                 & 0,04              & 0,1                 & 0,06                \\ \hline
        Окись азота           & 0,06              & 3 & 0,08                & 0,07              & 0,07                & 0,09                \\ \hline
        Бензо($\alpha$)-пирен & $1 \cdot 10^{-6}$ & 1 & $1,2 \cdot 10^{-6}$ & $1 \cdot 10^{-6}$ & $2,1 \cdot 10^{-6}$ & $1,8 \cdot 10^{-6}$ \\ \hline
        Сероуглерод           & 0,005             & 2 & 0,009               & 0,004             & 0,04                & 0,02                \\ \hline
        Аммиак                & 0,04              & 4 & 0,04                & 0,05              & 0,07                & 0,06                \\ \hline
        Формальдегид          & 0,003             & 2 & 0,005               & 0,009             & 0,04                & 0,11                \\ \hline
        Сажа                  & 0,05              & 3 & 0,08                & 0,05              & 0,04                & 0,5                 \\ \hline
        Фтористый водор.      & 0,005             & 3 & 0,007               & 0,009             & 0,008               & 0,011               \\ \hline
        Окись углерода        & 3,0               & 4 & 2,7                 & 0,8               & 1,0                 & 2,9                 \\ \hline
        Пред. углеводор.      & 5,0               & 3 & 2,2                 & 3,7               & 4,9                 & 4,1                 \\ \hline
        Ксилол                & 0,1               & 2 & 0,006               & 0,001             & 0,001               & 0,01                \\ \hline
        Толуол                & 0,4               & 3 & 0,001               & 0,02              & 0,08                & 0,03                \\ \hline
    \end{tabularx}
\end{table}


\section*{ \boxed{\text{Задание 1.}} }
\begin{quote}
    Рассчитайте комплексный индекс загрязнения атмосферы для городов разных регионов России (см. \cref{table:src-data}). 
\end{quote}

\subsection*{\underline{1.1. Расчет ИЗА}}

\begin{table}[H]
    \centering
    \caption{Расчет ИЗА для городов}
    \label{table:iza}
    \begin{tabularx}{\linewidth}{|c|X|c|c|c|c|}
        \hline
        \multirow{2}{*}{\textbf{№}} & \multirow{2}{*}{\textbf{Вещество}} & \multicolumn{4}{c|}{\textbf{ИЗА}}                                                    \\ \cline{3-6}
                                    &                                    & \textbf{Таганрог}                 & \textbf{Азов} & \textbf{Шахты} & \textbf{Ростов} \\ \hline
        1                           & Пыль                               & 2.67                              & 0.80          & 2.67           & 4.67            \\ \hline
        2                           & Двуокись серы                      & 0.75                              & 2.79          & 1.00           & 0.75            \\ \hline
        3                           & Двуокись азота                     & 3.29                              & 1.00          & 3.29           & 1.69            \\ \hline
        4                           & Окись азота                        & 1.33                              & 1.17          & 1.17           & 1.50            \\ \hline
        5                           & Бензо($\alpha$)пирен               & 1.36                              & 1.00          & 3.53           & 2.72            \\ \hline
        6                           & Сероуглерод                        & 2.15                              & 0.75          & 14.93          & 6.06            \\ \hline
        7                           & Аммиак                             & 1.00                              & 1.22          & 1.65           & 1.44            \\ \hline
        8                           & Формальдегид                       & 1.94                              & 4.17          & 29.00          & 108.03          \\ \hline
        9                           & Сажа                               & 1.60                              & 1.00          & 0.80           & 10.00           \\ \hline
        10                          & Фтористый водор.                   & 1.40                              & 1.80          & 1.60           & 2.20            \\ \hline
        11                          & Окись углерода                     & 0.91                              & 0.30          & 0.37           & 0.97            \\ \hline
        12                          & Пред. углеводор.                   & 0.44                              & 0.74          & 0.98           & 0.82            \\ \hline
        13                          & Ксилол                             & 0.03                              & 0.00          & 0.00           & 0.05            \\ \hline
        14                          & Толуол                             & 0.00                              & 0.05          & 0.20           & 0.08            \\ \hline
    \end{tabularx}
\end{table}



\subsection*{\underline{1.2. Расчет ИЗА$_5$}}
\begin{enumerate}
    \item \textbf{Таганрог}\\
          Наибольший вклад вносят: двуокись азота ($3.29$), пыль ($2.67$), сероуглерод ($2.15$), формальдегид ($1.94$) и сажа ($1.60$).
          $$\text{ИЗА}_5 = 3.29 + 2.67 + 2.15 + 1.94 + 1.60 = 11.65$$

    \item \textbf{Азов}\\
          Наибольший вклад вносят: формальдегид ($4.17$), двуокись серы ($2.79$), фтористый водород ($1.80$), аммиак ($1.22$) и окись азота ($1.17$).
          $$\text{ИЗА}_5 = 4.17 + 2.79 + 1.80 + 1.22 + 1.17 = 11.15$$

    \item \textbf{Шахты}\\
          Наибольший вклад вносят: формальдегид ($29.00$), сероуглерод ($14.93$), бензо($\alpha$)пирен ($3.53$), двуокись азота ($3.29$) и пыль ($2.67$).
          $$\text{ИЗА}_5 = 29.00 + 14.93 + 3.53 + 3.29 + 2.67 = 53.42$$

    \item \textbf{Ростов}\\
          Наибольший вклад вносят: формальдегид ($108.03$), сажа ($10.00$), сероуглерод ($6.06$), пыль ($4.67$) и бензо($\alpha$)пирен ($2.72$).
          $$\text{ИЗА}_5 = 108.03 + 10.00 + 6.06 + 4.67 + 2.72 = 131.48$$

    \item \textbf{Итоговые значения ИЗА$_5$}:

          \begin{itemize}
              \item Таганрог --- 11.65
              \item Азов --- 11.15
              \item Шахты --- 53.42
              \item Ростов --- 131.48
          \end{itemize}
\end{enumerate}




\section*{ \boxed{\text{Задание 2.}} }
\begin{quote}
    Установите степень загрязнения приземного слоя воздуха каждого города и региона.
\end{quote}

Согласно \cref{table:table-2-teor} классификация уровней загрязнения по ИЗА$_5$:
\begin{itemize}
    \item Таганрог --- Загрязненная атмосфера.
    \item Азов --- Загрязненная атмосфера.
    \item Шахты --- Экстремально загрязненная атмосфера.
    \item Ростов --- Экстремально загрязненная атмосфера.
\end{itemize}

\section*{ \boxed{\text{Задание 3.}} }
\begin{quote}
    Дайте сравнительную характеристику степени загрязнения атмосферы городов, с указанием перечня приоритетных загрязнителей в каждом городе.
\end{quote}

Анализ комплексного индекса ИЗА$_5$ показывает значительные различия в уровне загрязнения атмосферы исследуемых городов.

\textbf{Азов (ИЗА$_5 = 11.15$) и Таганрог (ИЗА$_5 = 11.65$)}.
Их уровень загрязнения сопоставим и относится к категории \textit{загрязненная атмосфера}.
Основные загрязнители в Азове --- формальдегид и двуокись серы, в Таганроге --- двуокись азота и пыль.
Эти вещества вносят наибольший вклад в общий индекс загрязнения, что указывает на необходимость контроля и снижения их концентраций для улучшения качества воздуха.

\textbf{Шахты (ИЗА$_5 = 53.42$)} и \textbf{Ростов (ИЗА$_5 = 131.48$)}.
Оба города имеют \textit{экстремально загрязненную атмосферу}.
Ключевым загрязнителем в обоих городах является формальдегид, который в Шахтах и Ростове вносит наибольший вклад в общий индекс загрязнения.
В Шахтах значительный вклад также вносит сероуглерод, а в Ростове --- сажа.
Примечательно, что вклад формальдегида в Ростове составляет более 82\% от общего ИЗА$_5$, что свидетельствует о его доминирующей роли в загрязнении атмосферы этого города.
Эти данные свидетельствуют о критической экологической ситуации в этих городах и необходимости принятия срочных мер по снижению выбросов данных веществ.

Города делятся на две группы: Азов и Таганрог с сопоставимым \textit{загрязненным} уровнем и Шахты и Ростов с \textit{экстремально загрязненной} атмосферой.

Ранжирование городов по возрастанию степени загрязнения:
$$ \text{Азов} \lesssim \text{Таганрог} \ll \text{Шахты} \lll \text{Ростов} $$.
