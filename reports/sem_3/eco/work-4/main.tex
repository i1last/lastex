\section*{Теоретические сведения}

\subsection*{Общие сведения и источники загрязнения}

Поверхность земли испытывает значительную по массе и опасности антропогенную нагрузку. Загрязнение почвы происходит из-за попадания в неё нехарактерных физических, химических или биологических агентов.

\textbf{Основные источники антропогенного загрязнения земли:}
\begin{itemize}
    \item твердые и жидкие отходы добывающей, перерабатывающей и химической промышленности, теплоэнергетики и транспорта;
    \item отходы потребления (в первую очередь твердые бытовые отходы – ТБО);
    \item сельскохозяйственные отходы и применяемые в агротехнике ядохимикаты;
    \item атмосферные выпадения токсичных веществ;
    \item аварийные сбросы и выбросы загрязняющих веществ.
\end{itemize}

\subsection*{Классификация и нормирование загрязняющих веществ}

При нормировании химических веществ в почве учитывается не только прямая опасность, но и последствия вторичного загрязнения контактирующих сред: воды, воздуха и растений. Основным критерием оценки является \textbf{предельно допустимая концентрация (ПДК)} — максимальное количество вредного вещества в почве, которое не оказывает негативного влияния на здоровье человека и окружающую среду.

По степени возможного отрицательного влияния загрязняющие вещества подразделяются на 3 класса:
\begin{itemize}
    \item[\textbf{I --}] \textbf{высокоопасные}: мышьяк, кадмий, ртуть, селен, свинец, цинк, фтор, бенз(а)пирен;
    \item[\textbf{II --}] \textbf{умеренно опасные}: бор, кобальт, никель, молибден, медь, сурьма, хром;
    \item[\textbf{III --}] \textbf{малоопасные}: барий, ванадий, вольфрам, марганец, стронций.
\end{itemize}

Важным свойством почвы является её \textbf{буферность} — способность противостоять изменению реакции почвенного раствора (pH) при воздействии кислот, щелочей или их солей. Буферность влияет на подвижность химических элементов и их воздействие на контактирующие среды.

\subsection*{Показатели оценки загрязнения почв}

Для количественной оценки уровня химического загрязнения почв используются следующие показатели:

\begin{enumerate}
    \item \textbf{Коэффициент концентрации химического вещества ($K_c$)} — показывает, во сколько раз фактическое содержание вещества в почве ($C_i$) превышает его фоновую концентрацию ($C_\phi$).
          \begin{equation}
              K_c = \frac{C_i}{C_\phi},
              \label{formula:concentration_coefficient}
          \end{equation}
          где:
          \begin{itemize}
              \item $C_i$ — фактическое содержание вещества в почве, мг/кг;
              \item $C_\phi$ — фоновая концентрация загрязняющего вещества в почве, мг/кг.
          \end{itemize}

    \item \textbf{Суммарный показатель загрязнения ($Z_c$)} — комплексный показатель, учитывающий эффект от нескольких загрязняющих веществ. Рассчитывается как сумма коэффициентов концентрации химических элементов за вычетом поправки на их количество.
          \begin{equation}
              Z_c = \sum_{i=1}^{n} K_{ci} - (n - 1),
              \label{formula:total_pollution_index}
          \end{equation}
          где:
          \begin{itemize}
              \item $K_{ci}$ — коэффициент концентрации $i$-го химического элемента;
              \item $n$ — число суммируемых элементов.
          \end{itemize}
\end{enumerate}

\subsection*{Оценочные шкалы}

Оценка опасности загрязнения почв по показателю $Z_c$ и выбор соответствующих мероприятий производятся на основе нормативных таблиц.

\begin{table}[H]
    \centering
    \caption{Ориентировочная оценочная шкала опасности загрязнения почв по суммарному показателю загрязнения ($Z_c$).}
    \label{table:health_impact}

    \begin{tblr}{
            colspec = {| Q[c, m] | Q[c, m] | X[j, h] | },
            width = \linewidth,
            hlines = 1, vlines = 1,
            row{1} = {font=\bfseries},
        }
        Значение $\bm{Z_c}$ & {Категория                                                                                                                                                                                \\загрязн. почв} & Изменения показателей здоровья населения в очагах загрязнения \\
        Менее 16            & Допустимая   & Наиболее низкий уровень заболеваемости детей и минимальная частота встречаемости функциональных отклонений.                                                                \\
        16...32             & {Умеренно                                                                                                                                                                                 \\опасная} & Увеличение уровня общей заболеваемости. \\
        32...128            & Опасная      & Увеличение уровня общей заболеваемости, числа часто болеющих детей, детей с хроническими заболеваниями, нарушениями функционального состояния сердечно-сосудистой системы. \\
        Более 128           & {Чрезвычайно                                                                                                                                                                              \\опасная} & Увеличение уровня общей заболеваемости детского населения, женщин с нарушениями репродуктивной функции (увеличение токсикозов беременности, числа преждевременных родов).
    \end{tblr}
\end{table}


\begin{longtblr}[
        caption = {Оценка почв, загрязненных химическими веществами.},
        label = {table:soil_assessment_tblr}
    ]{
        width = \linewidth,
        colspec = { | X[1,c,m] | X[3,j,h] | X[3,j,h] | X[4,j,h] | },
        hlines = 1,
        rowhead = 0,
        column{1} = {font=\bfseries},
        row{1} = {font=\bfseries, valign=m},
    }
    Категория                                     & Характеристика загрязнения                                                                                                                                                            & Возможность использования территории                                                                  & Предлагаемые мероприятия                                                                                                                                                                                                                \\
    \rotatebox{90}{\fbox{I} Допустимая}           & Содержание химических веществ в почве превышает фоновое, но не превышает ПДК.                                                                                                         & Под любые культуры.                                                                                   & Контроль уровня воздействия источников загрязнения и доступности токсикантов для растений (известкование, внесение удобрений и т.п.).                                                                                                   \\
    \rotatebox{90}{\fbox{II} Умеренно опасная}    & Содержание химических веществ в почве превышает их ПДК при лимитирующем общесанитарном, миграционном водном и воздушном показателях, но ниже допустимого уровня по транслокационному. & Под любые культуры при условии контроля их качества.                                                  & Аналогично I категории. Дополнительный контроль веществ в зоне дыхания с.-х. рабочих и в воде местных источников.                                                                                                                       \\
    \rotatebox{90}{\fbox{III} Высоко опасная}     & Содержание химических веществ в почве превышает их ПДК при транслокационном показателе вредности.                                                                                     & Под технические культуры. Использование под с.-х. культуры ограничено учетом растений-концентраторов. & Кроме мероприятий I категории, обязательный контроль за содержанием токсикантов в продуктах питания и кормах. Рекомендуется перемешивать продукцию с выращенной на чистой почве. Ограничение использования зеленой массы на корм скоту. \\
    \rotatebox{90}{\fbox{IV} Чрезвычайно опасная} & Содержание химических веществ превышает ПДК в почве по всем показателям вредности.                                                                                                    & Под технические культуры. Лесозащитные полосы.                                                        & Мероприятия по снижению уровня загрязнения и связыванию токсикантов в почве. Контроль за содержанием токсикантов в зоне дыхания с.-х. рабочих и в воде местных источников.                                                              \\
\end{longtblr}



\newpage
\centeredsection{РАСЧЕТЫ}

\begin{table}[H]
    \centering
    \caption{Исходные данные для варианта 4.}
    \label{table:source-data}

    \begin{tblr}{
            colspec = {
                    X[c]
                    X[c]
                    X[c]
                    X[c]
                    X[c]
                    X[c]
                },
            width = \linewidth,
            row{1-3} = {font=\bfseries},
            row{6} =  {font=\bfseries},
            hlines, vlines
        }
                                              & \SetCell[c=5]{c} Загрязняющее вещество                                                                                                       \\
        \SetCell[r=2]{c} Класс опасности      & Мышьяк                                                                                              & Ртуть & Хром (VI) & Марганец & Ванадий \\
                                              & I                                                                                                   & I     & II        & III      & III     \\
        $C_\phi, \frac{\text{мг}}{\text{кг}}$ & 6.0                                                                                                 & 2.9   & 0.11      & 1100     & 30      \\
        ПДК$, \frac{\text{мг}}{\text{кг}}$    & 5.0                                                                                                 & 2.1   & 0.05      & 1500.0   & 150.0   \\
        \hline
        Вариант                               & \SetCell[c=5]{c} Фактическое содержание вещества в почве $\bm{C_\phi}, \frac{\text{мг}}{\text{кг}}$                                          \\
        4                                     & 9.4                                                                                                 & 3.7   & 5.12      & 2340.0   & 230.0   \\
    \end{tblr}
\end{table}


\subsection*{ \boxed{\text{ Задание 1. }} }
\begin{quote}
    \textit{Рассчитать коэффициент концентрации химического вещества $К_с$ в почве по \cref{formula:concentration_coefficient}}.
\end{quote}

\begin{table}[H]
    \centering
    % \caption{caption}
    \label{table:step_1}

    \begin{tblr}{
            colspec = { Q[l, m]  Q[r, m] Q[l, m] Q[l, m]},
            hlines = 0, vlines = 0,
        }
        Вещество  & \SetCell[c=3]{c, m} $K_{ci}$                     \\
        \toprule
        Мышьяк    & $9.4     & / 6.0  & = 1.57$  \\
        Ртуть     & $3.7     & / 2.9  & = 1.28$  \\
        Хром (VI) & $5.12    & / 0.11 & = 46.55$ \\
        Марганец  & $2340.0  & / 1100 & = 2.13$  \\
        Ванадий   & $230.0   & / 30.0 & = 7.67$  \\
    \end{tblr}
\end{table}

\subsection*{ \boxed{\text{ Задание 2. }} }
\begin{quote}
    \textit{Рассчитать суммарный показатель загрязнения $Z_c$ по \cref{formula:total_pollution_index}}.
\end{quote}
\begin{equation*}
    Z_c = (1.57 + 1.28 + 46.55 + 2.13 + 7.67) - (5 - 1) = 58.20 - 4 = 54.20
\end{equation*}

\subsection*{ \boxed{\text{ Задание 3. }} }
\begin{quote}
    \textit{Оценить опасность загрязнения почв по показателю $Z_c$ с использованием \cref{table:health_impact} и \cref{table:soil_assessment_tblr}}.
\end{quote}

Поскольку $Z_c = 54.20$, уровень загрязнения почвы относится к категории \textbf{Опасная} (32...128). Данный уровень загрязненности почвы способствует увеличению уровня общей заболеваемости, числа часто болеющих детей, детей с хроническими заболеваниями, нарушениями функционального состояния сердечно-сосудистой системы.

Согласно \cref{table:soil_assessment_tblr}, почва относится к категории \textbf{III Высоко опасная}, так как содержание химических веществ превышает их ПДК.
Теорриторию с почвой данной категории можно использовать под технические культуры. Использование под сельскохозяйственные культуры ограничено учетом растений-концентраторов. 

Рекомендуемые мероприятия включают обязательный контроль за содержанием токсикантов в продуктах питания и кормах, а также рекомендуется перемешивать продукцию с выращенной на чистой почве. Ограничение использования зеленой массы на корм скоту.
