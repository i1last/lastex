\section*{Цель работы}

Познакомиться с понятиями предельно допустимой концентрации (ПДК), индексом загрязнения водных объектов (ИЗВ) и интегральным индексом экологического состояния водоемов.




\section*{Теоретические сведения}
\subsection*{Виды водопользования и нормирование качества воды}

В зависимости от назначения, выделяют следующие основные виды водопользования:
\begin{itemize}
    \item \textit{Хозяйственно-питьевое} — использование водных объектов как источников водоснабжения для населения и предприятий пищевой промышленности.
    \item \textit{Культурно-бытовое} — использование для купания, спорта и отдыха.
    \item \textit{Рыбохозяйственное} — использование для добычи (вылова) или искусственного воспроизводства водных биоресурсов.
\end{itemize}

Для оценки и контроля качества воды вводится понятие \textit{предельно допустимой концентрации (ПДК)} — максимальной концентрации вредного вещества, которая не оказывает прямого или косвенного негативного влияния на организм человека в течение всей жизни и на здоровье последующих поколений.

ПДК устанавливается для двух основных категорий водоемов:
\begin{enumerate}
    \item \textit{Для хозяйственно-питьевого и культурно-бытового водопользования (ПДК\textsubscript{в})}, с учетом трех показателей вредности:
          \begin{itemize}
              \item органолептического;
              \item общесанитарного;
              \item санитарно-токсикологического.
          \end{itemize}

    \item \textit{Для рыбохозяйственного водопользования (ПДК\textsubscript{вр})}, с учетом пяти показателей вредности:
          \begin{itemize}
              \item органолептического;
              \item санитарного;
              \item санитарно-токсикологического;
              \item токсикологического;
              \item рыбохозяйственного.
          \end{itemize}
\end{enumerate}

\subsection*{Оценка загрязнения и суммарное воздействие}

Для комплексной оценки качества воды часто используется \textit{индекс загрязнения воды (ИЗВ)}, который рассчитывается по шести гидрохимическим показателям. Три из них являются обязательными:
\begin{itemize}
    \item концентрация растворенного кислорода ($O_2$);
    \item водородный показатель ($pH$);
    \item биологическое потребление кислорода ($БПК_5$).
\end{itemize}

Защита окружающей среды от загрязнения регламентируется требованием, согласно которому фактическая концентрация $C_i$ каждого вредного вещества не должна превышать его ПДК.
$$
    C_i \le \text{ПДК}_i
$$
где:
\begin{itemize}
    \item $C_i$ — фактическая концентрация $i$-го вредного вещества;
    \item $\text{ПДК}_i$ — предельно допустимая концентрация $i$-го вредного вещества.
\end{itemize}

При совместном присутствии в воде нескольких вредных веществ, обладающих однонаправленным действием, их безразмерная суммарная концентрация не должна превышать единицу.
$$
    \frac{C_1}{\text{ПДК}_1} + \frac{C_2}{\text{ПДК}_2} + \dots + \frac{C_n}{\text{ПДК}_n} = \sum_{i=1}^{n} \frac{C_i}{\text{ПДК}_i} \le 1
$$

\subsection*{Наиболее токсичные вещества}
\begin{itemize}
    \item \textit{Ртуть (Hg)} — токсикант кумулятивного действия, относится к 1 классу чрезвычайно опасных веществ. В воде переходит в высокотоксичную метилированную форму. Поражает центральную нервную систему, почки, печень и ЖКТ. Вызывает болезнь Минамата.
    \item \textit{Мышьяк (As)} — опасный яд и канцероген. Даже в малых концентрациях вреден для человека. Способен накапливаться в организме. Отравление проявляется нарушениями в работе сердца, ЖКТ и нервной системы (параличи, судороги).
    \item \textit{Свинец (Pb)} — один из самых распространенных и опасных загрязнителей. Основным источником антропогенного загрязнения долгое время был автотранспорт из-за использования тетраэтилсвинца в топливе. Вызывает острое поражение почек, колики, гемолиз. Хроническое отравление приводит к неврологическим расстройствам, анемии, проблемам с репродуктивной функцией.
\end{itemize}


\section*{Практическое задание}
\begin{quote}
    \textit{На берегу озера площадью $S$ км${}^2$ и средней глубиной $h$ м располо-жено промышленное предприятие, использующее воду озера для технических нужд и затем сбрасывающее загрязнённую воду в озеро. Цикл работы предприятия непрерывный (круглосуточный). Объём сброса сточной воды – $L$ л/сек. Рассчитать, каким будет загрязнение озера через 1 год. Сделать выводы о промышленном загрязнении озера и дать рекомендации по сохранению озера.}
\end{quote}

\subsection*{Порядок расчета загрязнения водоема}

Для решения задачи по оценке загрязнения озера через год после сброса сточных вод необходимо выполнить следующие шаги:

\begin{enumerate}
    \item \textit{Определить начальный объем воды в озере ($V_\text{оз}$)}. Он рассчитывается как произведение площади озера $S$ на его среднюю глубину $h$.
          $$
              V_\text{оз} = S \cdot h
          $$

    \item \textit{Вычислить объем сточных вод ($V_\text{ст}$)}, поступающих в озеро за 1 год (365 дней). Если объем сброса задан в $L$ (л/сек), то годовой объем равен:
          $$
              V_\text{ст} = L \cdot 3600 \cdot 24 \cdot 365
          $$

    \item \textit{Определить количество каждого вредного вещества ($m_i$)}, поступившего в озеро со сточной водой за год. Этот параметр обычно задается в условии задачи.

    \item \textit{Вычислить итоговую концентрацию каждого вещества ($C_i$)} в озере. Концентрация — это отношение массы вещества к общему объему воды в озере ($V_\text{общ} = V_\text{оз} + V_\text{ст}$).
          $$
              C_i = \frac{m_i}{V_\text{общ}} = \frac{\text{количество ВВ}_i \text{ в озере}}{\text{объём воды в озере}}
          $$
          \textit{Примечание: Необходимо обеспечить согласованность единиц измерения (например, мг/л или г/м³).}

    \item \textit{Определить общее загрязнение озера ($C_\text{общ}$)} по формуле суммарной безразмерной концентрации.
          $$
              C_\text{общ} = \sum_{i=1}^{n} \frac{C_i}{\text{ПДК}_i}
          $$

    \item \textit{Сделать выводы}. Сравнить полученное значение $C_\text{общ}$ с единицей.
          \begin{itemize}
              \item Если $C_\text{общ} \le 1$, то суммарная концентрация загрязнителей находится в пределах нормы.
              \item Если $C_\text{общ} > 1$, то вода в озере считается загрязненной, и необходимо разрабатывать рекомендации по сохранению водоема.
          \end{itemize}
\end{enumerate}



\newpage
\centeredsection{РАСЧЕТЫ} % На защите будут вопросы по типу что за вещества такие загрязняющие и почему они вредны
\begin{table}[H]
    \centering
    \caption{Исходные данные для варианта 4.}
    \label{table:source-data}
    \begin{tabularx}{\linewidth}{|c|c|c|c|C|C|C|}
        \hline
        \multirow{2}{*}{№}                                      & \multirow{2}{*}{$S$, км${}^2$} & \multirow{2}{*}{$h$, м} & \multirow{2}{*}{$L,\frac{\text{л}}{\text{с}}$} & \multicolumn{3}{|c|}{Концентрация ВВ в сточной воде, $\frac{\text{мг}}{\text{л}}$}                    \\
        \cline{5-7}
                                                                &                                &                         &                                                & Мышьяк                                                                             & Ртуть   & Свинец \\
        \hline

        4                                                       & 5.2                            & 2.5                     & 10                                             & 0.020                                                                              & 0.00090 & 0.10   \\
        \hline \hline
        \multicolumn{4}{|c|}{ПДК, $\frac{\text{мг}}{\text{л}}$} & 0.01                           & 0.0005                  & 0.03                                                                                                                                                   \\
        \hline
    \end{tabularx}
\end{table}


\subsection*{ \boxed{\text{ Шаг 1. }} }
Определяем объем воды в озере:
$$
    V_\text{оз} = S \cdot h = 5.2 \text{ км}^2 \cdot 2.5 \text{ м} = 5.2 \cdot 10^6 \text{ м}^2 \cdot 2.5 \text{ м} = 13 \cdot 10^6 \text{ м}^3
$$

\subsection*{ \boxed{\text{ Шаг 2. }} }
Вычисляем объем сточных вод, поступающих в озеро за 1 год:
\begin{align*}
    V_\text{ст} = L \frac{\text{л}}{\text{с}} \cdot 60^2 \cdot 24 \cdot 365
           = 10 \frac{\text{л}}{\text{с}} \cdot 3600 \cdot 24 \cdot 365 = \\
           = 3.1536 \cdot 10^8 \text{ л}
           = 3.1536 \cdot 10^5 \text{ м}^3
\end{align*}

\subsection*{ \boxed{\text{ Шаг 3. }} }
Определяем количество каждого вредного вещества, поступившего в озеро со сточной водой за год:

\begin{itemize}
    \item Мышьяк:
          $$
              m_{As} = C_{ст} \cdot V_{ст} = 0.020 \frac{\text{мг}}{\text{л}} \cdot 3.1536 \cdot 10^8 \text{ л} = 6.3072 \cdot 10^6 \text{ мг} = 6307.2 \text{ г}
          $$

    \item Ртуть:
          $$
              m_{Hg} = C_{ст} \cdot V_{ст} = 0.00090 \frac{\text{мг}}{\text{л}} \cdot 3.1536 \cdot 10^8 \text{ л} = 2.83824 \cdot 10^5 \text{ мг} = 283.824 \text{ г}
          $$

    \item Свинец:
          $$
              m_{Pb} = C_{ст} \cdot V_{ст} = 0.10 \frac{\text{мг}}{\text{л}} \cdot 3.1536 \cdot 10^8 \text{ л} = 3.1536 \cdot 10^7 \text{ мг} = 31~536 \text{ г}
          $$
\end{itemize}

\subsection*{ \boxed{\text{ Шаг 4. }} }
Вычисляем итоговую концентрацию каждого вещества в озере:
$$
    V_\text{общ} = V_\text{оз} + V_\text{ст} = 13 \cdot 10^6 \text{ м}^3 + 3.1536 \cdot 10^5 \text{ м}^3 = 13.31536 \cdot 10^6 \text{ м}^3
$$
\begin{itemize}
    \item Мышьяк:
          $$
              C_{As} = \frac{m_{As}}{V_\text{общ}} = \frac{6307.2 \text{ г}}{13.31536 \cdot 10^6 \text{ м}^3} = 0.00047 \frac{\text{мг}}{\text{л}}
          $$

    \item Ртуть:
          $$
              C_{Hg} = \frac{m_{Hg}}{V_\text{общ}} = \frac{283.824 \text{ г}}{13.31536 \cdot 10^6 \text{ м}^3} = 0.000021 \frac{\text{мг}}{\text{л}}
          $$

    \item Свинец:
          $$
              C_{Pb} = \frac{m_{Pb}}{V_\text{общ}} = \frac{31~536 \text{ г}}{13.31536 \cdot 10^6 \text{ м}^3} = 0.00237 \frac{\text{мг}}{\text{л}}
          $$
\end{itemize}

\subsection*{ \boxed{\text{ Шаг 5. }} }
Определяем общее загрязнение озера:
\begin{align*}
    C_\text{общ} &=
    \frac{C_{As}}{\text{ПДК}_{As}} + \frac{C_{Hg}}{\text{ПДК}_{Hg}} + \frac{C_{Pb}}{\text{ПДК}_{Pb}} = \\
    &= \frac{0.00047}{0.01} + \frac{0.000021}{0.0005} + \frac{0.00237}{0.03} = \\
    &= 0.047 + 0.042 + 0.079 = 0.168
\end{align*}


\subsection*{ \boxed{\text{ Шаг 6. Выводы}} }
Поскольку $C_\text{общ} = 0.168 \le 1$, суммарная концентрация загрязнителей находится в пределах нормы. Однако, учитывая потенциальную опасность мышьяка, ртути и свинца, рекомендуется предпринять меры по снижению их концентрации в сточных водах перед сбросом в озеро, чтобы предотвратить возможное накопление этих веществ в экосистеме озера и обеспечить долгосрочную безопасность водоема.
