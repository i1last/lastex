\section*{Теоретические сведения}

\subsection*{Парниковый эффект и парниковые газы}

Парниковый эффект — явление, при котором температура у поверхности Земли повышается вследствие поглощения и переизлучения атмосферой инфракрасного излучения. Основными парниковыми газами (ПГ), регулируемыми Киотским протоколом и законодательством РФ, являются: диоксид углерода ($CO_2$), метан ($CH_4$), закись азота ($N_2O$), гидрофторуглероды (ГФУ), перфторуглероды (ПФУ), гексафторид серы ($SF_6$) и трифторид азота ($NF_3$).

Для сопоставления влияния различных газов на климат используется понятие \textbf{потенциала глобального потепления (ПГП, GWP)} — коэффициента, определяющего степень воздействия одной единицы массы конкретного ПГ относительно диоксида углерода.

Расчет выбросов производится в единицах $CO_2$-эквивалента по формуле:
\begin{equation}
    E_{CO_2e, y} = \sum_{i=1}^{n} E_{i,y} \cdot GWP_i,
    \label{eq:co2_equiv}
\end{equation}
где $E_{i,y}$ — масса выброса $i$-го газа, $GWP_i$ — потенциал глобального потепления (для $CO_2$ $GWP=1$).

\subsection*{Методика расчета выбросов от стационарного сжигания топлива}

Количественное определение выбросов $CO_2$ от стационарных источников рассчитывается по формуле:
\begin{equation}
    E_{CO_2} = \sum_{i=1}^{n} FC_{iy} \cdot EF_{CO_2, iy} \cdot OF_{iy},
    \label{eq:stationary_calc}
\end{equation}
где:
\begin{itemize}
    \item $FC_{iy}$ — расход топлива $i$ за период $y$ (в единицах массы, объема или энергии);
    \item $EF_{CO_2, iy}$ — коэффициент выбросов $CO_2$ (т $CO_2$/ед. топлива);
    \item $OF_{iy}$ — коэффициент окисления топлива (принимается равным 1.0).
\end{itemize}

Для перевода натуральных единиц топлива в энергетические (ТДж, где $1 \text{ТДж} = 10^{12} \text{Дж}$) используется соотношение:
\begin{equation}
    FC_{iy} [\text{ТДж}] = FC'_{iy} [\text{т}] \cdot NCV_{iy} [\text{МДж/кг}] \cdot 10^{-3}.
    \label{eq:energy_conversion}
\end{equation}

Справочные коэффициенты для расчета представлены в \cref{table:coef_stationary}.

\begin{table}[H]
    \centering
    \caption{Коэффициенты перевода расхода топлива и выбросов $CO_2$ (выборка для стационарных источников).}
    \label{table:coef_stationary}
    \begin{tblr}{
            colspec = { | c | X[l,m] | c | c | c | c | },
            width = \linewidth,
            hlines, vlines,
            row{1-2} = {font=\bfseries, c},
        }
        \SetCell[r=2]{c} № & \SetCell[r=2]{c} Вид топлива & \SetCell[r=2]{c} Ед. изм. & $NCV_{iy}$ & $EF_{CO_2}$ & $EF_{CO_2}$ \\
        & & & ТДж/тыс. т & т $CO_2$/т.у.т. & т $CO_2$/ТДж \\
        4  & Топливо дизельное     & тонна    & 42.5  & 2.17 & 74.1  \\
        8  & Газ горючий природный & тыс. м$^3$ & 33.08 & 1.59 & 54.4  \\
        18 & Торф топливный        & тонна    & 10.0  & 3.11 & 106.0 \\
    \end{tblr}
\end{table}

\subsection*{Методика расчета выбросов от транспорта}

Выбросы $CO_2$ от передвижных источников (сжигание топлива в двигателях) рассчитываются по формуле:
\begin{equation}
    E_{CO_2, y} = \sum FC_{i,b,y} \cdot EF_{i,b},
    \label{eq:mobile_calc}
\end{equation}
где $FC_{i,b,y}$ — расход топлива вида $i$ транспортом типа $b$ (в тоннах), $EF_{i,b}$ — коэффициент выбросов (т $CO_2$/т топлива).

Если учет топлива ведется в объемных единицах (литры) через пробег, масса топлива определяется как:
\begin{equation}
    FC_{i,b,y} [\text{т}] = \sum \left( \frac{N \cdot L \cdot q}{100} \right) \cdot \rho_i \cdot 10^{-3},
    \label{eq:fuel_mass_calc}
\end{equation}
где:
\begin{itemize}
    \item $N$ — количество единиц транспорта;
    \item $L$ — годовой пробег единицы, км (в исходных данных часто указывается в тыс. км, требуется перевод);
    \item $q$ — расход топлива, л/100 км;
    \item $\rho_i$ — плотность топлива, кг/л.
\end{itemize}

\begin{table}[H]
    \centering
    \caption{Коэффициенты выбросов и плотность топлива для транспорта.}
    \label{table:coef_mobile}
    \begin{tblr}{
            colspec = { | c | X[l,m] | c | c | },
            width = \linewidth,
            hlines, vlines,
            row{1} = {font=\bfseries, c},
        }
        № & Вид топлива & Плотность $\rho$, кг/л & $EF_{CO_2}$, т $CO_2$/т \\
        4 & Бензин АИ-95 & 0.750 & 3.026 \\
        12 & Компримированный прир. газ & 0.668 & 1.840 \\
        13 & Сжиженный природный газ & 0.424 & 2.710 \\
    \end{tblr}
\end{table}

\subsection*{План выполнения расчетов}

Для оценки суммарных выбросов парниковых газов предприятия необходимо выполнить следующие этапы:

1.  \textbf{Расчет выбросов от стационарных источников ($E_{stat}$).}
    Для каждого вида топлива выбирается метод расчета в зависимости от единиц измерения исходных данных:
    \begin{itemize}
        \item Для топлива, заданного в единицах массы (тонны), производится перевод в энергетические единицы (ТДж) с использованием низшей теплоты сгорания $NCV$, затем применяется коэффициент $EF$ (т $CO_2$/ТДж).
        \item Для топлива, заданного в условном топливе (т.у.т.), расчет производится напрямую с использованием коэффициента $EF$ (т $CO_2$/т.у.т.).
        \item Для топлива, заданного в энергетических единицах (ТДж), расчет производится прямым умножением на $EF$ (т $CO_2$/ТДж).
    \end{itemize}

2.  \textbf{Расчет выбросов от передвижных источников ($E_{mob}$).}
    Для каждого вида транспорта:
    \begin{itemize}
        \item Рассчитывается суммарный объем потребленного топлива $V$ (л) на основе количества машин $N$, пробега $L$ и удельного расхода $q$. Учитывается перевод пробега из тыс. км в км.
        \item Объем топлива переводится в массу $M$ (т) с использованием плотности $\rho$.
        \item Выбросы определяются произведением массы топлива на коэффициент $EF$ (т $CO_2$/т).
    \end{itemize}

3.  \textbf{Определение итогового выброса.}
    Суммирование результатов по всем источникам: $E_{\Sigma} = \sum E_{stat} + \sum E_{mob}$.


\newpage
\centeredsection{РАСЧЕТЫ}


\begin{table}[H]
    \centering
    \caption{Исходные данные для варианта 4. Стационарные источники.}
    \label{table:source-data-var4-stationary}
    \begin{tblr}{
            colspec = { | X[l,m] | X[c,m] | X[c,m] | },
            width = \linewidth,
            hlines, vlines,
            row{1} = {font=\bfseries},
        }
        Тип топлива           & Ед. изм. & Кол-во за год \\
        Топливо дизельное     & тонна    & 34            \\
        Газ горючий природный & т.у.т.   & 115           \\
        Торф топливный        & ТДж      & 9.21          \\
    \end{tblr}
\end{table}

\begin{table}[H]
    \centering
    \caption{Исходные данные для варианта 4. Передвижные источники.}
    \label{table:source-data-var4-mobile}
    \begin{tblr}{
            colspec = { | X[l,m] | X[c,m] | X[c,m] | X[c,m] | },
            width = \linewidth,
            hlines, vlines,
            row{1} = {font=\bfseries},
        }
        Тип топлива (транспорт)                & $\bm N$ (шт.) & Расход $\bm q$ ($\bm{ \frac{\text{л}}{100 \text{ км}}}$) & Пробег $\bm L$ (тыс. км) \\
        Бензин АИ-95                           & 5             & 15                                                       & 34                       \\
        Компримированный природный газ         & 4             & 7.6                                                      & 62                       \\
        Сжиженный природный газ                & 8             & 10.2                                                     & 12                       \\
    \end{tblr}
\end{table}


\subsection*{ \boxed{\text{ Задание 1. Расчет стационарных источников }} }

Расчет производится на основании \cref{eq:stationary_calc} и данных \cref{table:coef_stationary}.

\textbf{1. Топливо дизельное (исходные данные в тоннах).}
Переведем расход в энергетические единицы (ТДж), используя $NCV = 42.5$ ТДж/тыс.т (МДж/кг):
$$FC_{TJ} = 34 \cdot 42.5 \cdot 10^{-3} = 1.445 \text{ ТДж}$$
Выброс $CO_2$ (при $EF = 74.1$ т $CO_2$/ТДж):
$$E_1 = 1.445 \cdot 74.1 = 107.07 \text{ т } CO_2$$

\textbf{2. Газ горючий природный (исходные данные в т.у.т.).}
Используем коэффициент для т.у.т. ($EF = 1.59$ т $CO_2$/т.у.т.):
$$E_2 = 115 \cdot 1.59 = 182.85 \text{ т } CO_2$$

\textbf{3. Торф топливный (исходные данные в ТДж).}
Расчет прямой ($EF = 106.0$ т $CO_2$/ТДж):
$$E_3 = 9.21 \cdot 106.0 = 976.26 \text{ т } CO_2$$

\textbf{Суммарный выброс от стационарных источников:}
$$E_{stat} = 107.07 + 182.85 + 976.26 = 1266.18 \text{ т } CO_2$$

\subsection*{ \boxed{\text{ Задание 2. Расчет передвижных источников }} }

Расчет производится по \cref{eq:mobile_calc,eq:fuel_mass_calc} с использованием коэффициентов из \cref{table:coef_mobile}.
\textit{Примечание: $L$ переводится из тыс. км в единицы (умножение на 1000), $q$ дано на 100 км.}

\textbf{1. Бензин АИ-95.}
$$V_1 = 5 \cdot (34 \cdot 1000) \cdot \frac{15}{100} = 25500 \text{ л}$$
Масса топлива ($\rho = 0.750$ кг/л):
$$M_1 = 25500 \cdot 0.750 \cdot 10^{-3} = 19.125 \text{ т}$$
Выброс $CO_2$ ($EF = 3.026$ т $CO_2$/т):
$$E_{tr1} = 19.125 \cdot 3.026 = 57.87 \text{ т } CO_2$$

\textbf{2. Компримированный природный газ.}
$$V_2 = 4 \cdot (62 \cdot 1000) \cdot \frac{7.6}{100} = 18848 \text{ л}$$
Масса топлива ($\rho = 0.668$ кг/л):
$$M_2 = 18848 \cdot 0.668 \cdot 10^{-3} = 12.590 \text{ т}$$
Выброс $CO_2$ ($EF = 1.840$ т $CO_2$/т):
$$E_{tr2} = 12.590 \cdot 1.840 = 23.17 \text{ т } CO_2$$

\textbf{3. Сжиженный природный газ.}
$$V_3 = 8 \cdot (12 \cdot 1000) \cdot \frac{10.2}{100} = 9792 \text{ л}$$
Масса топлива ($\rho = 0.424$ кг/л):
$$M_3 = 9792 \cdot 0.424 \cdot 10^{-3} = 4.152 \text{ т}$$
Выброс $CO_2$ ($EF = 2.710$ т $CO_2$/т):
$$E_{tr3} = 4.152 \cdot 2.710 = 11.25 \text{ т } CO_2$$

\textbf{Суммарный выброс от передвижных источников:}
$$E_{mob} = 57.87 + 23.17 + 11.25 = 92.29 \text{ т } CO_2$$

\subsection*{ \boxed{\text{ Результат }} }

Общее количество выбросов парниковых газов ($CO_2$-эквивалент) предприятием составляет:
$$E_{\Sigma} = E_{stat} + E_{mob} = 1266.18 + 92.29 = \boxed{1358.47 \text{ т } CO_2}$$

В ходе выполнения практической работы установлено:

\begin{enumerate}
    \item {Общий объем выбросов} парниковых газов предприятием за отчетный период составил {1358.47 т $CO_2$-экв.}
    
    \item В структуре выбросов доминируют \textbf{стационарные источники}, на долю которых приходится 1266.18 т $CO_2$, что составляет \textbf{93.2\%} от суммарного объема выбросов. Вклад передвижных источников незначителен — 92.29 т $CO_2$ (\textbf{6.8\%}).

    \item Основным источником выбросов среди стационарных установок является сжигание \textbf{торфа топливного} (976.26 т $CO_2$)
    
    \item Среди передвижных источников наибольший вклад вносит автотранспорт, использующий \textbf{бензин АИ-95} (57.87 т $CO_2$), что составляет более 60\% от выбросов транспорта.
\end{enumerate}

