\section*{Цель работы}
Исследовать эффективность работы фотоэлектрической панели в зависимости от угла падения светового потока, построить семейство вольт-амперных характеристик (ВАХ) и определить зависимость выходной мощности и коэффициента заполнения от угла наклона.

\section*{Схема эксперимента}
Схема подключения фотоэлектрического модуля (ФЭМ) аналогична использованной в работе №1 и включает в себя панель, нагрузочный потенциометр, амперметр и вольтметр. Измерения проводились при фиксированной мощности источника света 50 Вт для различных углов наклона панели относительно горизонта ($0^\circ \dots 90^\circ$).

\section*{Протокол измерений}

В таблицах \ref{tab:pmax}, \ref{tab:sc} и \ref{tab:oc} представлены данные измерений для режимов максимальной мощности, короткого замыкания и холостого хода соответственно.

% --- РЕЖИМ МАКСИМАЛЬНОЙ МОЩНОСТИ (Pmax) ---
\begin{table}[H]
    \caption{Режим максимальной мощности ($P_{max}$), 50 Вт}
    \label{tab:pmax}
    \centering
    
    % Моно Pmax
    \begin{subtable}[t]{0.48\textwidth}
        \centering
        \caption{Монокристаллическая панель}
        \begin{tabular}{|c|c|c|c|}
            \hline
            $\alpha, ^\circ$ & $I, \text{мА}$ & $U, \text{В}$ & $P, \text{мВт}$ \\
            \hline
            90 & 27.4 & 12.14 & 332.64 \\
            75 & 27.1 & 12.02 & 325.74 \\
            60 & 25.5 & 11.31 & 288.41 \\
            45 & 22.4 & 9.98  & 223.55 \\
            30 & 17.3 & 7.74  & 133.90 \\
            15 & 12.3 & 5.53  & 68.02 \\
            0  & 9.3  & 4.12  & 38.32 \\
            \hline
        \end{tabular}
    \end{subtable}
    \hfill
    % Поли Pmax
    \begin{subtable}[t]{0.48\textwidth}
        \centering
        \caption{Поликристаллическая панель}
        \begin{tabular}{|c|c|c|c|}
            \hline
            $\alpha, ^\circ$ & $I, \text{мА}$ & $U, \text{В}$ & $P, \text{мВт}$ \\
            \hline
            90 & 42.7 & 14.93 & 637.51 \\
            75 & 38.7 & 13.53 & 523.61 \\
            60 & 29.8 & 10.41 & 310.22 \\
            45 & 21.7 & 7.58  & 164.49 \\
            30 & 14.9 & 5.22  & 77.78 \\
            15 & 10.2 & 3.55  & 36.21 \\
            0  & 7.1  & 2.66  & 18.89 \\
            \hline
        \end{tabular}
    \end{subtable}
\end{table}

% --- РЕЖИМ КОРОТКОГО ЗАМЫКАНИЯ (КЗ) ---
\begin{table}[H]
    \caption{Режим короткого замыкания (КЗ), 50 Вт}
    \label{tab:sc}
    \centering
    
    % Моно КЗ
    \begin{subtable}[t]{0.48\textwidth}
        \centering
        \caption{Монокристаллическая панель}
        \begin{tabular}{|c|c|c|c|}
            \hline
            $\alpha, ^\circ$ & $I, \text{мА}$ & $U, \text{В}$ & $P, \text{мВт}$ \\
            \hline
            90 & 42.8 & 0 & 0 \\
            75 & 42.0 & 0 & 0 \\
            60 & 35.9 & 0 & 0 \\
            45 & 27.1 & 0 & 0 \\
            30 & 19.1 & 0 & 0 \\
            15 & 14.0 & 0 & 0 \\
            0  & 9.6  & 0 & 0 \\
            \hline
        \end{tabular}
    \end{subtable}
    \hfill
    % Поли КЗ
    \begin{subtable}[t]{0.48\textwidth}
        \centering
        \caption{Поликристаллическая панель}
        \begin{tabular}{|c|c|c|c|}
            \hline
            $\alpha, ^\circ$ & $I, \text{мА}$ & $U, \text{В}$ & $P, \text{мВт}$ \\
            \hline
            90 & 45.4 & 0 & 0 \\
            75 & 42.0 & 0 & 0 \\
            60 & 39.3 & 0 & 0 \\
            45 & 23.6 & 0 & 0 \\
            30 & 15.6 & 0 & 0 \\
            15 & 11.0 & 0 & 0 \\
            0  & 7.5  & 0 & 0 \\
            \hline
        \end{tabular}
    \end{subtable}
\end{table}

% --- РЕЖИМ ХОЛОСТОГО ХОДА (ХХ) ---
\begin{table}[H]
    \caption{Режим холостого хода (ХХ), 50 Вт}
    \label{tab:oc}
    \centering
    
    % Моно ХХ
    \begin{subtable}[t]{0.48\textwidth}
        \centering
        \caption{Монокристаллическая панель}
        \begin{tabular}{|c|c|c|c|}
            \hline
            $\alpha, ^\circ$ & $I, \text{мА}$ & $U, \text{В}$ & $P, \text{мВт}$ \\
            \hline
            90 & 0 & 16.05 & 0 \\
            75 & 0 & 15.75 & 0 \\
            60 & 0 & 15.50 & 0 \\
            45 & 0 & 15.26 & 0 \\
            30 & 0 & 14.95 & 0 \\
            15 & 0 & 14.60 & 0 \\
            0  & 0 & 14.25 & 0 \\
            \hline
        \end{tabular}
    \end{subtable}
    \hfill
    % Поли ХХ
    \begin{subtable}[t]{0.48\textwidth}
        \centering
        \caption{Поликристаллическая панель}
        \begin{tabular}{|c|c|c|c|}
            \hline
            $\alpha, ^\circ$ & $I, \text{мА}$ & $U, \text{В}$ & $P, \text{мВт}$ \\
            \hline
            90 & 0 & 19.54 & 0 \\
            75 & 0 & 19.18 & 0 \\
            60 & 0 & 18.69 & 0 \\
            45 & 0 & 18.19 & 0 \\
            30 & 0 & 17.50 & 0 \\
            15 & 0 & 16.72 & 0 \\
            0  & 0 & 15.93 & 0 \\
            \hline
        \end{tabular}
    \end{subtable}
\end{table}

\section*{Графические зависимости}

\subsection*{Семейство ВАХ для различных углов наклона}
На \cref{fig:iv_family} представлено семейство вольт-амперных характеристик, построенных по трем характеристическим точкам (КЗ, ММ, ХХ) для каждого угла.

\begin{figure}[hab]
    \centering
    \includegraphics[width=\linewidth]{code/results/iv_family.png}
    \caption{Семейство ВАХ для моно- и поликристаллических панелей}
    \label{fig:iv_family}
\end{figure}

\subsection*{Зависимость мощности от угла наклона}
График зависимости $P = f(\alpha)$ представлен на \cref{fig:power_angle}.

\begin{figure}[hab]
    \centering
    \includegraphics[width=0.8\linewidth]{code/results/power_vs_angle.png}
    \caption{Зависимость максимальной мощности от угла наклона панели}
    \label{fig:power_angle}
\end{figure}

\subsection*{Зависимость коэффициента заполнения от угла наклона}
График зависимости $K = f(\alpha)$ представлен на \cref{fig:fill_factor}, где
$$
K(\alpha) = \frac{P_\alpha}{U_{\text{ХХ},\alpha} \cdot I_{\text{КЗ},\alpha}}
$$

\begin{figure}[hab]
    \centering
    \includegraphics[width=0.8\linewidth]{code/results/fill_factor.png}
    \caption{Изменение коэффициента заполнения ВАХ при изменении угла}
    \label{fig:fill_factor}
\end{figure}

\section*{Выводы}

В ходе работы исследовано влияние угла падения света на энергетические характеристики солнечных панелей.
\begin{itemize}
    \item \textit{Максимальная мощность:} Для обоих типов панелей максимальная мощность достигается при угле наклона $90^\circ$. При уменьшении угла до $0^\circ$ мощность падает нелинейно. Поликристаллическая панель показала более высокую пиковую мощность ($P_{max} \approx 637$ мВт) по сравнению с монокристаллической ($P_{max} \approx 332$ мВт).
    
    \item \textit{Коэффициент заполнения ($K$):} Расчет показал, что коэффициент заполнения поликристаллической панели ($K \approx 0.72$) выше, чем у монокристаллической ($K \approx 0.48$). Хотя теоретически монокристаллические модули обладают более высоким $K$, полученный результат может быть связан с износом конкретного образца монокристаллической панели, используемого в лабораторной установке.

    \item \textit{Влияние типа панели:} Поликристаллическая панель в данном эксперименте продемонстрировала большую чувствительность к углу наклона: падение мощности при переходе от $90^\circ$ к $60^\circ$ у неё более выражено, чем у монокристаллической.}
\end{itemize}