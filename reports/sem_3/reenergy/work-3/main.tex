\section*{Цель работы}
Исследовать зависимость выходной мощности фотоэлектрического модуля (ФЭМ) от площади освещенной поверхности путем последовательного затенения отдельных фотоэлементов.

\section*{Схема эксперимента}
Схема подключения (рис. \ref{fig:circuit3}) включает в себя солнечную панель, нагрузку (потенциометр), амперметр и вольтметр. В ходе эксперимента использовался режим максимальной мощности ($P_{max}$), определенный в предыдущих работах. Затенение производилось путем полного перекрытия непрозрачным материалом от 1 до 9 фотоэлементов.

\begin{figure}[h]
    \centering
    \includegraphics[width=0.5\textwidth]{figs/scheme.png} % Используем ту же схему, что и ранее
    \caption{Схема установки для исследования влияния затенения}
    \label{fig:circuit3}
\end{figure}

\section*{Протокол измерений и расчеты}

В таблице \ref{tab:shading_protocol} представлены данные измерений. Расчет мощности произведен по формуле $P = I \cdot U$. Относительная мощность рассчитывалась как:
$$ \eta = \frac{P_i}{P_0} \cdot 100\% $$
где $P_0$ — мощность полностью открытой панели.

\begin{table}[H]
    \caption{Влияние затенения на характеристики панелей (Источник 20 Вт)}
    \label{tab:shading_protocol}
    \centering
    
    % Subtable 1: Mono
    \begin{subtable}[t]{\textwidth}
        \centering
        \caption{Монокристаллическая панель}
        \begin{tabular}{|c|c|c|c|c|}
            \hline
            $N_\text{закр}$ & $I, \text{мА}$ & $U, \text{В}$ & $P, \text{мВт}$ & $P/P_{max}, \%$ \\
            \hline
            0 & 13.20 & 10.50 & 138.60 & 100.0 \\
            1 & 6.87  & 4.58  & 31.46  & 22.7 \\
            2 & 1.65  & 1.36  & 2.24   & 1.6 \\
            3 & 0.81  & 0.72  & 0.58   & 0.4 \\
            4 & 0.66  & 0.46  & 0.30   & 0.2 \\
            5 & 0.58  & 0.51  & 0.30   & 0.2 \\
            6 & 0.42  & 0.33  & 0.14   & 0.1 \\
            7 & 0.35  & 0.28  & 0.10   & <0.1 \\
            8 & 0.33  & 0.24  & 0.08   & <0.1 \\
            9 & 0.25  & 0.21  & 0.05   & <0.1 \\
            \hline
        \end{tabular}
    \end{subtable}
    \hfill
    % Subtable 2: Poly
    \begin{subtable}[t]{\textwidth}
        \centering
        \caption{Поликристаллическая панель}
        \begin{tabular}{|c|c|c|c|c|}
            \hline
            $N_\text{закр}$ & $I, \text{мА}$ & $U, \text{В}$ & $P, \text{мВт}$ & $P/P_{max}, \%$ \\
            \hline
            0 & 14.80 & 10.85 & 160.58 & 100.0 \\
            1 & 7.12  & 5.55  & 39.52  & 24.6 \\
            2 & 4.11  & 3.14  & 12.91  & 8.0 \\
            3 & 2.52  & 1.78  & 4.49   & 2.8 \\
            4 & 1.96  & 1.53  & 3.00   & 1.9 \\
            5 & 1.10  & 0.82  & 0.90   & 0.6 \\
            6 & 0.89  & 0.64  & 0.57   & 0.4 \\
            7 & 0.86  & 0.62  & 0.53   & 0.3 \\
            8 & 0.82  & 0.59  & 0.48   & 0.3 \\
            9 & 0.80  & 0.55  & 0.44   & 0.3 \\
            \hline
        \end{tabular}
    \end{subtable}
\end{table}

\section*{Графические зависимости}

\subsection*{Влияние затенения на ток и напряжение}
На графиках \ref{fig:current_shade} и \ref{fig:voltage_shade} показано падение тока и напряжения при увеличении числа закрытых элементов.

\begin{figure}[H]
    \centering
    \includegraphics[width=0.8\linewidth]{code/results/current_shading.png}
    \caption{Зависимость тока нагрузки от числа закрытых элементов}
    \label{fig:current_shade}
\end{figure}

\begin{figure}[H]
    \centering
    \includegraphics[width=0.8\linewidth]{code/results/voltage_shading.png}
    \caption{Зависимость напряжения на нагрузке от числа закрытых элементов}
    \label{fig:voltage_shade}
\end{figure}

\subsection*{Деградация выходной мощности}
График падения мощности представлен на рис. \ref{fig:power_shade}.

\begin{figure}[H]
    \centering
    \includegraphics[width=0.8\linewidth]{code/results/power_shading.png}
    \caption{Зависимость выходной мощности от степени затенения}
    \label{fig:power_shade}
\end{figure}

\section*{Выводы}

В ходе работы исследовано влияние частичного затенения на эффективность работы солнечных панелей.

\begin{itemize}
    \item \textit{Критическое падение мощности:} Эксперимент показал непропорционально высокую зависимость мощности от площади затенения. При закрытии всего одного элемента (из общего массива) мощность монокристаллической панели упала до $22.7\%$ от номинала, а поликристаллической — до $24.6\%$.
    \item \textit{Физическая суть явления:} Солнечные элементы в панели соединены последовательно. Согласно законам Кирхгофа, ток в последовательной цепи ограничен током самого «слабого» звена. Затененный элемент перестает генерировать ток и становится паразитным сопротивлением (нагрузкой), на котором рассеивается мощность остальных ячеек.
    \item \textit{Сравнение типов панелей:}
        \begin{itemize}
            \item Монокристаллическая панель продемонстрировала более резкое падение характеристик: при закрытии двух элементов мощность составила всего $1.6\%$ от исходной.
            \item Поликристаллическая панель показала чуть более плавное снижение (при 2 закрытых элементах осталось $8.0\%$ мощности), однако общий тренд на катастрофическое снижение эффективности сохраняется для обоих типов.
        \end{itemize}
\end{itemize}
