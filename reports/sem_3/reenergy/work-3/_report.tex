% === ПОДКЛЮЧЕНИЕ ОБЩЕЙ ПРЕАМБУЛЫ ===
%========================================================================================
% КЛАСС ДОКУМЕНТА И ОСНОВНЫЕ ПАРАМЕТРЫ
%========================================================================================
\documentclass[a4paper,14pt,russian]{extarticle}

% Расширяет возможности размеров стандартных классов
\usepackage{extsizes}

% Разрешаем ставить больее длинные пробелы, если иначе не выходит сделать
% более верное разбиение абзаца на строки
\sloppy





%========================================================================================
% ЯЗЫК, ШРИФТЫ И КОДИРОВКА (Современный подход для LuaLaTeX)
%========================================================================================
\usepackage{cmap}           % Обеспечивает правильное отображение шрифтов и символов в PDF
\usepackage{fontspec}       % Пакет для работы с любыми системными шрифтами.
                            % Заменяет устаревшие inputenc и fontenc.
\usepackage[russian]{babel} % Поддержка русского языка: переносы, названия и т.д.

% --- Настройка шрифтов и интервалов
% Явно определяем семейство шрифтов "Times New Roman", указывая путь к файлам
\setmainfont{Times New Roman}[
    Path            = /usr/local/share/fonts/truetype/times-new-roman/, % Путь внутри контейнера
    Extension       = .ttf,
    UprightFont     = times,
    BoldFont        = timesbd,
    ItalicFont      = timesi,
    BoldItalicFont  = timesbi
]
% \setmainfont{TeX Gyre Termes} % Свободный аналог Times New Roman, включенный в TeX Live

% Указываем babel использовать основной шрифт для всех языков по умолчанию
\babelprovide[main]{russian}


\usepackage{setspace}         % Пакет для гибкого управления интервалами
\onehalfspacing               % Установка полуторного межстрочного интервала





%========================================================================================
% СТРАНИЦЫ
%========================================================================================
\usepackage[left=3cm, right=1cm, top=2cm, bottom=2cm]{geometry} % Поля документа
\usepackage{indentfirst}        % Красная строка для первого абзаца после заголовка
\setlength{\parindent}{1.25cm}  % Отступ красной строки - 1.25 см

\usepackage[toc,page]{appendix} % Поддержка приложение в отчете
\usepackage{pdflscape}          % Поддержка горизонтальных страниц

% --- Поддержка колонтитулов
\usepackage{enotez}

\makeatletter % Создание нового стиля для сносок на странице
\def\enotez@endnotes@footer{%
    \begin{center}
        \rule{0.5\linewidth}{0.4pt}
        \enotez@theendnotes
    \end{center}
}

% --- Настройка содержанмя
\usepackage{tocloft}

% Убираем точки между заголовком и номером страницы
\renewcommand{\cftsecleader      }{\hfil} 
\renewcommand{\cftsubsecleader   }{\hfil}
\renewcommand{\cftsubsubsecleader}{\hfil}
% Делаем заголовок оглавления без жирного шрифта
\renewcommand{\cfttoctitlefont}{\normalfont\bfseries\Large\centering}
\renewcommand{\cfttoctitlefont}{\normalfont\Large\bfseries\centering}
% Меняем название на "СОДЕРЖАНИЕ"
\renewcommand{\contentsname}{{\large\uppercase{СОДЕРЖАНИЕ}}}
% Убираем жирный шрифт с разделов и номеров страниц
\renewcommand{\cftsecfont       }{\normalfont} 
\renewcommand{\cftsecpagefont   }{\normalfont}
\renewcommand{\cftsubsecfont    }{\normalfont}
\renewcommand{\cftsubsecpagefont}{\normalfont}


%========================================================================================
% ЗАГОЛОВКИ
%========================================================================================
\usepackage{titlesec}

%titleformat{<команда>     }{<стиль>             }{<номер>           }{<отступ>}{<текст до>}
\titleformat{\section      }{\bfseries\normalsize}{\thesection.      }{1em     }{          }
\titleformat{\subsection   }{\bfseries\normalsize}{\thesubsection.   }{1em     }{          }
\titleformat{\subsubsection}{\bfseries\normalsize}{\thesubsubsection.}{1em     }{          }

\newcommand{\centeredsection}[1]{
    \noindent
    \begin{center}
        \textbf{\normalsize #1}
    \end{center}
    \par
}




%========================================================================================
% МАТЕМАТИКА
%========================================================================================
\usepackage{amsmath}        % Основные математические окружения
\usepackage{amsfonts}       % Математические шрифты
\usepackage{amssymb}        % Дополнительные математические символы
\usepackage{mathtools}      % Расширение для amsmath с исправлением ошибок и новыми командами
\usepackage{icomma}         % Корректная работа запятой как десятичного разделителя в формулах





%========================================================================================
% ГРАФИКА, ТАБЛИЦЫ И ПЛАВАЮЩИЕ ОБЪЕКТЫ
%========================================================================================
\usepackage{graphicx}       % Для вставки изображений
\usepackage{float}          % Для точного позиционирования объектов с опцией [H]
\usepackage{caption}        % Гибкая настройка подписей к рисункам и таблицам

% --- Настройка подписей к рисункам
\captionsetup[figure]{
    justification=centering,   % Выравнивание по центру
    labelsep=endash,           % Разделитель "Рисунок 1 –" (тире)
    singlelinecheck=false,     % Принудительное центрирование даже для коротких подписей
    font=normalsize,           % Обычный размер шрифта (не курсив)
    skip=6pt                   % Отступ после подписи
}

% --- Настройка подписей к таблицам
\captionsetup[table]{
    position=top,              % Подпись над таблицей
    justification=raggedright, % Выравнивание по левому краю
    labelsep=endash,           % Разделитель "Таблица 1 –" (тире)
    singlelinecheck=false,     % Принудительное выравнивание по левому краю
    font=normalsize,           % Обычный размер шрифта (не курсив)
    skip=6pt                   % Отступ перед таблицей
}

% --- Пакеты для качественных таблиц
\usepackage{multirow}       % Улучшенное форматирование таблиц
\usepackage{tabularx}       % Таблицы с автоматическим расчетом ширины колонок
\usepackage{array}          % Расширяет возможности работы с таблицами и выравниваниями
\usepackage{booktabs}       % Профессиональное оформление таблиц (горизонтальные линии \toprule, \midrule, \bottomrule)
\usepackage{makecell}       % Многострочные ячейки в таблицах

% --- Настройка новых типов колонок для tabularx
\renewcommand{\tabularxcolumn}[1]{m{#1}}
\newcolumntype{C}{>{\centering\arraybackslash}X}
\newcolumntype{L}{>{\arraybackslash}X}
\newcolumntype{R}{>{\raggedleft\arraybackslash}X}





%========================================================================================
% ИСХОДНЫЙ КОД
%========================================================================================
\usepackage{listings}       % Для вставки листингов кода
\usepackage{xcolor}         % Для определения цветов

% --- Настройка стиля для листингов
\definecolor{codegray}{gray}{0.95}
\definecolor{codepurple}{rgb}{0.58,0,0.82}
\definecolor{backcolour}{rgb}{0.98,0.98,0.98}

\lstdefinestyle{mystyle}{
    backgroundcolor=\color{backcolour},
    commentstyle=\color{green!50!black},
    keywordstyle=\color{blue},
    numberstyle=\tiny\color{gray},
    stringstyle=\color{codepurple},
    basicstyle=\footnotesize\ttfamily,
    breakatwhitespace=false,
    breaklines=true,
    captionpos=b,
    keepspaces=true,
    numbers=left,
    numbersep=5pt,
    showspaces=false,
    showstringspaces=false,
    showtabs=false,
    tabsize=2,
    frame=single,
    framerule=0.5pt,
    rulecolor=\color{black!20},
    title=\lstname
}
\lstset{style=mystyle} % Применяем стиль по умолчанию

% --- Новая команда для вставки кода из файла
% Использование: \insertcode[caption={Подпись}, label={lbl:code}]{путь/к/файлу.py}
% Код будет автоматически отформатирован, подсвечен синтаксис Python, и пронумерованы строки.
% Язык можно поменять, например: \insertcode[language=C++]{путь/к/файлу.cpp}.
% Т.е. language=Python - это язык по умолчанию, который можно свободно переопределять.
\newcommand{\insertcode}[2][]{\lstinputlisting[language=Python, #1]{#2}}





%========================================================================================
% ССЫЛКИ И НАВИГАЦИЯ
%========================================================================================
\usepackage{hyperref}       % Создание кликабельных ссылок в документе
\hypersetup{
    colorlinks=true,
    linkcolor=black,
    urlcolor=blue,
    citecolor=black
}

\usepackage[russian]{cleveref} % "Умные" ссылки (\cref вместо \ref)
% cleveref автоматически подставляет "рис.", "табл.", "формула"
% Было:  Как видно из рис. \ref{fig:graph} и табл. \ref{tab:my_results}... -> Результат: "Как видно из рис. 1 и табл. 1..."
% Стало: Как видно из \cref{fig:graph}     и \cref{tab:my_results}...      -> Результат: "Как видно из рис. 1 и табл. 1..."
% Настраиваем названия для cleveref
\crefname{figure}{рис.}{рис.}
\Crefname{figure}{Рис.}{Рис.}
\crefname{table}{табл.}{табл.}
\Crefname{table}{Табл.}{Табл.}
\crefname{section}{разд.}{разд.}
\Crefname{section}{Разд.}{Разд.}
\crefname{equation}{формуле}{формулам}
\Crefname{equation}{Формуле}{Формулам}


% === КОНФИГУРАЦИЯ ДАННЫХ ДЛЯ ТИТУЛЬНОГО ЛИСТА ===
\newcommand{\Department}{ЭТПТ}     % Кафедра <\Department>
\newcommand{\WorkType}{практической работе №3}    % По <\WorkType> (напр: По лабораторной работе №1)
\newcommand{\Discipline}{возобновляемая энергетика}  % По дисциплине <\Discipline>
\newcommand{\WorkTitle}{ИССЛЕДОВАНИЕ ЗАВИСИМОСТИ МОЩНОСТИ ФОТОЭЛЕКТРИЧЕСКОГО МОДУЛЯ ОТ ОСВЕЩЕННОЙ ЧАСТИ ЕЕ ПЛОЩАДИ}
\newcommand{\Group}{4494}
\newcommand{\Variant}{--}
\newcommand{\StudentName}{Рахметов А. Р.}
\newcommand{\TeacherName}{Ермекова М. Р., Козулина Т. П.}
\newcommand{\Year}{2025}

% Включение библиографии
% bibtex автоматически будет включен в процесс сборки документа
% при обнаружении файла references.bib в директории с основным _report.tex файлом
% \addbibresource{references.bib}  % Имя файла references.bib не следует менять! Используется оптимизация скрипта сборки документа

% === СБОРКА ДОКУМЕНТА ===
\begin{document}

 

% --- ТИТУЛЬНЫЙ ЛИСТ ---
% Это шаблон, он не используется напрямую.
% Вместо текста здесь стоят команды, которые будут определены в config.tex

\begin{center}
    МИНОБРНАУКИ РОССИИ \\
    СПбГЭТУ «ЛЭТИ» ИМ. В.И. УЛЬЯНОВА (ЛЕНИНА) \\
    Кафедра \Department % Имя кафедры
\end{center}

\vfill

\begin{center}
    \textbf{
        \MakeUppercase{Отчёт} \\
        По \WorkType \\ % Номер работы
        По дисциплине «\Discipline» \\
        Тема: \WorkTitle \\ % Тема работы
    }
\end{center}

\vfill

\noindent
\begin{tabularx}{\textwidth}{l X r}
    Студент гр. \Group, вар. \Variant & \hrulefill & \StudentName \\\\
    Преподаватель    & \hrulefill & \TeacherName
\end{tabularx}

\vfill

\begin{center}
    Санкт-Петербург \\
    \Year \\
\end{center}

\thispagestyle{empty}
\newpage




% --- ОГЛАВЛЕНИЕ ---
% \tableofcontents
% \thispagestyle{empty}
% \newpage



% --- ВВЕДЕНИЕ ---
% \newpage
% \addcontentsline{toc}{section}{Введение}
% \phantomsection
% \centeredsection{\uppercase{Введение}}
% В современном мире задачи транспортной логистики и автоматизации навигации играют ключевую роль во многих отраслях, от грузоперевозок до разработки беспилотных транспортных средств. Одной из фундаментальных проблем в этой области является построение оптимального маршрута с учетом различных ограничений, таких как запас хода транспортного средства. Эффективное решение этой задачи позволяет сократить временные и энергетические затраты, что обуславливает \textit{актуальность} данной курсовой работы. Разработка алгоритмов, способных находить кратчайший путь в графе с весами, является классической задачей теории графов, имеющей широкое практическое применение.

% Комментарий к абзацу об актуальности:
% Здесь мы идем от общего к частному. Начинаем с широкой области («транспортная логистика», «автоматизация навигации»), затем сужаем ее до конкретной проблемы («построение оптимального маршрута с учетом ограничений»), и, наконец, связываем это с методами решения («задачи теории графов»). Это показывает, что ваша работа вписана в более широкий научный и практический контекст. Ключевое слово *«актуальность»* выделено курсивом для акцента.

Целью данной курсовой работы является разработка программы на математическом языке Matlab для поиска и визуализации оптимального маршрута движения объекта с ограниченным запасом хода между двумя заданными точками на местности с набором пунктов дозаправки.

% Комментарий к цели работы:
% Цель — это одно, максимум два предложения, которые четко и однозначно формулируют конечный результат вашей работы. Используется глагол в неопределенной форме («разработка», «исследование», «создание»). Формулировка цели должна точно соответствовать теме вашей курсовой работы.

Для достижения поставленной цели необходимо было решить следующие задачи:
\begin{enumerate}
    \item Проанализировать исходные данные: характеристики подвижного объекта, координаты начальной, конечной и промежуточных точек (пунктов дозаправки).
    \item Представить карту местности в виде неориентированного взвешенного графа, где вершины соответствуют точкам на местности, а ребра — возможным перемещениям между ними.
    \item Реализовать алгоритм построения графа с учетом ограничения на максимальное расстояние, проходимое объектом без дозаправки.
    \item Реализовать алгоритм поиска кратчайшего пути в графе. В соответствии с заданием (см. \cref{table:source_data}), следует использовать алгоритм Форда-Беллмана.
    \item Разработать функцию для формирования NMEA-подобных сообщений, описывающих движение по оптимальному маршруту.
    \item Создать модуль для графической визуализации исходного графа, всех возможных путей и найденного оптимального маршрута.
    \item Обеспечить сохранение результатов расчетов (длина пути, NMEA-сообщения) в текстовый файл.
\end{enumerate}

% \paragraph{Комментарий к задачам:}
% Задачи — это конкретные шаги, которые вы предприняли для достижения цели. Они должны быть представлены в виде нумерованного списка и отражать логику вашей работы и структуру основной части отчета. По сути, каждый пункт списка задач может стать основой для подраздела в основной главе. Формулировки задач также начинаются с глагола («проанализировать», «представить», «реализовать»).

Объектом исследования является процесс нахождения оптимального пути в дискретной среде с ограничениями.

Предметом исследования являются алгоритмы на графах, в частности алгоритм Форда-Беллмана, и методы их программной реализации в среде Matlab для решения прикладных навигационных задач.

% \paragraph{Комментарий к объекту и предмету:}
% Это формальный, но важный элемент введения.
% *   *Объект* — это более широкое явление или процесс, который вы изучаете. Это ответ на вопрос «что исследуется?».
% *   *Предмет* — это конкретная часть объекта, его свойства или методы его изучения, которые рассматриваются в вашей работе. Это ответ на вопрос «какие аспекты объекта исследуются?».

% Курсовая работа состоит из введения, основной части, заключения, списка использованных источников и приложений.
% Во введении обосновывается актуальность темы, ставятся цель и задачи исследования.
% Основная часть содержит постановку задачи, описание математической модели, описание алгоритмов и программной реализации.
% В заключении приводятся основные выводы по проделанной работе.
% В приложениях содержится листинг кода разработанных программных модулей.

% \paragraph{Комментарий к структуре работы:}
% Этот абзац кратко описывает, из каких частей состоит ваш отчет. Он служит своего рода «содержанием в прозе» и помогает проверяющему быстро сориентироваться в документе.



% --- ОСНОВНОЕ СОДЕРЖАНИЕ ---
\section*{Цель работы}
Исследовать эффективность работы фотоэлектрической панели в зависимости от угла падения светового потока, построить семейство вольт-амперных характеристик (ВАХ) и определить зависимость выходной мощности и коэффициента заполнения от угла наклона.

\section*{Схема эксперимента}
Схема подключения фотоэлектрического модуля (ФЭМ) аналогична использованной в работе №1 и включает в себя панель, нагрузочный потенциометр, амперметр и вольтметр. Измерения проводились при фиксированной мощности источника света 50 Вт для различных углов наклона панели относительно горизонта ($0^\circ \dots 90^\circ$).

\section*{Протокол измерений}

В таблицах \ref{tab:pmax}, \ref{tab:sc} и \ref{tab:oc} представлены данные измерений для режимов максимальной мощности, короткого замыкания и холостого хода соответственно.

% --- РЕЖИМ МАКСИМАЛЬНОЙ МОЩНОСТИ (Pmax) ---
\begin{table}[H]
    \caption{Режим максимальной мощности ($P_{max}$), 50 Вт}
    \label{tab:pmax}
    \centering
    
    % Моно Pmax
    \begin{subtable}[t]{0.48\textwidth}
        \centering
        \caption{Монокристаллическая панель}
        \begin{tabular}{|c|c|c|c|}
            \hline
            $\alpha, ^\circ$ & $I, \text{мА}$ & $U, \text{В}$ & $P, \text{мВт}$ \\
            \hline
            90 & 27.4 & 12.14 & 332.64 \\
            75 & 27.1 & 12.02 & 325.74 \\
            60 & 25.5 & 11.31 & 288.41 \\
            45 & 22.4 & 9.98  & 223.55 \\
            30 & 17.3 & 7.74  & 133.90 \\
            15 & 12.3 & 5.53  & 68.02 \\
            0  & 9.3  & 4.12  & 38.32 \\
            \hline
        \end{tabular}
    \end{subtable}
    \hfill
    % Поли Pmax
    \begin{subtable}[t]{0.48\textwidth}
        \centering
        \caption{Поликристаллическая панель}
        \begin{tabular}{|c|c|c|c|}
            \hline
            $\alpha, ^\circ$ & $I, \text{мА}$ & $U, \text{В}$ & $P, \text{мВт}$ \\
            \hline
            90 & 42.7 & 14.93 & 637.51 \\
            75 & 38.7 & 13.53 & 523.61 \\
            60 & 29.8 & 10.41 & 310.22 \\
            45 & 21.7 & 7.58  & 164.49 \\
            30 & 14.9 & 5.22  & 77.78 \\
            15 & 10.2 & 3.55  & 36.21 \\
            0  & 7.1  & 2.66  & 18.89 \\
            \hline
        \end{tabular}
    \end{subtable}
\end{table}

% --- РЕЖИМ КОРОТКОГО ЗАМЫКАНИЯ (КЗ) ---
\begin{table}[H]
    \caption{Режим короткого замыкания (КЗ), 50 Вт}
    \label{tab:sc}
    \centering
    
    % Моно КЗ
    \begin{subtable}[t]{0.48\textwidth}
        \centering
        \caption{Монокристаллическая панель}
        \begin{tabular}{|c|c|c|c|}
            \hline
            $\alpha, ^\circ$ & $I, \text{мА}$ & $U, \text{В}$ & $P, \text{мВт}$ \\
            \hline
            90 & 42.8 & 0 & 0 \\
            75 & 42.0 & 0 & 0 \\
            60 & 35.9 & 0 & 0 \\
            45 & 27.1 & 0 & 0 \\
            30 & 19.1 & 0 & 0 \\
            15 & 14.0 & 0 & 0 \\
            0  & 9.6  & 0 & 0 \\
            \hline
        \end{tabular}
    \end{subtable}
    \hfill
    % Поли КЗ
    \begin{subtable}[t]{0.48\textwidth}
        \centering
        \caption{Поликристаллическая панель}
        \begin{tabular}{|c|c|c|c|}
            \hline
            $\alpha, ^\circ$ & $I, \text{мА}$ & $U, \text{В}$ & $P, \text{мВт}$ \\
            \hline
            90 & 45.4 & 0 & 0 \\
            75 & 42.0 & 0 & 0 \\
            60 & 39.3 & 0 & 0 \\
            45 & 23.6 & 0 & 0 \\
            30 & 15.6 & 0 & 0 \\
            15 & 11.0 & 0 & 0 \\
            0  & 7.5  & 0 & 0 \\
            \hline
        \end{tabular}
    \end{subtable}
\end{table}

% --- РЕЖИМ ХОЛОСТОГО ХОДА (ХХ) ---
\begin{table}[H]
    \caption{Режим холостого хода (ХХ), 50 Вт}
    \label{tab:oc}
    \centering
    
    % Моно ХХ
    \begin{subtable}[t]{0.48\textwidth}
        \centering
        \caption{Монокристаллическая панель}
        \begin{tabular}{|c|c|c|c|}
            \hline
            $\alpha, ^\circ$ & $I, \text{мА}$ & $U, \text{В}$ & $P, \text{мВт}$ \\
            \hline
            90 & 0 & 16.05 & 0 \\
            75 & 0 & 15.75 & 0 \\
            60 & 0 & 15.50 & 0 \\
            45 & 0 & 15.26 & 0 \\
            30 & 0 & 14.95 & 0 \\
            15 & 0 & 14.60 & 0 \\
            0  & 0 & 14.25 & 0 \\
            \hline
        \end{tabular}
    \end{subtable}
    \hfill
    % Поли ХХ
    \begin{subtable}[t]{0.48\textwidth}
        \centering
        \caption{Поликристаллическая панель}
        \begin{tabular}{|c|c|c|c|}
            \hline
            $\alpha, ^\circ$ & $I, \text{мА}$ & $U, \text{В}$ & $P, \text{мВт}$ \\
            \hline
            90 & 0 & 19.54 & 0 \\
            75 & 0 & 19.18 & 0 \\
            60 & 0 & 18.69 & 0 \\
            45 & 0 & 18.19 & 0 \\
            30 & 0 & 17.50 & 0 \\
            15 & 0 & 16.72 & 0 \\
            0  & 0 & 15.93 & 0 \\
            \hline
        \end{tabular}
    \end{subtable}
\end{table}

\section*{Графические зависимости}

\subsection*{Семейство ВАХ для различных углов наклона}
На \cref{fig:iv_family} представлено семейство вольт-амперных характеристик, построенных по трем характеристическим точкам (КЗ, ММ, ХХ) для каждого угла.

\begin{figure}[hab]
    \centering
    \includegraphics[width=\linewidth]{code/results/iv_family.png}
    \caption{Семейство ВАХ для моно- и поликристаллических панелей}
    \label{fig:iv_family}
\end{figure}

\subsection*{Зависимость мощности от угла наклона}
График зависимости $P = f(\alpha)$ представлен на \cref{fig:power_angle}.

\begin{figure}[hab]
    \centering
    \includegraphics[width=0.8\linewidth]{code/results/power_vs_angle.png}
    \caption{Зависимость максимальной мощности от угла наклона панели}
    \label{fig:power_angle}
\end{figure}

\subsection*{Зависимость коэффициента заполнения от угла наклона}
График зависимости $K = f(\alpha)$ представлен на \cref{fig:fill_factor}, где
$$
K(\alpha) = \frac{P_\alpha}{U_{\text{ХХ},\alpha} \cdot I_{\text{КЗ},\alpha}}
$$

\begin{figure}[hab]
    \centering
    \includegraphics[width=0.8\linewidth]{code/results/fill_factor.png}
    \caption{Изменение коэффициента заполнения ВАХ при изменении угла}
    \label{fig:fill_factor}
\end{figure}

\section*{Выводы}

В ходе работы исследовано влияние угла падения света на энергетические характеристики солнечных панелей.
\begin{itemize}
    \item \textit{Максимальная мощность:} Для обоих типов панелей максимальная мощность достигается при угле наклона $90^\circ$. При уменьшении угла до $0^\circ$ мощность падает нелинейно. Поликристаллическая панель показала более высокую пиковую мощность ($P_{max} \approx 637$ мВт) по сравнению с монокристаллической ($P_{max} \approx 332$ мВт).
    
    \item \textit{Коэффициент заполнения ($K$):} Расчет показал, что коэффициент заполнения поликристаллической панели ($K \approx 0.72$) выше, чем у монокристаллической ($K \approx 0.48$). Хотя теоретически монокристаллические модули обладают более высоким $K$, полученный результат может быть связан с износом конкретного образца монокристаллической панели, используемого в лабораторной установке.

    \item \textit{Влияние типа панели:} Поликристаллическая панель в данном эксперименте продемонстрировала большую чувствительность к углу наклона: падение мощности при переходе от $90^\circ$ к $60^\circ$ у неё более выражено, чем у монокристаллической.}
\end{itemize}
% --- ОСНОВНОЕ СОДЕРЖАНИЕ ---



% --- ЗАКЛЮЧЕНИЕ ---
% \newpage
% \addcontentsline{toc}{section}{Заключение}
% \phantomsection
% \centeredsection{\uppercase{Заключение}}
% В ходе выполнения данной курсовой работы была успешно решена задача разработки и реализации программного обеспечения на языке Matlab для поиска и визуализации оптимального маршрута движения объекта с ограниченным запасом хода.

Для решения поставленной задачи была разработана математическая модель, представляющая территорию в виде неориентированного взвешенного графа. На основе этой модели был реализован программный код, включающий модули для построения графа с учетом ограничений, поиска кратчайшего пути с помощью алгоритма Форда-Беллмана, генерации навигационных сообщений и графического отображения результатов.

В результате проделанной работы были получены следующие основные результаты:
\begin{itemize}
    \item Разработана \textit{программа} в среде Matlab, позволяющая автоматизировать процесс поиска оптимального маршрута.
    \item Сформирована \textit{математическая модель} задачи на основе теории графов, формализующая условия и ограничения.
    \item Реализован программный модуль для поиска кратчайшего пути на графе с использованием алгоритма Форда-Беллмана.
    \item Разработан алгоритм и реализующая его функция для формирования NMEA-подобных сообщений, описывающих движение по найденному маршруту.
    \item Создан модуль визуализации, который наглядно представляет построенный граф, все возможные пути и итоговый оптимальный маршрут, что упрощает анализ результатов.
    \item Проведено тестирование программы на конкретном примере, подтвердившее корректность работы реализованных алгоритмов и всей программы в целом.
\end{itemize}

В результате, все поставленные в работе задачи были выполнены в полном объеме, а основная цель курсовой работы — достигнута. Разработанный программный продукт является законченным решением, готовым к использованию для решения аналогичных навигационно-логистических задач.




% --- СПИСОК ИСПОЛЬЗОВАННЫХ ИСТОЧНИКОВ ---
% \newpage
% \addcontentsline{toc}{section}{Список использованных источников}
% \phantomsection
% \centeredsection{\uppercase{Список использованных источников}}
% \printbibliography


% --- ПРИЛОЖЕНИЯ ---
% \begin{appendices}
%     \section{Блок-схема алгоритма Форда-Беллмана}
\label{app:block_diagram_FB}
\begin{figure}[H]
    \centering
    \includegraphics[keepaspectratio, height=\freeht, width=\linewidth]{block_diagram/FB.drawio.pdf}
    % \caption{caption}
    % \label{fig:image}
\end{figure}

\section{Блок-схема buildGraph}
\label{app:block_diagram_BG}
\begin{figure}[H]
    \centering
    \includegraphics[keepaspectratio, height=\freeht, width=\linewidth]{block_diagram/BG.drawio.pdf}
    % \caption{caption}
    % \label{fig:image}
\end{figure}

\section{main.m}
\label{app:main.m}

\begin{codemultipage}
    % \captionof{listing}{caption\label{lst:label}}
    \inputminted{matlab}{code/main.m}
\end{codemultipage}



\section{buildGraph.m}
\label{app:buildGraph.m}

\begin{codemultipage}
    % \captionof{listing}{caption\label{lst:label}}
    \inputminted{matlab}{code/functions/buildGraph.m}
\end{codemultipage}



\section{findOptimalPathByFordBellman.m}
\label{app:findOptimalPathByFordBellman.m}

\begin{codemultipage}
    % \captionof{listing}{caption\label{lst:label}}
    \inputminted{matlab}{code/functions/findOptimalPathByFordBellman.m}
\end{codemultipage}



\section{generateNmeaMessages.m}
\label{app:generateNmeaMessages.m}

\begin{codemultipage}
    % \captionof{listing}{caption\label{lst:label}}
    \inputminted{matlab}{code/functions/generateNmeaMessages.m}
\end{codemultipage}



\section{plotRoute.m}
\label{app:plotRoute.m}

\begin{codemultipage}
    % \captionof{listing}{caption\label{lst:label}}
    \inputminted{matlab}{code/functions/plotRoute.m}
\end{codemultipage}



\section{saveResults.m}
\label{app:saveResults.m}

\begin{codemultipage}
    % \captionof{listing}{caption\label{lst:label}}
    \inputminted{matlab}{code/functions/saveResults.m}
\end{codemultipage}



\section{refueling\_points.txt}
\label{app:refueling_points.txt}

\begin{codemultipage}
    % \captionof{listing}{caption\label{lst:label}}
    \inputminted{matlab}{code/data/refueling_points.txt}
\end{codemultipage}


\section{graph\_distances.csv}
\label{app:graph_distances.csv}

\begin{codemultipage}
    % \captionof{listing}{caption\label{lst:label}}
    \inputminted{text}{code/results/graph_distances.csv}
\end{codemultipage}
% \end{appendices}



\end{document}