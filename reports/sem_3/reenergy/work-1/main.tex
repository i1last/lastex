\section*{Цель работы}
Построить вольт-амперные характеристики (ВАХ) солнечных моно- и поликристаллических панелей в зависимости от интенсивности светового потока, рассчитать мощность и определить коэффициент заполнения ВАХ.

\section*{Схема эксперимента}
Схема подключения фотоэлектрического модуля (ФЭМ) для снятия характеристик представлена на \cref{fig:circuit}.

\begin{figure}[hb]
    \centering
    \includegraphics[width=0.5\textwidth]{figs/scheme.png}
    \caption{Схема подключения для снятия ВАХ солнечной панели}
    \label{fig:circuit}
\end{figure}


\section*{Протокол измерений и расчет мощности}

В \cref{table:protocol} представлены измеренные значения тока ($I$) и напряжения ($U$), а также рассчитанная мощность ($P$) по формуле $ P = I \cdot U $ для монокристаллической солнечной панели \cref{table:mono} и поликристаллической --- \cref{table:mono}.

\begin{table}[H]
    \centering
    \caption{Протокол наблюдений}
    \label{table:protocol}

    \begin{subtable}[a]{0.48\textwidth}
        \centering
        \caption{Монокристаллическая, 50 Вт}
        \label{table:mono}
        \begin{tblr}{
            colspec = {c c c c},
            hlines, vlines,
            cells = {halign=center},
            row{1} = {font=\bfseries}
        }
            № опыта & $\bm I$, мА & $\bm U$, В & $\bm P$, мВт \\
            1 (XX) & 0 & 15.8 & 0 \\
            2 & 6.14 & 13.64 & 83.75 \\
            3 & 8.1 & 13.48 & 109.19 \\
            4 & 12.3 & 13.16 & 161.87 \\
            5 & 20.2 & 13.3 & 268.66 \\
            6 & 29.01 & 10.4 & 301.70 \\
            7 & 31.11 & 8.15 & 253.55 \\
            8 & 33.24 & 5.4 & 179.50 \\
            9 & 33.6 & 3.94 & 132.38 \\
            10 & 33.82 & 3.47 & 117.36 \\
            11 & 33.91 & 3.02 & 102.41 \\
            12 (КЗ) & 35.7 & 0 & 0 \\
        \end{tblr}
    \end{subtable}
    \hfill
    \begin{subtable}[a]{0.48\textwidth}
        \centering
        \caption{Поликристаллическая, 50 Вт}
        \label{table:poly}
        \begin{tblr}{
            colspec = {c c c c},
            hlines, vlines,
            cells = {halign=center},
            row{1} = {font=\bfseries}
        }
            № опыта & $\bm I$, мА & $\bm U$, В & $\bm P$, мВт \\
            1 (XX) & 0 & 19.36 & 0 \\
            2 & 9.21 & 18.92 & 174.25 \\
            3 & 11.26 & 18.86 & 212.36 \\
            4 & 15.02 & 18.74 & 281.47 \\
            5 & 17.48 & 18.62 & 325.48 \\
            6 & 32.5 & 17.46 & 567.45 \\
            7 & 40.1 & 14.1 & 565.41 \\
            8 & 41.3 & 12.15 & 501.80 \\
            9 & 43.3 & 8.11 & 351.16 \\
            10 & 44.4 & 4.2 & 186.48 \\
            11 & 44.5 & 2.95 & 131.28 \\
            12 (КЗ) & 49 & 0 & 0 \\
        \end{tblr}
    \end{subtable}
\end{table}


\section*{Графические зависимости}

\subsection*{Вольт-амперные характеристики (ВАХ)}
Зависимость тока от напряжения $I = f(U)$ см. на \cref{fig:iv_curve}.

\begin{figure}[hab]
    \centering
    % Подключаем график ВАХ
    \includegraphics[width=\linewidth]{code/results/iv_curves.png}
    \caption{ВАХ солнечных панелей}
    \label{fig:iv_curve}
\end{figure}

\subsection*{Мощностные характеристики}
Зависимость мощности от тока нагрузки $P = f(I)$ см. на \cref{fig:pi_curve}.

\begin{figure}[hab]
    \centering
    % Подключаем график мощности
    \includegraphics[width=\linewidth]{code/results/pi_curves.png}
    \caption{Зависимость мощности от тока нагрузки}
    \label{fig:pi_curve}
\end{figure}

\section*{Расчет коэффициента заполнения}

Коэффициент заполнения ВАХ ($K$) рассчитывается по формуле:
$$ K = \frac{P_{max}}{U_\text{хх} \cdot I_\text{кз}} $$

\subsection*{Для монокристаллической панели:}
\begin{itemize}
    \item $P_{max} \approx 301.70$ мВт
    \item $U_\text{хх} = 15.8$ В
    \item $I_\text{кз} = 35.7$ мА
\end{itemize}
$$ K_\text{моно} = \frac{301.70}{15.8 \cdot 35.7} = \frac{301.70}{564.06} \approx {0.53} $$

\subsection*{Для поликристаллической панели:}
\begin{itemize}
    \item $P_{max} \approx 567.45$ мВт
    \item $U_\text{хх} = 19.36$ В
    \item $I_\text{кз} = 49.0$ мА
\end{itemize}
$$ K_\text{поли} = \frac{567.45}{19.36 \cdot 49.0} = \frac{567.45}{948.64} \approx {0.60} $$

\section*{Выводы}
В ходе работы были построены ВАХ и мощностные характеристики для двух типов панелей при мощности источника света 50 Вт.

Поликристаллическая панель в данных условиях показала большую максимальную мощность ($P_{max} \approx 567$ мВт) по сравнению с монокристаллической ($P_{max} \approx 302$ мВт).

Коэффициент заполнения ВАХ у поликристаллической панели выше ($0.60 > 0.53$), что свидетельствует о более эффективном режиме работы в рамках данного эксперимента.

На графиках отчетливо видна точка максимальной мощности, соответствующая перегибу вольт-амперной характеристики.
