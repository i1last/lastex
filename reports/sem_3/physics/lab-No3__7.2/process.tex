


\clearpage
\centeredsection{\MakeUppercase{Обработка результатов измерений}}

\subsection*{ \boxed{\text{ Задание 1. }} }
\begin{quote}
    \textit{Используя результаты измерений из \cref{tab:protocol_V-A} построить на миллиметровой бумаге графики зависимостей фототока ($I_\text{ф}$) от напряжения ($U$): $I_\text{ф} = f(U)_{\Phi=\text{const}, \lambda=\text{const}}$ для трех значений длин волн ($\lambda_1$, $\lambda_2$, $\lambda_3$) (вольтамперные характеристики фоторезистора)}
\end{quote}

Фототок $I_\text{ф}$ для каждого измерения вычисляется как разность между полным током $I$, зарегистрированным амперметром при освещении, и темновым током $I_\text{т}$ при том же напряжении $U$:
$$I_\text{ф} = I - I_\text{т}$$
Результаты расчетов сведены в таблицу \ref{tab:results_V-A}.

\begin{table}[H]
    \centering
    \caption{Рассчитанные вольт-амперные характеристики фоторезистора ($I_\text{ф} = f(U)_{\Phi=\text{const}, \lambda=\text{const}}$), $J/J_0 = 1.1$, $[I_\forall] = [\text{мкА}]$.}
    \label{tab:results_V-A}
    \begin{tabularx}{\linewidth}{|c|c|*{12}{C|}}
        \hline
        \multicolumn{2}{|c|}{\multirow{2}{*}{Параметр}} & \multicolumn{12}{c|}{Напряжение $U$, В}                                                                                                                                                                                              \\
        \cline{3-14}
        \multicolumn{2}{|c|}{}                          & 0.5                                     & 1            & 1.5          & 2            & 2.5           & 3             & 3.5           & 4             & 4.5           & 5             & 5.5           & 6                             \\
        \hline \hline
        \multicolumn{2}{|c|}{$I_\text{т}$}              & 0.1                                     & 0.2          & 0.3          & 0.3          & 0.4           & 0.5           & 0.5           & 0.5           & 0.5           & 0.6           & 0.7           & 0.7                           \\
        \hline \hline
        \multirow{2}{*}{$\lambda_1 = 430$}              & $I$                                     & 0.3          & 0.7          & 1.0          & 1.3           & 1.5           & 1.9           & 2.2           & 2.6           & 2.8           & 3.2           & 3.5           & 3.8           \\
        \cline{2-14}
                                                        & $\mathbf{I_\text{ф}}$                   & \textbf{0.2} & \textbf{0.5} & \textbf{0.7} & \textbf{1.0}  & \textbf{1.1}  & \textbf{1.4}  & \textbf{1.7}  & \textbf{2.1}  & \textbf{2.3}  & \textbf{2.6}  & \textbf{2.8}  & \textbf{3.1}  \\
        \hline \hline
        \multirow{2}{*}{$\lambda_2 = 470$}              & $I$                                     & 0.5          & 1.0          & 1.5          & 2.0           & 2.5           & 3.0           & 3.5           & 4.0           & 4.5           & 5.0           & 5.6           & 6.1           \\
        \cline{2-14}
                                                        & $\mathbf{I_\text{ф}}$                   & \textbf{0.4} & \textbf{0.8} & \textbf{1.2} & \textbf{1.7}  & \textbf{2.1}  & \textbf{2.5}  & \textbf{3.0}  & \textbf{3.5}  & \textbf{4.0}  & \textbf{4.4}  & \textbf{4.9}  & \textbf{5.4}  \\
        \hline \hline
        \multirow{2}{*}{$\lambda_3 = 520$}              & $I$                                     & 3.4          & 7.2          & 9.8          & 13.5          & 16.6          & 19.9          & 23.1          & 26.5          & 29.6          & 32.9          & 36.3          & 39.4          \\
        \cline{2-14}
                                                        & $\mathbf{I_\text{ф}}$                   & \textbf{3.3} & \textbf{7.0} & \textbf{9.5} & \textbf{13.2} & \textbf{16.2} & \textbf{19.4} & \textbf{22.6} & \textbf{26.0} & \textbf{29.1} & \textbf{32.3} & \textbf{35.6} & \textbf{38.7} \\
        \hline
    \end{tabularx}
\end{table}

По данным из таблицы \ref{tab:results_V-A} построены графики зависимостей $I_\text{ф}(U)$: \cref{app:fig_V-A_lambda1}, \cref{app:fig_V-A_lambda2}, \cref{app:fig_V-A_lambda3}.





\subsection*{ \boxed{\text{ Задание 2. }} }
\begin{quote}
    \textit{Построить на миллиметровой бумаге световые кривые\\ $I_\text{ф} = f(\Phi)_{U=\text{const},~\lambda=\text{const}}$ по результатам измерений из \cref{tab:protocol_light}}
\end{quote}

Для расчета фототока в \cref{tab:results_light} необходимо определить темновой ток $I_\text{т}$ при напряжении $U = 6.00$ В. Из данных таблицы при относительном световом потоке $J/J_0 = 0$ полный ток $I = 0.6$ мкА. Это значение соответствует темновому току.
$$I_\text{т}(U=6.00 \text{ В}) = 0.6 \text{ мкА}$$
Далее фототок рассчитывается по той же формуле $I_\text{ф} = I - I_\text{т}$.

\begin{table}[H]
    \centering
    \caption{Рассчитанные световые характеристики фоторезистора ($I_\text{ф} = f(\Phi)_{U=\text{const}}$), $U = 6.00$ В, $I_\text{т} = 0.6$ мкА, $[I_\forall] = [\text{мкА}]$.}
    \label{tab:results_light}
    \begin{tabularx}{\linewidth}{|c|l|*{13}{C|}}
        \hline
        \multicolumn{2}{|c|}{\multirow{2}{*}{Параметр}} & \multicolumn{13}{c|}{Относительный световой поток $J/J_0$}                                                                                                                                                                                                    \\
        \cline{3-15}
        \multicolumn{2}{|c|}{}                          & 0                                                          & 0.1          & 0.2          & 0.3          & 0.4          & 0.5          & 0.6          & 0.7          & 0.8          & 0.9          & 1.0          & 1.1          & 1.2                         \\
        \hline \hline
        \multirow{2}{*}{$\lambda_1 = 430$}              & $I$                                                        & 0.6          & 0.7          & 0.8          & 0.9          & 1.0          & 1.3          & 1.7          & 2.0          & 2.4          & 2.9          & 3.3          & 3.8          & 4.6          \\
        \cline{2-15}
                                                        & $\mathbf{I_\text{ф}}$                                      & \textbf{0.0} & \textbf{0.1} & \textbf{0.2} & \textbf{0.3} & \textbf{0.4} & \textbf{0.7} & \textbf{1.1} & \textbf{1.4} & \textbf{1.8} & \textbf{2.3} & \textbf{2.7} & \textbf{3.2} & \textbf{4.0} \\
        \hline \hline
        \multirow{2}{*}{$\lambda_2 = 470$}              & $I$                                                        & 0.6          & 0.8          & 1.0          & 1.3          & 1.7          & 2.2          & 2.7          & 3.3          & 4.0          & 4.7          & 5.3          & 6.1          & 6.8          \\
        \cline{2-15}
                                                        & $\mathbf{I_\text{ф}}$                                      & \textbf{0.0} & \textbf{0.2} & \textbf{0.4} & \textbf{0.7} & \textbf{1.1} & \textbf{1.6} & \textbf{2.1} & \textbf{2.7} & \textbf{3.4} & \textbf{4.1} & \textbf{4.7} & \textbf{5.5} & \textbf{6.2} \\
        \hline
    \end{tabularx}
\end{table}

По данным из \cref{tab:results_light} построены графики зависимостей $I_\text{ф}(J/J_0)$ для двух длин волн: \cref{app:ex2_lambda1} и \cref{app:ex2_lambda2}.



\subsection*{ \boxed{\text{ Задание 3. }} }
\begin{quote}
    \textit{По результатам измерений из \cref{tab:protocol_spectral} построить на миллиметровой бумаге спектральную характеристику фоторезистора $I_\text{ф} = f(\lambda)_{\Phi=\text{const}}$. Определить по спектральной характеристике край собственного поглощения $\lambda_\text{кр}$. По \cref{formula:bandgap} оценить ширину $\Delta \varepsilon = hc / \lambda_\text{кр}$ запрещенной зоны полупроводника, из которого изготовлен фоторезистор. Пользуясь \cref{table:bandgaps}, определить материал полупроводника.}
\end{quote}

Для построения спектральной характеристики ($I_\text{ф} = f(\lambda)$) используем данные из \cref{tab:protocol_spectral}. Темновой ток при $U=6.00$ В принимаем равным $I_\text{т} = 0.6$ мкА (из предыдущего задания).

\begin{table}[H]
    \centering
    \caption{Рассчитанная спектральная характеристика фоторезистора ($I_\text{ф} = f(\lambda)_{\Phi=\text{const}}$), $J/J_0 = 1.000$, $U = 6.00$ В, $I_\text{т} = 0.6$ мкА}
    \label{tab:results_spectral}
    \begin{tabularx}{\linewidth}{|c|*{8}{C|}}
        \hline
        $\lambda$, нм                       & 430          & 470          & 520           & 565           & 590           & 660           & 700           & 860          \\
        \hline \hline
        $I$, мкА                            & 3.3          & 5.3          & 36.8          & 41.5          & 53.4          & 53.5          & 33.6          & 0.9          \\
        \hline
        $\mathbf{I_\text{ф}}$, \textbf{мкА} & \textbf{2.7} & \textbf{4.7} & \textbf{36.2} & \textbf{40.9} & \textbf{52.8} & \textbf{52.9} & \textbf{33.0} & \textbf{0.3} \\
        \hline
    \end{tabularx}
\end{table}

По данным таблицы \ref{tab:results_spectral} построен график спектральной характеристики $I_\text{ф}(\lambda)$: \cref{app:ex3_spectral}.

\subsubsection*{Определение красной границы фотоэффекта}
Красную границу фотоэффекта $\lambda_\text{кр}$ можно оценить, найдя длину волны, при которой фототок на длинноволновом спаде характеристики уменьшается до половины своего максимального значения.
\begin{enumerate}
    \item Максимальное значение фототока из \cref{tab:results_spectral}:
          $$I_{\text{ф, max}} = 52.9 \text{ мкА (при } \lambda = 660 \text{ нм)}$$
    \item Половина максимального значения:
          $$\frac{I_{\text{ф, max}}}{2} = \frac{52.9}{2} = 26.45 \text{ мкА}$$
    \item Это значение находится между точками, соответствующими $\lambda_1 = 700$ нм ($I_{\text{ф}1} = 33.0$ мкА) и $\lambda_2 = 860$ нм ($I_{\text{ф}2} = 0.3$ мкА). Применим формулу линейной интерполяции для нахождения $\lambda_\text{кр}$:
          $$\lambda_\text{кр} = \lambda_1 \cdot \frac{(I_{\text{ф, 1/2}} - I_{\text{ф}1})(\lambda_2 - \lambda_1)}{I_{\text{ф}2} - I_{\text{ф}1}}$$
          $$\lambda_\text{кр} = 700 \cdot \frac{(26.45 - 33.0)(860 - 700)}{0.3 - 33.0}$$
          $$\lambda_\text{кр} \approx 732 \text{ нм}$$
\end{enumerate}

\subsubsection*{Расчет ширины запрещенной зоны}
Ширина запрещенной зоны $\Delta\varepsilon$ связана с красной границей фотоэффекта $\lambda_\text{кр}$ соотношением:
$$\Delta\varepsilon = \frac{hc}{\lambda_\text{кр}}$$
где $h = 6.626 \cdot 10^{-34}$ Дж$\cdot$с — постоянная Планка, $c = 3 \cdot 10^8$ м/с — скорость света в вакууме.

Выполним расчет:
$$
    \begin{aligned}
        \Delta\varepsilon & = \frac{(6.626 \cdot 10^{-34} \text{ Дж}\cdot\text{с}) \cdot (3 \cdot 10^8 \text{ м/с})}{732 \cdot 10^{-9} \text{ м}} \\
                          & \approx \frac{19.878 \cdot 10^{-26}}{732 \cdot 10^{-9}} \text{ Дж} \approx 2.715 \cdot 10^{-20} \text{ Дж}
    \end{aligned}
$$
Переведем энергию в электрон-вольты (эВ), зная, что $1$ эВ $= 1.602 \cdot 10^{-19}$ Дж:
$$\Delta\varepsilon = \frac{2.715 \cdot 10^{-20} \text{ Дж}}{1.602 \cdot 10^{-19} \text{ Дж/эВ}} \approx 1.695 \text{ эВ}$$
Округляя, получаем:
$$\boxed{\Delta\varepsilon \approx 1.70 \text{ эВ}}$$

\subsubsection*{Определение материала полупроводника}
Сравним полученное значение ширины запрещенной зоны $\Delta\varepsilon \approx 1.70$ эВ с табличными данными для различных полупроводников (\cref{table:bandgaps}). Наиболее близкое значение соответствует \textit{селениду кадмия (CdSe)}, для которого $\Delta\varepsilon = 1.70$ эВ.



\section*{Выводы}
В ходе лабораторной работы были исследованы основные характеристики фоторезистора: вольт-амперная, световая и спектральная. Установлено, что вольт-амперные характеристики фоторезистора при постоянной интенсивности и длине волны излучения являются линейными, что соответствует теоретическим представлениям о внутреннем фотоэффекте. Проводимость фоторезистора, определяемая наклоном ВАХ, существенно зависит от длины волны падающего света.

Анализ световых характеристик показал, что зависимость фототока от интенсивности падающего излучения близка к линейной в исследованном диапазоне, с наблюдаемой тенденцией к насыщению при максимальных значениях светового потока.

Снятая спектральная характеристика продемонстрировала селективную чувствительность фоторезистора. Максимум фоточувствительности наблюдается в диапазоне длин волн $\lambda \approx 590-660$ нм, со значительным спадом чувствительности как в коротковолновой, так и в длинноволновой области спектра.

На основе анализа длинноволнового спада спектральной характеристики была определена красная граница внутреннего фотоэффекта для данного полупроводника, которая составила $\lambda_{\text{кр}} \approx 732$ нм. По значению красной границы была рассчитана ширина запрещенной зоны полупроводника:
$$\boxed{\Delta\varepsilon \approx 1.70 \text{ эВ}}$$

Сравнение полученного значения с табличными данными позволяет сделать вывод, что материалом исследуемого фоторезистора является \textit{селенид кадмия (CdSe)}, для которого справочное значение ширины запрещенной зоны составляет $\Delta\varepsilon = 1.70$ эВ. Таким образом, цели лабораторной работы были выполнены.