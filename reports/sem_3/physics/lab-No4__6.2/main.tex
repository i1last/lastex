\section*{Цель работы}
Изучение явления внешнего фотоэффекта; экспериментальное исследование вольт-амперной, световой и спектральной характеристик вакуумного фотоэлемента.

\section*{Приборы и принадлежности}
Модульный учебный комплекс МУК-ОК, в состав которого входят стенд с объектами исследования С3-ОК01, амперметр-вольтметр АВ-1, источник питания ИПС1, соединительные проводники.

\section*{Исследуемые закономерности}
\textit{Внешний фотоэффект} — явление испускания электронов (фотоэлектронов) веществом под действием падающего светового излучения. Регистрируемый с помощью амперметра световой ток является суперпозицией «истинного» фототока $I_\text{ф}$ и темнового тока $I_\text{т}$.
$$ I = I_\text{ф} + I_\text{т} $$

В данной лабораторной работе исследуются следующие основные характеристики фотоэлемента:
\begin{enumerate}
    \item \textit{Вольт-амперная характеристика} — зависимость силы фототока от напряжения между катодом и анодом для выбранных значений падающего светового потока $\Phi$ с определенной длиной волны $\lambda$.
          $$ I_\text{ф} = f(U)|_{\Phi=\text{const}, \lambda=\text{const}} $$
          С увеличением напряжения фототок возрастает и при некотором напряжении $U_A$ достигает насыщения $I_{\text{ф.н}}$, когда все выбитые электроны достигают анода.

    \item \textit{Световая характеристика} — зависимость фототока насыщения $I_{\text{ф.н}}$ от светового потока $\Phi$ при неизменном его спектральном составе.
          $$ I_{\text{ф.н}} = f(\Phi)|_{U=\text{const}, \lambda=\text{const}} $$
          Эта характеристика является линейной (при условии отсутствия объемного заряда).

    \item \textit{Спектральная характеристика} — зависимость фототока насыщения $I_{\text{ф.н}}$ от длины волны $\lambda$ падающего света при неизменной величине светового потока.
          $$ I_{\text{ф.н}} = f(\lambda)|_{\Phi=\text{const}, U=\text{const}} $$
\end{enumerate}

\begin{figure}[H]
    \centering
    \begin{minipage}{0.48\linewidth}
        \centering
        \def\thefigure{6.6}
        \protect\phantomsection
        \includegraphics[width=0.9\linewidth]{figs/6-6.png}
        \caption{Вольт-амперные характеристики фотоэлемента}
        \label{fig:va_char}
    \end{minipage}\hfill
    \begin{minipage}{0.48\linewidth}
        \centering
        \def\thefigure{6.7}
        \protect\phantomsection
        \includegraphics[width=0.9\linewidth]{figs/6-7.png}
        \caption{Световая характеристика вакуумного фотоэлемента}
        \label{fig:light_char}
    \end{minipage}
\end{figure}

\newpage
\centeredsection{\MakeUppercase{Указания по проведению эксперимента}}

\begin{figure}[H]
    \centering
    \begin{minipage}{0.48\linewidth}
        \centering
        \def\thefigure{6.4}
        \protect\phantomsection
        \includegraphics[width=0.9\linewidth]{figs/6-4.png}
        \caption{Электрическая схема экспериментальной установки}
        \label{fig:circuit}
    \end{minipage}\hfill
    \begin{minipage}{0.48\linewidth}
        \centering
        \def\thefigure{6.5}
        \protect\phantomsection
        \includegraphics[width=0.9\linewidth]{figs/6-5.png}
        \caption{Модульный учебный комплекс МУК-ОК}
        \label{fig:setup}
    \end{minipage}
\end{figure}

\begin{enumerate}
    \item \textbf{Собрать электрическую схему} с помощью соединительных проводников на стенде С3-ОК01 (см. \cref{fig:circuit}).

    \item \textbf{Настроить приборы.} На блоке АВ-1 выбрать режим измерения вольтметра ($0...20$ В) и амперметра (– $20$ мкА).

    \item \textbf{Снять темновую характеристику} $I_\text{т} = f(U)$ при относительном световом потоке $\Phi \sim J/J_0 = 0.1$ и длине волны $\lambda_4$. Измерения проводить в диапазоне напряжений $U = 0...20$ В с шагом $\Delta U = 1$ В. Результаты занести в \cref{tab:protocol_V-A}.

    \item \textbf{Снять семейство вольт-амперных характеристик} $I=f(U)$ для трех значений светового потока $\Phi$ (например, $(J/J_0)_1=0.3$, $(J/J_0)_2=0.7$, $(J/J_0)_3=1.1$) при длине волны $\lambda_4$. Результаты занести в \cref{tab:protocol_V-A}.

    \item \textbf{Снять спектральную характеристику.} Измерить световой ток $I$ для всех длин волн $\lambda_0 ... \lambda_7$ при фиксированном напряжении насыщения $U_\text{н}$ (например, $20$ В) и постоянном световом потоке (например, $J/J_0=1.1$). Для каждой длины волны измерить темновой ток $I_\text{т}$ (при $J/J_0=0.1$, $U_\text{н}=20$ В). Результаты занести в \cref{tab:protocol_spectral}.
\end{enumerate}

\newpage
\thispagestyle{empty}
\centeredsection{ПРОТОКОЛ НАБЛЮДЕНИЙ}

\begin{table}[H]
    \centering
    \caption{Вольт-амперные характеристики фотоэлемента ($I_\text{ф} = f(U)_{\Phi=\text{const}}$), $\lambda = \lambda_4$.}
    \label{tab:protocol_V-A}
    \begin{tabularx}{\linewidth}{|C||C||C|C||C|C||C|C|}
        \hline
        % Заголовок, занимающий 8 столбцов
        \multirow{2}{*}{$U$, В} & \multirow{2}{*}{$I_\text{т}$} & \multicolumn{2}{c||}{$(J/J_0)_1 = \underline{0.3}$} & \multicolumn{2}{c||}{$(J/J_0)_2 = \underline{0.7}$} & \multicolumn{2}{c||}{$(J/J_0)_3 = \underline{1.1}$}                                      \\
        \cline{3-8}
        % Вторая строка заголовка для 6 столбцов (первые два заняты \multirow)
                                &                               & $I$                                                 & $I_\text{ф}$                                        & $I$                                                 & $I_\text{ф}$ & $I$  & $I_\text{ф}$ \\
        \hline \hline
        % Строки данных, каждая содержит 8 ячеек (7 символов '&')
        0                       & 0.05                          & 0.12                                                & 0.07                                                & 0.20                                                & 0.15         & 0.26 & 0.21         \\ \hline
        1                       & 0.09                          & 0.24                                                & 0.15                                                & 0.36                                                & 0.27         & 0.53 & 0.44         \\ \hline
        2                       & 0.23                          & 0.39                                                & 0.16                                                & 0.47                                                & 0.24         & 0.69 & 0.46         \\ \hline
        3                       & 0.35                          & 0.40                                                & 0.05                                                & 0.58                                                & 0.23         & 0.75 & 0.40         \\ \hline
        4                       & 0.45                          & 0.47                                                & 0.02                                                & 0.64                                                & 0.19         & 0.85 & 0.40         \\ \hline
        5                       & 0.55                          & 0.52                                                & -0.03                                               & 0.71                                                & 0.16         & 0.94 & 0.39         \\ \hline
        6                       & 0.70                          & 0.67                                                & -0.03                                               & 0.79                                                & 0.09         & 1.07 & 0.37         \\ \hline
        7                       & 0.77                          & 0.75                                                & -0.02                                               & 0.89                                                & 0.12         & 1.15 & 0.38         \\ \hline
        8                       & 0.85                          & 0.85                                                & 0.00                                                & 0.99                                                & 0.14         & 1.24 & 0.39         \\ \hline
        9                       & 0.98                          & 0.98                                                & 0.00                                                & 1.05                                                & 0.07         & 1.31 & 0.33         \\ \hline
        10                      & 1.07                          & 1.17                                                & 0.10                                                & 1.08                                                & 0.01         & 1.36 & 0.29         \\ \hline
        11                      & 1.23                          & 1.23                                                & 0.00                                                & 1.13                                                & -0.10        & 1.48 & 0.25         \\ \hline
        12                      & 1.29                          & 1.26                                                & -0.03                                               & 1.18                                                & -0.11        & 1.55 & 0.26         \\ \hline
        13                      & 1.36                          & 1.35                                                & -0.01                                               & 1.25                                                & -0.11        & 1.61 & 0.25         \\ \hline
        14                      & 1.53                          & 1.42                                                & -0.11                                               & 1.35                                                & -0.18        & 1.69 & 0.16         \\ \hline
        15                      & 1.59                          & 1.56                                                & -0.03                                               & 1.43                                                & -0.16        & 1.75 & 0.16         \\ \hline
        16                      & 1.68                          & 1.63                                                & -0.05                                               & 1.51                                                & -0.17        & 1.82 & 0.14         \\ \hline
        17                      & 1.74                          & 1.75                                                & 0.01                                                & 1.61                                                & -0.13        & 1.91 & 0.17         \\ \hline
        18                      & 1.88                          & 1.85                                                & -0.03                                               & 1.69                                                & -0.19        & 1.98 & 0.10         \\ \hline
        19                      & 1.93                          & 1.93                                                & 0.00                                                & 1.78                                                & -0.15        & 2.06 & 0.13         \\ \hline
        20                      & 1.98                          & 1.99                                                & 0.01                                                & 1.84                                                & -0.14        & 2.12 & 0.14         \\ \hline
    \end{tabularx}
\end{table}

\vfill
\noindent
Рахметов А. Р., гр. 4494 ~~\hrulefill~~ «\rule{1cm}{0.4pt}» \rule{3cm}{0.4pt} 20\rule{0.75cm}{0.4pt} г.

\newpage
\thispagestyle{empty}
\centeredsection{ПРОДОЛЖЕНИЕ ПРОТОКОЛА НАБЛЮДЕНИЙ}
\begin{table}[H]
    \centering
    \caption{Спектральная характеристика фотоэлемента ($I_\text{ф.н} = f(\lambda)_{\Phi=\text{const},~U=\text{const}}$), $J/J_0 = \underline{1.1}$, $U = \underline{20}$ В.}
    \label{tab:protocol_spectral}
    \begin{tabularx}{\linewidth}{|c|*{8}{C|}}
        \hline
        $\{\lambda_i\}_{i=0}^7$    & 430  & 470  & 520  & 565  & 590  & 660  & 700  & 860  \\
        \hline \hline
        $I$          & 4.41 & 5.13 & 3.82 & 2.60 & 2.02 & 1.67 & 1.48 & 1.44 \\
        \hline
        $I_\text{т}$ & 1.55 & 1.62 & 1.74 & 1.54 & 1.46 & 1.41 & 1.42 & 1.39 \\
        \hline
        $I_\text{ф}$ & 2.86 & 3.51 & 2.08 & 1.06 & 0.56 & 0.26 & 0.06 & 0.05 \\
        \hline
    \end{tabularx}
\end{table}

\textit{Примечание: } $[\lambda_\forall] = [\text{нм}]$; \hfill $[I_\forall] = [\text{мкА}]$; \hfill $I_\text{ф} = I - I_\text{т}$

\vfill
\noindent
Рахметов А. Р., гр. 4494 ~~\hrulefill~~ «\rule{1cm}{0.4pt}» \rule{3cm}{0.4pt} 20\rule{0.75cm}{0.4pt} г.