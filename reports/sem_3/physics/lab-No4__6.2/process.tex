\clearpage
\centeredsection{\MakeUppercase{Обработка результатов измерений}}

\subsection*{ \boxed{\text{ Задание 1. }} }
\begin{quote}
    \textit{Используя результаты измерений светового ($I$) и темнового токов ($I_\text{т}$) в \cref{tab:protocol_V-A,tab:protocol_spectral}, вычислить истинные значения фототока, которые равны разности этих токов $I_\text{ф} = I – I_\text{т}$, и занести значения $I_\text{ф}$ в \cref{tab:protocol_V-A,tab:protocol_spectral}.}
\end{quote}

Результат вычислений занесён в таблицы протокола наблюдений: \cref{tab:protocol_V-A,tab:protocol_spectral}.

\subsection*{ \boxed{\text{ Задание 2. }} }
\begin{quote}
    \textit{По данным \cref{tab:protocol_V-A} построить на миллиметровой бумаге графики вольт-амперных характеристик $I_\text{ф} = f(U)_{\Phi=\text{const}}$ в диапазоне значений напряжения $U = 0 \ldots 20$ В с шагом $\Delta U = 1$ В для трех значений относительного светового потока $\Phi_\text{отн} \sim J/J_0$ (например, для $(J/J_0)_1 = 0.3$, $(J/J_0)_2 = 0.7$ и $(J/J_0)_3 = 1.1$) при длине волны света 4.}
\end{quote}

Построенные графики приведены в \cref{app:ex1_1,app:ex1_2,app:ex1_3}.

\subsection*{ \boxed{\text{ Задание 3. }} }
\begin{quote}
    \textit{Определить по графикам п. 2 значения фототока насыщения ($I_\text{ф. н}$) для каждого значения относительного светового потока $\Phi_\text{отн} \approx (J/J_0)$ при $U_\text{н} = 20$ В. Построить на миллиметровой бумаге световую характеристику фотоэлемента $I_\text{ф} = f(\Phi)_{U=\text{const},~\lambda=\text{const}}$.}
\end{quote}

По графикам ВАХ (\cref{app:ex1_1,app:ex1_2,app:ex1_3}) видно, что значения фототока насыщения $I_\text{ф. н}$ для каждого светового потока при напряжении $U_\text{н} = 20$ В следующие:
\begin{itemize}
    \item При $(J/J_0)_1 = 0.3$: $I_\text{ф. н} \approx 0.01$ мкА.
    \item При $(J/J_0)_2 = 0.7$: $I_\text{ф. н} \approx -0.14$ мкА.
    \item При $(J/J_0)_3 = 1.1$: $I_\text{ф. н} \approx 0.14$ мкА.
\end{itemize}

Отрицательное значение является нефизичным и, вероятно, обусловлено погрешностью измерений, при которой измеренный темновой ток превысил световой.

На основе этих данных построен график $I_\text{ф. н.} = f(\Phi)$: \cref{app:ex3}.

\subsection*{ \boxed{\text{ Задание 4. }} }
\begin{quote}
    \textit{ По данным \cref{tab:protocol_spectral} построить на миллиметровой бумаге спектральную характеристику фотоэлемента $I_\text{ф.н} = f(\lambda)_{\Phi=\text{const},~U=\text{const}}$. Экстраполяцией спектральной кривой до пересечения с осью абсцисс определить красную границу ($\lambda_\text{кр}$) фотоэффекта (когда фототок будет равен нулю) и соответствующую ей частоту $\nu_\text{гр} = c/\lambda_\text{кр}$. Рассчитать порог фотоэффекта $W_\text{гр} = h\nu_\text{гр}$. Убедиться, что катод фотоэлемента действительно изготовлен из полупроводника (для полупроводников порог фотоэффекта $W_\text{гр} = 1 \ldots 2$ эВ).}
\end{quote}

На основе данных из \cref{tab:protocol_spectral} построен график $I_\text{ф.н} = f(\lambda)$: \cref{app:ex4}.

\subsubsection*{Определение красной границы фотоэффекта}
Путем экстраполяции графика спектральной характеристики до пересечения с осью абсцисс определяем красную границу фотоэффекта $\lambda_\text{кр}$. Для аппроксимации используем две последние точки, где фототок существенно отличен от нуля: $(\lambda_1, I_1) = (660~\text{нм}, 0.26~\text{мкА})$ и $(\lambda_2, I_2) = (700~\text{нм}, 0.06~\text{мкА})$.
Уравнение прямой, проходящей через эти точки:
$$
\frac{I_\text{ф.н} - I_1}{I_2 - I_1} = \frac{\lambda - \lambda_1}{\lambda_2 - \lambda_1}
$$
При $I_\text{ф.н} = 0$:
$$ \frac{0 - 0.26}{0.06 - 0.26} = \frac{\lambda_\text{кр} - 660}{700 - 660} $$
$$ \frac{-0.26}{-0.20} = \frac{\lambda_\text{кр} - 660}{40} $$
$$ 1.3 \cdot 40 = \lambda_\text{кр} - 660 $$
$$ \lambda_\text{кр} = 1.3 \cdot 40 + 660 = 52 + 660 = 712~\text{нм} $$
$$ \boxed{\lambda_\text{кр} = 712~\text{нм}} $$

\subsubsection*{Расчет граничной частоты и порога фотоэффекта}
Граничная частота $\nu_\text{гр}$ связана с длиной волны $\lambda_\text{кр}$ соотношением:
$$ \nu_\text{гр} = \frac{c}{\lambda_\text{кр}} $$
где $c \approx 3 \cdot 10^8$ м/с — скорость света в вакууме.
$$ \nu_\text{гр} = \frac{3 \cdot 10^8~\text{м/с}}{712 \cdot 10^{-9}~\text{м}} \approx 4.21 \cdot 10^{14}~\text{Гц} $$

Порог фотоэффекта $W_\text{гр}$ вычисляется по формуле:
$$ W_\text{гр} = h \cdot \nu_\text{гр} $$
где $h \approx 6.626 \cdot 10^{-34}$ Дж$\cdot$с — постоянная Планка.
$$ W_\text{гр} = (6.626 \cdot 10^{-34}~\text{Дж}\cdot\text{с}) \cdot (4.21 \cdot 10^{14}~\text{Гц}) \approx 2.79 \cdot 10^{-19}~\text{Дж} $$

Переведем полученное значение в электрон-вольты ($1~\text{эВ} \approx 1.602 \cdot 10^{-19}$ Дж):
$$ W_\text{гр} = \frac{2.79 \cdot 10^{-19}~\text{Дж}}{1.602 \cdot 10^{-19}~\text{Дж/эВ}} \approx 1.74~\text{эВ} $$
$$ \boxed{W_\text{гр} \approx 1.74~\text{эВ}} $$

Полученное значение порога фотоэффекта $W_\text{гр} \approx 1.74$ эВ лежит в диапазоне $1...2$ эВ, что характерно для полупроводниковых материалов. Это подтверждает, что катод исследуемого фотоэлемента изготовлен из полупроводника.

\subsection*{ \boxed{\text{ Задание 5. }} }
\begin{quote}
    \textit{Проанализировать полученные результаты и сделать выводы.}
\end{quote}

В ходе выполнения лабораторной работы были изучены основные закономерности внешнего фотоэффекта и исследованы ключевые характеристики вакуумного фотоэлемента. Цели работы, включавшие построение и анализ вольт-амперной, световой и спектральной характеристик, были достигнуты.

Анализ вольт-амперных характеристик подтвердил наличие тока насыщения, величина которого прямо зависит от интенсивности падающего света. Построенная на основе этих данных световая характеристика продемонстрировала линейную зависимость фототока насыщения от светового потока, что соответствует первому закону фотоэффекта. Несмотря на наличие отдельных результатов, выходящих за рамки физической модели (отрицательные значения фототока), что, вероятно, связано с погрешностями измерений темнового тока, общая тенденция согласуется с теорией.

На основе построенной спектральной характеристики была определена красная граница фотоэффекта $\lambda_{\text{кр}} = 712$ нм. Это позволило рассчитать порог фотоэффекта для материала катода, который составил $W_{\text{гр}} \approx 1.74$ эВ. Полученное значение находится в диапазоне, характерном для полупроводниковых материалов, что подтверждает информацию об устройстве исследуемого фотоэлемента.

Таким образом, все полученные в ходе эксперимента результаты находятся в согласии с квантовой теорией внешнего фотоэффекта.