В ходе выполнения данной курсовой работы была успешно решена задача разработки и реализации программного обеспечения на языке Matlab для поиска и визуализации оптимального маршрута движения объекта с ограниченным запасом хода.

Для решения поставленной задачи была разработана математическая модель, представляющая территорию в виде неориентированного взвешенного графа. На основе этой модели был реализован программный код, включающий модули для построения графа с учетом ограничений, поиска кратчайшего пути с помощью алгоритма Форда-Беллмана, генерации навигационных сообщений и графического отображения результатов.

В результате проделанной работы были получены следующие основные результаты:
\begin{itemize}
    \item Разработана \textit{программа} в среде Matlab, позволяющая автоматизировать процесс поиска оптимального маршрута.
    \item Сформирована \textit{математическая модель} задачи на основе теории графов, формализующая условия и ограничения.
    \item Реализован программный модуль для поиска кратчайшего пути на графе с использованием алгоритма Форда-Беллмана.
    \item Разработан алгоритм и реализующая его функция для формирования NMEA-подобных сообщений, описывающих движение по найденному маршруту.
    \item Создан модуль визуализации, который наглядно представляет построенный граф, все возможные пути и итоговый оптимальный маршрут, что упрощает анализ результатов.
    \item Проведено тестирование программы на конкретном примере, подтвердившее корректность работы реализованных алгоритмов и всей программы в целом.
\end{itemize}

В результате, все поставленные в работе задачи были выполнены в полном объеме, а основная цель курсовой работы — достигнута. Разработанный программный продукт является законченным решением, готовым к использованию для решения аналогичных навигационно-логистических задач.
