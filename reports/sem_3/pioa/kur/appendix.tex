% ========================================================================================
% ФАЙЛ С ПРИЛОЖЕНИЯМИ
% ----------------------------------------------------------------------------------------
% ПРИМЕЧАНИЕ: Команды ,  и 
% расставляются вручную для полного контроля над версткой документа.
% Если после компиляции разрыв страницы окажется в другом месте,
% просто переместите эти команды.
% ========================================================================================



% --- ПРИЛОЖЕНИЕ А ---
\section{Задание на курсовую работу}
\label{app:task_sheet}

На \cref{fig:task_sheet} представлен скан-образ индивидуального задания на курсовую работу.

\begin{figure}[H]
    \centering
    % ВНИМАНИЕ: Убедитесь, что путь к файлу 'images/cond_kur.png' корректен.
    \includegraphics[width=0.9\textwidth]{images/cond_kur.png}
    \caption{Задание на курсовую работу (Вариант 1.9.ФБ)}
    \label{fig:task_sheet}
\end{figure}
% Данное приложение, скорее всего, уместится на одной странице,
% поэтому команды  и  не требуются.


% --- ПРИЛОЖЕНИЕ Б ---
 % Начинаем следующее приложение с новой страницы
\section{Результаты работы программы}
\label{app:results}

В данном приложении представлены результаты работы программы для тестового набора данных, разработанного для проверки всех сценариев.

\begin{listing}[H]
    % ВНИМАНИЕ: Убедитесь, что путь к файлу 'data/refueling_points.txt' корректен.
    \inputminted{text}{data/refueling_points.txt}
    \caption{Содержимое файла входных данных \texttt{refueling\_points.txt}}
    \label{lst:input_data}
\end{listing}

\begin{listing}[H]
    % ВНИМАНИЕ: Убедитесь, что путь к файлу 'results/result.txt' корректен.
    \inputminted{text}{results/result.txt}
    \caption{Содержимое файла результатов \texttt{result.txt}}
    \label{lst:output_data}
\end{listing}

% ПРЕДПОЛАГАЕМЫЙ РАЗРЫВ СТРАНИЦЫ: График, скорее всего, начнется на новой странице.



\begin{figure}[H]
    \centering
    % ВНИМАНИЕ: Убедитесь, что путь к файлу 'results/optimal_route.png' корректен.
    \includegraphics[width=\textwidth]{results/optimal_route.png}
    \caption{График, сгенерированный программой}
    \label{fig:result_plot}
\end{figure}

% Ставим "Окончание" в конце этого многостраничного приложения.



% --- ПРИЛОЖЕНИЕ В ---
 % Начинаем следующее приложение с новой страницы
\section{Листинги программ}
\label{app:listings}

\begin{listing}[H]
    % ВНИМАНИЕ: Убедитесь, что пути к файлам кода верны.
    \inputminted{matlab}{main.m}
    \caption{Листинг основного скрипта \texttt{main.m}}
    \label{lst:main_script}
\end{listing}

\begin{listing}[H]
    \inputminted{matlab}{functions/buildGraph.m}
    \caption{Листинг функции \texttt{buildGraph.m}}
    \label{lst:buildGraph}
\end{listing}

% ПРЕДПОЛАГАЕМЫЙ РАЗРЫВ СТРАНИЦЫ



\begin{listing}[H]
    \inputminted{matlab}{functions/findOptimalPathByFordBellman.m}
    \caption{Листинг функции \texttt{findOptimalPathByFordBellman.m}}
    \label{lst:fordBellman}
\end{listing}

\begin{listing}[H]
    \inputminted{matlab}{functions/generateNmeaMessages.m}
    \caption{Листинг функции \texttt{generateNmeaMessages.m}}
    \label{lst:generateNmea}
\end{listing}

% ПРЕДПОЛАГАЕМЫЙ РАЗРЫВ СТРАНИЦЫ



\begin{listing}[H]
    \inputminted{matlab}{functions/plotRoute.m}
    \caption{Листинг функции \texttt{plotRoute.m}}
    \label{lst:plotRoute}
\end{listing}

\begin{listing}[H]
    \inputminted{matlab}{functions/saveResults.m}
    \caption{Листинг функции \texttt{saveResults.m}}
    \label{lst:saveResults}
\end{listing}

% Ставим "Окончание" в конце этого многостраничного приложения.
