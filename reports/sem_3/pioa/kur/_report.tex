% === ПОДКЛЮЧЕНИЕ ОБЩЕЙ ПРЕАМБУЛЫ ===
%========================================================================================
% КЛАСС ДОКУМЕНТА И ОСНОВНЫЕ ПАРАМЕТРЫ
%========================================================================================
\documentclass[a4paper,14pt,russian]{extarticle}

% Расширяет возможности размеров стандартных классов
\usepackage{extsizes}

% Разрешаем ставить больее длинные пробелы, если иначе не выходит сделать
% более верное разбиение абзаца на строки
\sloppy





%========================================================================================
% ЯЗЫК, ШРИФТЫ И КОДИРОВКА (Современный подход для LuaLaTeX)
%========================================================================================
\usepackage{cmap}           % Обеспечивает правильное отображение шрифтов и символов в PDF
\usepackage{fontspec}       % Пакет для работы с любыми системными шрифтами.
                            % Заменяет устаревшие inputenc и fontenc.
\usepackage[russian]{babel} % Поддержка русского языка: переносы, названия и т.д.

% --- Настройка шрифтов и интервалов
% Явно определяем семейство шрифтов "Times New Roman", указывая путь к файлам
\setmainfont{Times New Roman}[
    Path            = /usr/local/share/fonts/truetype/times-new-roman/, % Путь внутри контейнера
    Extension       = .ttf,
    UprightFont     = times,
    BoldFont        = timesbd,
    ItalicFont      = timesi,
    BoldItalicFont  = timesbi
]
% \setmainfont{TeX Gyre Termes} % Свободный аналог Times New Roman, включенный в TeX Live

% Указываем babel использовать основной шрифт для всех языков по умолчанию
\babelprovide[main]{russian}


\usepackage{setspace}         % Пакет для гибкого управления интервалами
\onehalfspacing               % Установка полуторного межстрочного интервала





%========================================================================================
% СТРАНИЦЫ
%========================================================================================
\usepackage[left=3cm, right=1cm, top=2cm, bottom=2cm]{geometry} % Поля документа
\usepackage{indentfirst}        % Красная строка для первого абзаца после заголовка
\setlength{\parindent}{1.25cm}  % Отступ красной строки - 1.25 см

\usepackage[toc,page]{appendix} % Поддержка приложение в отчете
\usepackage{pdflscape}          % Поддержка горизонтальных страниц

% --- Поддержка колонтитулов
\usepackage{enotez}

\makeatletter % Создание нового стиля для сносок на странице
\def\enotez@endnotes@footer{%
    \begin{center}
        \rule{0.5\linewidth}{0.4pt}
        \enotez@theendnotes
    \end{center}
}

% --- Настройка содержанмя
\usepackage{tocloft}

% Убираем точки между заголовком и номером страницы
\renewcommand{\cftsecleader      }{\hfil} 
\renewcommand{\cftsubsecleader   }{\hfil}
\renewcommand{\cftsubsubsecleader}{\hfil}
% Делаем заголовок оглавления без жирного шрифта
\renewcommand{\cfttoctitlefont}{\normalfont\bfseries\Large\centering}
\renewcommand{\cfttoctitlefont}{\normalfont\Large\bfseries\centering}
% Меняем название на "СОДЕРЖАНИЕ"
\renewcommand{\contentsname}{{\large\uppercase{СОДЕРЖАНИЕ}}}
% Убираем жирный шрифт с разделов и номеров страниц
\renewcommand{\cftsecfont       }{\normalfont} 
\renewcommand{\cftsecpagefont   }{\normalfont}
\renewcommand{\cftsubsecfont    }{\normalfont}
\renewcommand{\cftsubsecpagefont}{\normalfont}


%========================================================================================
% ЗАГОЛОВКИ
%========================================================================================
\usepackage{titlesec}

%titleformat{<команда>     }{<стиль>             }{<номер>           }{<отступ>}{<текст до>}
\titleformat{\section      }{\bfseries\normalsize}{\thesection.      }{1em     }{          }
\titleformat{\subsection   }{\bfseries\normalsize}{\thesubsection.   }{1em     }{          }
\titleformat{\subsubsection}{\bfseries\normalsize}{\thesubsubsection.}{1em     }{          }

\newcommand{\centeredsection}[1]{
    \noindent
    \begin{center}
        \textbf{\normalsize #1}
    \end{center}
    \par
}




%========================================================================================
% МАТЕМАТИКА
%========================================================================================
\usepackage{amsmath}        % Основные математические окружения
\usepackage{amsfonts}       % Математические шрифты
\usepackage{amssymb}        % Дополнительные математические символы
\usepackage{mathtools}      % Расширение для amsmath с исправлением ошибок и новыми командами
\usepackage{icomma}         % Корректная работа запятой как десятичного разделителя в формулах





%========================================================================================
% ГРАФИКА, ТАБЛИЦЫ И ПЛАВАЮЩИЕ ОБЪЕКТЫ
%========================================================================================
\usepackage{graphicx}       % Для вставки изображений
\usepackage{float}          % Для точного позиционирования объектов с опцией [H]
\usepackage{caption}        % Гибкая настройка подписей к рисункам и таблицам

% --- Настройка подписей к рисункам
\captionsetup[figure]{
    justification=centering,   % Выравнивание по центру
    labelsep=endash,           % Разделитель "Рисунок 1 –" (тире)
    singlelinecheck=false,     % Принудительное центрирование даже для коротких подписей
    font=normalsize,           % Обычный размер шрифта (не курсив)
    skip=6pt                   % Отступ после подписи
}

% --- Настройка подписей к таблицам
\captionsetup[table]{
    position=top,              % Подпись над таблицей
    justification=raggedright, % Выравнивание по левому краю
    labelsep=endash,           % Разделитель "Таблица 1 –" (тире)
    singlelinecheck=false,     % Принудительное выравнивание по левому краю
    font=normalsize,           % Обычный размер шрифта (не курсив)
    skip=6pt                   % Отступ перед таблицей
}

% --- Пакеты для качественных таблиц
\usepackage{multirow}       % Улучшенное форматирование таблиц
\usepackage{tabularx}       % Таблицы с автоматическим расчетом ширины колонок
\usepackage{array}          % Расширяет возможности работы с таблицами и выравниваниями
\usepackage{booktabs}       % Профессиональное оформление таблиц (горизонтальные линии \toprule, \midrule, \bottomrule)
\usepackage{makecell}       % Многострочные ячейки в таблицах

% --- Настройка новых типов колонок для tabularx
\renewcommand{\tabularxcolumn}[1]{m{#1}}
\newcolumntype{C}{>{\centering\arraybackslash}X}
\newcolumntype{L}{>{\arraybackslash}X}
\newcolumntype{R}{>{\raggedleft\arraybackslash}X}





%========================================================================================
% ИСХОДНЫЙ КОД
%========================================================================================
\usepackage{listings}       % Для вставки листингов кода
\usepackage{xcolor}         % Для определения цветов

% --- Настройка стиля для листингов
\definecolor{codegray}{gray}{0.95}
\definecolor{codepurple}{rgb}{0.58,0,0.82}
\definecolor{backcolour}{rgb}{0.98,0.98,0.98}

\lstdefinestyle{mystyle}{
    backgroundcolor=\color{backcolour},
    commentstyle=\color{green!50!black},
    keywordstyle=\color{blue},
    numberstyle=\tiny\color{gray},
    stringstyle=\color{codepurple},
    basicstyle=\footnotesize\ttfamily,
    breakatwhitespace=false,
    breaklines=true,
    captionpos=b,
    keepspaces=true,
    numbers=left,
    numbersep=5pt,
    showspaces=false,
    showstringspaces=false,
    showtabs=false,
    tabsize=2,
    frame=single,
    framerule=0.5pt,
    rulecolor=\color{black!20},
    title=\lstname
}
\lstset{style=mystyle} % Применяем стиль по умолчанию

% --- Новая команда для вставки кода из файла
% Использование: \insertcode[caption={Подпись}, label={lbl:code}]{путь/к/файлу.py}
% Код будет автоматически отформатирован, подсвечен синтаксис Python, и пронумерованы строки.
% Язык можно поменять, например: \insertcode[language=C++]{путь/к/файлу.cpp}.
% Т.е. language=Python - это язык по умолчанию, который можно свободно переопределять.
\newcommand{\insertcode}[2][]{\lstinputlisting[language=Python, #1]{#2}}





%========================================================================================
% ССЫЛКИ И НАВИГАЦИЯ
%========================================================================================
\usepackage{hyperref}       % Создание кликабельных ссылок в документе
\hypersetup{
    colorlinks=true,
    linkcolor=black,
    urlcolor=blue,
    citecolor=black
}

\usepackage[russian]{cleveref} % "Умные" ссылки (\cref вместо \ref)
% cleveref автоматически подставляет "рис.", "табл.", "формула"
% Было:  Как видно из рис. \ref{fig:graph} и табл. \ref{tab:my_results}... -> Результат: "Как видно из рис. 1 и табл. 1..."
% Стало: Как видно из \cref{fig:graph}     и \cref{tab:my_results}...      -> Результат: "Как видно из рис. 1 и табл. 1..."
% Настраиваем названия для cleveref
\crefname{figure}{рис.}{рис.}
\Crefname{figure}{Рис.}{Рис.}
\crefname{table}{табл.}{табл.}
\Crefname{table}{Табл.}{Табл.}
\crefname{section}{разд.}{разд.}
\Crefname{section}{Разд.}{Разд.}
\crefname{equation}{формуле}{формулам}
\Crefname{equation}{Формуле}{Формулам}


% === КОНФИГУРАЦИЯ ДАННЫХ ДЛЯ ТИТУЛЬНОГО ЛИСТА ===
\newcommand{\Department}{КСУ}
\newcommand{\WorkType}{курсовой работе}
\newcommand{\Discipline}{программирование и основы алгоритмизации}
\newcommand{\WorkTitle}{СОСТАВЛЕНИЕ АЛГОРИТМА И НАПИСАНИЕ ПРОГРАММ ОБРАБОТКИ МАССИВА ДАННЫХ}
\newcommand{\Group}{4494}
\newcommand{\Variant}{1.9.ФБ}
\newcommand{\StudentName}{Рахметов А. Р.}
\newcommand{\TeacherName}{Шпекторов А. Г.}
\newcommand{\Year}{2025}

% === СБОРКА ДОКУМЕНТА ===
\begin{document}



% --- ТИТУЛЬНЫЙ ЛИСТ ---
% Это шаблон, он не используется напрямую.
% Вместо текста здесь стоят команды, которые будут определены в config.tex

\begin{center}
    МИНОБРНАУКИ РОССИИ \\
    СПбГЭТУ «ЛЭТИ» ИМ. В.И. УЛЬЯНОВА (ЛЕНИНА) \\
    Кафедра \Department % Имя кафедры
\end{center}

\vfill

\begin{center}
    \textbf{
        \MakeUppercase{Отчёт} \\
        По \WorkType \\ % Номер работы
        По дисциплине «\Discipline» \\
        Тема: \WorkTitle \\ % Тема работы
    }
\end{center}

\vfill

\noindent
\begin{tabularx}{\textwidth}{l X r}
    Студент гр. \Group, вар. \Variant & \hrulefill & \StudentName \\\\
    Преподаватель    & \hrulefill & \TeacherName
\end{tabularx}

\vfill

\begin{center}
    Санкт-Петербург \\
    \Year \\
\end{center}

\thispagestyle{empty}
\newpage




% --- ОГЛАВЛЕНИЕ ---
\begin{center}
    \tableofcontents
\end{center}
\thispagestyle{empty}
\newpage



% --- ВВЕДЕНИЕ ---
\newpage
\addcontentsline{toc}{section}{Введение}
\phantomsection
\centeredsection{\uppercase{Введение}}
В современном мире задачи транспортной логистики и автоматизации навигации играют ключевую роль во многих отраслях, от грузоперевозок до разработки беспилотных транспортных средств. Одной из фундаментальных проблем в этой области является построение оптимального маршрута с учетом различных ограничений, таких как запас хода транспортного средства. Эффективное решение этой задачи позволяет сократить временные и энергетические затраты, что обуславливает \textit{актуальность} данной курсовой работы. Разработка алгоритмов, способных находить кратчайший путь в графе с весами, является классической задачей теории графов, имеющей широкое практическое применение.

% Комментарий к абзацу об актуальности:
% Здесь мы идем от общего к частному. Начинаем с широкой области («транспортная логистика», «автоматизация навигации»), затем сужаем ее до конкретной проблемы («построение оптимального маршрута с учетом ограничений»), и, наконец, связываем это с методами решения («задачи теории графов»). Это показывает, что ваша работа вписана в более широкий научный и практический контекст. Ключевое слово *«актуальность»* выделено курсивом для акцента.

Целью данной курсовой работы является разработка программы на математическом языке Matlab для поиска и визуализации оптимального маршрута движения объекта с ограниченным запасом хода между двумя заданными точками на местности с набором пунктов дозаправки.

% Комментарий к цели работы:
% Цель — это одно, максимум два предложения, которые четко и однозначно формулируют конечный результат вашей работы. Используется глагол в неопределенной форме («разработка», «исследование», «создание»). Формулировка цели должна точно соответствовать теме вашей курсовой работы.

Для достижения поставленной цели необходимо было решить следующие задачи:
\begin{enumerate}
    \item Проанализировать исходные данные: характеристики подвижного объекта, координаты начальной, конечной и промежуточных точек (пунктов дозаправки).
    \item Представить карту местности в виде неориентированного взвешенного графа, где вершины соответствуют точкам на местности, а ребра — возможным перемещениям между ними.
    \item Реализовать алгоритм построения графа с учетом ограничения на максимальное расстояние, проходимое объектом без дозаправки.
    \item Реализовать алгоритм поиска кратчайшего пути в графе. В соответствии с заданием (см. \cref{table:source_data}), следует использовать алгоритм Форда-Беллмана.
    \item Разработать функцию для формирования NMEA-подобных сообщений, описывающих движение по оптимальному маршруту.
    \item Создать модуль для графической визуализации исходного графа, всех возможных путей и найденного оптимального маршрута.
    \item Обеспечить сохранение результатов расчетов (длина пути, NMEA-сообщения) в текстовый файл.
\end{enumerate}

% \paragraph{Комментарий к задачам:}
% Задачи — это конкретные шаги, которые вы предприняли для достижения цели. Они должны быть представлены в виде нумерованного списка и отражать логику вашей работы и структуру основной части отчета. По сути, каждый пункт списка задач может стать основой для подраздела в основной главе. Формулировки задач также начинаются с глагола («проанализировать», «представить», «реализовать»).

Объектом исследования является процесс нахождения оптимального пути в дискретной среде с ограничениями.

Предметом исследования являются алгоритмы на графах, в частности алгоритм Форда-Беллмана, и методы их программной реализации в среде Matlab для решения прикладных навигационных задач.

% \paragraph{Комментарий к объекту и предмету:}
% Это формальный, но важный элемент введения.
% *   *Объект* — это более широкое явление или процесс, который вы изучаете. Это ответ на вопрос «что исследуется?».
% *   *Предмет* — это конкретная часть объекта, его свойства или методы его изучения, которые рассматриваются в вашей работе. Это ответ на вопрос «какие аспекты объекта исследуются?».

% Курсовая работа состоит из введения, основной части, заключения, списка использованных источников и приложений.
% Во введении обосновывается актуальность темы, ставятся цель и задачи исследования.
% Основная часть содержит постановку задачи, описание математической модели, описание алгоритмов и программной реализации.
% В заключении приводятся основные выводы по проделанной работе.
% В приложениях содержится листинг кода разработанных программных модулей.

% \paragraph{Комментарий к структуре работы:}
% Этот абзац кратко описывает, из каких частей состоит ваш отчет. Он служит своего рода «содержанием в прозе» и помогает проверяющему быстро сориентироваться в документе.



% --- ОСНОВНОЕ СОДЕРЖАНИЕ ---
\clearpage
\section{Анализ задачи и математическая модель}
\label{sec:analysis_and_model}

\subsection{Постановка задачи}
\label{subsec:problem_statement}

В рамках данной курсовой работы требуется разработать программу, предназначенную для нахождения оптимального маршрута движения объекта по территории с заданными пунктами дозаправки. Решение задачи должно включать формирование графовой модели местности, поиск кратчайшего пути и представление результатов в текстовом и графическом виде.

Исходя из задания, были определены следующие входные данные и ограничения (см. \cref{table:source_data}).

\begin{table}[hb]
    \centering
    \caption{Исходные данные}
    \label{table:source_data}

    \begin{tblr}{
            colspec = {| X[2, l, m] | X[1, c, m] | },
            width = \linewidth,
            hlines = 1, vlines = 1,
            row{1} = {font=\bfseries},
            row{5} = {font=\bfseries},
            row{10} = {font=\bfseries},
        }
        \SetCell[c=2]{c} Характеристики подвижного объекта                \\
        Скорость движения                      & $v = 18$ км/ч.           \\
        Время работы от одно заряда АКБ        & $t_{max} = 7$ ч.         \\
        Время, необходимое для полной зарядки  & $t_{recharge} = 24$ мин. \\
        \hline
        \SetCell[c=2]{c} Описание территории                              \\
        Координаты начальной точки маршрута    & $P_{start} = (-9, 7)$.   \\
        Координаты конечной точки маршрута     & $P_{end} = (8, -6)$.     \\
        Множество координат пунктов дозаправки & $\{P_i\}$                \\
        Масштаб карты                          & $M = (40 : 1)$ км        \\
        \hline
        \SetCell[c=2]{c} Алгоритм поиска кратчайшего пути                 \\
        \SetCell[c=2]{c} Алгоритм Форда-Беллмана                          \\
    \end{tblr}
\end{table}

Требуемые выходные данные:
\begin{enumerate}
    \item Оптимальный по длине маршрут, представленный в виде последовательности узлов (точек) от $P_{start}$ до $P_{end}$.
    \item Общая протяженность найденного оптимального маршрута $L_{optimal}$.
    \item Последовательность NMEA-подобных сообщений формата \mintinline{x}!$UTHDG!, описывающих каждый сегмент оптимального пути.
    \item Графическое представление, включающее все возможные пути между точками, а также выделенный оптимальный маршрут.
\end{enumerate}

Основным ограничением при построении маршрута является максимальная дальность хода объекта на одном полном заряде аккумулятора. Данная величина, $D_{max}$, вычисляется как произведение скорости на время автономной работы:
\begin{equation}
    D_{max} = v \cdot t_{max} = 18 \text{ км/ч} \cdot 7 \text{ ч} = 126 \text{ км}.
    \label{formula:max_distance}
\end{equation}
Перемещение между двумя любыми точками возможно только в том случае, если расстояние между ними не превышает $D_{max}$.

\subsection{Обоснование выбора математического аппарата}
\label{subsec:math_apparatus_justification}

Поставленная задача по своей структуре является классической задачей поиска оптимального пути на множестве дискретных точек с заданными ограничениями. Для ее формализации и решения наиболее эффективным является применение \textit{теории графов}.

Данный выбор обусловлен следующими соображениями:
\begin{itemize}
    \item Точки на местности (начальная, конечная и пункты дозаправки) естественным образом представляются в виде \textit{вершин} (или узлов) графа.
    \item Возможные перемещения между парами точек, удовлетворяющие ограничению по дальности хода, могут быть смоделированы как \textit{ребра}, соединяющие соответствующие вершины.
    \item Расстояние между точками, которое необходимо минимизировать, является естественной \textit{весовой характеристикой} для каждого ребра.
\end{itemize}

Таким образом, исходная навигационная задача сводится к задаче нахождения кратчайшего пути во взвешенном неориентированном графе. Такой подход позволяет применить для решения хорошо изученные и формально описанные алгоритмы, такие как алгоритм Дейкстры, Флойда или, как указано в задании, алгоритм Форда-Беллмана.

\subsection{Разработка математической модели}
\label{subsec:math_model_development}

На основе выбранного математического аппарата представим карту местности и условия задачи в виде неориентированного взвешенного графа $G = (V, E)$, где $V$ — множество вершин, а $E$ — множество ребер.

Множество вершин графа $V$ формируется из всех заданных точек на местности:
\begin{equation}
    V = \{v_1, v_2, \dots, v_N\} = \{P_{start}\} \cup \{P_i\} \cup \{P_{end}\},
    \label{formula:vertex_set}
\end{equation}
где $N$ — общее количество точек (начальная, конечная и все пункты дозаправки). Каждая вершина $v_k \in V$ характеризуется своими декартовыми координатами $(x_k, y_k)$ в системе координат карты.

Множество ребер $E$. Граф не является полным, так как перемещение возможно не между всеми парами вершин. Ребро $(v_i, v_j)$ существует в множестве $E$ тогда и только тогда, когда физическое расстояние между точками, соответствующими этим вершинам, не превышает максимальную дальность хода $D_{max}$.

Расстояние $L(v_i, v_j)$ между двумя вершинами $v_i = (x_i, y_i)$ и $v_j = (x_j, y_j)$ вычисляется с учетом масштаба карты $M$ по следующей формуле:
\begin{equation}
    L(v_i, v_j) = M \cdot \sqrt{(x_j - x_i)^2 + (y_j - y_i)^2}.
    \label{formula:distance}
\end{equation}
Таким образом, условие существования ребра $(v_i, v_j)$ формально записывается как:
\begin{equation}
    (v_i, v_j) \in E \iff L(v_i, v_j) \le D_{max}.
    \label{formula:edge_condition}
\end{equation}
Поскольку граф является \textit{неориентированным}, расстояние $L(v_i, v_j) = L(v_j, v_i)$.

Каждому ребру $(v_i, v_j) \in E$ ставится в соответствие вес $w(v_i, v_j)$, равный физическому расстоянию между узлами.
\begin{equation}
    w(v_i, v_j) = L(v_i, v_j).
    \label{formula:edge_weight}
\end{equation}

С учетом разработанной модели исходная задача нахождения оптимального маршрута сводится к задаче нахождения пути с минимальной суммой весов ребер от начальной вершины $v_{start}$, соответствующей точке $P_{start}$, до конечной вершины $v_{end}$, соответствующей точке $P_{end}$, в построенном графе $G$.





\clearpage
\section{Алгоритмы и методы решения}
\label{sec:algorithms_and_methods}

В данной главе рассматриваются алгоритмы, использованные для решения поставленных задач. Описывается метод поиска кратчайшего пути в графе, а также алгоритм расчета навигационных параметров и формирования NMEA-сообщений для оптимального маршрута.

\subsection{Алгоритм поиска кратчайшего пути}
\label{subsec:shortest_path_algorithm}

Согласно заданию, для нахождения кратчайшего пути в построенном графе $G=(V, E)$ был применен \textit{алгоритм Форда-Беллмана}. Этот алгоритм предназначен для поиска кратчайшего пути от одной вершины ко всем остальным во взвешенном ориентированном или неориентированном графе. Ключевой особенностью алгоритма является его способность корректно работать с ребрами, имеющими отрицательные веса. В рассматриваемой задаче веса ребер (расстояния) всегда неотрицательны, что не отменяет справедливость применения данного алгоритма.

Основой алгоритма является процесс \textit{релаксации} (ослабления) ребер. Для каждой вершины $v \in V$ хранится значение $d[v]$ — текущая известная длина кратчайшего пути от начальной вершины $v_{start}$ до $v$. Изначально $d[v_{start}] = 0$, а для всех остальных вершин $v \neq v_{start}$ значение $d[v] = \infty$.

Процесс релаксации для ребра $(u, v)$ с весом $w(u, v)$ заключается в проверке условия:
\begin{equation}
    d[v] > d[u] + w(u, v).
    \label{formula:relaxation_condition}    
\end{equation}  % WIP: Добавить рисунок, иллюстрирующий данное неравенство

Если неравенство выполняется, это означает, что найден более короткий путь до вершины $v$ через вершину $u$. В этом случае значение $d[v]$ обновляется:
\begin{equation}
    d[v] := d[u] + w(u, v).
    \label{formula:relaxation_update}    
\end{equation}
Также для восстановления пути сохраняется информация о том, что предшественником $v$ на кратчайшем пути теперь является $u$.

Алгоритм Форда-Беллмана выполняет релаксацию для каждого ребра графа $|V|-1$ раз. Такое количество итераций гарантирует нахождение кратчайшего пути в графе без циклов отрицательного веса.

\subsubsection*{Шаги алгоритма.}  % WIP: Необходимо так же добавить блок-схему алгоритма
\label{subsubsec:algorithm_steps}
\begin{enumerate}
    \item \textit{Инициализация.} Для каждой вершины $v \in V$ устанавливается начальное расстояние $d[v] = \infty$ и предшественник $p[v] = \text{null}$. Для начальной вершины $v_{start}$ устанавливается $d[v_{start}] = 0$.
    
    \item \textit{Релаксация ребер.} Выполняется цикл, который повторяется $|V|-1$ раз. Внутри этого цикла выполняется перебор всех ребер $(u, v) \in E$ и для каждого из них производится операция релаксации.
    
    \item \textit{Проверка на наличие циклов отрицательного веса (опционально).} После $|V|-1$ итерации выполняется еще одна итерация релаксации по всем ребрам. Если на этой итерации удается улучшить путь хотя бы до одной вершины, это свидетельствует о наличии в графе цикла отрицательного веса. В данной задаче этот шаг не является необходимым, так как все веса положительны.
    
    \item \textit{Восстановление пути.} После завершения всех итераций, если путь до конечной вершины $v_{end}$ был найден ($d[v_{end}] \neq \infty$), оптимальный маршрут восстанавливается в обратном порядке, двигаясь от $v_{end}$ к $v_{start}$ по сохраненным предшественникам.
\end{enumerate}

\subsection{Алгоритм расчета параметров маршрута и формирования NMEA-сообщений}
\label{subsec:nmea_generation_algorithm}

После нахождения оптимального пути в виде последовательности вершин необходимо рассчитать детальные параметры движения для каждого сегмента маршрута и сформировать NMEA-подобные сообщения. Согласно заданию, используется специальный формат сообщения \mintinline{x}!$UTHDG!.

\subsubsection*{Формат сообщения.}
\label{subsubsec:message_format}

Каждое сообщение описывает один сегмент пути (перемещение между двумя соседними вершинами в оптимальном маршруте) и имеет следующую структуру: \mintinline{x}!$UTHDG,XX,Y.Y,DD.DD,S1,Z.Z,S2!, где:
\begin{description}
    \item[\mintinline{x}!XX!] — часы с момента старта (целое число).
    \item[\mintinline{x}!Y.Y!] — минуты, прошедшие с начала текущего часа (вещественное число).
    \item[\mintinline{x}!DD.DD!] — длина текущего сегмента маршрута в километрах (вещественное число).
    \item[\mintinline{x}!S1!] — флаг наличия поворота. Принимает значение $T$ (\textit{Turn}), если направление движения изменилось по сравнению с предыдущим сегментом, и $N$ (\textit{No turn}) в противном случае.
    \item[\mintinline{x}!Z.Z!] — азимут, или угол направления движения на данном сегменте, в градусах (вещественное число).
    \item[\mintinline{x}!S2!] — флаг окончания всего маршрута. Принимает значение $E$ (\textit{End}) для последнего сегмента пути и $N$ (\textit{Not end}) для всех остальных.
\end{description}

\subsubsection*{Алгоритм формирования сообщений.}
\label{subsubsec:message_generation_algorithm}

Для каждого сегмента пути между вершинами $v_i=(x_i, y_i)$ и $v_{i+1}=(x_{i+1}, y_{i+1})$ из оптимального маршрута производятся следующие вычисления:

\begin{enumerate}
    \item \textit{Расчет длины сегмента (\texttt{DD.DD}):}
    Длина сегмента вычисляется как евклидово расстояние между точками с учетом масштаба карты $M$:
    \begin{equation}
        DD = M \cdot \sqrt{(x_{i+1} - x_i)^2 + (y_{i+1} - y_i)^2}.
        \label{formula:segment_length}
    \end{equation}
    
    \item \textit{Расчет времени в пути (\texttt{XX}, \texttt{Y.Y}):}
    Сначала вычисляется время, необходимое для прохождения текущего сегмента: $\Delta t = DD / v$. Затем это время добавляется к общему времени, накопленному с начала движения, $T_{total}$.
    \begin{equation}
        T_{total} := T_{total} + \Delta t.
        \label{formula:total_time_update}
    \end{equation}
    Компоненты \texttt{XX} и \texttt{Y.Y} вычисляются из $T_{total}$:
    \begin{align}
        XX &= \lfloor T_{total} \rfloor, \\
        YY &= (T_{total} - XX) \cdot 60.
        \label{formula:time_components}
    \end{align}
    
    \item \textit{Расчет азимута (\texttt{Z.Z}):}
    Азимут (угол направления) вычисляется на основе приращений координат $\Delta x = x_{i+1} - x_i$ и $\Delta y = y_{i+1} - y_i$. Для корректного определения угла во всех четвертях используется арктангенс с двумя аргументами:
    \begin{equation}
        \theta_{\text{rad}} = \operatorname{atan2}(\Delta y, \Delta x) = \operatorname{rad2deg}(\theta_{\text{rad}}).
        \label{formula:azimuth}
    \end{equation}
    Полученное значение в радианах переводится в градусы и приводится к диапазону $[0, 360^\circ)$:
    \begin{equation}
        \theta_{\text{deg}} = \theta_{\text{rad}} \cdot \frac{180}{\pi}.
        \label{formula:azimuth_degrees}
    \end{equation}
    Если $\theta_{\text{deg}} < 0$, то $\theta_{\text{deg}} := \theta_{\text{deg}} + 360$. Итоговое значение является азимутом $ZZ$.
    
    \item \textit{Определение флагов (\texttt{S1}, \texttt{S2}):}
    Флаги определяются на основе логических проверок:
    \begin{itemize}
        \item $S1 = T$, если азимут текущего сегмента $ZZ_i$ не равен азимуту предыдущего сегмента $ZZ_{i-1}$. В противном случае $S1 = N$. Для первого сегмента флаг всегда $T$.
        \item $S2 = E$, если текущий сегмент является последним в маршруте. В противном случае $S2 = N$.
    \end{itemize}
\end{enumerate}




\clearpage
\section{Программная реализация и анализ результатов}
\label{section:implementation_and_results}

В данной главе описывается структура разработанного программного комплекса, среда разработки, а также представлены и проанализированы результаты, полученные в ходе выполнения программы с заданными начальными условиями.

\subsection{Структура программного комплекса}
\label{subsec:software_structure}

Программный комплекс был разработан в среде математического моделирования \textit{Matlab}. Выбор данной среды обусловлен ее широкими возможностями для работы с матрицами, удобными средствами для визуализации данных и наличием встроенных функций для реализации сложных вычислительных алгоритмов.

Проект имеет модульную структуру, что обеспечивает гибкость и упрощает его понимание и возможную дальнейшую модификацию. Структура включает один главный исполняемый скрипт и набор вспомогательных функций, каждая из которых решает конкретную подзадачу.

\begin{description}
    \item[\texttt{main.m}] — главный скрипт программы. Он выполняет следующие действия:
    \begin{itemize}
        \item Инициализация начальных условий (стандартных или введенных пользователем).
        \item Чтение координат заправочных пунктов из файла данных.
        \item Последовательный вызов вспомогательных функций для построения графа, поиска пути, генерации NMEA-сообщений и визуализации.
        \item Сохранение и вывод итоговых результатов.
    \end{itemize}
    (См. \cref{appendix:main.m}).

    \item[\texttt{buildGraph.m}] — функция, отвечающая за реализацию математической модели. На вход она получает массив координат всех точек и параметры объекта, а на выходе формирует структуру графа, содержащую матрицу смежности и матрицу весов (расстояний) ребер в соответствии с ограничением по дальности хода. (См. \cref{appendix:buildGraph.m}).

    \item[\texttt{findOptimalPathByFordBellman.m}] — функция, реализующая алгоритм Форда-Беллмана для поиска кратчайшего пути. В качестве входных данных принимает структуру графа и индексы начальной и конечной вершин. Возвращает последовательность индексов вершин оптимального пути и его общую длину. (См. \cref{appendix:findOptimalPathByFordBellman.m}).

    \item[\texttt{generateNmeaMessages.m}] — функция для генерации NMEA-сообщений. На основе найденного оптимального пути и координат узлов она рассчитывает параметры движения для каждого сегмента и формирует массив строк в формате \mintinline{x}!$UTHDG!. (См. \cref{appendix:generateNmeaMessages.m}).

    \item[\texttt{plotRoute.m}] — функция для графической визуализации. Она строит график, на котором отображаются все узлы, все возможные пути (ребра графа), а также выделяет цветом найденный оптимальный маршрут, начальную и конечную точки. (См. \cref{appendix:plotRoute.m}).

    \item[\texttt{saveResults.m}] — функция, сохраняющая результаты работы программы (длину пути и NMEA-сообщения) в текстовый файл. (См. \cref{appendix:saveResults.m}).
\end{description}  % WIP: Приложения должны существовать

Такая декомпозиция на функции соответствует принципам структурного программирования и делает код более читаемым и поддерживаемым.

\subsection{Тестирование и результаты работы}
\label{subsec:testing_and_results}

Программный комплекс был протестирован с использованием стандартных начальных условий, указанных в файле \texttt{main.m} и в постановке задачи (\cref{table:source_data}). Координаты заправочных пунктов были загружены из файла \mintinline{x}!refueling_points.txt!. (Также существует файл \mintinline{x}!refueling_points_extended.txt!, особенностью которого являются нецелочисленные координаты заправочных станций).

\subsubsection*{Итоговые расчетные данные.}
\label{subsubsec:final_computed_data}

После выполнения программы были получены следующие результаты. Общая длина оптимального маршрута от начальной точки $(-9, 7)$ до конечной точки $(8, -6)$ составила:
\begin{equation}
    L_{optimal} = 922.93 \text{ км}.
    \label{formula:L_optimal}
\end{equation}

Сгенерированные NMEA-сообщения, описывающие каждый сегмент оптимального пути, и длина оптимального пути представлены в \cref{lst:nmea_messages}.

\begin{listing}[H]
    \caption{Сгенерированный результирующий отчёт}
    \label{lst:nmea_messages}
    \inputminted[
        % firstline      = 4,
        % lastline       = 10,
        % highlightlines = {1-10},
    ]{txt}{code/results/result.txt}
\end{listing}




\subsubsection*{Анализ визуализации}
\label{subsubsec:visualization_analysis}

Результат графической визуализации маршрута представлен на \cref{fig:optimal_route}.  % WIP

\begin{figure}[hb]
    \centering
    \includegraphics[width=1.0\linewidth]{code/results/optimal_route.png}
    \caption{Визуализация маршрута на графе}
    \label{fig:optimal_route}
\end{figure}

На рисунке черными точками обозначены все возможные узлы (заправочные станции), зеленой и пурпурной --- стартовая и конечная соответственно. Тонкими пунктирными линиями показаны все возможные перемещения между узлами, удовлетворяющие ограничению по дальности хода ($D_{max} \le 126$ км). Найденный оптимальный путь выделен жирной красной линией.

Как видно из графа, алгоритм успешно нашел непрерывный маршрут от старта к финишу. Маршрут состоит из последовательности отрезков, длина каждого из которых не превышает максимально допустимую. Алгоритм обходит некоторые короткие, но ведущие в неверном направлении пути, выбирая ту последовательность узлов, которая в сумме дает минимальное итоговое расстояние. Визуализация подтверждает корректность работы реализованного алгоритма поиска кратчайшего пути.
% --- ОСНОВНОЕ СОДЕРЖАНИЕ ---



% --- ЗАКЛЮЧЕНИЕ ---
\newpage
\addcontentsline{toc}{section}{Заключение}
\phantomsection
\centeredsection{\uppercase{Заключение}}
В ходе выполнения данной курсовой работы была успешно решена задача разработки и реализации программного обеспечения на языке Matlab для поиска и визуализации оптимального маршрута движения объекта с ограниченным запасом хода.

Для решения поставленной задачи была разработана математическая модель, представляющая территорию в виде неориентированного взвешенного графа. На основе этой модели был реализован программный код, включающий модули для построения графа с учетом ограничений, поиска кратчайшего пути с помощью алгоритма Форда-Беллмана, генерации навигационных сообщений и графического отображения результатов.

В результате проделанной работы были получены следующие основные результаты:
\begin{itemize}
    \item Разработана \textit{программа} в среде Matlab, позволяющая автоматизировать процесс поиска оптимального маршрута.
    \item Сформирована \textit{математическая модель} задачи на основе теории графов, формализующая условия и ограничения.
    \item Реализован программный модуль для поиска кратчайшего пути на графе с использованием алгоритма Форда-Беллмана.
    \item Разработан алгоритм и реализующая его функция для формирования NMEA-подобных сообщений, описывающих движение по найденному маршруту.
    \item Создан модуль визуализации, который наглядно представляет построенный граф, все возможные пути и итоговый оптимальный маршрут, что упрощает анализ результатов.
    \item Проведено тестирование программы на конкретном примере, подтвердившее корректность работы реализованных алгоритмов и всей программы в целом.
\end{itemize}

В результате, все поставленные в работе задачи были выполнены в полном объеме, а основная цель курсовой работы — достигнута. Разработанный программный продукт является законченным решением, готовым к использованию для решения аналогичных навигационно-логистических задач.


...Для поиска по графу рекомендуется применять алгоритм Форда-Беллмана \cite{korn_algorithms}. 
Подробное описание алгоритма можно найти в интернет-источниках \cite{wiki_ford_bellman}. 
Все расчеты были выполнены в среде Matlab \cite{dyakonov_matlab}...

% --- СПИСОК ИСПОЛЬЗОВАННЫХ ИСТОЧНИКОВ ---
\newpage
\addcontentsline{toc}{section}{Список использованных источников}
\phantomsection
% \centeredsection{\uppercase{Список использованных источников}}
% \addbibresource{references.bib}  % Имя файла references.bib не следует менять! Используется оптимизация скрипта сборки документа
% \printbibliography
\renewcommand{\bibsection}{\centering\textbf{СПИСОК ИСПОЛЬЗОВАННЫХ ИСТОЧНИКОВ}} % смена названия библиографии по умолчанию
\bibliography{references}



% --- ПРИЛОЖЕНИЯ ---
\begin{appendices}
    % \section{Блок-схема алгоритма Форда-Беллмана}
\label{app:block_diagram_FB}
\begin{figure}[H]
    \centering
    \includegraphics[keepaspectratio, height=\freeht, width=\linewidth]{block_diagram/FB.drawio.pdf}
    % \caption{caption}
    % \label{fig:image}
\end{figure}

\section{Блок-схема buildGraph}
\label{app:block_diagram_BG}
\begin{figure}[H]
    \centering
    \includegraphics[keepaspectratio, height=\freeht, width=\linewidth]{block_diagram/BG.drawio.pdf}
    % \caption{caption}
    % \label{fig:image}
\end{figure}

\section{main.m}
\label{app:main.m}

\begin{codemultipage}
    % \captionof{listing}{caption\label{lst:label}}
    \inputminted{matlab}{code/main.m}
\end{codemultipage}



\section{buildGraph.m}
\label{app:buildGraph.m}

\begin{codemultipage}
    % \captionof{listing}{caption\label{lst:label}}
    \inputminted{matlab}{code/functions/buildGraph.m}
\end{codemultipage}



\section{findOptimalPathByFordBellman.m}
\label{app:findOptimalPathByFordBellman.m}

\begin{codemultipage}
    % \captionof{listing}{caption\label{lst:label}}
    \inputminted{matlab}{code/functions/findOptimalPathByFordBellman.m}
\end{codemultipage}



\section{generateNmeaMessages.m}
\label{app:generateNmeaMessages.m}

\begin{codemultipage}
    % \captionof{listing}{caption\label{lst:label}}
    \inputminted{matlab}{code/functions/generateNmeaMessages.m}
\end{codemultipage}



\section{plotRoute.m}
\label{app:plotRoute.m}

\begin{codemultipage}
    % \captionof{listing}{caption\label{lst:label}}
    \inputminted{matlab}{code/functions/plotRoute.m}
\end{codemultipage}



\section{saveResults.m}
\label{app:saveResults.m}

\begin{codemultipage}
    % \captionof{listing}{caption\label{lst:label}}
    \inputminted{matlab}{code/functions/saveResults.m}
\end{codemultipage}



\section{refueling\_points.txt}
\label{app:refueling_points.txt}

\begin{codemultipage}
    % \captionof{listing}{caption\label{lst:label}}
    \inputminted{matlab}{code/data/refueling_points.txt}
\end{codemultipage}


\section{graph\_distances.csv}
\label{app:graph_distances.csv}

\begin{codemultipage}
    % \captionof{listing}{caption\label{lst:label}}
    \inputminted{text}{code/results/graph_distances.csv}
\end{codemultipage} % Приложения
\end{appendices}



\end{document}