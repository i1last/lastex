В современном мире задачи транспортной логистики и автоматизации навигации играют ключевую роль во многих отраслях, от грузоперевозок до разработки беспилотных транспортных средств. Одной из фундаментальных проблем в этой области является построение оптимального маршрута с учетом различных ограничений, таких как запас хода транспортного средства. Эффективное решение этой задачи позволяет сократить временные и энергетические затраты, что обуславливает \textit{актуальность} данной курсовой работы. Разработка алгоритмов, способных находить кратчайший путь в графе с весами, является классической задачей теории графов, имеющей широкое практическое применение.

% Комментарий к абзацу об актуальности:
% Здесь мы идем от общего к частному. Начинаем с широкой области («транспортная логистика», «автоматизация навигации»), затем сужаем ее до конкретной проблемы («построение оптимального маршрута с учетом ограничений»), и, наконец, связываем это с методами решения («задачи теории графов»). Это показывает, что ваша работа вписана в более широкий научный и практический контекст. Ключевое слово *«актуальность»* выделено курсивом для акцента.

Целью данной курсовой работы является разработка программы на математическом языке Matlab для поиска и визуализации оптимального маршрута движения объекта с ограниченным запасом хода между двумя заданными точками на местности с набором пунктов дозаправки.

% Комментарий к цели работы:
% Цель — это одно, максимум два предложения, которые четко и однозначно формулируют конечный результат вашей работы. Используется глагол в неопределенной форме («разработка», «исследование», «создание»). Формулировка цели должна точно соответствовать теме вашей курсовой работы.

Для достижения поставленной цели необходимо было решить следующие задачи:
\begin{enumerate}
    \item Проанализировать исходные данные: характеристики подвижного объекта, координаты начальной, конечной и промежуточных точек (пунктов дозаправки).
    \item Представить карту местности в виде неориентированного взвешенного графа, где вершины соответствуют точкам на местности, а ребра — возможным перемещениям между ними.
    \item Реализовать алгоритм построения графа с учетом ограничения на максимальное расстояние, проходимое объектом без дозаправки.
    \item Реализовать алгоритм поиска кратчайшего пути в графе. В соответствии с заданием (см. \cref{table:source_data}), следует использовать алгоритм Форда-Беллмана.
    \item Разработать функцию для формирования NMEA-подобных сообщений, описывающих движение по оптимальному маршруту.
    \item Создать модуль для графической визуализации исходного графа, всех возможных путей и найденного оптимального маршрута.
    \item Обеспечить сохранение результатов расчетов (длина пути, NMEA-сообщения) в текстовый файл.
\end{enumerate}

% \paragraph{Комментарий к задачам:}
% Задачи — это конкретные шаги, которые вы предприняли для достижения цели. Они должны быть представлены в виде нумерованного списка и отражать логику вашей работы и структуру основной части отчета. По сути, каждый пункт списка задач может стать основой для подраздела в основной главе. Формулировки задач также начинаются с глагола («проанализировать», «представить», «реализовать»).

Объектом исследования является процесс нахождения оптимального пути в дискретной среде с ограничениями.

Предметом исследования являются алгоритмы на графах, в частности алгоритм Форда-Беллмана, и методы их программной реализации в среде Matlab для решения прикладных навигационных задач.

% \paragraph{Комментарий к объекту и предмету:}
% Это формальный, но важный элемент введения.
% *   *Объект* — это более широкое явление или процесс, который вы изучаете. Это ответ на вопрос «что исследуется?».
% *   *Предмет* — это конкретная часть объекта, его свойства или методы его изучения, которые рассматриваются в вашей работе. Это ответ на вопрос «какие аспекты объекта исследуются?».

% Курсовая работа состоит из введения, основной части, заключения, списка использованных источников и приложений.
% Во введении обосновывается актуальность темы, ставятся цель и задачи исследования.
% Основная часть содержит постановку задачи, описание математической модели, описание алгоритмов и программной реализации.
% В заключении приводятся основные выводы по проделанной работе.
% В приложениях содержится листинг кода разработанных программных модулей.

% \paragraph{Комментарий к структуре работы:}
% Этот абзац кратко описывает, из каких частей состоит ваш отчет. Он служит своего рода «содержанием в прозе» и помогает проверяющему быстро сориентироваться в документе.
