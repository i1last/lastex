\clearpage
\section{Анализ задачи и математическая модель}
\label{sec:analysis_and_model}

\subsection{Постановка задачи}
\label{subsec:problem_statement}

В рамках данной курсовой работы требуется разработать программу, предназначенную для нахождения оптимального маршрута движения объекта по территории с заданными пунктами дозаправки. Решение задачи должно включать формирование графовой модели местности, поиск кратчайшего пути и представление результатов в текстовом и графическом виде.

Исходя из задания, были определены следующие входные данные и ограничения (см. \cref{table:source_data}).

\begin{table}[hb]
    \centering
    \caption{Исходные данные}
    \label{table:source_data}

    \begin{tblr}{
            colspec = {| X[2, l, m] | X[1, c, m] | },
            width = \linewidth,
            hlines = 1, vlines = 1,
            row{1} = {font=\bfseries},
            row{5} = {font=\bfseries},
            row{10} = {font=\bfseries},
        }
        \SetCell[c=2]{c} Характеристики подвижного объекта                \\
        Скорость движения                      & $v = 18$ км/ч.           \\
        Время работы от одно заряда АКБ        & $t_{max} = 7$ ч.         \\
        Время, необходимое для полной зарядки  & $t_{recharge} = 24$ мин. \\
        \hline
        \SetCell[c=2]{c} Описание территории                              \\
        Координаты начальной точки маршрута    & $P_{start} = (-9, 7)$.   \\
        Координаты конечной точки маршрута     & $P_{end} = (8, -6)$.     \\
        Множество координат пунктов дозаправки & $\{P_i\}$                \\
        Масштаб карты                          & $M = (40 : 1)$ км        \\
        \hline
        \SetCell[c=2]{c} Алгоритм поиска кратчайшего пути                 \\
        \SetCell[c=2]{c} Алгоритм Форда-Беллмана                          \\
    \end{tblr}
\end{table}

Требуемые выходные данные:
\begin{enumerate}
    \item Оптимальный по длине маршрут, представленный в виде последовательности узлов (точек) от $P_{start}$ до $P_{end}$.
    \item Общая протяженность найденного оптимального маршрута $L_{optimal}$.
    \item Последовательность NMEA-подобных сообщений формата \mintinline{x}!$UTHDG!, описывающих каждый сегмент оптимального пути.
    \item Графическое представление, включающее все возможные пути между точками, а также выделенный оптимальный маршрут.
\end{enumerate}

Основным ограничением при построении маршрута является максимальная дальность хода объекта на одном полном заряде аккумулятора. Данная величина, $D_{max}$, вычисляется как произведение скорости на время автономной работы:
\begin{equation}
    D_{max} = v \cdot t_{max} = 18 \text{ км/ч} \cdot 7 \text{ ч} = 126 \text{ км}.
    \label{formula:max_distance}
\end{equation}
Перемещение между двумя любыми точками возможно только в том случае, если расстояние между ними не превышает $D_{max}$.

\subsection{Обоснование выбора математического аппарата}
\label{subsec:math_apparatus_justification}

Поставленная задача по своей структуре является классической задачей поиска оптимального пути на множестве дискретных точек с заданными ограничениями. Для ее формализации и решения наиболее эффективным является применение \textit{теории графов}.

Данный выбор обусловлен следующими соображениями:
\begin{itemize}
    \item Точки на местности (начальная, конечная и пункты дозаправки) естественным образом представляются в виде \textit{вершин} (или узлов) графа.
    \item Возможные перемещения между парами точек, удовлетворяющие ограничению по дальности хода, могут быть смоделированы как \textit{ребра}, соединяющие соответствующие вершины.
    \item Расстояние между точками, которое необходимо минимизировать, является естественной \textit{весовой характеристикой} для каждого ребра.
\end{itemize}

Таким образом, исходная навигационная задача сводится к задаче нахождения кратчайшего пути во взвешенном неориентированном графе. Такой подход позволяет применить для решения хорошо изученные и формально описанные алгоритмы, такие как алгоритм Дейкстры, Флойда или, как указано в задании, алгоритм Форда-Беллмана.

\subsection{Разработка математической модели}
\label{subsec:math_model_development}

На основе выбранного математического аппарата представим карту местности и условия задачи в виде неориентированного взвешенного графа $G = (V, E)$, где $V$ — множество вершин, а $E$ — множество ребер.

Множество вершин графа $V$ формируется из всех заданных точек на местности:
\begin{equation}
    V = \{v_1, v_2, \dots, v_N\} = \{P_{start}\} \cup \{P_i\} \cup \{P_{end}\},
    \label{formula:vertex_set}
\end{equation}
где $N$ — общее количество точек (начальная, конечная и все пункты дозаправки). Каждая вершина $v_k \in V$ характеризуется своими декартовыми координатами $(x_k, y_k)$ в системе координат карты.

Множество ребер $E$. Граф не является полным, так как перемещение возможно не между всеми парами вершин. Ребро $(v_i, v_j)$ существует в множестве $E$ тогда и только тогда, когда физическое расстояние между точками, соответствующими этим вершинам, не превышает максимальную дальность хода $D_{max}$.

Расстояние $L(v_i, v_j)$ между двумя вершинами $v_i = (x_i, y_i)$ и $v_j = (x_j, y_j)$ вычисляется с учетом масштаба карты $M$ по следующей формуле:
\begin{equation}
    L(v_i, v_j) = M \cdot \sqrt{(x_j - x_i)^2 + (y_j - y_i)^2}.
    \label{formula:distance}
\end{equation}
Таким образом, условие существования ребра $(v_i, v_j)$ формально записывается как:
\begin{equation}
    (v_i, v_j) \in E \iff L(v_i, v_j) \le D_{max}.
    \label{formula:edge_condition}
\end{equation}
Поскольку граф является \textit{неориентированным}, расстояние $L(v_i, v_j) = L(v_j, v_i)$.

Каждому ребру $(v_i, v_j) \in E$ ставится в соответствие вес $w(v_i, v_j)$, равный физическому расстоянию между узлами.
\begin{equation}
    w(v_i, v_j) = L(v_i, v_j).
    \label{formula:edge_weight}
\end{equation}

С учетом разработанной модели исходная задача нахождения оптимального маршрута сводится к задаче нахождения пути с минимальной суммой весов ребер от начальной вершины $v_{start}$, соответствующей точке $P_{start}$, до конечной вершины $v_{end}$, соответствующей точке $P_{end}$, в построенном графе $G$.





\clearpage
\section{Алгоритмы и методы решения}
\label{sec:algorithms_and_methods}

В данной главе рассматриваются алгоритмы, использованные для решения поставленных задач. Описывается метод поиска кратчайшего пути в графе, а также алгоритм расчета навигационных параметров и формирования NMEA-сообщений для оптимального маршрута.

\subsection{Алгоритм поиска кратчайшего пути}
\label{subsec:shortest_path_algorithm}

Согласно заданию, для нахождения кратчайшего пути в построенном графе $G=(V, E)$ был применен \textit{алгоритм Форда-Беллмана}. Этот алгоритм предназначен для поиска кратчайшего пути от одной вершины ко всем остальным во взвешенном ориентированном или неориентированном графе. Ключевой особенностью алгоритма является его способность корректно работать с ребрами, имеющими отрицательные веса. В рассматриваемой задаче веса ребер (расстояния) всегда неотрицательны, что не отменяет справедливость применения данного алгоритма.

Основой алгоритма является процесс \textit{релаксации} (ослабления) ребер. Для каждой вершины $v \in V$ хранится значение $d[v]$ — текущая известная длина кратчайшего пути от начальной вершины $v_{start}$ до $v$. Изначально $d[v_{start}] = 0$, а для всех остальных вершин $v \neq v_{start}$ значение $d[v] = \infty$.

Процесс релаксации для ребра $(u, v)$ с весом $w(u, v)$ заключается в проверке условия:
\begin{equation}
    d[v] > d[u] + w(u, v).
    \label{formula:relaxation_condition}    
\end{equation}  % WIP: Добавить рисунок, иллюстрирующий данное неравенство

Если неравенство выполняется, это означает, что найден более короткий путь до вершины $v$ через вершину $u$. В этом случае значение $d[v]$ обновляется:
\begin{equation}
    d[v] := d[u] + w(u, v).
    \label{formula:relaxation_update}    
\end{equation}
Также для восстановления пути сохраняется информация о том, что предшественником $v$ на кратчайшем пути теперь является $u$.

Алгоритм Форда-Беллмана выполняет релаксацию для каждого ребра графа $|V|-1$ раз. Такое количество итераций гарантирует нахождение кратчайшего пути в графе без циклов отрицательного веса.

\subsubsection*{Шаги алгоритма.}  % WIP: Необходимо так же добавить блок-схему алгоритма
\label{subsubsec:algorithm_steps}
\begin{enumerate}
    \item \textit{Инициализация.} Для каждой вершины $v \in V$ устанавливается начальное расстояние $d[v] = \infty$ и предшественник $p[v] = \text{null}$. Для начальной вершины $v_{start}$ устанавливается $d[v_{start}] = 0$.
    
    \item \textit{Релаксация ребер.} Выполняется цикл, который повторяется $|V|-1$ раз. Внутри этого цикла выполняется перебор всех ребер $(u, v) \in E$ и для каждого из них производится операция релаксации.
    
    \item \textit{Проверка на наличие циклов отрицательного веса (опционально).} После $|V|-1$ итерации выполняется еще одна итерация релаксации по всем ребрам. Если на этой итерации удается улучшить путь хотя бы до одной вершины, это свидетельствует о наличии в графе цикла отрицательного веса. В данной задаче этот шаг не является необходимым, так как все веса положительны.
    
    \item \textit{Восстановление пути.} После завершения всех итераций, если путь до конечной вершины $v_{end}$ был найден ($d[v_{end}] \neq \infty$), оптимальный маршрут восстанавливается в обратном порядке, двигаясь от $v_{end}$ к $v_{start}$ по сохраненным предшественникам.
\end{enumerate}

\subsection{Алгоритм расчета параметров маршрута и формирования NMEA-сообщений}
\label{subsec:nmea_generation_algorithm}

После нахождения оптимального пути в виде последовательности вершин необходимо рассчитать детальные параметры движения для каждого сегмента маршрута и сформировать NMEA-подобные сообщения. Согласно заданию, используется специальный формат сообщения \mintinline{x}!$UTHDG!.

\subsubsection*{Формат сообщения.}
\label{subsubsec:message_format}

Каждое сообщение описывает один сегмент пути (перемещение между двумя соседними вершинами в оптимальном маршруте) и имеет следующую структуру: \mintinline{x}!$UTHDG,XX,Y.Y,DD.DD,S1,Z.Z,S2!, где:
\begin{description}
    \item[\mintinline{x}!XX!] — часы с момента старта (целое число).
    \item[\mintinline{x}!Y.Y!] — минуты, прошедшие с начала текущего часа (вещественное число).
    \item[\mintinline{x}!DD.DD!] — длина текущего сегмента маршрута в километрах (вещественное число).
    \item[\mintinline{x}!S1!] — флаг наличия поворота. Принимает значение $T$ (\textit{Turn}), если направление движения изменилось по сравнению с предыдущим сегментом, и $N$ (\textit{No turn}) в противном случае.
    \item[\mintinline{x}!Z.Z!] — азимут, или угол направления движения на данном сегменте, в градусах (вещественное число).
    \item[\mintinline{x}!S2!] — флаг окончания всего маршрута. Принимает значение $E$ (\textit{End}) для последнего сегмента пути и $N$ (\textit{Not end}) для всех остальных.
\end{description}

\subsubsection*{Алгоритм формирования сообщений.}
\label{subsubsec:message_generation_algorithm}

Для каждого сегмента пути между вершинами $v_i=(x_i, y_i)$ и $v_{i+1}=(x_{i+1}, y_{i+1})$ из оптимального маршрута производятся следующие вычисления:

\begin{enumerate}
    \item \textit{Расчет длины сегмента (\texttt{DD.DD}):}
    Длина сегмента вычисляется как евклидово расстояние между точками с учетом масштаба карты $M$:
    \begin{equation}
        DD = M \cdot \sqrt{(x_{i+1} - x_i)^2 + (y_{i+1} - y_i)^2}.
        \label{formula:segment_length}
    \end{equation}
    
    \item \textit{Расчет времени в пути (\texttt{XX}, \texttt{Y.Y}):}
    Сначала вычисляется время, необходимое для прохождения текущего сегмента: $\Delta t = DD / v$. Затем это время добавляется к общему времени, накопленному с начала движения, $T_{total}$.
    \begin{equation}
        T_{total} := T_{total} + \Delta t.
        \label{formula:total_time_update}
    \end{equation}
    Компоненты \texttt{XX} и \texttt{Y.Y} вычисляются из $T_{total}$:
    \begin{align}
        XX &= \lfloor T_{total} \rfloor, \\
        YY &= (T_{total} - XX) \cdot 60.
        \label{formula:time_components}
    \end{align}
    
    \item \textit{Расчет азимута (\texttt{Z.Z}):}
    Азимут (угол направления) вычисляется на основе приращений координат $\Delta x = x_{i+1} - x_i$ и $\Delta y = y_{i+1} - y_i$. Для корректного определения угла во всех четвертях используется арктангенс с двумя аргументами:
    \begin{equation}
        \theta_{\text{rad}} = \operatorname{atan2}(\Delta y, \Delta x) = \operatorname{rad2deg}(\theta_{\text{rad}}).
        \label{formula:azimuth}
    \end{equation}
    Полученное значение в радианах переводится в градусы и приводится к диапазону $[0, 360^\circ)$:
    \begin{equation}
        \theta_{\text{deg}} = \theta_{\text{rad}} \cdot \frac{180}{\pi}.
        \label{formula:azimuth_degrees}
    \end{equation}
    Если $\theta_{\text{deg}} < 0$, то $\theta_{\text{deg}} := \theta_{\text{deg}} + 360$. Итоговое значение является азимутом $ZZ$.
    
    \item \textit{Определение флагов (\texttt{S1}, \texttt{S2}):}
    Флаги определяются на основе логических проверок:
    \begin{itemize}
        \item $S1 = T$, если азимут текущего сегмента $ZZ_i$ не равен азимуту предыдущего сегмента $ZZ_{i-1}$. В противном случае $S1 = N$. Для первого сегмента флаг всегда $T$.
        \item $S2 = E$, если текущий сегмент является последним в маршруте. В противном случае $S2 = N$.
    \end{itemize}
\end{enumerate}




\clearpage
\section{Программная реализация и анализ результатов}
\label{section:implementation_and_results}

В данной главе описывается структура разработанного программного комплекса, среда разработки, а также представлены и проанализированы результаты, полученные в ходе выполнения программы с заданными начальными условиями.

\subsection{Структура программного комплекса}
\label{subsec:software_structure}

Программный комплекс был разработан в среде математического моделирования \textit{Matlab}. Выбор данной среды обусловлен ее широкими возможностями для работы с матрицами, удобными средствами для визуализации данных и наличием встроенных функций для реализации сложных вычислительных алгоритмов.

Проект имеет модульную структуру, что обеспечивает гибкость и упрощает его понимание и возможную дальнейшую модификацию. Структура включает один главный исполняемый скрипт и набор вспомогательных функций, каждая из которых решает конкретную подзадачу.

\begin{description}
    \item[\texttt{main.m}] — главный скрипт программы. Он выполняет следующие действия:
    \begin{itemize}
        \item Инициализация начальных условий (стандартных или введенных пользователем).
        \item Чтение координат заправочных пунктов из файла данных.
        \item Последовательный вызов вспомогательных функций для построения графа, поиска пути, генерации NMEA-сообщений и визуализации.
        \item Сохранение и вывод итоговых результатов.
    \end{itemize}
    (См. \cref{appendix:main.m}).

    \item[\texttt{buildGraph.m}] — функция, отвечающая за реализацию математической модели. На вход она получает массив координат всех точек и параметры объекта, а на выходе формирует структуру графа, содержащую матрицу смежности и матрицу весов (расстояний) ребер в соответствии с ограничением по дальности хода. (См. \cref{appendix:buildGraph.m}).

    \item[\texttt{findOptimalPathByFordBellman.m}] — функция, реализующая алгоритм Форда-Беллмана для поиска кратчайшего пути. В качестве входных данных принимает структуру графа и индексы начальной и конечной вершин. Возвращает последовательность индексов вершин оптимального пути и его общую длину. (См. \cref{appendix:findOptimalPathByFordBellman.m}).

    \item[\texttt{generateNmeaMessages.m}] — функция для генерации NMEA-сообщений. На основе найденного оптимального пути и координат узлов она рассчитывает параметры движения для каждого сегмента и формирует массив строк в формате \mintinline{x}!$UTHDG!. (См. \cref{appendix:generateNmeaMessages.m}).

    \item[\texttt{plotRoute.m}] — функция для графической визуализации. Она строит график, на котором отображаются все узлы, все возможные пути (ребра графа), а также выделяет цветом найденный оптимальный маршрут, начальную и конечную точки. (См. \cref{appendix:plotRoute.m}).

    \item[\texttt{saveResults.m}] — функция, сохраняющая результаты работы программы (длину пути и NMEA-сообщения) в текстовый файл. (См. \cref{appendix:saveResults.m}).
\end{description}  % WIP: Приложения должны существовать

Такая декомпозиция на функции соответствует принципам структурного программирования и делает код более читаемым и поддерживаемым.

\subsection{Тестирование и результаты работы}
\label{subsec:testing_and_results}

Программный комплекс был протестирован с использованием стандартных начальных условий, указанных в файле \texttt{main.m} и в постановке задачи (\cref{table:source_data}). Координаты заправочных пунктов были загружены из файла \mintinline{x}!refueling_points.txt!. (Также существует файл \mintinline{x}!refueling_points_extended.txt!, особенностью которого являются нецелочисленные координаты заправочных станций).

\subsubsection*{Итоговые расчетные данные.}
\label{subsubsec:final_computed_data}

После выполнения программы были получены следующие результаты. Общая длина оптимального маршрута от начальной точки $(-9, 7)$ до конечной точки $(8, -6)$ составила:
\begin{equation}
    L_{optimal} = 922.93 \text{ км}.
    \label{formula:L_optimal}
\end{equation}

Сгенерированные NMEA-сообщения, описывающие каждый сегмент оптимального пути, и длина оптимального пути представлены в \cref{lst:nmea_messages}.

\begin{listing}[H]
    \caption{Сгенерированный результирующий отчёт}
    \label{lst:nmea_messages}
    \inputminted[
        % firstline      = 4,
        % lastline       = 10,
        % highlightlines = {1-10},
    ]{txt}{code/results/result.txt}
\end{listing}




\subsubsection*{Анализ визуализации}
\label{subsubsec:visualization_analysis}

Результат графической визуализации маршрута представлен на \cref{fig:optimal_route}.  % WIP

\begin{figure}[hb]
    \centering
    \includegraphics[width=1.0\linewidth]{code/results/optimal_route.png}
    \caption{Визуализация маршрута на графе}
    \label{fig:optimal_route}
\end{figure}

На рисунке черными точками обозначены все возможные узлы (заправочные станции), зеленой и пурпурной --- стартовая и конечная соответственно. Тонкими пунктирными линиями показаны все возможные перемещения между узлами, удовлетворяющие ограничению по дальности хода ($D_{max} \le 126$ км). Найденный оптимальный путь выделен жирной красной линией.

Как видно из графа, алгоритм успешно нашел непрерывный маршрут от старта к финишу. Маршрут состоит из последовательности отрезков, длина каждого из которых не превышает максимально допустимую. Алгоритм обходит некоторые короткие, но ведущие в неверном направлении пути, выбирая ту последовательность узлов, которая в сумме дает минимальное итоговое расстояние. Визуализация подтверждает корректность работы реализованного алгоритма поиска кратчайшего пути.