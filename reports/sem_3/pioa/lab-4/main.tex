\section*{Цель работы.}
Изучение работы с функциями в математическом пакете MatLab.

\section*{Задания.}
\begin{enumerate}
    \item Создать функцию в соответствии с \cref{fig:source}. Сделать ее нечетной (посколько график \cref{fig:source} проходит через точку $(0, 0)$).
    \item Создать прототип функции, сохраняющей внешний вид рисунка из \cref{fig:source}, но с произвольными числами, с учетом нечетности.
    \item Создать функции для выполнения операций из \cref{table:source}, но с произвольными матрицами. Добавить в функции вариативность действий (сумма/произведение).
    \item Создать скрипт для вызова функций из п. 1-3 и вывода результатов.
\end{enumerate}

\section*{Требования к выполнению заданий.}
\begin{enumerate}
    \item Для выполнения п. 2 нужно создать функцию, возвращающую объект типа \textit{function\_handle}.
    \item Функция для поиска максимальных/минимальных элементов должна быть адаптирована к разному количеству выходных аргументов, и уметь возвращать сам элемент/строку/столбец, и его индекс/номер.
    \item Функция для вычисления частных сумм/произведений должна иметь дополнительный входной символьный аргумент: если символ \mintinline{text}!‘s’! --- нужно считать сумму, если \mintinline{text}!‘p’! --- произведение.
    \item Вычисление сумм и произведений должно быть реализовано в виде локальных функций.
\end{enumerate}

\begin{figure}[b]
    \centering
    \includegraphics[width=0.4\textwidth]{figs/source.png}
    \caption{Исходный график кусочно-линейной функции}
    \label{fig:source}
\end{figure}

\section*{Экспериментальные результаты.}
\subsection*{ \boxed{\text{ Задание 1. }} }

\begin{quote}
    \textit{Создать функцию в соответствии с \cref{fig:source}. Сделать ее нечетной (посколько график \cref{fig:source} проходит через точку $(0, 0)$). }
\end{quote}

\begin{listing}[H]
    \caption{Нечетный вариант функции из \cref{fig:source}}
    \label{lst:ex1.m}
    \inputminted[
        % firstline      = 1,
        % lastline       = 10,
        % highlightlines = {1-10},
    ]{matlab}{code/ex1.m}
\end{listing}

В данном задании используется функция \textit{sign}, которая в данном случае используется как инструмент для получения знака переменной \textit{text}.
Её использование вместе с \textit{abs} позволяет с относительной простотой отражать положительную часть функции относительно начала координат для получения нечетной функции.


\subsection*{ \boxed{\text{ Задание 2. }} }

\begin{quote}
    \textit{Создать прототип функции, сохраняющей внешний вид рисунка из \cref{fig:source}, но с произвольными числами, с учетом нечетности.}
\end{quote}

\begin{table}[hbt]
    \centering
    \caption{Исходные данные для варианта 3}
    \label{table:source}
    \begin{tabularx}{\linewidth}{cCC}
        Исходные данные & Результат поиска & Результат вычисления \\
        \toprule

        $$
            \begin{bmatrix}
                -1.1 & 4    & -11 & 6    \\
                3    & 0    & 5   & 2.3  \\
                6.2  & -0.3 & 3   & -6.4
            \end{bmatrix}
        $$
        & столбец с минимальным элементом & накопительная сумма элементов столбцов \\
    \end{tabularx}
\end{table}

\begin{listing}[H]
    \caption{Прототип функции}
    \label{lst:ex2.m}
    \inputminted[
        % firstline      = 1,
        % lastline       = 10,
        highlightlines = {6-8},
    ]{matlab}{code/ex2.m}
\end{listing}

Реализация задания 1 в функции для задания 2 заключается в параметризации всех констант, которые определяют вид функции.
Были добавлены параметры для интервалов разрывов функции $a$ и $b$ (точки $(2, -2)$ и $(6,-2)$ соотв. на \cref{fig:source}), а также значений горизонтальных участков $m$ и $n$ (промежутки $[2; 6]$ и $[6; 8]$ на \cref{fig:source}).

Примечательно, что функция возвращает свой прототип (конструкция \mintinline{text}!f = @ (x) ...!). Пример использования функции приведен на строках 6--8 в \cref{lst:ex2.m}. 

Это позволяет вызывать функцию с любыми значениями этих параметров, сохраняя при этом общий вид функции, представленный на \cref{fig:source}.


\subsection*{ \boxed{\text{ Задание 3. }} }
\begin{quote}
    \textit{Создать функции для выполнения операций из \cref{table:source}, но с произвольными матрицами. Добавить в функции вариативность действий (сумма/произведение).}
\end{quote}

\begin{listing}[H]
    \caption{Функция поиска над произв. матрицей}
    \label{lst:ex3_1.m}
    \inputminted[
        % firstline      = 1,
        % lastline       = 10,
        % highlightlines = {1-10},
    ]{matlab}{code/ex3_1.m}
\end{listing}

\begin{codemultipage}
    \captionof{listing}{Функция подсчета над произв. матрицей\label{lst:ex3_2.m}}
    \inputminted{matlab}{code/ex3_2.m}
\end{codemultipage}

Для создания данных функций за основу были взяты наработки из предыдущих лабораторных работ.
В \cref{lst:ex3_1.m} представлена функция нахождения минимального элемента и его столбца. Она поддерживает переменное число выходных аргументов. В случае, если он один, фукнция запишет в него структуру с полями \textit{value} (значение минимального элемента) и \textit{column} (номер столбца, в котором находится минимальный элемент). Если выходных аргументов два, то в первый будет записан минимальный элемент, а во второй --- номер столбца.

Вторая функция, представленная в \cref{lst:ex3_2.m}, реализует вычисление частичных сумм или произведений над столбцами матрицы в зависимости от входного символьного аргумента \textit{operation}. Если он равен \mintinline{text}!'sum'!, то будет вычислена накопительная сумма, если \mintinline{text}!'mul'! --- накопительное произведение.

\subsection*{ \boxed{\text{ Задание 4. }} }
\begin{quote}
    \textit{Создать скрипт для вызова функций из п. 1-3 и вывода результатов.}
\end{quote}

\begin{codemultipage}
    \captionof{listing}{Скрипт для работы с функциями\label{lst:ex4.m}}
    \inputminted{matlab}{code/ex4.m}
\end{codemultipage}

В данном скрипте под функцией \mintinline{text}!ex3! понимается функция из задания 3, а под \mintinline{text}!ex2! --- функция из задания 2.
Алгоритм запрашивает входные параметры для функций у пользователя, вызывает функции с этими параметрами, а затем выводит результаты на экран.

\section*{Выводы.}

Алгоритм работает исправно и корректно выполняет все поставленные задачи (см. \cref{fig:result.png}).

Были решены все задачи. Цель лабораторной работы была успешно достигнута, продемонстрировав владение всеми необходимыми знаниями для работы с функциями в MatLab.

\begin{figure}[hbt]
    \centering
    \includegraphics[width=\linewidth, height=0.75\textheight, keepaspectratio]{figs/result.png}
    \caption{Вызов программы}
    \label{fig:result.png}
\end{figure}

\begin{figure}[hbt]
    \centering
    \includegraphics[width=1.00\textwidth]{figs/result_figure.png}
    \caption{График при параметрах из \cref{fig:result.png}}
    \label{fig:result_figure.png}
\end{figure}

