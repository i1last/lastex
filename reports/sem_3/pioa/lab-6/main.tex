\section*{Цель работы.}
Изучение методов работы с объектно-ориентированной графикой в среде MATLAB.

\newcommand{\exOne}{В соответствие с вариантом задания (\cref{table:9_var3}) составить блок-схему алгоритма определения произвольного решения уравнения $ax+b=f(x)$.}

\newcommand{\exTwo}{Написать и отладить программу решения уравнения графическим способом. Оценить точность решений. В случае низкой точности (более 0.01) найти численное решение.}

\newcommand{\exThree}{Отобразить решения, удовлетворяющие требуемой точности, средствами объектно-ориентированной графики.}

\newcommand{\exFour}{Подобрать коэффициенты таким образом, чтобы уравнение имело единственное решение.}

\section*{Задания и требоования к их выполнению.}
\begin{enumerate}
    \item \exOne
    \item \exTwo
    \item \exThree
    \item \exFour
\end{enumerate}

\begin{table}[hb]
    \centering
    \caption{Условия для варианта 3}
    \label{table:9_var3}

    \begin{tblr}{
        colspec = {| Q[c, m] | X[c, m] | X[c, m] | X[c, m] | },
        hlines = 1, vlines = 1,
        row{1} = {font=\bfseries},
    }
        № варианта & $\bm{a}$ & $\bm{b}$ & $\bm{f(x)}$ \\
        3 & -0.6 & 4.6 & $\sin\frac{x}{2} - \cos3x+3$ \\
    \end{tblr}
\end{table}



\section*{Экспериментальные результаты.}
\subsection*{ \boxed{\text{ Задание 1. }} }

\begin{quote}
    \textit{\exOne}
\end{quote}

На \cref{fig:block_diagram} представлена блок-схема алгоритма, реализованного для решения поставленной задачи. Алгоритм включает в себя инициализацию параметров, построение графиков для визуального анализа, численное нахождение корней с использованием начальных приближений и отображение результатов.

\begin{figure}[H]
    \centering
    \includegraphics[width=1\textwidth]{shemes/diagram.png}
    \caption{Блок-схема алгоритма решения уравнения $ax+b=f(x)$}
    \label{fig:block_diagram}
\end{figure}

\subsection*{ \boxed{\text{ Задание 2. }} }
\begin{quote}
    \textit{\exTwo}
\end{quote}

Для решения уравнения была разработана программа \texttt{main.m} (\cref{lst:main.m}). Первым шагом является графическое решение: программа строит графики левой ($y_1(x) = ax+b$) и правой ($y_2(x) = f(x)$) частей уравнения. Визуальный анализ графиков позволяет определить количество решений и их примерное расположение.

Поскольку графический метод дает низкую точность, для нахождения корней с точностью не хуже $0.01$ применяется численный метод. В программе используется встроенная функция \textit{fzero}, которая находит корень уравнения $g(x) - f(x) = 0$ на основе начального приближения, полученного с графика.

Точность каждого найденного решения $x_{sol}$ оценивается путем вычисления абсолютной разницы $|g(x_{sol}) - f(x_{sol})|$. Результаты вычислений и проверки точности выводятся в командное окно (\cref{fig:callMain}).

\begin{codemultipage}
    \captionof{listing}{Код основной программы\label{lst:main.m}}
    \nopagebreak
    \inputminted{matlab}{code/main.m}
\end{codemultipage}

\begin{figure}[h!]
    \centering

    \begin{subfigure}[a]{0.48\textwidth}
        \centering
        \includegraphics[width=\textwidth]{figs/callMain_part1.png}
        % \caption{caption1}
        \label{fig:callMain_part1.png}
    \end{subfigure}
    \hfill
    \begin{subfigure}[a]{0.48\textwidth}
        \centering
        \includegraphics[width=\textwidth]{figs/callMain_part2.png}
        % \caption{caption2}
        \label{fig:callMain_part2.png}
    \end{subfigure}

    \caption{Процесс вызова main.m}
    \label{fig:callMain}
\end{figure}




\subsection*{ \boxed{\text{ Задание 3. }} }
\begin{quote}
    \textit{\exThree}
\end{quote}

Программа, представленная в \cref{lst:main.m}, является решением задач 2 и 3. Она была модифицирована для использования указателей (handles) на графические объекты.

При создании окна (\texttt{figure}), осей (\texttt{axes}) и линий (\texttt{plot}) их указатели сохраняются в переменные. Далее, для настройки внешнего вида графика используется функция \texttt{set}, которая принимает указатель на объект и пары \textit{"свойство-значение"}. Такой подход позволяет детально управлять каждым элементом графика: изменять цвет и стиль линий, настраивать вид маркеров для найденных решений, форматировать сетку и подписи.

Результат работы программы представлен на \cref{fig:plotMain.png}. На графике четко видны исходные функции и найденные точки их пересечения, оформленные с помощью объектно-ориентированного подхода.

\begin{figure}[H]
    \centering
    \includegraphics[width=1\textwidth]{figs/plotMain.png}
    \caption{Результат работы main.m}
    \label{fig:plotMain.png}
\end{figure}

\subsection*{ \boxed{\text{ Задание 4. }} }
\begin{quote}
    \textit{\exFour}
\end{quote}

Для того чтобы уравнение имело единственное решение, необходимо подобрать коэффициенты $a$ и $b$ соответствующим образом. Функция $f(x)$ является ограниченной и колеблющейся, а ее наклон (производная) также имеет ограниченное значение. Прямая $g(x)=ax+b$ имеет постоянный наклон $a$.

Если выбрать наклон прямой $a$ таким образом, чтобы его абсолютное значение было значительно больше максимального наклона кривой $f(x)$, то прямая гарантированно пересечет медленно изменяющуюся кривую только один раз. Коэффициент $b$ при этом будет лишь смещать точку пересечения.

Был написан скрипт \texttt{oneRoot.m} (\cref{lst:oneRoot.m}), в котором был выбран коэффициент $a = -8$. Результат его работы показан на \cref{fig:plotOneRoot.png}, где наглядно продемонстрировано наличие единственного решения.

\begin{listing}[H]
    \caption{Единственное пересечение графиков}
    \label{lst:oneRoot.m}
    \inputminted[
        % firstline      = 1,
        % lastline       = 10,
        % highlightlines = {1-10},
    ]{matlab}{code/oneRoot.m}
\end{listing}

\begin{figure}[H]
    \centering
    \includegraphics[width=1\textwidth]{figs/plotOneRoot.png}
    \caption{Результат работы oneRoot.m}
    \label{fig:plotOneRoot.png}
\end{figure}


\section*{Выводы.}
В ходе выполнения данной лабораторной работы была достигнута поставленная цель по изучению методов работы с объектно-ориентированной графикой в среде MatLab.

Было решено уравнение $\left[g(x) \coloneqq ax+b\right]=\left[f(x) \coloneqq \sin\frac{x}{2} - \cos3x+3 \right]$ с использованием комбинированного подхода: начальный графический анализ для определения количества и примерного расположения корней, и последующее численное уточнение с помощью функции \textit{fzero} для достижения требуемой точности.

С помощью получения указателей на графические объекты и использования функции \textit{set} удалось осуществить детальную настройку внешнего вида графика, включая стили линий, маркеры решений и параметры осей.

Также были успешно подобраны коэффициенты уравнения для обеспечения существования единственного решения, что было подтверждено графически. Работа продемонстрировала эффективность инструментов MatLab для решения инженерных и математических задач.